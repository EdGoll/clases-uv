\documentclass{beamer}

\mode<presentation>
{
	\usetheme{CambridgeUS}
	\setbeamercovered{transparent}
}
\usepackage[spanish]{babel}
\usepackage[latin1]{inputenc}
\usepackage{color}
\usepackage{hyperref}
\usepackage{multicol}
\usepackage{algorithm,algorithmic}

\title[\textbf{ICI 4242 - Aut\'omatas y compiladores}]{\textbf{ICI 4242 - Aut\'omatas y compiladores}}

\subtitle{Expresiones regulares}

\author[Rodrigo Olivares]
{
	Rodrigo Olivares \\
	\vspace{0.5mm}
	Mg. en Ingenier\'ia Inform\'atica \\
	\vspace{0.5mm}
	\texttt{\normalsize rodrigo.olivares@uv.cl}
}

\institute[PUCV]

\date{1er Semestre} 

\subject{Expresiones regulares}

%\AtBeginSection
%{
%	\begin{frame}<beamer>
%	\frametitle{Contenido}
%	\tableofcontents[currentsection,currentsubsection]
%	\end{frame}
%}
%
%\AtBeginSubsection
%{
%	\begin{frame}<beamer>
%	\frametitle{Contenido}
%	\tableofcontents[currentsection,currentsubsection]
%	\end{frame}
%}
%
%\beamerdefaultoverlayspecification{<+->}

\begin{document}

	\begin{frame}
		\titlepage
	\end{frame}

%	\begin{frame}
%		\frametitle{Contenido}
%		\tableofcontents%[pausesections]
%	\end{frame}

	\section{Concepto}

		\subsection{Definici\'on}

		\begin{frame}
			\frametitle{Concepto}
			\framesubtitle{Definici\'on}

			\begin{block}{Objetivo}
			    El objetivo de las expresiones regulares es representar todos los posibles lenguajes definidos sobre un alfabeto $\Sigma$, en base a una serie de lenguajes primitivos, y operadores de composici\'on.
			\end{block}
		\end{frame}

		\begin{frame}
			\frametitle{Concepto}
			\framesubtitle{Definici\'on}

			\begin{block}{Lenguajes primitivos}
			    \begin{itemize}
			        \item[-] El lenguaje vac\'io.
			        \item[-] El lenguaje formado por la palabra vac\'ia.
			        \item[-] Los lenguajes correspondientes a los distintos s\'imbolos del alfabeto.
			    \end{itemize}
			\end{block}
			\begin{block}{Operadores de composici\'on}
			    \begin{itemize}
			        \item[-] La uni\'on 
			        \item[-] La concatenaci\'on.
			        \item[-] La clausura o cierre.
			    \end{itemize}
			\end{block}
		\end{frame}

		\begin{frame}
			\frametitle{Concepto}
			\framesubtitle{Definici\'on}

			\begin{exampleblock}{Ejemplo}
			    \begin{itemize}
			        \item[1.-] Lenguaje formado por las cadenas que terminan en 01:
			        \begin{table}[H]
			            \begin{center}
			                \begin{tabular}{ccl}
			                    $\{0,1\}^{*}.\{01\}$ & $=$ & \\
			                    & $=$ & $(\{0\} \cup \{1\})^{*}.\{01\}$  \\
			                    & $=$ & $(0+1)^{*}01$ \emph{Expresi\'on regular}
			                \end{tabular}
			            \end{center}
			        \end{table}
			        \item[2.-] Lenguaje formado por palabras de longitud par sobre a's y b's:
			        \begin{table}[H]
			            \begin{center}
			                \begin{tabular}{ccl}
			                    $\{aa,ab,ba,bb\}^{*}$ & $=$ & \\
			                    & $=$ & $(\{aa\} \cup \{ab\} \cup \{ba\} \cup \{bb\})^{*}$  \\
			                    & $=$ & $(aa+ab+ba+bb)^{*}$ \emph{Expresi\'on regular}
			                \end{tabular}
			            \end{center}
			        \end{table}
			    \end{itemize}
			\end{exampleblock}
		\end{frame}

		\begin{frame}
			\frametitle{Concepto}
			\framesubtitle{Definici\'on}

			\begin{block}{Definici\'on}
			   \begin{itemize}
			       \item[] Dado un alfabeto $\Sigma$, las expresiones regulares sobre $\Sigma$ se definen de forma recursiva por las siguientes reglas:
			       \begin{itemize}
			           \item[1.-] Las siguientes expresiones son expresiones primitivas:
			           \begin{itemize}
			               \item[-] $\O$
			               \item[-] $\lambda$
			               \item[-] $\sigma$, siendo $\sigma \in \Sigma$
			           \end{itemize}
			           \item[2.-] Sean $\alpha$ y $\beta$ expresiones regulares, entonces son expresiones regulares derivadas:
			           \begin{itemize}
			               \item[-] $\alpha + \beta$ (uni\'on)
			               \item[-] $\alpha.\beta$ (o simplemente $\alpha\beta$) (concatenaci\'on)
			               \item[-] $\alpha^{*}$ (cierre)
			               \item[-] ($\alpha$)
			           \end{itemize}
			           \item[] No hay m\'as expresiones regulares sobre $\Sigma$ que las construidas mediante estas reglas.
			       \end{itemize}
			   \end{itemize}
			\end{block}
		\end{frame}

        \begin{frame}
			\frametitle{Concepto}
			\framesubtitle{Definici\'on}

			\begin{block}{Precedencia de operadores}
			    \begin{itemize}
			        \item[1.-] ()
			        \item[2.-] $^{*}$ (cierre)
			        \item[3.-] . (concatenaci\'on)
			        \item[4.-] $+$ (uni\'on)
			    \end{itemize}
			\end{block}

			\begin{exampleblock}{Algunos ejemplos de expresiones regulares}
			   \begin{itemize}
			       \item[-] $(0+1)^{*}01$
			       \item[-] $(aa+ab+ba+bb)^{*}$
			       \item[-] $a^{*}(a+b)$
			       \item[-] $(aa)^{*}(bb)^{*}b$
			   \end{itemize}
			\end{exampleblock}
		\end{frame}

        \subsection{Lenguaje descrito}

        \begin{frame}
			\frametitle{Concepto}
			\framesubtitle{Lenguaje descrito}

			\begin{block}{Lenguaje descrito - Definici\'on}
			    Sea \textbf{r} una expresi\'on regular sobre $\sigma$. El \textbf{lenguaje descrito por r}, $L(r)$, se define recursivamente de la siguiente forma:
			    \begin{table}[H]
			            \begin{center}
			                \begin{tabular}{rlllll}
			                    1.- & Si \textbf{r} = $\O$                         & $\Rightarrow$ & $L(\O)$                     & = & $\O$\\
			                    2.- & Si \textbf{r} = $\lambda$                 & $\Rightarrow$ & $L(\lambda)$             & = & $\{\lambda\}$\\
			                    3.- & Si \textbf{r} = $a$, $a \in \Sigma$   & $\Rightarrow$ & $L(a)$                        & = & $\{a\}$ \\
			                    4.- & Si \textbf{r} = $\alpha + \beta$        & $\Rightarrow$ & $L(\alpha + \beta)$    & = & $L(\alpha) \cup L(\beta)$ \\
			                    5.- & Si \textbf{r} = $\alpha.\beta$           & $\Rightarrow$ & $L(\alpha.\beta)$       & = & $L(\alpha).L(\beta)$ \\
			                    6.- & Si \textbf{r} = $\alpha^{*}$              & $\Rightarrow$ & $L(\alpha^{*})$          & = & $L(\alpha)^{*}$\\
			                    7.- & Si \textbf{r} = $(\alpha)$                 & $\Rightarrow$ & $L((\alpha))$              & = & $L(\alpha)$ \\
			                \end{tabular}
			            \end{center}
			        \end{table}
			        donde $\alpha$ y $\beta$ son expresiones regulares.
			\end{block}
		\end{frame}

        \begin{frame}
			\frametitle{Concepto}
			\framesubtitle{Lenguaje descrito}

			\begin{exampleblock}{Ejemplo}
			    Mostrar el lenguaje descrito por una ER mediante notaci\'on por comprensi\'on (conjuntista):
			    \begin{table}[H]
			            \begin{center}
			                \begin{tabular}{rcl}
			                    $L(a^{*}(a+b))$ 
			                    & = & $L(a^{*})L((a+b))$ \\
			                    & = & $L(a)^{*}L(a+b)$ \\
			                    & = & $L(a)^{*}(L(a) \cup L(b))$ \\
			                    & = & $\{a\}^{*}(\{a\} \cup \{b\})$ \\
			                    & = & $\{\lambda,a,aa,aaa,\ldots\}\{a,b\}$ \\
			                    & = & $\{a,aa,aaa,\ldots,b,ab,aab,aaab,\ldots\}$ \\
			                    & = & $\{a^{n}~|~n \geq 1\} \cup \{a^{n}b~|~n \geq 0\}$ \\
			                \end{tabular}
			            \end{center}
			        \end{table}
			\end{exampleblock}
		\end{frame}

        \begin{frame}
			\frametitle{Concepto}
			\framesubtitle{Lenguaje descrito}

			\begin{exampleblock}{Ejercicios}
			    Mostrar el lenguaje descrito por una ER mediante notaci\'on por comprensi\'on (conjuntista):
			    \begin{itemize}
			        \item[1.-] $L((aa)^{*}(bb)^{*}b)$ \textquestiondown~= ? $\{a^{2n}b^{2m+1}~|~n,m \geq 0\}$.
			        \item[2.-] Si $\Sigma = \{a,b,c\} \Rightarrow L((a+b+c)^{*})$ \textquestiondown~= ? $\Sigma^{*}$.
			        \item[3.-] $L(a^{*}.(b+c))$.
			        \item[4.-] $L(0^{*}.1.0^{*})$.
			        \item[5.-] $L((a+b+c+\cdots+z)^{*}.(a+b)^{*})$.
			    \end{itemize}
			\end{exampleblock}
		\end{frame}

        \subsection{Propiedades}

        \begin{frame}
			\frametitle{Concepto}
			\framesubtitle{Propiedades}

			\begin{exampleblock}{Definici\'on - Equivalencia}
			    Dos expresiones regulares $r_{1}$ y $r_{2}$ se dicen equivalentes, $r_{1} = r_{2}$, si describen el mismo lenguaje, esto es, si $L(r_{1})=L(r_{2})$. En base a esta definici\'on se pueden establecer las siguientes equivalencias y propiedades:
			\end{exampleblock}
		\end{frame}

        \begin{frame}
			\frametitle{Concepto}
			\framesubtitle{Propiedades}

			\begin{block}{Definici\'on - Equivalencia}
			    \begin{itemize}
			        \item[] Respecto a las operaciones + (uni\'on) y . (concatenaci\'on):
			        \begin{itemize}
			            \item[1.-] Son asociativas: 
			            \begin{itemize}
			                \item[$\rightarrow$] $\alpha + (\beta + \gamma) = (\alpha + \beta) + \gamma = \alpha + \beta + \gamma$
			                \item[$\rightarrow$] $\alpha.(\beta.\gamma) = (\alpha.\beta).\gamma = \alpha.\beta.\gamma$
			            \end{itemize}
			            \item[2.-] $+$ es conmutativa e idempotente: 
			            \begin{itemize}
			                \item[$\rightarrow$] $\alpha + \beta = \beta + \alpha$
			                \item[$\rightarrow$] $\alpha + \alpha = \alpha$
			            \end{itemize}
			            \item[3.-] Distributividad:
			            \begin{itemize}
			                \item[$\rightarrow$] $\alpha.(\beta + \gamma)=\alpha.\beta+\alpha.\gamma$
			                \item[$\rightarrow$] $(\alpha + \beta).\gamma = \alpha.\gamma + \beta.\gamma$
			            \end{itemize}
			        \end{itemize}
			    \end{itemize}
			\end{block}
		\end{frame}

        \begin{frame}
			\frametitle{Concepto}
			\framesubtitle{Propiedades}

			\begin{block}{Definici\'on - Equivalencia}
			    \begin{itemize}
			        \item[] Respecto a las operaciones + (uni\'on) y . (concatenaci\'on):
			        \begin{itemize}
			            \item[4.-] Elemento neutro:
			            \begin{itemize}
			                \item[$\rightarrow$] $\alpha.\lambda = \lambda.\alpha = \alpha$
			                \item[$\rightarrow$] $\alpha + \O = \O + \alpha = \alpha$
			            \end{itemize}
			            \item[5.-] Lenguaje Vac\'io:
			            \begin{itemize}
			                \item[$\rightarrow$] $\O.\alpha = \alpha.\O = \O$
			            \end{itemize}
			            \item[6.-] Uni\'on vac\'ia:
			            \begin{itemize}
			                \item[$\rightarrow$] Si $\lambda \in L(\alpha)$, entonces $\alpha + \lambda = \alpha $
			            \end{itemize}
			        \end{itemize}
			    \end{itemize}
			\end{block}
		\end{frame}

        \begin{frame}
			\frametitle{Concepto}
			\framesubtitle{Propiedades}

			\begin{block}{Definici\'on - Equivalencia}
			    \begin{itemize}
			        \item[] Respecto a la operaci\'on * (cierre o clausura):
			        \begin{itemize}
			            \item[7.-] $\alpha^{*} = \lambda + \alpha.\alpha^{*}$
			            \item[8.-] $\lambda^{*} = \lambda$
			            \item[9.-] $\O^{*} = \lambda$
			            \item[10.-] $\alpha^{*}.\alpha^{*} = \alpha^{*}$
			            \item[11.-] $\alpha.\alpha^{*} = \alpha^{*}.\alpha$
			            \item[12.-] $(\alpha^{*})^{*} = \alpha^{*}$
			            \item[13.-] $(\alpha^{*}+\beta^{*})^{*} = (\alpha^{*}.\beta^{*})^{*} = (\alpha + \beta)^{*} = (\alpha^{*}.\beta)^{*}.\alpha^{*}$
			            \item[14.-] $(\alpha.\beta)^{*}.\alpha = \alpha.(\beta.\alpha)^{*}$
			        \end{itemize}
			    \end{itemize}
			\end{block}
		\end{frame}

        \begin{frame}
			\frametitle{Concepto}
			\framesubtitle{Propiedades}

			\begin{exampleblock}{Ejemplos}
			    Para comprobar si dos expresiones son equivalentes se puede intentar transformarlos mediante estas reglas en una misma expresi\'on. $\Sigma=\{a,b,c\}$ 
			    \begin{table}[H]
			                \begin{tabular}{rcll}
			                   \multicolumn{4}{l}{$c^{*}.c+c^{*}$ \textquestiondown~= ? $c^{*}$} \\ 
			                   & & & \\
			                    $c^{*}.c + c^{*}$
			                    & = & $c^{*}.c + c^{*} + \lambda$ & (por 6) \\
			                    & = & $c.c^{*} + c^{*} + \lambda$ & (por 11) \\
                               & = & $\lambda + c.c^{*} + c^{*}$ & (por 2) \\
                               & = & $c^{*} + c^{*}$ & (por 7) \\
                               & = & $c^{*}$ & (por 2) \\
			                \end{tabular}
			        \end{table}
			\end{exampleblock}
		\end{frame}

        \begin{frame}
			\frametitle{Concepto}
			\framesubtitle{Propiedades}

			\begin{exampleblock}{Ejemplos}
			    Para comprobar si dos expresiones son equivalentes se puede intentar transformarlos mediante estas reglas en una misma expresi\'on. $\Sigma=\{a,b,c\}$ 
			    \begin{table}[H]
			                \begin{tabular}{rcll}
			                   \multicolumn{4}{l}{$c+c^{*}$ \textquestiondown~= ? $c^{*}$} \\ 
			                   & & & \\
			                    $c + c^{*}$
			                    & = & $c + \lambda + c.c^{*}$ & (por 7) \\
			                    & = & $\lambda + c + c.c^{*}$ & (por 2) \\
                               & = & $\lambda + c.\lambda + c.c^{*}$ & (por 4) \\
                               & = & $\lambda + c.(\lambda + c^{*})$ & (por 3) \\
                               & = & $\lambda + c.c^{*}$ & (por 6) \\
                               & = & $c^{*}$ & (por 7) \\
			                \end{tabular}
			        \end{table}
			\end{exampleblock}
		\end{frame}

        \begin{frame}
			\frametitle{Concepto}
			\framesubtitle{Propiedades}

			\begin{exampleblock}{Ejercicios}
			    \begin{itemize}
			        \item[1.-] $((c + b.a)^{*}.a^{*})^{*}$ \textquestiondown~= ? $((c + b.a) + a)^{*}$
			        \item[2.-] Dado dos expresiones regulares $r = b.c+a.c^{*}.a.c+a.c^{*}.c+a$ y $s=(b + a.c^{*}a).c+a.c^{*}$. \textquestiondown Representan \emph{r} y \emph{s} el mismo lenguaje?
			        \item[3.-] Sea $r = (a^{*}.(b + c)^{*}+b^{*})^{*}$ y $s = (a + b + c)^{*}$,, demuestre $r$ \textquestiondown~= ? $s$.
			    \end{itemize}
			\end{exampleblock}
		\end{frame}

		\begin{frame}
			\frametitle{Preguntas}

			\hspace{4cm}\huge{Preguntas ?}
		
		\end{frame}
	\end{document}

\usetheme{default}
\usetheme{JuanLesPins}
\usetheme{Goettingen}
\usetheme{Szeged}
\usetheme{Warsaw}

\usecolortheme{crane}

\usefonttheme{serif}
\usefonttheme{structuresmallcapsserif}
