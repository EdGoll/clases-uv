\documentclass{beamer}

\mode<presentation>
{
    \usetheme{CambridgeUS}
    \setbeamercovered{transparent}
}
\usepackage[spanish]{babel}
\usepackage[latin1]{inputenc}
\usepackage{color}
\usepackage{hyperref}
\usepackage{algorithm,algorithmic}

\usepackage{listings}

\definecolor{mygreen}{rgb}{0,0.6,0}
\definecolor{mygray}{rgb}{0.5,0.5,0.5}
\definecolor{mymauve}{rgb}{0.58,0,0.82}

\lstset{ %
  backgroundcolor=\color{white},   % choose the background color; you must add \usepackage{color} or \usepackage{xcolor}
  basicstyle=\footnotesize,        % the size of the fonts that are used for the code
  breakatwhitespace=false,         % sets if automatic breaks should only happen at whitespace
  breaklines=true,                 % sets automatic line breaking
  captionpos=b,                    % sets the caption-position to bottom
  commentstyle=\color{gray},    % comment style
  deletekeywords={...},            % if you want to delete keywords from the given language
  escapeinside={\%*}{*)},          % if you want to add LaTeX within your code
  extendedchars=true,              % lets you use non-ASCII characters; for 8-bits encodings only, does not work with UTF-8
  frame=none,                       % adds a frame around the code
  keepspaces=true,                 % keeps spaces in text, useful for keeping indentation of code (possibly needs columns=flexible)
  keywordstyle=\color{blue},       % keyword style
  language=Java,                 % the language of the code
  otherkeywords={else},           % if you want to add more keywords to the set
  numbers=none,                    % where to put the line-numbers; possible values are (none, left, right)
  numbersep=5pt,                   % how far the line-numbers are from the code
  numberstyle=\tiny\color{mygray}, % the style that is used for the line-numbers
  rulecolor=\color{black},         % if not set, the frame-color may be changed on line-breaks within not-black text (e.g. comments (green here))
  showspaces=false,                % show spaces everywhere adding particular underscores; it overrides 'showstringspaces'
  showstringspaces=false,          % underline spaces within strings only
  showtabs=false,                  % show tabs within strings adding particular underscores
  stepnumber=2,                    % the step between two line-numbers. If it's 1, each line will be numbered
  stringstyle=\color{mymauve},     % string literal style
  tabsize=2,                       % sets default tabsize to 2 spaces
  title=\lstname                   % show the filename of files included with \lstinputlisting; also try caption instead of title
}


\title[\textbf{ICI 4242 - Aut\'omatas y compiladores}]{\textbf{ICI 4242 - Aut\'omatas y compiladores}}

\subtitle{Descripci\'on de la asignatura}

\author[Rodrigo Olivares]
{
    Rodrigo Olivares \\
    \vspace{0.5mm}
    Msc. en Ingenier\'ia Inform\'atica \\
    \vspace{0.5mm}
    \texttt{\normalsize rodrigo.olivares@uv.cl}
}

\institute[PUCV]

\date{1er Semestre} 

\subject{Descripci\'on de la asignatura}

%\AtBeginSection
%{
%    \begin{frame}<beamer>
%    \frametitle{Contenido}
%    \tableofcontents[currentsection,currentsubsection]
%    \end{frame}
%}

%\AtBeginSubsection
%{
%    \begin{frame}<beamer>
%    \frametitle{Contenido}
%    \tableofcontents[currentsection,currentsubsection]
%    \end{frame}
%}

%\beamerdefaultoverlayspecification{<+->}

\begin{document}

    \begin{frame}
        \titlepage
    \end{frame}

%    \begin{frame}
%        \frametitle{Contenido}
%        \tableofcontents[pausesections]
%    \end{frame}

    \section{Antecedentes de la asignatura}

        \subsection{Descripci\'on}

        \begin{frame}
            \frametitle{Antecedentes de la asignatura}
            \framesubtitle{Descripci\'on}

            \begin{itemize}
                \item Nombre: \textbf{ICI 4242 - Aut\'omatas y compiladores}
                \item Horario: 
                \begin{itemize}
                    \item \textbf{Lunes: 11:40 - 13:10 hrs.}
                    \item \textbf{Viernes: 11:40 - 13:10 hrs.}
                \end{itemize}
                \item Clases expositivas y laboratorio.
                \item Ayudant\'ia: \emph{por definir}
            \end{itemize}
        \end{frame}

        \subsection{Unidades tem\'aticas}

        \begin{frame}
            \frametitle{Antecedentes de la asignatura}
            \framesubtitle{Unidades tem\'aticas}

            \begin{itemize}
                \item Unidad I: \textbf{Lenguajes y gram\'aticas formales}
                \begin{itemize}
                    \item Alfabetos y cadenas.
                    \item Lenguajes formales.
                    \item Gram\'aticas formales.
                \end{itemize}
                \item Unidad II: \textbf{Aut\'omatas finitos}
                \begin{itemize}
                    \item Aut\'omatas finitos deterministas.
                    \item Aut\'omatas finitos no deterministas.
                    \item Aut\'omatas finitos no deterministas con $\lambda$-transiciones.
                    \item Aut\'omatas de Pila.
                    \item M\'aquinas de Turing
                \end{itemize}
                \item Unidad III: \textbf{An\'alisis l\'exico}
                \begin{itemize}
                    \item Funciones del analizador l\'exico.
                    \item Tokens.
                    \item Herramientas para implementar analizadores l\'xicos
                    \item Implementaci\'on de analizador l\'exico.
                \end{itemize}
            \end{itemize}
        \end{frame}

        \begin{frame}
            \frametitle{Antecedentes de la asignatura}
            \framesubtitle{Unidades tem\'aticas}

            \begin{itemize}
                \item Unidad IV: \textbf{An\'alisis sint\'actico}
                \begin{itemize}
                    \item Funciones del analizador sint\'actico.
                    \item \'Arboles de sintaxis abstracta.
                    \item Herramientas para implementar analizadores sint\'acticos.
                    \item Implementaci\'on de analizador sint\'actico.
                \end{itemize}
                \item Unidad V: \textbf{An\'alisis sem\'antico}
                \begin{itemize}
                    \item Funciones del analizador sem\'antico.
                    \item Tablas de s\'imbolos.
                    \item Implementaci\'on de analizador sem\'antico.
                \end{itemize}
                \item Unidad VI: \textbf{Generaci\'on de c\'odigo}
                \begin{itemize}
                    \item Funciones del generador de c\'odigo.
                    \item Representaciones intermedias.
                    \item Implementaci\'on de generador de c\'odigo.
                \end{itemize}
            \end{itemize}
        \end{frame}

    \section{Calificaciones}

        \subsection{Cronograma}

        \begin{frame}
            \frametitle{Calificaciones}
            \framesubtitle{Cronograma}

            \textbf{Cronograma de las calificaciones}

            \begin{itemize}
                \item C\'atedra 1: \textbf{Viernes 4 de Mayo}
                \item C\'atedra 2: \textbf{Lunes 18 de Junio}
                \item Control: \textbf{Lunes 23 de Abril}
                \item Tarea 1: \textbf{Viernes 04 de Mayo}
                \item Tarea 2: \textbf{Viernes 22 de Junio}
                \item Examen: \textbf{Viernes 29 de Junio}
            \end{itemize}
        \end{frame}

        \subsection{Ponderaci\'on}

        \begin{frame}
            \frametitle{Calificaciones}
            \framesubtitle{Ponderaci\'on}

            \begin{center}
                \lstinputlisting[caption={}]{pond.java}
            \end{center}    
        \end{frame}

    \section{Consideraciones}

        \subsection{Software y documentaci\'on}

        \begin{frame}
            \frametitle{Consideraciones}
            \framesubtitle{Software}

            \begin{itemize}
                    \item[-] \textbf{JFLAP} disponible en:
                \begin{itemize}
                    \item[] \url{http://www.jflap.org}
                \end{itemize} 
                \item[-] \textbf{ANTLR} disponible en:
                \begin{itemize}
                    \item[] \url{http://www.antlr.org}
                    \item[] \textbf{Documentaci\'on} disponible en: \url{https://github.com/antlr/antlr4/blob/master/doc/index.md} 
                \end{itemize} 
                \item[-] Entorno de desarrollo \textbf{Eclipse}, disponible en:
                \begin{itemize}
                    \item[] \url{https://www.eclipse.org}
                \end{itemize} 
            \end{itemize}

        \end{frame}

    \section{Bibliograf\'ia}
    
        \begin{frame}
            \frametitle{Bibliograf\'ia}

            \textbf{Obligatoria}
 
            \begin{thebibliography}{10} 
                \beamertemplatebookbibitems
                \bibitem{1} [Compilers, Principles, Techniques and Tools] Aho, A. et al. \newblock \emph{Addison-Wesley}, 1988.
                \bibitem{2} [Principles of Compiler Design] Aho, A. et al. \newblock \emph{Addison-Wesley}, 1988.
            \end{thebibliography} 

            \vspace{10pt}

            \textbf{Complementaria}

            \begin{thebibliography}{10} 
                \beamertemplatebookbibitems
                \bibitem{1}[Introduction to Automata Theory, Languages, and Computation] John E. Hopcroft, Rajeev Motwani, Jeffrey D. Ullman \newblock \emph{Editorial Alfaomega}, 2011.
             \end{thebibliography} 
        \end{frame}

        \begin{frame}
            \frametitle{Preguntas}

            \hspace{4cm}\huge{Preguntas ?}
        
        \end{frame}
    \end{document}

\usetheme{default}
\usetheme{JuanLesPins}
\usetheme{Goettingen}
\usetheme{Szeged}
\usetheme{Warsaw}

\usecolortheme{crane}

\usefonttheme{serif}
\usefonttheme{structuresmallcapsserif}