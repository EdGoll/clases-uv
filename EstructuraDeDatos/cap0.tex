\documentclass{beamer} % [handout] para imprimir eliminando transiciones

%\usefonttheme[onlymath]{serif}
%\usepackage{fontspec}
%\defaultfontfeatures{Mapping=tex-text}
%\setsansfont[Ligatures={Common}]{Futura}
%\setmonofont[Scale=0.8]{Monaco}

\usepackage{beamerthemesplit}
\usepackage[utf8]{inputenc}
\usepackage[spanish]{babel}
\mode<presentation>
\usetheme{default}
\usecolortheme{dolphin}
\usepackage{alltt}    % \begin{alltt}
\usepackage{amssymb}  % mathematical symbols
\usepackage{comment}
\usepackage{multicol} % \multicols
\usepackage{tabto}    % \tabto
\usepackage{verbatim} % comentarios

\title{Estructuras de datos}   %[titulo corto]
\author{Eduardo Godoy Llanca} %[nombre corto]
\date{2018}                    %[fecha corta]
\institute{Universidad de Valparaíso}                 %[instituto corto]

\newcommand{\HRule}{\rule{\linewidth}{0.2mm}\\[1ex]}
\newcommand{\blue}[1]{\textcolor{blue}{#1}}
\newcommand{\red}[1]{\textcolor{red}{#1}}
\newcommand{\redb}[1]{{\color{red!70!black}{#1}}}
\newcommand{\green}[1]{{\color{green!70!black}{#1}}}
\newcommand{\gray}[1]{{\color{gray!50!white}{#1}}}
\newcommand{\textgreek}[1]{\begingroup\fontencoding{LGR}\selectfont#1\endgroup}
% \alert{texto destacado en rojo}
% \color{green} Color en verde
% \structure{texto en lila}

\begin{document}


%\begin{frame}%[plain]
%  \titlepage
%\end{frame}
%
% [opciones]:
% plain: oculta barra de navegacion, deja + espacio para contenido
% fragile: usar comandos como verbatim
% b,c,t: alineacion vertical
% label=nombre_etiqueta
% allowframebreaks: divide contenido en varios frames si es demasiado largo
% shrink: para escribir mucho texto en una transparencia, reduciendo tamano de fuente

%%%%%%%%%% PORTADA %%%%%%%%%%
\begin{frame}[plain]
  \begin{figure}[h]
    \begin{minipage}{0.3\textwidth}
    \includegraphics[width=.9\textwidth]{./image/logo-UV.png}
    \end{minipage}
    \begin{minipage}{0.65\textwidth}
     $~$\\[3.6ex]
     \footnotesize{Escuela de Ingeniería Civil Informática}\\
     \footnotesize{Facultad de Ingeniería}
    \end{minipage}
  \end{figure}
  \begin{center}
    \vspace{1ex}
    \HRule
    \Large{Estructuras de datos}\\{\small Introducción}\\[-1ex]
    \HRule\vspace{1ex}
    \large{Eduardo Godoy Llanca}\\[.5ex]\footnotesize{eduardo.gl@gmail.com}\\[6ex] {\tiny 2017-II}\\[6ex]
  \end{center}
\end{frame}

%%%%%%%%%% INDEX %%%%%%%%%%
\begin{frame}
 \frametitle{Index}
 \scriptsize 			% reducir tamano de letra
 \tableofcontents		%[pausesections]
\end{frame}

%%%%%%%%%%% ACTUAL INDEX %%%%%%%%%%
%\AtBeginSection[] %generar indice automaticamente
%{
%\begin{frame}<beamer>%[plain]
% \frametitle{Index}
% \framesubtitle{subtitulo}
% \scriptsize
% \tableofcontents[currentsection, currentsubsection]
%\end{frame}
%}

%==============================
\section{Contenido}

%------------------------------
\subsection{Introducción: Análisis de Algoritmos.}
\subsection{Unidad I: Tipos de Datos Abstractos.}
\subsection{Unidad II: Algoritmos de ordenamiento.}
\subsection{Unidad III: Árboles.}
\subsection{Unidad IV: Estructuras Aleatorias.}

%==============================
\section{Evaluaciones}

%------------------------------
\subsection{Certamenes}
\begin{frame}
  \begin{block}{Certamenes}
  \begin{enumerate}
    \item Introducción + Unidad I + Unidad II: 22 de Ocubre.
    \item Unidad III: 26 de Noviembre.
    \item Unidad IV: 7 de Enero.
  \end{enumerate}
  \begin{itemize}
    \item Ponderación: 80\%
  \end{itemize}

\end{block}
\end{frame}

\subsection{Tareas}
\begin{frame}
  \begin{block}{Tareas}
  \begin{itemize}
    \item Tarea 1, TDA: 1 de Ocubre.
    \item Tarea 2, Algoritmos de Ordenamiento: 15 de Ocubre.
    \item Tarea 3, Arboles: 19 de Noviembre.
    \item Tarea 4, Estructuras Aleatorias: 29 de Diciembre.
  \end{itemize}
  \begin{itemize}
    \item Ponderación: 20\%
  \end{itemize}
\end{block}
\end{frame}

%------------------------------

%------------------------------

\begin{frame}
 \begin{block}{Bibliografía}
  \begin{itemize}
    \item Weiss, M., Estructura de datos y algoritmos,\\ Addison-Wesley, 1995.
    \item Aho, Hopcroft y Ullman, Estructuras de datos y algoritmos, Addison-Wesley, 1988.
  \end{itemize}
 \end{block}
 \begin{block}{Recursos}
  \begin{itemize}
    \item Wikipedia y Wikimedia Commons.
  \end{itemize}
 \end{block}
\end{frame}

\end{document}
