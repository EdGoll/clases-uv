%\documentclass{article}
\documentclass[letter,12pt,oneside]{book}
\usepackage[utf8]{inputenc}
\usepackage[spanish]{babel}
\usepackage[LGR,T1]{fontenc}
\usepackage{amssymb}                % símbolos especiales
\usepackage{amsmath, amsthm}        % ambiente \newtheorem
\usepackage{color}
\usepackage{fancyhdr}
\usepackage{graphicx}
\usepackage{tabto}                  % \tabto
\setlength{\textheight}{21cm}
\setlength{\textwidth}{17cm}
\setlength{\topmargin}{0cm}
\setlength{\oddsidemargin}{0cm}
\setlength{\evensidemargin}{0cm}

% \newtheorem{nombre}{caption}[within]
\theoremstyle{definition}
\newtheorem{corolary}{Corolario}[]
\newtheorem{lemma}{Lema}[]
\newtheorem{theorem}{Teorema}[]
\newtheorem{example}{Ejemplo}[section]

% \newenvironment{nombre}[argumentos]{begindef}{enddef}
\newenvironment{lista}{\begin{list}{\textbullet}{\itemindent -1ex \itemsep -1ex}}{\end{list}}

\rhead{\begin{picture}(0,0) \put(-120,0){\includegraphics[width=40mm]{./image/logo-UV.png}} \end{picture}}
\lhead{\vspace{-0.3cm}Universidad de Valparaíso\\Facultad de Ingeniería\\Escuela de Ingeniería Civil Informática\vspace{0.3cm}}

\pagestyle{fancy}

\begin{document}
%\maketitle

\begin{center}
$~$
\end{center}

\noindent
Nombre: \rule{.6\textwidth}{.5pt} Rut: \rule{.24\textwidth}{.5pt}

\begin{center}
 {\Large
  {\color{white}.}\\
  Estructuras de datos\\[1ex]
  Control de diagnóstico}\\[1.2ex]
  Prof: Fabián Riquelme Csori\\
  2017-II
\end{center}

\begin{enumerate}
    \item Considere estos dos códigos:
    
    \begin{tabular}{p{40ex}p{40ex}}
        \texttt{for(i=0; i<2n; i++)\{}  
                    & \texttt{for(i=0; i<n; i++)} \\
        $~~~~$ \texttt{printf("hola mundo$\backslash$n");} 
                    & $~~~~$ \texttt{for(i=0; i<n; i++)\{} \\
        $~~~~$ \texttt{return 0;} 
                    & $~~~~~~~~$ \texttt{printf("hola mundo$\backslash$n");} \\
        \texttt{\}} & $~~~~~~~~$ \texttt{return 0;} \\
                    & $~~~~$ \texttt{\}}\\
    \end{tabular}
    
    Suponiendo que ambos códigos se ejecutan en una misma máquina, ¿cuál debería acabar antes y por qué? \tabto{80ex}[30 pts]
    \vspace{3cm}
    
    \item Describa tres maneras distintas de definir una matriz de $m\times n$ elementos en un lenguaje de programación de su elección. \tabto{80ex}[30 pts]
    \vspace{4cm}
    
\end{enumerate}

\newpage

\begin{center}
 {\Large
  {\color{white}.}\\[5ex]
  Estructuras de datos\\[1ex]
  Control de diagnóstico -- Pauta}\\[1.2ex]
  Prof: Fabián Riquelme Csori\\
  2017-II
\end{center}

\begin{enumerate}
    \item El printf del primer algoritmo se ejecuta $2n$ veces.\\
    El segundo algoritmo se puede interpretar de dos maneras. Primero, como que las variables $i$ de los dos for son independientes, en cuyo caso el printf se ejecuta $n^2$ veces. Segundo, como que la $i$ del segundo for es dependiente del primer for, en cuyo caso el printf se ejecuta $n$ veces.\\
    De acuerdo con esto, para un $n$ suficientemente grande, bajo la primera interpretación, el primer algoritmo termina antes, y bajo la segunda interpretación, el segundo algoritmo termina antes.
    \item Mediante arreglos de arreglos, reemplazando los arreglos por listas o vectores, como una lista de pares ordenados, etc.
    \end{enumerate}

\end{document}
