%\documentclass{article}
\documentclass[letter,12pt,oneside]{book}
\usepackage[utf8]{inputenc}
\usepackage[spanish]{babel}
\usepackage[LGR,T1]{fontenc}
\usepackage{amssymb}                % símbolos especiales
\usepackage{amsmath, amsthm}        % ambiente \newtheorem
\usepackage{color}
\usepackage{fancyhdr}
\usepackage{graphicx}
\usepackage{tabto}                  % \tabto
\setlength{\textheight}{22cm}
\setlength{\textwidth}{17cm}
\setlength{\topmargin}{-1cm}
\setlength{\oddsidemargin}{0cm}
\setlength{\evensidemargin}{0cm}

% \newtheorem{nombre}{caption}[within]
\theoremstyle{definition}
\newtheorem{corolary}{Corolario}[]
\newtheorem{lemma}{Lema}[]
\newtheorem{theorem}{Teorema}[]
\newtheorem{example}{Ejemplo}[section]

% \newenvironment{nombre}[argumentos]{begindef}{enddef}
\newenvironment{lista}{\begin{list}{\textbullet}{\itemindent -1ex \itemsep -1ex}}{\end{list}}

\rhead{\begin{picture}(0,0) \put(-120,0){\includegraphics[width=40mm]{./image/logo-UV}} \end{picture}}
\lhead{\vspace{-0.3cm}Universidad de Valparaíso\\Facultad de Ingeniería\\Escuela de Ingeniería Civil Informática\vspace{0.3cm}}

\pagestyle{fancy}

\begin{document}
%\maketitle

\begin{center}
$~$
\end{center}

\noindent
Nombre: \rule{.6\textwidth}{.5pt} Rut: \rule{.24\textwidth}{.5pt}

\begin{center}
 {\Large
  {\color{white}.}\\
  Estructuras de datos\\[1ex]
  Control 2}\\[1.2ex]
  Prof: Fabián Riquelme Csori\\
  2017-II
\end{center}

\begin{enumerate}
    \item Además de las FIFO (first-in, first-out) y las LIFO (last-in, first-out) existen las FINO (first-in, never-out). Para definir el correspondiente TDA, determine:
    \begin{enumerate}
        \item Cuatro operaciones para trabajar sobre él. \tabto{76ex} [12 pts]
        \item Cuatro axiomas que regulen el funcionamiento de dichas operaciones. \tabto{76ex} [12 pts]
    \end{enumerate}
    
    \item ¿En qué casos es mejor una estructura de datos Vector que una Lista? \tabto{81ex} [12 pts]
    
    \item Considere el siguiente pseudocódigo, que define una lista doblemente enlazada de nodos:

    \texttt{struct Node \{}\\
    $~~~~$\texttt{data;} \tabto{23ex}// Dato almacenado en el nodo\\
    $~~~~$\texttt{next;} \tabto{23ex}// Puntero al nodo siguiente (NULL para el último nodo)\\
    \texttt{\}}\\
    \texttt{struct List \{}\\
    $~~~~$\texttt{Node FirstNode;} \tabto{23ex}// La lista apunta al primer nodo; NULL si está vacía\\
    $~~~~$\texttt{Node LastNode;} \tabto{23ex}// La lista apunta al último nodo; NULL si está vacía\\
    \texttt{\}}
    
    Para insertar un nodo \texttt{newNode} después de un nodo \texttt{node}, se define la función \texttt{insertAfter}:
    
    \texttt{function insertAfter(Node node, Node newNode) \{}\\
    $~~~~$\texttt{newNode.next := node.next;}\\
    $~~~~$\texttt{node.next := newNode;}\\
    \texttt{\}}
    
    Para insertar un nodo \texttt{newNode} al inicio de la lista \texttt{list}, se define la función \texttt{insertBeginning}:
    
    \texttt{function insertBeginning(List list, Node newNode) \{}\\
    $~~~~$\texttt{newNode.next   := list.firstNode;}\\
    $~~~~$\texttt{list.firstNode := newNode;}\\
    \texttt{\}}

    A partir de lo anterior, defina:
    \begin{enumerate}
        \item Una función \texttt{insertEnd}, que inserta un nodo \texttt{newNode} al final de \texttt{list}. \tabto{76ex} [12 pts]
        \item Una función \texttt{removeAfter}, que elimina un nodo \texttt{Node}. \tabto{76ex} [12 pts]
    \end{enumerate}
\end{enumerate}

\newpage

\begin{center}
 {\Large
  {\color{white}.}\\
  Estructuras de datos\\[1ex]
  Control 2 - Pauta}\\[1.2ex]
  Prof: Fabián Riquelme Csori\\
  2017-II
\end{center}

\begin{enumerate}
    \item[1. $a)$] Son respuestas válidas size(), isEmpty(), push(element), top() o create().\\
    Las operaciones tipo pop() o de acceso directo a cualquier índice no deberían estar permitidas. \tabto{81ex} [12 pts]
    \item[1. $b)$] Cualquiera análoga a las vistas en clases para Pilas o Colas son válidas, salvo aquellas que consideren las operaciones pop().\tabto{81ex} [12 pts]
    \item[2.] Cuando no se requieren tantas operaciones de inserción, extracción o desplazamiento de datos durante la ejecución.\\
    Cuando se requiere acceder frecuentemente a los datos. \tabto{81ex} [12 pts]
    \item[3. $a)$] La función queda definida por:\\[1.2ex]
    \texttt{function insertEnd(List list, Node newNode) \{}\\
    $~~~~$\texttt{(list.LastNode).next := newNode;}\\
    \texttt{\}} \tabto{81ex} [12 pts]
    \item[3. $b)$] La función queda definida por:\\[1.2ex]
    \texttt{function removeAfter(Node node) \{}\\
    $~~~~$\texttt{aux := node.next;}\\
    $~~~~$\texttt{node.next := node.next.next;}\\
    $~~~~$\texttt{destroy aux;}\\
    \texttt{\}} \tabto{81ex} [12 pts]
\end{enumerate}

\end{document}
