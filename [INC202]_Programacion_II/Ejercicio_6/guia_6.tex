\documentclass{article}

\usepackage[spanish]{babel}
\usepackage[latin1]{inputenc}
\usepackage{colortbl}
\usepackage{amsmath}
\usepackage{graphicx}

\begin{document}

	\centerline{\sc \large Gu\'ia de ejercicios: Programaci\'on 2}
	\centerline{\sc \normalsize Escuela de Ingenier\'ia Civil Inform\'atica}
	\centerline{\sc \normalsize  Universidad de Valpara\'iso}
	%\centerline{\sc \small 26 de Octubre de 2015}

	\vspace{1pc}

	\begin{enumerate}
	    \item[] Desarrolle una aplicaci\'on en Java que realice la lectura del archivo \emph{dataset-chilebank.csv} y permita responder la siguientes consultas:
	    \begin{enumerate}
	        \item \textquestiondown Cu\'antas personas no poseen deudas?
            \item \textquestiondown Cu\'antas personas dependientes con t\'itulo universitario tienen deuda 0?
            \item \textquestiondown Cu\'antas personas independientes con t\'itulo universitario tienen cupo 0?
	        \item Muestre la lista anterior.
	        \item \textquestiondown Cu\'antos estudiantes sobre 25 a\~nos tienen deuda 0?
	        \item \textquestiondown Cu\'antos estudiantes casados tienen cupo 0?
	        \item Muestre el listado de los estudiantes del caso anterior.
	        \item \textquestiondown Cu\'antos empresarios tienen dueda superior a 1 mill\'on?
	        	\item \textquestiondown Cu\'antos empresarios independientes tienen deuda?
	        \item \textquestiondown Cu\'antos empresarios tienen dueda inferior a 1 mill\'on?
	        	\item \textquestiondown Cu\'antos empresarios dependientes no tienen deuda?
	        \item Muestre los listados anteriores.
	    \end{enumerate}
	\end{enumerate}
\end{document}