\documentclass[10pt]{article}
\usepackage{graphicx}
\usepackage{amssymb}
\usepackage{tikz}
\usepackage{float}
\usepackage{epstopdf}
\usepackage{enumitem}
\usepackage{multicol,multirow}
\DeclareGraphicsRule{.tif}{png}{.png}{`convert #1 `dirname #1`/`basename #1 .tif`.png}
\newcommand*\circled[1]{\tikz[baseline=(char.base)]{\node[shape=circle,blue,draw,inner sep=.5pt] (char) {#1};}}
\usepackage{listings}

\definecolor{mygreen}{rgb}{0,0.6,0}
\definecolor{mygray}{rgb}{0.5,0.5,0.5}
\definecolor{mymauve}{rgb}{0.58,0,0.82}

\lstset{ %
  backgroundcolor=\color{white},   % choose the background color; you must add \usepackage{color} or \usepackage{xcolor}
  basicstyle=\footnotesize,        % the size of the fonts that are used for the code
  breakatwhitespace=false,         % sets if automatic breaks should only happen at whitespace
  breaklines=true,                 % sets automatic line breaking
  captionpos=b,                    % sets the caption-position to bottom
  commentstyle=\color{gray},    % comment style
  deletekeywords={...},            % if you want to delete keywords from the given language
  escapeinside={\%*}{*)},          % if you want to add LaTeX within your code
  extendedchars=true,              % lets you use non-ASCII characters; for 8-bits encodings only, does not work with UTF-8
  frame=none,	                   % adds a frame around the code
  keepspaces=true,                 % keeps spaces in text, useful for keeping indentation of code (possibly needs columns=flexible)
  keywordstyle=\color{blue},       % keyword style
  language=Java,                 % the language of the code
  otherkeywords={else},           % if you want to add more keywords to the set
  numbers=none,                    % where to put the line-numbers; possible values are (none, left, right)
  numbersep=5pt,                   % how far the line-numbers are from the code
  numberstyle=\tiny\color{mygray}, % the style that is used for the line-numbers
  rulecolor=\color{black},         % if not set, the frame-color may be changed on line-breaks within not-black text (e.g. comments (green here))
  showspaces=false,                % show spaces everywhere adding particular underscores; it overrides 'showstringspaces'
  showstringspaces=false,          % underline spaces within strings only
  showtabs=false,                  % show tabs within strings adding particular underscores
  stepnumber=2,                    % the step between two line-numbers. If it's 1, each line will be numbered
  stringstyle=\color{mymauve},     % string literal style
  tabsize=2,	                   % sets default tabsize to 2 spaces
  title=\lstname                   % show the filename of files included with \lstinputlisting; also try caption instead of title
}

% For a visual definition of these parameters, see
\textwidth = 6.5 in
\textheight = 9 in
\oddsidemargin = 0.0 in
\evensidemargin = 0.0 in
\topmargin = 0.0 in             
\headheight = 0.0 in            
\headsep = 0.0 in
            
\parskip = 0.2in                % vertical space between paragraphs
% Delete the % in the following line if you don't want to have the first line of every paragraph indented
%\parindent = 0.0in

\begin{document}
\begin{center}
    {\Large Pauta Certamen 1, Programaci\'on II} \\
    \emph{\small Prof. Rodrigo Olivares} \\
    \emph{\scriptsize Noviembre 8, 2016}
\end{center}
\vspace*{-35pt}
\begin{center}
    \rule{1\textwidth}{.3pt}
\end{center}
\vspace*{-42pt}
\begin{center}
    \rule{1\textwidth}{2pt}
\end{center}

{ \scriptsize

\begin{enumerate}    
    
    \item \emph{30pts.} De las siguentes afirmaciones, encierre en un c\'irculo la o las alternativas correctas.

    \begin{multicols}{2}

    \begin{enumerate}[label=(\alph*)]
        \item[i.] Algunos enfoques de la orientaci\'on a objeto son:
        \item[\circled{(a)}] El enfoque reusable.
        \item[\circled{(b)}] El enfoque abstracto.
        \item[(c)] El enfoque imperativo.
        \item[(d)] El enfoque procedural.
        \item[(e)] Ninguna de las anteriores.
    \end{enumerate}

    \begin{enumerate}[label=(\alph*)]
        \item[ii.] En cuanto a la programaci\'on orientada a objeto (POO):
        \item[(a)] Se apoya en el paradigma estructural.
        \item[(b)] Se base en la interacci\'on de funciones.
        \item[\circled{(c)}] Proporciona herramientas para modelar el mundo real.
        \item[(d)] Es una propiedad de los lenguajes orientados a objetos.
        \item[(e)] Ninguna de las anteriores.
    \end{enumerate}

    \begin{enumerate}[label=(\alph*)]
        \item[iii.] Una clase es: 
        \item[(a)] Una colecci\'on de objetos.
        \item[(b)] Una herencia del mundo real.
        \item[\circled{(c)}] Una categorizaci\'on de objetos.
        \item[(d)] Un puntero a memoria.
        \item[(e)] Ninguna de las anteriores.
    \end{enumerate}

    \begin{enumerate}[label=(\alph*)]
        \item[iv.] Respecto a una clase:
        \item[\circled{(a)}] Se declaran utilizando la palabra resevada class.
        \item[\circled{(b)}] Debe tener el mismo nombre que el archivo.
        \item[(c)] Se declaran static los atributos de miembros del objeto.
        \item[(d)] Todas deben incluir el m\'etodo main.
        \item[(e)] Ninguna de las anteriores.
    \end{enumerate}

    \begin{enumerate}[label=(\alph*)]
        \item[v.] Respecto a una clase:
        \item[\circled{(a)}] Puede o no tener atributos.
        \item[\circled{(b)}] Puede o no tener m\'etodos.
        \item[\circled{(c)}] Puede o no tener constructor.
        \item[(d)] Puede o no tener un nombre.
        \item[(e)] Puede o no tener un tipo.
    \end{enumerate}
    
    \begin{enumerate}[label=(\alph*)]
        \item[vi.] Un objeto es: 
        \item[(a)] Una categorizaci\'on de la clase.
        \item[\circled{(b)}] La instancia de una clase.
        \item[\circled{(c)}] Una abstracci\'on del mundo real.        
        \item[(d)] Un m\'etodo de interaci\'on entre clases. 
        \item[(e)] Siempre est\'atico.
    \end{enumerate}

    \begin{enumerate}[label=(\alph*)]
        \item[vii.] El principio de abstracci\'on: 
        \item[(a)] Es una t\'ecnica que protege el estado de una entidad.
        \item[\circled{(b)}] Es parte del paradigma de orientaci\'on a objeto.
        \item[(c)] En Java, se logra utilizando los modificadores de acceso.
        \item[(d)] Es absorber el conocimiento de una entidad.
        \item[(e)] Ninguna de las anteriores.
    \end{enumerate}

    \begin{enumerate}[label=(\alph*)]
        \item[viii.] Respecto los constructores:
        \item[(a)] Son declarados private
        \item[(b)] Deben ser m\'etodos de tipo void.
        \item[\circled{(c)}] Deben ser normbrados igual que las clases.
        \item[(d)] El compilador siempre crea el constructor vac\'io.
        \item[\circled{(e)}] Se utiliza para instanciar objetos.
    \end{enumerate}

    \begin{enumerate}[label=(\alph*)]
        \item[ix.] El polimorfismo: 
        \item[(a)] Una funcionalidad implementada con distintos nombres.
        \item[\circled{(b)}] El mismo nombre implementa distintas funcionalidades.
        \item[(c)] Es una caracter\'istica de Java.
        \item[\circled{(d)}] Es una caracter\'istica de POO.
        \item[(e)] Un ejemplo es el s\'imbolo $\rightarrow$.
    \end{enumerate}

    \begin{enumerate}[label=(\alph*)]
        \item[x.] El met\'odo \emph{main}:
        \item[\circled{(a)}] Debe ser void.
        \item[\circled{(b)}] Debe ser static.
        \item[\circled{(c)}] Debe incluir argumentos de entrada.
        \item[(d)] Debe retornar un valor.
        \item[\circled{(e)}] Debe incluirse en un programa.
    \end{enumerate}

\end{multicols}

\newpage 

/************************ \textbf{Jugador.java} ************************/

\lstinputlisting[caption={}]{certamen/Jugador.java}

/************************ \textbf{Equipo.java} ************************/

\lstinputlisting[caption={}]{certamen/Equipo.java}

/************************ \textbf{Liga.java} ************************/

\lstinputlisting[caption={}]{certamen/Liga.java}

/************************ \textbf{LigaImp.java} ************************/

\lstinputlisting[caption={}]{certamen/LigaImp.java}

\end{enumerate}

}

\end{document} 