\documentclass[10pt]{article}
\usepackage{graphicx}
\usepackage{amssymb}
\usepackage{tikz}
\usepackage{epstopdf}
\usepackage{enumitem}
\usepackage{multicol,multirow}
\DeclareGraphicsRule{.tif}{png}{.png}{`convert #1 `dirname #1`/`basename #1 .tif`.png}
\newcommand*\circled[1]{\tikz[baseline=(char.base)]{\node[shape=circle,blue,draw,inner sep=.5pt] (char) {#1};}}

% For a visual definition of these parameters, see
\textwidth = 6.5 in
\textheight = 9 in
\oddsidemargin = 0.0 in
\evensidemargin = 0.0 in
\topmargin = 0.0 in             
\headheight = 0.0 in            
\headsep = 0.0 in
            
\parskip = 0.2in                % vertical space between paragraphs
% Delete the % in the following line if you don't want to have the first line of every paragraph indented
%\parindent = 0.0in

\begin{document}
\begin{center}
    {\Large Pauta Certamen 1, Programaci\'on II} \\
    \emph{\small Prof. Rodrigo Olivares} \\
    \emph{\small Ayud. Juan Carlos Tapia} \\
    \emph{\scriptsize Septiembre 22, 2015}
\end{center}
\vspace*{-35pt}
\begin{center}
    \rule{1\textwidth}{.3pt}
\end{center}
\vspace*{-42pt}
\begin{center}
    \rule{1\textwidth}{2pt}
\end{center}

\vspace*{-15pt}
{\small \textbf{Instrucciones}:}
\vspace*{-15pt}

{\scriptsize
\begin{itemize}
    \item[-] El puntaje m\'aximo del certamen es 100\%, siendo el 60\% el m\'inimo requerido para aprobar.
    \item[-] Responda cada pregunta en la hoja indicada, agregando su nombre. Si no responde alguna pregunta, debe entregar la hoja con su nombre e indicar que \textbf{no responde}.
    \item[-] El certamen es \underline{\textbf{individual}}. Cualquier intento de copia, ser\'a sancionado con nota \textbf{1,0}.
\end{itemize}
}
\vspace*{10pt}

\vspace*{-30pt}

\begin{enumerate}

    \item \emph{30pts.} De las siguentes afirmaciones, encierre en un c\'irculo la o las alternativas correctas.
    {\scriptsize
    \begin{multicols}{2}

    \begin{enumerate}[label=(\alph*)]
        \item[i.] Algunos enfoques de la orientaci\'on a objeto son:
        \item[(a)] El enfoque reusabable.
        \item[(b)] El enfoque abstracto.
        \item[(c)] El enfoque imperativo.
        \item[(d)] El enfoque procedural.
        \item[\circled{(e)}] Ninguna de las anteriores.
    \end{enumerate}

    \begin{enumerate}[label=(\alph*)]
        \item[ii.] En cuanto a la programaci\'on orientada a objeto:
        \item[(a)] Se apoya en el paradigma estructural.
        \item[\circled{(b)}] Divide el programa en peque\~nas unidades de c\'odigo.
        \item[\circled{(c)}] Proporciona herramientas para modelar el mundo real.
        \item[(d)] Es una herramienta del lenguaje JAVA.
        \item[(e)] Ninguna de las anteriores.
    \end{enumerate}

    \begin{enumerate}[label=(\alph*)]
        \item[iii.] Una clase es: 
        \item[(a)] Un arreglo de objetos.
        \item[(b)] Una abstracci\'on del mundo real.
        \item[(c)] Una herramienta de programaci\'on.
        \item[(d)] Un punto a memoria.
        \item[\circled{(e)}] Ninguna de las anteriores.
    \end{enumerate}

    \begin{enumerate}[label=(\alph*)]
        \item[iv.] Respecto a una clase:
        \item[(a)] Se declaran utilizando la palabra resevada clase.
        \item[\circled{(b)}] Debe tener el mismo nombre que el archivo.
        \item[(c)] Son declaradas est\'aticas con static.
        \item[(d)] Todas deben incluir el m\'etodo main.
        \item[(e)] Ninguna de las anteriores.
    \end{enumerate}

    \begin{enumerate}[label=(\alph*)]
        \item[v.] Respecto a una clase:
        \item[\circled{(a)}] Puede o no tener atributos.
        \item[\circled{(b)}] Puede o no tener m\'etodos.
        \item[\circled{(c)}] Puede o no tener constructor.
        \item[(d)] Puede o no tener un nombre.
        \item[(e)] Puede o no tener un tipo.
    \end{enumerate}
    
    \begin{enumerate}[label=(\alph*)]
        \item[vi.] Un objeto es: 
        \item[(a)] Un tipo de dato de la clase.
        \item[\circled{(b)}] La instancia de una clase.
        \item[\circled{(c)}] Una abstracci\'on del mundo real.        
        \item[(d)] Un sub-conjunto de atributos y m\'etodos de la clase.
        \item[(e)] Siempre est\'atico.
    \end{enumerate}

    \begin{enumerate}[label=(\alph*)]
        \item[vii.] El principio de ocultamiento: 
        \item[\circled{(a)}] Es una t\'ecnica que protege el estado de una entidad.
        \item[\circled{(b)}] Es indispensable en el paradigma de orientaci\'on a objeto.
        \item[\circled{(c)}] En Java, se logra utilizando los modificadores de acceso.
        \item[(d)] Es encapsular el conocmineto de una entidad.
        \item[(e)] Ninguna de las anteriores.
    \end{enumerate}

    \begin{enumerate}[label=(\alph*)]
        \item[viii.] Respecto a la herencia, las clases:
        \item[(a)] Heredan s\'olo los m\'etodos privados.
        \item[\circled{(b)}] Heredan el compotamiento completo de la clase padre.
        \item[(c)] Heredan s\'olo el comportamiento que se desea utilizar.
        \item[(d)] En Java, se implementan con la palabra implements.
        \item[\circled{(e)}] En Java, se implementan con la palabra extends.
    \end{enumerate}

    \begin{enumerate}[label=(\alph*)]
        \item[ix.] El polimorfismo: 
        \item[\circled{(a)}] El mismo nombre implementa distintas funcionalidades.
        \item[(b)] Una funcionalidad implementada con distintos nombres.
        \item[(c)] Es una caracter\'istica de JAVA.
        \item[\circled{(d)}] Es una caracter\'istica de POO.
        \item[\circled{(e)}] Un ejemplo es el s\'imbolo \%.
    \end{enumerate}

    \begin{enumerate}[label=(\alph*)]
        \item[x.] El met\'odo \emph{main}:
        \item[(a)] Puede no ser void.
        \item[\circled{(b)}] Debe ser static.
        \item[(c)] Puede no llevar argumentos de entrada.
        \item[(d)] Debe retornar un valor.
        \item[\circled{(e)}] Debe incluirse en un programa.
    \end{enumerate}

\end{multicols}
}
\newpage

\item \emph{30pts.} Desarrolle una clase Calculadora que permita gestionar las operaciones elementales: suma, resta, multiplicaci\'on, divisi\'on y m\'odulo. Recuerde utilizar los principios de orientaci\'on a objetos.

\begin{verbatim}
public class Calculadora {

    public double numero;

    public double suma(double num) {
        return numero + num;
    }

    public double resta(double num) {
        return numero - num;
    }

    public double multiplicacion(double num) {
        return numero * num;
    }

    public double division(double num) {
        if (num == 0) {
            return Double.NaN;
        } else {
            return numero / num;
        }
    }

    public double modulo(double num) {
        if (num == 0) {
            return Double.NaN;
        } else {
            return numero % num;
        }
    }
    
    public static void main(String[] args) {
        Calculadora calc = new Calculadora();
        
        calc.numero = 50;
        double num = 10;
        
        System.out.println(calc.numero + " + " + num + " = " + calc.suma(num));
        System.out.println(calc.numero + " - " + num + " = " + calc.resta(num));
        System.out.println(calc.numero + " * " + num + " = " + calc.multiplicacion(num));
        System.out.println(calc.numero + " / " + num + " = " + calc.division(num));
        System.out.println(calc.numero + " % " + num + " = " + calc.modulo(num));
    }
}

\end{verbatim}

\newpage

\item \emph{40pts.} Un pal\'indrome es una palabra, n\'umero o frase que puede ser le\'ia de igual forma hacia adelante y hacia atr\'as, por ejemplo:
		\begin{itemize}
			\item[] ``anita lava la tina".
			\item[] ``1234567890987654321".
			\item[] ``la ruta nos aporto otro paso natural".
		\end{itemize}
Desarrolle una clase (en JAVA) que posea los m\'etodos necesarios para determinar si una palabra es o no un pal\'indrome.
\end{enumerate}


\begin{verbatim}
public class Palindrome {

    public String frase;

    private boolean isPalindrome() {
        String fraseLimpia = "";
        String fraseInvertida = "";
        if (frase != null) {
            fraseLimpia = frase.replace(" ", "");
            for (int i = 0; i < fraseLimpia.length(); i++) {
                fraseInvertida = fraseLimpia.charAt(i) + fraseInvertida;
            }
        }
        return fraseLimpia.equalsIgnoreCase(fraseInvertida);
    }

    public static void main(String[] args) {
        Palindrome pal = new Palindrome();

        pal.frase = "Anita lava la tina";

        if (pal.isPalindrome()) {
            System.out.println("La frase: \"" + pal.frase + "\" es palindrome");
        } else {
            System.out.println("La frase: \"" + pal.frase + "\" no es palindrome");
        }
    }
}

\end{verbatim}

\end{document} 
