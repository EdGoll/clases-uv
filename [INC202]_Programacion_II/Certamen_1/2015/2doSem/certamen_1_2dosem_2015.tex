\documentclass[10pt]{article}
\usepackage{graphicx}
\usepackage{amssymb}
\usepackage{epstopdf}
\usepackage{enumitem}
\usepackage{multicol,multirow}
\DeclareGraphicsRule{.tif}{png}{.png}{`convert #1 `dirname #1`/`basename #1 .tif`.png}

% For a visual definition of these parameters, see
\textwidth = 6.5 in
\textheight = 9 in
\oddsidemargin = 0.0 in
\evensidemargin = 0.0 in
\topmargin = 0.0 in             
\headheight = 0.0 in            
\headsep = 0.0 in
            
\parskip = 0.2in                % vertical space between paragraphs
% Delete the % in the following line if you don't want to have the first line of every paragraph indented
%\parindent = 0.0in

\begin{document}
\begin{center}
    {\Large Certamen 1, Programaci\'on II} \\
    \emph{\small Prof. Rodrigo Olivares} \\
    \emph{\small Ayud. Juan Carlos Tapia} \\
    \emph{\scriptsize Septiembre 22, 2015}
\end{center}
\vspace*{-35pt}
\begin{center}
    \rule{1\textwidth}{.3pt}
\end{center}
\vspace*{-42pt}
\begin{center}
    \rule{1\textwidth}{2pt}
\end{center}

\vspace*{-15pt}
{\small \textbf{Instrucciones}:}
\vspace*{-15pt}

{\scriptsize
\begin{itemize}
    \item[-] El puntaje m\'aximo del certamen es 100\%, siendo el 60\% el m\'inimo requerido para aprobar.
    \item[-] Responda cada pregunta en la hoja indicada, agregando su nombre. Si no responde alguna pregunta, debe entregar la hoja con su nombre e indicar que \textbf{no responde}.
    \item[-] El certamen es \underline{\textbf{individual}}. Cualquier intento de copia, ser\'a sancionado con nota \textbf{1,0}.
\end{itemize}
\vspace*{10pt}

\vspace*{-30pt}

\begin{enumerate}

    \item \emph{30pts.} De las siguentes afirmaciones, encierre en un c\'irculo la o las alternativas correctas.
    \begin{multicols}{2}

    \begin{enumerate}[label=(\alph*)]
        \item[i.] Algunos enfoques de la orientaci\'on a objeto son:
        \item El enfoque reusabable.
        \item El enfoque abstracto.
        \item El enfoque imperativo.
        \item El enfoque procedural.
        \item Ninguna de las anteriores.
    \end{enumerate}

    \begin{enumerate}[label=(\alph*)]
        \item[ii.] En cuanto a la programaci\'on orientada a objeto:
        \item Se apoya en el paradigma estructural.
        \item Divide el programa en peque\~nas unidades de c\'odigo.
        \item Proporciona herramientas para modelar el mundo real.
        \item Es una herramienta del lenguaje JAVA.
        \item Ninguna de las anteriores.
    \end{enumerate}

    \begin{enumerate}[label=(\alph*)]
        \item[iii.] Una clase es: 
        \item Un arreglo de objetos.
        \item Una abstracci\'on del mundo real.
        \item Una herramienta de programaci\'on.
        \item Un punto a memoria.
        \item Ninguna de las anteriores.
    \end{enumerate}

    \begin{enumerate}[label=(\alph*)]
        \item[iv.] Respecto a una clase:
        \item Se declaran utilizando la palabra resevada clase.
        \item Debe tener el mismo nombre que el archivo.
        \item Son declaradas est\'aticas con static.
        \item Todas deben incluir el m\'etodo main.
        \item Ninguna de las anteriores.
    \end{enumerate}

    \begin{enumerate}[label=(\alph*)]
        \item[v.] Respecto a una clase:
        \item Puede o no tener atributos.
        \item Puede o no tener m\'etodos.
        \item Puede o no tener constructor.
        \item Puede o no tener un nombre.
        \item Puede o no tener un tipo.
    \end{enumerate}
    
    \begin{enumerate}[label=(\alph*)]
        \item[vi.] Un objeto es: 
        \item Un tipo de dato de la clase.
        \item La instancia de una clase.
        \item Una abstracci\'on del mundo real.        
        \item Un sub-conjunto de atributos y m\'etodos de la clase.
        \item Siempre est\'atico.
    \end{enumerate}

    \begin{enumerate}[label=(\alph*)]
        \item[vii.] El principio de ocultamiento: 
        \item Es una t\'ecnica que protege el estado de una entidad.
        \item Es indispensable en el paradigma de orientaci\'on a objeto.
        \item En Java, se logra utilizando los modificadores de acceso.
        \item Es encapsular el conocmineto de una entidad.
        \item Ninguna de las anteriores.
    \end{enumerate}

    \begin{enumerate}[label=(\alph*)]
        \item[viii.] Respecto a la herencia, las clases:
        \item Heredan s\'olo los m\'etodos privados.
        \item Heredan el compotamiento completo de la clase padre.
        \item Heredan s\'olo el comportamiento que se desea utilizar.
        \item En Java, se implementan con la palabra implements.
        \item En Java, se implementan con la palabra extends.
    \end{enumerate}

    \begin{enumerate}[label=(\alph*)]
        \item[ix.] El polimorfismo: 
        \item El mismo nombre implemenao distintas funcionalidades.
        \item Una funcionalidad implementada con distintos nombres.
        \item Es una caracter\'istica de JAVA.
        \item Es una caracter\'istica de POO.
        \item Un ejemplo es el s\'imbolo \%.
    \end{enumerate}

    \begin{enumerate}[label=(\alph*)]
        \item[x.] El met\'odo \emph{main}:
        \item Puede no ser void.
        \item Debe ser static.
        \item Puede no llevar argumentos de entrada.
        \item Debe retornar un valor.
        \item Debe incluirse en un programa.
    \end{enumerate}

\end{multicols}

\newpage

\item \emph{30pts.} Desarrolle una clase Calculadora que permita gestionar las operaciones elementales: suma, resta, multiplicaci\'on, divisi\'on y m\'odulo. Recuerde utilizar los principios de orientaci\'on a objetos.

\newpage

\item \emph{40pts.} Un pal\'indrome es una palabra, n\'umero o frase que puede ser le\'ia de igual forma hacia adelante y hacia atr\'as, por ejemplo:
		\begin{itemize}
			\item[] ``anita lava la tina".
			\item[] ``1234567890987654321".
			\item[] ``la ruta nos aporto otro paso natural".
		\end{itemize}
Desarrolle una clase (en JAVA) que posea los m\'etodos necesarios para determinar si una palabra es o no un pal\'indrome.
\end{enumerate}
}
\end{document} 
