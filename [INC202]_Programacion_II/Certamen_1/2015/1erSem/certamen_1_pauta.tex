\documentclass[10pt]{article}
\usepackage{graphicx}
\usepackage{amssymb}
\usepackage{tikz}
\usepackage{epstopdf}
\usepackage{enumitem}
\usepackage{multicol,multirow}
\DeclareGraphicsRule{.tif}{png}{.png}{`convert #1 `dirname #1`/`basename #1 .tif`.png}
\newcommand*\circled[1]{\tikz[baseline=(char.base)]{\node[shape=circle,blue,draw,inner sep=2pt] (char) {#1};}}

% For a visual definition of these parameters, see
\textwidth = 6.5 in
\textheight = 9 in
\oddsidemargin = 0.0 in
\evensidemargin = 0.0 in
\topmargin = 0.0 in             
\headheight = 0.0 in            
\headsep = 0.0 in
            
\parskip = 0.2in                % vertical space between paragraphs
% Delete the % in the following line if you don't want to have the first line of every paragraph indented
%\parindent = 0.0in

\begin{document}
\begin{center}
    {\Large Pauta Certamen 1, Programaci\'on II} \\
    \emph{\small Prof. Rodrigo Olivares} \\
    \emph{\small Ayud. Diego Agullo} \\
    \emph{\scriptsize Abril 21, 2015}
\end{center}
\vspace*{-35pt}
\begin{center}
    \rule{1\textwidth}{.3pt}
\end{center}
\vspace*{-42pt}
\begin{center}
    \rule{1\textwidth}{2pt}
\end{center}

\vspace*{-15pt}
{\small \textbf{Instrucciones}:}
\vspace*{-15pt}

{\scriptsize
\begin{itemize}
    \item[-] El puntaje m\'aximo del certamen es 100\%, siendo el 60\% el m\'inimo requerido para aprobar.
    \item[-] Responda cada pregunta en la hoja indicada, agregando su nombre. Si no responde alguna pregunta, debe entregar la hoja con su nombre e indicar que \textbf{no responde}.
    \item[-] El certamen es \underline{\textbf{individual}}. Cualquier intento de copia, ser\'a sancionado con nota \textbf{1,0}.
\end{itemize}
\vspace*{10pt}

\vspace*{-30pt}

\begin{enumerate}

    \item \emph{20pts.} De las siguentes afirmaciones, encierre en un c\'irculo la o las alternativas correctas.
    \begin{multicols}{2}

    \begin{enumerate}[label=(\alph*)]
        \item[i.] La orientaci\'on a objeto es: 
        \item[\circled{(a)}] Un paradigma de programaci\'on.
        \item[(b)] Un paradigma de programaci\'on estucturado.
        \item[(c)] Una herramienta de programaci\'on.
        \item[(d)] Un lenguaje de programaci\'on.
        \item[(e)] Ninguna de las anteriores.
    \end{enumerate}

    \begin{enumerate}[label=(\alph*)]
        \item[ii.] Algunos enfoques de la orientaci\'on a objeto son:
        \item[(a)] El enfoque de reusabilidad.
        \item[\circled{(b)}] El enfoque revolucionario.
        \item[\circled{(c)}] El enfoque evolutivo.
        \item[(d)] El enfoque imperativo.
        \item[(e)] El enfoque procedural.
    \end{enumerate}

    \begin{enumerate}[label=(\alph*)]
        \item[iii.] En cuanto a la programaci\'on orientada a objeto:
        \item[(a)] Se apoya en el paradigma procedural.
        \item[\circled{(b)}] Divide el programa en peque\~nas unidades de c\'odigo.
        \item[\circled{(c)}] Proporciona t\'ecnicas para modelar el mundo real.
        \item[(d)] Es un lenguaje de programaci\'on.
        \item[(e)] Ninguna de las anteriores.
    \end{enumerate}

    \begin{enumerate}[label=(\alph*)]
        \item[iv.] Una clase es: 
        \item[(a)] Una colecci\'on de objetos.
        \item[\circled{(b)}] Una abstracci\'on del mundo real.
        \item[\circled{(c)}] Una herramienta de programaci\'on.
        \item[(d)] Un tipo de dato.
        \item[(e)] Ninguna de las anteriores.
    \end{enumerate}

    \begin{enumerate}[label=(\alph*)]
        \item[v.] Respecto a una clase:
        \item[(a)] S\'olo pueden ser p\'ublicas.
        \item[\circled{(b)}] Debe tener el mismo nombre que el archivo.
        \item[\circled{(c)}] Una clase puede o no tener un constructor.
        \item[\circled{(d)}] Una clase puede o no tener atributos.
        \item[(e)] Ninguna de las anteriores.
        \item[]
    \end{enumerate}

    \begin{enumerate}[label=(\alph*)]
        \item[vi.] Un objeto es: 
        \item[\circled{(a)}] Una abstracci\'on del mundo real.
        \item[(b)] Un tipo de dato.
        \item[\circled{(c)}] La instancia de una clase.
        \item[\circled{(d)}] Un conjunto de atributos y m\'etodos.
        \item[(e)] Una plantilla para generar m\'as objetos.
    \end{enumerate}

    \begin{enumerate}[label=(\alph*)]
        \item[vii.] El principio de ocultamiento: 
        \item[(a)] Es una t\'ecnica que protege el estado de una entidad.
        \item[(b)] Es dispensable en el paradigma de orientaci\'on a objeto.
        \item[\circled{(c)}] En Java, se logra utilizando los modificadores de acceso.
        \item[(d)] Es encapsular el conocmineto de una entidad.
        \item[(e)] Ninguna de las anteriores.
    \end{enumerate}

    \begin{enumerate}[label=(\alph*)]
        \item[viii.] Respecto a la herencia, las clases:
        \item[(a)] Heredan s\'olo los m\'etodos de igual nombre.
        \item[\circled{(b)}] Heredan el compotamiento completo de la clase padre.
        \item[(c)] Heredan s\'olo el comportamiento que se desea utilizar.
        \item[(d)] En Java, se implementan con la palabra implements.
        \item[\circled{(e)}] En Java, se implementan con la palabra extends.
    \end{enumerate}

    \begin{enumerate}[label=(\alph*)]
        \item[ix.] Un constructor: 
        \item[\circled{(a)}] Tiene siempre el mismo nombre de la clase.
        \item[\circled{(b)}] Puede o no incluirse en la clase.
        \item[(c)] Debe tener par\'ametos de entrada.
        \item[(d)] No puede ser sobrecargado.
        \item[(e)] Debe incluir el tipo de dato de retorno.
    \end{enumerate}

    \begin{enumerate}[label=(\alph*)]
        \item[x.] El met\'odo \emph{main}:
        \item[\circled{(a)}] Siempre debe ser void.
        \item[\circled{(b)}] Siempre debe ser static.
        \item[\circled{(c)}] Siempre debe llevar argumentos de entrada.
        \item[(d)] Puede retornar un valor.
        \item[(e)] Puede no incluirse en un programa.
    \end{enumerate}

\end{multicols}

\newpage

\item \emph{25pts.} La siguiente clase describe un cierto comportamiento para c\'alculos aritm\'eticos:

\begin{verbatim}
public class Calculo {

    private ArrayList lista = new ArrayList();
    private Random rand = new Random();
    private int size;

    public void crearTamanio() {
        size = rand.nextInt(99) + 1;
    }

    public void llenar() {
        for (int lista = 0; lista < size; lista++) {
            this.lista.add(rand.nextInt(99));
        }
    }

    public int suma() {
        int suma = 0;
        for (int lista = 0; lista < size; lista++) {
            suma += Integer.parseInt(this.lista.get(lista).toString());
        }
        return suma;
    }

    public float promedio() {
        return (this.suma() / size);
    }
}
\end{verbatim}

    \begin{enumerate}[label=(\alph*)]
        \item \emph{5pts.} Construya el m\'etodo \emph{main} que permita crear el tama\~no de la lista, luego la llene y muestre las operaciones atim\'eticas.
        \item \emph{7pts.} Construya un ruteo para la clase anterior (incluyendo el m\'etodo \emph{main}).
        \item \emph{13pts.} Agregue un m\'etodo a la clase Calculo que permita obtener la \emph{mediana} (el dato del medio, de una lista ordenada).
    \end{enumerate}

\newpage

\item \emph{25pts.} Desarrollo las clases necesarias para gestionar un arreglo de n\'umeros de un tama\~no determinado por teclado. El arreglo contendr\'a n\'umeros aleatorios entre 1 y 1000. Se deber\'an mostrar todos aquellos n\'umeros que terminen en un d\'igito que se indique por teclado (se debe controlar que el valor introducido sea correcto). Por ejemplo, en un arreglo de 10 elementos, indicamos mostrar los n\'umeros terminados en 5. La salida ser\'ia: 155, 25, 5, etc.

\newpage

\item \emph{30pts.} Desarrolle las clases nececesarias que permitan clasificar smart-phones dependiento de su sistema operativo (iOS, Andriod, Windows Phone, etc). Los datos asociados a los atributos de un smart-phone deben ser ingresados por la entrada est\'andar (consisere al menos 3 atributos). Al momento de terminar el ingreso, se debe mostrar la lista de cada uno de ellos, indicando la cantidad. Recuerde utilizar todos los conceptos de orientaci\'on a objetos vistos en clase.

\end{enumerate}
}
\end{document} 
