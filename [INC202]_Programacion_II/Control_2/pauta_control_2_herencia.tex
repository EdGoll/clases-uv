\documentclass[10pt]{article}
\usepackage{graphicx}
\usepackage{amssymb}
\usepackage{verbatim}
\usepackage{epstopdf}
\usepackage{color, xcolor}
\DeclareGraphicsRule{.tif}{png}{.png}{`convert #1 `dirname #1`/`basename #1 .tif`.png}

\usepackage{listings}

\lstset{ %
  backgroundcolor=\color{white},   % choose the background color; you must add \usepackage{color} or \usepackage{xcolor}
  basicstyle=\footnotesize,        % the size of the fonts that are used for the code
  breakatwhitespace=false,         % sets if automatic breaks should only happen at whitespace
  breaklines=true,                 % sets automatic line breaking
  captionpos=b,                    % sets the caption-position to bottom
  commentstyle=\color{gray},    % comment style
  deletekeywords={...},            % if you want to delete keywords from the given language
  escapeinside={\%*}{*)},          % if you want to add LaTeX within your code
  extendedchars=true,              % lets you use non-ASCII characters; for 8-bits encodings only, does not work with UTF-8
  frame=none,                       % adds a frame around the code
  keepspaces=true,                 % keeps spaces in text, useful for keeping indentation of code (possibly needs columns=flexible)
  keywordstyle=\color{blue},       % keyword style
  language=Java,                 % the language of the code
  otherkeywords={else},           % if you want to add more keywords to the set
  numbers=none,                    % where to put the line-numbers; possible values are (none, left, right)
  numbersep=5pt,                   % how far the line-numbers are from the code
  numberstyle=\tiny\color{gray}, % the style that is used for the line-numbers
  rulecolor=\color{black},         % if not set, the frame-color may be changed on line-breaks within not-black text (e.g. comments (green here))
  showspaces=false,                % show spaces everywhere adding particular underscores; it overrides 'showstringspaces'
  showstringspaces=false,          % underline spaces within strings only
  showtabs=false,                  % show tabs within strings adding particular underscores
  stepnumber=2,                    % the step between two line-numbers. If it's 1, each line will be numbered
  stringstyle=\color{orange},     % string literal style
  tabsize=2,                       % sets default tabsize to 2 spaces
  title=\lstname                   % show the filename of files included with \lstinputlisting; also try caption instead of title
}

% For a visual definition of these parameters, see
\textwidth = 6.5 in
\textheight = 9 in
\oddsidemargin = 0.0 in
\evensidemargin = 0.0 in
\topmargin = 0.0 in			 
\headheight = 0.0 in			
\headsep = 0.0 in
			
\parskip = 0.2in				% vertical space between paragraphs
% Delete the % in the following line if you don't want to have the first line of every paragraph indented
%\parindent = 0.0in

\begin{document}

\begin{center}
	{\Large Pauta Control 2}\\
    Programaci\'on 2 \\
	\emph{\small Prof. Rodrigo Olivares} \\
\end{center}

\begin{enumerate}
    \item[ ] Considere las siguientes clases:
    \begin{center}
        \lstinputlisting[caption={}]{ClasePrincipal.java}
    \end{center}    
    \begin{center}
        \lstinputlisting[caption={}]{ClaseAbstracta.java}
    \end{center}    
    \item[] Defina la \texttt{ClaseInstanciadora}, la \texttt{ClasePadre} y la \texttt{Interfaz}, para dar respuesta el m\'etodo \texttt{main}.
\end{enumerate}
\textbf{Calificaci\'on:}
\begin{itemize}
\item ClaseInstanciadora (50 pts)
\begin{itemize}
\item Definici\'on de la clase 10 pts.
\item Herencia 8 pts.
\item Declaracion y uso del atributo \texttt{Object z} 3 pts.
\item Constructor 8 pts.
\item Invocaci\'on al constructor padre 8 pts.
\item SobreEscritura 3 pts.
\item Implementaci\'on del \texttt{metodoAbstracto} 10 pts.
\end{itemize}
\item ClasePadre (30 pts)
\begin{itemize}
\item Definici\'on de la clase 10 pts.
\item Constructor 10 pts.
\item Implementaci\'on del \texttt{metodoClasePadre} 10 pts.
\end{itemize}
\item Interfaz (20 pts)
\begin{itemize}
\item Definici\'on de la interfaz 10 pts.
\item Declaraci\'on del \texttt{metodoAbstracto} 10 pts.
\end{itemize}
\end{itemize}
\newpage
\lstinputlisting[caption={}]{ClaseInstanciadora.java}
\lstinputlisting[caption={}]{ClasePadre.java}
\lstinputlisting[caption={}]{Interfaz.java}
\end{document} 