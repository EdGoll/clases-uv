\documentclass{beamer}
%
% Choose how your presentation looks.
%
% For more themes, color themes and font themes, see:
% http://deic.uab.es/~iblanes/beamer_gallery/index_by_theme.html
%
\mode<presentation>
{
  \usetheme{Boadilla}     % or try Boadilla, Darmstadt, Madrid, Warsaw, default, ...
  \usecolortheme{default} % or try Boadilla, albatross, beaver, crane, default, ...
  \usefonttheme{default}  % or try Boadilla, serif, structurebold, default, ...
  \setbeamertemplate{navigation symbols}{}
  \setbeamertemplate{caption}[numbered]
} 

\usepackage[english]{babel}
\usepackage[utf8x]{inputenc}

\title[Software Colaborativo lantonistas]{Um modelo de software colaborativo com suporte à troca de
informações entre equipes médicas plantonistas}
%\author[CDTec] {Cristian Pereira, André Einhardt, Mateus Nascimento}
\author[CDTec] {Vinícius Tocantins Marques, Cristiano André da Costa, Jorge Luis Victoria Barbosa, Rodrigo da Rosa Righi}

\institute[]{Apresentado por:André Einhardt, Mateus Nascimento}
\date{Setembro 2015}


\begin{document}

\begin{frame}
    \begin{center}
  \titlepage 
      \begin{figure}[h]
  \includegraphics[width=2.3cm,height=2.3cm] {img/uv_logo_alta_rgb.jpg}
  
%\includegraphics[width=3.5cm,height=2.1cm,scale=1] {img/lups_logo_letras.png}
      \end{figure}
  \end{center}
\end{frame}

% Uncomment these lines for an automatically generated outline.
%\begin{frame}{Outline}
%  \tableofcontents
%\end{frame}

\section{Sumário}
  \begin{frame}{Sumário}
    \begin{itemize}
      \item Introdução e Proposta do Artigo
      \item Motivações e Objetivos
      \item Trabalhos Relacionados     
      \item Contribuição do Trabalho
      \item Considerações Finais
    \end{itemize}
  \vskip 5cm
\end{frame}

\section{Corpo Apresentação}
% ================== ÁREAS SENDO ALTERADAS POR CRISTIAN PEREIRA ==================
\begin{frame}{Introdução e Proposta do Artigo}
Modelo colaborativo denominado Doctor Collab
\begin{itemize}
    \item Facilita a rotina de profissionais da area da saúde durante a troca de equipe, permitindo a visualização de dados dos prontuários
    \item Produz uma rede de colaboração cruzando dados de fontes heterogêneas em prol do sucesso do atendimento médico
    \item Utiliza Gerenciamento de tarefas, Redes Bayesianas e ciência da situação
    \item Acesso a partir de dispositivos móveis
\end{itemize}
\vskip 3cm
\end{frame}

\begin{frame}{Motivações e Objetivos}
    \begin{block}{Objetivos}
      \begin{itemize}
        \item Propor um modelo colaborativo baseado em nuvel computacional móvel, auxiliando na troca de informações de médicos plantonistas.
        \item Ajudar na tomada de decisão de médicos plantonistas.
    \end{itemize}
  \end{block}
  \vskip 1cm
  \begin{block}{Trabalhos Relacionados}
    \begin{itemize}
        \item Empregar inferência através da utilização de Redes Bayesuanas (dá suporte à colaboração entre as equipes).
        \item Arquitetura baseada em núvem computacional acessada a partir de dispositivos móveis.
    \end{itemize}
  \end{block}
\end{frame}



\begin{frame}{Trabalhos Relacionados}
    \begin{block}{Objetivos}
      \begin{itemize}
        \item Propor um modelo colaborativo baseado em nuvel computacional móvel, auxiliando na troca de informações de médicos plantonistas.
        \item Ajudar na tomada de decisão de médicos plantonistas.
    \end{itemize}
  \end{block}
  \vskip 1cm
  \begin{block}{Trabalhos Relacionados}
    \begin{itemize}
        \item Empregar inferência através da utilização de Redes Bayesuanas (dá suporte à colaboração entre as equipes).
        \item Arquitetura baseada em núvem computacional acessada a partir de dispositivos móveis.
    \end{itemize}
  \end{block}
\end{frame}
\begin{frame}{Contribuição do Trabalho}
No modelo proposto algumas premissas foram definidas:
\begin{block}{Premissas}
  \begin{enumerate}[I]
    \item Processamento em nuvem computacional; 
    \item Suporte a colaboração na camada de aplicação; 
    \item Organização das informações em base ontológica; 
    \item Elasticidade, escalabilidade e disponibilidade fornecidas pela arquitetura em nuvem.
  \end{enumerate}
\end{block}
\begin{itemize}
    \item Diversas equipes com acesso a dados dos pacientes (base de dados em nuvem).
    \item Contribuição de forma colaborativa para evolução do quadro médico do paciente.
    \item Padrões pré-estabelecidos e gerenciamento de tarefas (objetivando a conciência situacional)
  \end{itemize}
\end{frame}

\begin{frame}{Considerações Finais}
    Visão geral do Doctor Collab
  \begin{figure}
   %\includegraphics[width=9cm,height=7.5cm] {img/VisaoModeloDoctorCollab.png}
  \end{figure}
\end{frame}


% =================================== FIM MATEUS =================================
\end{document}

              