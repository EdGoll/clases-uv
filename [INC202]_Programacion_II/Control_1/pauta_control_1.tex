\documentclass[10pt]{article}
\usepackage{graphicx}
\usepackage{amssymb}
\usepackage{verbatim}
\usepackage{epstopdf}
\usepackage{color, xcolor}
\DeclareGraphicsRule{.tif}{png}{.png}{`convert #1 `dirname #1`/`basename #1 .tif`.png}

\usepackage{listings}

\lstset{ %
  backgroundcolor=\color{white},   % choose the background color; you must add \usepackage{color} or \usepackage{xcolor}
  basicstyle=\footnotesize,        % the size of the fonts that are used for the code
  breakatwhitespace=false,         % sets if automatic breaks should only happen at whitespace
  breaklines=true,                 % sets automatic line breaking
  captionpos=b,                    % sets the caption-position to bottom
  commentstyle=\color{gray},    % comment style
  deletekeywords={...},            % if you want to delete keywords from the given language
  escapeinside={\%*}{*)},          % if you want to add LaTeX within your code
  extendedchars=true,              % lets you use non-ASCII characters; for 8-bits encodings only, does not work with UTF-8
  frame=none,                       % adds a frame around the code
  keepspaces=true,                 % keeps spaces in text, useful for keeping indentation of code (possibly needs columns=flexible)
  keywordstyle=\color{blue},       % keyword style
  language=Java,                 % the language of the code
  otherkeywords={else},           % if you want to add more keywords to the set
  numbers=none,                    % where to put the line-numbers; possible values are (none, left, right)
  numbersep=5pt,                   % how far the line-numbers are from the code
  numberstyle=\tiny\color{gray}, % the style that is used for the line-numbers
  rulecolor=\color{black},         % if not set, the frame-color may be changed on line-breaks within not-black text (e.g. comments (green here))
  showspaces=false,                % show spaces everywhere adding particular underscores; it overrides 'showstringspaces'
  showstringspaces=false,          % underline spaces within strings only
  showtabs=false,                  % show tabs within strings adding particular underscores
  stepnumber=2,                    % the step between two line-numbers. If it's 1, each line will be numbered
  stringstyle=\color{orange},     % string literal style
  tabsize=2,                       % sets default tabsize to 2 spaces
  title=\lstname                   % show the filename of files included with \lstinputlisting; also try caption instead of title
}
% For a visual definition of these parameters, see
\textwidth = 6.5 in
\textheight = 9 in
\oddsidemargin = 0.0 in
\evensidemargin = 0.0 in
\topmargin = 0.0 in			 
\headheight = 0.0 in			
\headsep = 0.0 in
			
\parskip = 0.2in				% vertical space between paragraphs
% Delete the % in the following line if you don't want to have the first line of every paragraph indented
%\parindent = 0.0in

\begin{document}

\begin{center}
	{\Large Pauta Control 1}\\
	Programaci\'on 2 \\
	\emph{\small Prof. Rodrigo Olivares}
\end{center}

\begin{enumerate}
    \item[ ] Utilizando los conceptos del Paradigma de Orientaci\'on a Objetos, construya una aplicaci\'on en Java que gestione las operaciones elementales de una calculadora: \emph{suma}, \emph{resta}, \emph{multiplicaci\'on}, \emph{divisi\'on} y \emph{m\'odulo}; las operaciones de trasnformaci\'on de sistema num\'erico: decimal, octal y binario; y al menos 3 funciones matem\'aticas (\emph{sin}, \emph{cos}, \emph{tag}, \emph{abs}, \emph{pow}, \emph{sqrt}, etc). La clase desarrollada, debe contener:
    \begin{enumerate}
	    \item Un atributo double para almacenar el valor resultante.
	    \item Los m\'etodos para cada operaci\'on.
	    \item El m\'etodo \emph{toString}() para retornar la operaci\'on realizada y el valor resultante. 
	    \item El m\'etodo principal.
	\end{enumerate} 
	\item[] Utilice los m\'etodos miembros de la clase Integer.
	\begin{itemize}
	    \item toBinaryString(int i): retorna un String Binario.
	    \item toOctalString(int i): retorna un String Octal.
	    \item toHexString(int i): retorna un String Hexadecimal.
	\end{itemize}
	\item[] Utilice el m\'etodo miembro de la clase Double.
	\begin{itemize}
	    \item intValue(): retorna el valor entero de un double.
	\end{itemize}
\end{enumerate}
\textbf{Calificaci\'on}
\begin{itemize}
\item Declaraci\'on de la clase = 2
\item Atributo resultado = 3
\item Constructor = 5
\item M\'etodo suma = 7
\item M\'etodo resta = 7
\item M\'etodo multiplicaci\'on = 7
\item M\'etodo divisi\'on = 7
\item M\'etodo m\'odulo = 7
\item M\'etodo sin = 7
\item M\'etodo cos = 7
\item M\'etodo tan = 7
\item M\'etodo binario = 7
\item M\'etodo octal = 7
\item M\'etodo hexadecimal = 7
\item M\'etodo toString = 5
\item M\'etodo main = 8
\end{itemize}
\newpage
\lstinputlisting[caption={}]{Calculadora.java}
\end{document} 