\documentclass[10pt]{article}
\usepackage{graphicx}
\usepackage{amssymb}
\usepackage{verbatim}
\usepackage{epstopdf}
\DeclareGraphicsRule{.tif}{png}{.png}{`convert #1 `dirname #1`/`basename #1 .tif`.png}

% For a visual definition of these parameters, see
\textwidth = 6.5 in
\textheight = 9 in
\oddsidemargin = 0.0 in
\evensidemargin = 0.0 in
\topmargin = 0.0 in			 
\headheight = 0.0 in			
\headsep = 0.0 in
			
\parskip = 0.2in				% vertical space between paragraphs
% Delete the % in the following line if you don't want to have the first line of every paragraph indented
%\parindent = 0.0in

\begin{document}

\begin{center}
	{\Large Control 1}\\
	Programaci\'on 2 \\
	\emph{\small Prof. Rodrigo Olivares} \\
	\emph{\scriptsize \date{13 de abril de 2017}}
\end{center}

\begin{enumerate}
    \item[ ] Utilizando los conceptos del Paradigma de Orientaci\'on a Objetos, construya una aplicaci\'on en Java que gestione las operaciones elementales de una calculadora: \emph{suma}, \emph{resta}, \emph{multiplicaci\'on}, \emph{divisi\'on} y \emph{m\'odulo}; las operaciones de trasnformaci\'on de sistema num\'erico: decimal, octal y binario; y al menos 3 funciones matem\'aticas (\emph{sin}, \emph{cos}, \emph{tag}, \emph{abs}, \emph{pow}, \emph{sqrt}, etc). La clase desarrollada, debe contener:
    \begin{enumerate}
	    \item Un atributo double para almacenar el valor resultante.
	    \item Los m\'etodos para cada operaci\'on.
	    \item El m\'etodo \emph{toString}() para retornar la operaci\'on realizada y el valor resultante. 
	    \item El m\'etodo principal.
	\end{enumerate} 
	\item[] Utilice los m\'etodos miembros de la clase Integer.
	\begin{itemize}
	    \item toBinaryString(int i): retorna un String Binario.
	    \item toOctalString(int i): retorna un String Octal.
	    \item toHexString(int i): retorna un String Hexadecimal.
	\end{itemize}
	\item[] Utilice el m\'etodo miembro de la clase Double.
	\begin{itemize}
	    \item intValue(): retorna el valor entero de un double.
	\end{itemize}
\end{enumerate}

\end{document} 