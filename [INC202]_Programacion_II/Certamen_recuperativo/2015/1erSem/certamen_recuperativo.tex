\documentclass[10pt]{article}
\usepackage{graphicx}
\usepackage{amssymb}
\usepackage{epstopdf}
\usepackage{tikz}
\usepackage{enumitem}
\usepackage{multicol,multirow}
\DeclareGraphicsRule{.tif}{png}{.png}{`convert #1 `dirname #1`/`basename #1 .tif`.png}
\renewcommand{\tablename}{Tabla} 
\renewcommand{\figurename}{Figura} 
\newcommand*\circled[1]{\tikz[baseline=(char.base)]{\node[shape=circle,blue,draw,inner sep=1pt] (char) {#1};}}

% For a visual definition of these parameters, see
\textwidth = 6.5 in
\textheight = 9 in
\oddsidemargin = 0.0 in
\evensidemargin = 0.0 in
\topmargin = 0.0 in             
\headheight = 0.0 in            
\headsep = 0.0 in
            
\parskip = 0.2in                % vertical space between paragraphs
% Delete the % in the following line if you don't want to have the first line of every paragraph indented
%\parindent = 0.0in

\begin{document}
\begin{center}
    %{\Large Pauta Certamen Recuperativo, Programaci\'on II} \\
    {\Large Certamen Recuperativo, Programaci\'on II} \\
    \emph{\small Prof. Rodrigo Olivares} \\

\end{center}
\vspace*{-35pt}
\begin{center}
    \rule{1\textwidth}{.3pt}
\end{center}
\vspace*{-42pt}
\begin{center}
    \rule{1\textwidth}{2pt}
\end{center}

\vspace*{-15pt}
{\small \textbf{Instrucciones}:}
\vspace*{-15pt}

{\scriptsize
\begin{itemize}
    \item[-] El puntaje m\'aximo del certamen es 100\%, siendo el 60\% el m\'inimo requerido para aprobar.
    \item[-] El certamen es \underline{\textbf{individual}}. Cualquier intento de copia, ser\'a sancionado con nota \textbf{1,0}.
\end{itemize}
\vspace*{10pt}

\vspace*{-30pt}

\begin{enumerate}

    \item \emph{100pts.} De las siguentes afirmaciones, encierre en un c\'irculo la o las alternativas correctas (\emph{4pts c/u}).
    \begin{multicols}{2}

	\begin{enumerate}[label=(\alph*)]
        \item[i.] La orientaci\'on a objeto es: 
        \item[(a)] Un paradigma de programaci\'on procedural.
        \item[(b)] Un paradigma de programaci\'on estucturado.
        \item[(c)] Una herramienta de programaci\'on.
        \item[(d)] Un lenguaje de programaci\'on.
        \item[(e)] Ninguna de las anteriores.
    \end{enumerate}

    \begin{enumerate}[label=(\alph*)]
        \item[ii.] Algunos enfoques de la orientaci\'on a objeto son:
        \item[(a)] El enfoque de reusabilidad.
        \item[(b)] El enfoque revolucionario.
        \item[(c)] El enfoque evolutivo.
        \item[(d)] El enfoque imperativo.
        \item[(e)] El enfoque procedural.
    \end{enumerate}

    \begin{enumerate}[label=(\alph*)]
        \item[iii.] En cuanto a la programaci\'on orientada a objeto:
        \item[(a)] Se apoya en el paradigma procedural.
        \item[(b)] Divide el programa en peque\~nas unidades de c\'odigo.
        \item[(c)] Proporciona t\'ecnicas para modelar el mundo real.
        \item[(d)] Es un lenguaje de programaci\'on.
        \item[(e)] Es una herramienta de programaci\'on.
    \end{enumerate}

    \begin{enumerate}[label=(\alph*)]
        \item[iv.] Una clase es: 
        \item[(a)] Una colecci\'on de objetos.
        \item[(b)] Una abstracci\'on del mundo real.
        \item[(c)] Una herramienta de programaci\'on.
        \item[(d)] Un tipo de dato.
        \item[(e)] Ninguna de las anteriores.
    \end{enumerate}

    \begin{enumerate}[label=(\alph*)]
        \item[v.] Respecto a una clase:
        \item[(a)] S\'olo pueden ser p\'ublicas.
        \item[(b)] Debe tener el mismo nombre que el archivo.
        \item[(c)] Debe tener al menos un constructor.
        \item[(d)] Su constructor debe tener el mismo nombre.
        \item[(e)] Puede o no tener atributos.
    \end{enumerate}

    \begin{enumerate}[label=(\alph*)]
        \item[vi.] Un objeto es: 
        \item[(a)] Una abstracci\'on del mundo real.
        \item[(b)] Un tipo de dato.
        \item[(c)] La instancia de una clase.
        \item[(d)] Un conjunto de atributos y m\'etodos.
        \item[(e)] Una plantilla para generar m\'as objetos.
    \end{enumerate}

    \begin{enumerate}[label=(\alph*)]
        \item[vii.] El principio de ocultamiento: 
        \item[(a)] Es una t\'ecnica que protege el estado de una entidad.
        \item[(b)] Es \'util en enfoques procedurales.
        \item[(c)] En Java, se logra utilizando los modificadores de acceso.
        \item[(d)] Es encapsular el conocimiento de una entidad.
        \item[(e)] Ninguna de las anteriores.
    \end{enumerate}

    \begin{enumerate}[label=(\alph*)]
        \item[viii.] Un constructor: 
        \item[(a)] No siempre tiene el mismo nombre de la clase.
        \item[(b)] Puede o no incluirse en la clase.
        \item[(c)] Debe tener par\'ametos de entrada.
        \item[(d)] No puede ser sobrecargado.
        \item[(e)] Debe incluir el tipo de dato de retorno.
    \end{enumerate}

    \begin{enumerate}[label=(\alph*)]
        \item[ix.] El met\'odo \emph{main}:
        \item[(a)] Siempre debe ser void.
        \item[(b)] Siempre debe ser static.
        \item[(c)] Siempre debe llevar argumentos de entrada.
        \item[(d)] Puede retornar un valor.
        \item[(e)] Puede no incluirse en un programa.
    \end{enumerate}

	\begin{enumerate}[label=(\alph*)]
        \item[x.] Una clase abstracta:
        \item[(a)] Tiene al menos un m\'etodo impementado.
        \item[(b)] Tiene al menos un m\'etodo abstracto.
        \item[(c)] Tiene todos sus m\'etodos abstractos
        \item[(d)] Es factible de ser implementada.
        \item[(e)] Es factible de ser extendida.
    \end{enumerate}

    \begin{enumerate}[label=(\alph*)]
        \item[xi.] Una interface:
        \item[(a)] Tiene al menos un m\'etodo impementado.
        \item[(b)] Tiene al menos un m\'etodo abstracto.
        \item[(c)] Tiene todos sus m\'etodos abstractos
        \item[(d)] Es factible de ser implementada.
        \item[(e)] Es factible de ser extendida.
    \end{enumerate}

    \begin{enumerate}[label=(\alph*)]
        \item[xii.] Respecto a la herencia, las clases:
        \item[(a)] Heredan s\'olo los m\'etodos de igual nombre.
        \item[(b)] Heredan el comportamiento completo de la clase padre.
        \item[(c)] Heredan s\'olo el comportamiento que se desea utilizar.
        \item[(d)] En Java, se implementan con la palabra implements.
        \item[(e)] En Java, se implementan con la palabra extends.
    \end{enumerate}

    \begin{enumerate}[label=(\alph*)]
        \item[xiii.] La herencia m\'ultiple:
        \item[(a)] Permite heredar diverso compartimiento.
        \item[(b)] Apoya el principio ocultamiento.
        \item[(c)] Apoya el principio de encapsulamiento.
        \item[(d)] En Java se desarrolla implementado clases abstractas.
        \item[(e)] En Java se desarrolla implementado interfaces.
    \end{enumerate}

    \begin{enumerate}[label=(\alph*)]
        \item[xiv.] Un TDA Bean:
        \item[(a)] No requiere constructor.
        \item[(b)] Puede tener sobre carga de constructor.
        \item[(c)] Apoya el principio ocultamiento.
        \item[(d)] Apoya el principio de encapsulamiento.
        \item[(e)] Tiene m\'etodos privados y atributos p\'ublicos.
    \end{enumerate}

    \begin{enumerate}[label=(\alph*)]
        \item[xv.] Una clase est\'atica:
        \item[(a)] Requiere instanciaci\'on.
        \item[(b)] No requiere instanciaci\'on.
        \item[(c)] No es viable crear instancias. 
        \item[(d)] Sus atributos y m\'etodos son miembros del objeto.
        \item[(e)] Ninguna de las anteriores
    \end{enumerate}

    \begin{enumerate}[label=(\alph*)]
        \item[xvi.] Respecto a las interfaces gr\'afica en Java:
        \item[(a)] Swing sustituye a AWT.
        \item[(b)] AWT sustituye a Swing.
        \item[(c)] Swing y AWT poseen los mismos componentes.
        \item[(d)] Swing incopora los AWT Components.
        \item[(e)] Ninguna de las anteriores
    \end{enumerate}

    \begin{enumerate}[label=(\alph*)]
        \item[xvii.] Referente a JFrame:
        \item[(a)] Habitualmente se usa para crear la ventana principal.
        \item[(b)] Su m\'etodo getContentPane() obtiene el panel principal.
        \item[(c)] Su m\'etodo add() permite agregar componentes al panel.
        \item[(d)] Su m\'etodo setSize() permite dimensionar la ventana.
        \item[(e)] Ninguna de las anteriores
    \end{enumerate}

    \begin{enumerate}[label=(\alph*)]
        \item[xviii.] Para realizar acciones desde un bot\'on Se requiere:
        \item[(a)] Crear una clase que implemente un ActionEvent.
        \item[(b)] Crear una clase que implemente un ActionListener.
        \item[(c)] Sobreescribir el m\'etodo actionList(ActionPerformed)
        \item[(d)] Sobreescribir el m\'etodo actionPerformed(ActionEvent)
        \item[(e)] Agregar la instancia de la clase oyente, al bot\'on.
    \end{enumerate}

    \begin{enumerate}[label=(\alph*)]
        \item[xix.] Algunos JComponents : 
        \item[(a)] JPanel, JScrollPanel, JDialog.
        \item[(b)] JPanel, JScrollPane, JDialog.
        \item[(c)] JFileChouser, JScrollPane, JLabel.
        \item[(d)] JList, JButton, JTextField.
        \item[(e)] JPassword, JFrame, JTextArea.
    \end{enumerate}

    \begin{enumerate}[label=(\alph*)]
        \item[xx.] \emph{4pts.} Sobre los JLabel: 
        \item[(a)] Su constructor recibe texto.
        \item[(b)] Su constructor recibe im\'agenes.
        \item[(c)] Su constructor recibe texto e im\'agenes, al mismo tiempo.
        \item[(d)] Su constructor recibe el tama\~no del texto a mostrar.
        \item[(e)] No pueden ser modificados en ejecuci\'on.
    \end{enumerate}

\end{multicols}

\end{enumerate}
}
\end{document} 
