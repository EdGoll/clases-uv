\documentclass{article}

\usepackage[spanish]{babel}
\usepackage[latin1]{inputenc}
\usepackage{colortbl}
\usepackage{amsmath}
\usepackage{graphicx}
\textwidth = 6.5 in
\textheight = 9 in
\oddsidemargin = 0.0 in
\evensidemargin = 0.0 in
\topmargin = 0.0 in
\headheight = 0.0 in
\headsep = 0.0 in
\begin{document}
\begin{center}
	{\Large Control 3}\\
	Programaci\'on 2 \\
	\emph{\small Prof. Eduardo Godoy} \\
	%\emph{\scriptsize \date{mayo de 2017}}
\end{center}
	\centerline{\sc \normalsize Escuela de Ingenier\'ia Civil Inform\'atica}
	\centerline{\sc \normalsize  Universidad de Valpara\'iso}
	%\centerline{\sc \small 26 de Octubre de 2015}

	\vspace{1pc}

\begin{itemize}
  \item \emph{30pts.} Una empresa de autobuses le ha solicitado a Ud, como alumno de la Escuela de Ingenier\'ia Civil Inform\'atica que desarrolle una simulaci\'on del proceso de subida y bajada de pasajeros. Los requisitos que le impone la empresa son:

  \begin{itemize}
      \item La simulaci\'on se realiza s\'olo considerando un autob\'us
      \item La capacidad m\'axima de un autob\'us es 45 pasajeros (s\'olo se consideran pasajeros sentados).
      \item Si el autob\'us est\'a vac\'io, no puede detenerse para bajar pasajeros.
      \item Si el auto est\'a lleno no puede detenerse para subir pasajeros.
      \item La cantidad de pasajeros que suben y bajan es aleatoria entre 1 y 10.
      \item La frecuencia de subida y bajada de pasajeros es aleatoria en 1 y 5 segundos.
  \end{itemize}

  \textbf{Condiciones de entrega:}
  \begin{itemize}
      \item[-] \textbf{Debe compilar}.
      \item[-] Debe considerar la creaci\'on de al menos dos thread para los proceso y un recurso compartido (clase que contiene los metodos sincronizados)
      \item[-] Enviar al correo eduardo.gl@gmail.com \textbf{S\'OLO} los archivos \textbf{*.java}, eliminado de ante mano la instrucci\'on \textbf{package}, en un comprimido ApellidoPaternoNombreC3.zip. El no cumplimiento del formato ser\'a penalizado con 10 punto de descuento.
      \item[-] Entrega jueves 9 antes de llas 2:00 am. Subir a la hora que se les indica en el certamen. El no cumplimiento con la hora de entrega, ser\'a penalizado con 1 punto de descuento por cada minuto de retraso.
  \end{itemize}
\end{itemize}


	\begin{table}[!ht]
		 {\scriptsize
			\begin{center}
					 \begin{tabular}{|p{4cm}|p{4cm}|p{4cm}|p{4cm}|}\hline
							\multicolumn{4}{|c|}{\textbf{\textquestiondown C\'omo ser\'e evaluado este Control?} } \\ \hline
							\multicolumn{1}{|c|}{\textbf{T\'opico}} &
							\multicolumn{1}{c|}{\textbf{Logrado}} &
							\multicolumn{1}{c|}{\textbf{Medianamente logrado}} &
							\multicolumn{1}{c|}{\textbf{No logrado}} \\ \hline
							T\'opico 1 - Clase Autobus. &
							\emph{5pts} Crea las clase MonitorAutobus con sus atributos y constructor requerido. &
							\emph{2pts} Crea clase con algunos  atributo o m\'etodos	requeridos en el. Crea m\'etodos o Atributos en otras clases no indicadas en el problema. &
							\emph{  0pts} No crea clases requeridas. \\ \hline

							T\'opico 1-a - Clase Autobus - M\'etodo de ingreso de pasajeros. &
							\emph{20pts} Define e implementa correctamente el m\'etodo relacionado \emph{ingreso de pasajero}  dentro de la clase MonitorAutobus obteniendo la salida requerida.  &
							\emph{10pts} Define e implementa m\'etodo acercandose parcialmente a la salida esperada. &
							\emph{ 0pts} No define m\'etodo.  M\'etodo definido pero no cumple con lo m\'inimo esperado.\\ \hline

							T\'opico 1-b - Clase Autobus - M\'etodo de salida de pasajeros. &
							\emph{20pts} Define e implementa correctamente el m\'etodo \emph{salida de pasajero}  dentro de la clase MonitorAutobus obteniendo la salida requerida. &
							\emph{10pts} Define e implementa m\'etodo acercandose parcialmente a la salida esperada. &
							\emph{ 0pts} No define m\'etodo.  M\'etodo definido pero no cumple con lo m\'inimo esperado. \\ \hline

							T\'opico 2 - ThreadSalida&
							\emph{20pts} Define e implementa la clase Thread para simular el ingreso de pasajeros con sus m\'etodos y  atributos requeridos.&
							\emph{15pts} Define e implementa m\'etodo acercandose parcialmente a la salida esperada.  &
							\emph{ 0pts} No define m\'etodo.  M\'etodo definido pero no cumple con lo m\'inimo esperado. \\ \hline

              T\'opico 3 - ThreadIngreso&
							\emph{20pts} Define e implementa la clase Thread para simular la salida de pasajeros con sus m\'etodos y  atributos requeridos.&
							\emph{10pts} Define e implementa m\'etodo acercandose parcialmente a la salida esperada.  &
							\emph{ 0pts} No define m\'etodo.  M\'etodo definido pero no cumple con lo m\'inimo esperado. \\ \hline

              T\'opico 4 - Clase Inicializadora &
							\emph{10pts} Define e implementa la clase de control que permite iniciar el proceso de simulaci\'on (clase impl) con su m\'etodo main incluido.&
							\emph{5pts} Define e implementa clase y m\'etodo acercandose parcialmente a la salida esperada.  &
							\emph{ 0pts} No define m\'etodo.  M\'etodo definido pero no cumple con lo m\'inimo esperado. \\ \hline

							Paradigma Orientaci\'on a Objetos  &
							\emph{5pts} Resuelve el problema utilizando el POO. &
							\emph{3pts} Utiliza parte del POO para resolver el problema. &
							\emph{0pts} No utiliza el POO para dar soluci\'on al problema.\\ \hline
							Total m\'aximo puntaje pregunta 2 &
							\emph{100pts} &
							\emph{50pts} &
							\emph{  0pts} \\ \hline
					\end{tabular}
			\end{center}}
	 \end{table}

\end{document}
