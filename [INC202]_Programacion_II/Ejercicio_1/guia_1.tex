\documentclass{article}

\usepackage{colortbl}
\usepackage{amsmath}

\begin{document}

	\centerline{\sc \large Gu\'ia 1 de ejercicios: Programaci\'on 2}
	\centerline{\sc \normalsize Escuela de Ingenier\'ia Civil Inform\'atica}
	\centerline{\sc \normalsize  Universidad de Valpara\'iso}

	\vspace{1pc}

	\begin{enumerate}
		\item Desarrolle una clase \textbf{MedicionPersona} que posea:
		\begin{enumerate}
			\item Caracter\'isticas: \emph{nombre(s)}, \emph{apellidos}, \emph{edad}, \emph{RUT}, \emph{sexo} (H hombre, M mujer), \emph{peso} y \emph{altura}. Considerar el tipo de dato id\'oneo para cada uno de los atributos y definir el valor por omisi\'on acorde a cada tipo.
			\item Comportamiento: \textbf{comprobarSexo()}, \textbf{esMayorDeEdad()}, \textbf{calcularIMC()} y \textbf{toString()} para mostrar los atributos del objeto. 
			\begin{enumerate}
				\item \emph{comprobarSexo}(): determina si la persona es \emph{hombre} o \emph{mujer}.
				\item \emph{esMayorDeEdad}(): determina si la persona es o no mayor de edad.
				\item \emph{calcularIMC}(): c\'alculo del \'indice de masa corporal ($\frac{peso}{altura^{2}}$). IMC ideal: $20 \leq IMC \leq 24,9$.
				\item \emph{toString}(): retorna toda la informaci\'on de medici\'on.
			\end{enumerate}
			\item[] Crear una clase externa que realice lo siguiente:
			\begin{enumerate}
				\item Instanciar 3 objetos con datos ingresados manualmente.
				\item Para cada objeto, se debe comprobar si la persona est\'a eb est\'a en su peso ideal, tiene sobrepeso o esta\'a por debajo de su peso ideal (con un mensaje en la salida est\'andar).
				\item Indicar para cada objeto si la persona es mayor de edad.
				\item Por \'ultimo, para cada objeto, se debe mostrar la informaci\'on de la medici\'on.
			\end{enumerate}
		\end{enumerate}
		\item Desarrolle una clase \textbf{Password} que posea:
		\begin{enumerate}
			\item Caracter\'isticas: \emph{longitud} y \emph{contrasenia}. Por omisi\'on, la longitud ser\'a de 8 (puede ser modificada).
			\item Comportamiento: \textbf{esFuerte()}.
			\begin{enumerate}
				\item \emph{generarPassword}(): genera la contrase\~na del objeto, con la longitud que tenga.
				\item \emph{esFuerte}(): entrega un booleano si es fuerte o no (para ser fuerte, debe tener m\'as de 2 may\'usculas, m\'as de 1 min\'uscula y m\'as de 5 n\'umeros). Utilice el m\'etodo \emph{isUpperCase(char)} de clase Character para determinar si la letra es may\'uscula.
			\end{enumerate}
			\item[] Crear una clase externa que realice lo siguiente:
			\begin{enumerate}
				\item Instanciar 3 objetos con datos ingresados manualmente.
				\item Para cada objeto, se debe agregar una contrase\~na y verificar si es o no fuerte.
			\end{enumerate}
		\end{enumerate}
		\item Desarrolle una clase que permita saber si una palabra, frase u oraci\'on es o no pal\'indrome. Para ser pal\'indrome, la palabra, frase u oraci\'on debe ser le\'ia tanto de izquierda a derecha como de derecha a izquierda (no se concideran los espacios en blanco, tildes, may\'usuclas/min\'usculas, etc). Ejemplos:
		\begin{enumerate}
			\item[-] ANITA LAVA LA TINA 
			\item[-] LA RUTA NOS APORTO OTRO PASO NATURAL
			\item[-] LUZ AZUL 
			\item[-] ALA
		\end{enumerate}
%		\item Dado el siguiente programa:
%		\begin{verbatim}
%		class A {
%
%		  int[] x = {1, 2}, y;
%
%		  public void f(int z[]) {
%		    z[1] += 2;
%		  }
%
%		  public void g() {
%		    A a = new A();
%		    a.x[0]++;
%		  }
%
%		  public static void main(String args[]) {
%		    A b = new A();
%		    b.y = b.x;
%		    b.f(b.y);
%		    b.g();
%		    System.out.println(b.x[0] + " " + b.x[1]);          (1)
%		    System.out.println(b.y[0] + " " + b.y[1]);          (2)
%		  }
%		}
%		\end{verbatim}
%		\begin{enumerate}
%			\item \textquestiondown Cu\'al es la salida de la l\'inea (1)?
%			\item \textquestiondown Cu\'al es la salida de la l\'inea (2)?
%		\end{enumerate}
		\item Desarrolle la clase \textbf{FechaHora} que posea:
		\begin{enumerate}
			\item Caracter\'isticas: \emph{anio, mes, dia, horas, minutos y segundos}.
			\item Comportamiento: \textbf{verificarFechaCorrecta()}, \textbf{verificarHoraCorrecta()}, \textbf{toString()}. 
			\begin{enumerate}
				\item \emph{verificarFechaCorrecta}(): entrega un booleano. Veririca si la fecha ingresada es correcta, por ejemplo \emph{30/2/2015} es falso.
				\item \emph{verificarHoraCorrecta}(): entrega un booleano. Veririca si la hora ingresada es correcta, por ejemplo \emph{25:75:84} es falso.
				\item \emph{toString}(): entrega una cadena de texto con la fecha y hora ingresada. Debe definir el formato de salida.
			\end{enumerate}
			\item[] Crear una clase externa que realice lo siguiente:
			\begin{enumerate}
				\item Instanciar un objeto de la clase FechaHora, con datos ingresados manualmente.
				\item Para el objeto, se debe verificar la correctitud de la fecha y la hora.
				\item Mostrar la feha y hora. El formato lo define el desarrollador.
			\end{enumerate}
		\end{enumerate}
		\item Desarrolle las clases Tri\'angulo, Rect\'angulo y C\'irculo que posea:
		\begin{enumerate}
			\item Caracter\'isticas: depende de la figura geom\'etrica. Determine el tipo de dato correcto para cada atributo.
			\item Comportamiento: \textbf{getPerimetro()}, \textbf{getArea()}, \textbf{getTipo()}. 
			\begin{enumerate}
				\item \emph{getPerimetro}(): calcular el per\'imetro para el tri\'angulo, rect\'angulo y c\'irculo.
				\item \emph{getArea}(): calcular el \'area para el tri\'angulo, rect\'angulo y c\'irculo.
				\item \emph{getTipo}(): entrega una cadena de texto que indica el tipo de la figura geom\'etrica, por ejemplo: Tri\'angulo: \emph{Equilatero, Is\'oceles o Escaleno}.
			\end{enumerate}
			\item[] Crear una clase externa que realice lo siguiente:
			\begin{enumerate}
				\item Instanciar 3 objetos con datos ingresados manualmente.
				\item Para cada objeto, se debe calcular el per\'imetro y \'area.
				\item Indicar tipo de la figura geom\'etrica.
			\end{enumerate}
		\end{enumerate}
%		\item Desarrolle una clase que permita gestionar la comunicaci\'on entre el alfabeto y el c\'odigo morse.
%		\begin{center}
%			\begin{tabular}{|c|c|c|c|c|c|} \hline
%			\multicolumn{1}{|>{\columncolor[rgb]{0.8, 0.8, 0.8}}c|}{\textbf{Signo}} &
%			\multicolumn{1}{|>{\columncolor[rgb]{0.8, 0.8, 0.8}}c|}{\textbf{C\'odigo}} &
%			\multicolumn{1}{|>{\columncolor[rgb]{0.8, 0.8, 0.8}}c|}{\textbf{Signo}} &
%			\multicolumn{1}{|>{\columncolor[rgb]{0.8, 0.8, 0.8}}c|}{\textbf{C\'odigo}} &
%			\multicolumn{1}{|>{\columncolor[rgb]{0.8, 0.8, 0.8}}c|}{\textbf{Signo}} &
%			\multicolumn{1}{|>{\columncolor[rgb]{0.8, 0.8, 0.8}}c|}{\textbf{C\'odigo}} \\ \hline
%			A & \textperiodcentered-    & N & -\textperiodcentered & 0 & - - - - -   \\ \hline
%			B & -\textperiodcentered\textperiodcentered\textperiodcentered & O & - - -   & 1 & \textperiodcentered- - - -   \\ \hline
%			C & -\textperiodcentered-\textperiodcentered & P & \textperiodcentered- -\textperiodcentered & 2 & \textperiodcentered\textperiodcentered- - -   \\ \hline
%			D & -\textperiodcentered\textperiodcentered   & Q & - -\textperiodcentered- & 3 & \textperiodcentered\textperiodcentered\textperiodcentered- -   \\ \hline
%			E & \textperiodcentered 	    & R & \textperiodcentered-\textperiodcentered   & 4 &	\textperiodcentered\textperiodcentered\textperiodcentered\textperiodcentered-   \\ \hline
%			F & \textperiodcentered\textperiodcentered-\textperiodcentered & S & \textperiodcentered\textperiodcentered\textperiodcentered   & 5 & \textperiodcentered\textperiodcentered\textperiodcentered\textperiodcentered\textperiodcentered   \\ \hline
%			G & - -\textperiodcentered   & T & - 	  & 6 & -\textperiodcentered\textperiodcentered\textperiodcentered\textperiodcentered   \\ \hline
%			H & \textperiodcentered\textperiodcentered\textperiodcentered\textperiodcentered & U & \textperiodcentered\textperiodcentered-   & 7 & - -\textperiodcentered\textperiodcentered\textperiodcentered   \\ \hline
%			I & \textperiodcentered\textperiodcentered     & V & \textperiodcentered\textperiodcentered\textperiodcentered- & 8 &	- - -\textperiodcentered\textperiodcentered   \\ \hline
%			J & \textperiodcentered- - - & W & \textperiodcentered- -   & 9 &	- - - -\textperiodcentered   \\ \hline
%			K & -\textperiodcentered-   & X & -\textperiodcentered\textperiodcentered- & . &	\textperiodcentered-\textperiodcentered-\textperiodcentered- \\ \hline
%			L & \textperiodcentered-\textperiodcentered\textperiodcentered & Y & -\textperiodcentered- - & , &	- -\textperiodcentered\textperiodcentered - - \\ \hline
%			M & - -     & Z & - -\textperiodcentered\textperiodcentered & ? & \textperiodcentered- -\textperiodcentered\textperiodcentered   \\ \hline
%		\end{tabular}
%	\end{center}
	\end{enumerate}
\end{document}