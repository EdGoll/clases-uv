\documentclass{article}

\usepackage[spanish]{babel}
\usepackage[latin1]{inputenc}
\usepackage{colortbl}
\usepackage{amsmath}

\begin{document}

	\centerline{\sc \large Gu\'ia 3 de ejercicios: Programaci\'on 2}
	\centerline{\sc \normalsize Escuela de Ingenier\'ia Civil Inform\'atica}
	\centerline{\sc \normalsize  Universidad de Valpara\'iso}

	\vspace{1pc}

	\begin{enumerate}
	    %\item Se desea calcular la \textbf{altura} y \textbf{edad} promedio de una cierta cantidad de personas. Para ello se le ha solicitado implementar un programa que permita el ingreso de los datos de las personas por la entrada est\'andar e imprima por pantalla la informaci\'on requerida. Desarrolle el programa implemetado la POO (\textbf{tips}: Clase Persona, Clase InformacionSolicitada, Clase Principal, etc). 
	    \item Una empresa de arriendo de autom\'oviles le ha solicitado desarrollar un programa que les permita ingresar los datos de cada veh\'iculo (sus atributos b\'asicos: patente, color, tipo, modelo y marca) y luego los muestre en pantalla. Adem\'as, el programa debe permitir ''arrendar'' un veh\'iculo disponible de la lista, mostrabdo el valor asociado del arriendo. Cuando el veh\'iculo es ''arrendado'', debe desaparecer de la lista y mostrar la cantidad actualizada de autom\'oviles disponibles y el listado de ellos (\textbf{tips}: Clase Automovil, Clase Arriendo, Clase Principal, etc).
	    \item Desarrolle un programa que permita clasificar smart-phones dependiento de su sistema operativo (iOS, Andriod, Windows Phone, etc). Los datos asociados a los atributos de un smart-phone deben ser ingresados por teclado y al momento de terminar el ingreo, se debe mostrar la lista de cada uno de ellos, indicando la cantidad.
	\end{enumerate}
\end{document}