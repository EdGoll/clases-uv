\documentclass{article}

\usepackage[spanish]{babel}
\usepackage[latin1]{inputenc}
\usepackage{colortbl}
\usepackage{amsmath}
\usepackage{graphicx}

\begin{document}

	\centerline{\sc \large Gu\'ia de ejercicios: Programaci\'on 2}
	\centerline{\sc \normalsize Escuela de Ingenier\'ia Civil Inform\'atica}
	\centerline{\sc \normalsize  Universidad de Valpara\'iso}
	\centerline{\sc \small 9 de Mayo de 2017}

	\vspace{1pc}

	\begin{enumerate}
	    \item[] Desarrolle una aplicaci\'on en Java que considere la lectura del archivo \emph{WorldCupBrazil2014\_dataset.txt} y permita responder la siguientes consultas:
	    \begin{enumerate}
	        \item \textquestiondown Cu\'antos partidos se jugaron?
	        \item \textquestiondown Cu\'antos partidos jug\'o Chile?
	        \item \textquestiondown Cu\'ales fueron los resultado de Chile en el mundial?
	        \item \textquestiondown Cu\'ales fueron los lugares donde jug\'o Chile?
	        \item \textquestiondown Cual fue el pa\'is m\'as goleador?
	        \item \textquestiondown Cual fue el pa\'is m\'as batido?
	        \item \textquestiondown Cu\'antos partidos terminaron empatados?
	        \item \textquestiondown Cual fue el partido con mayor diferencia de goles?
	        \item \textquestiondown Qu\'e pa\'is fue el campe\'on?
	        \item \textquestiondown Cu\'ales fueron los lugares donde se disputaron el tercer puesto y la final?
	    \end{enumerate}
	    \item[] Cambie el archivo dataset a \emph{WorldCup2010SouthAfrica\_dataset.txt} y eval\'ue las preguntas anteriores.
	\end{enumerate}
\end{document}