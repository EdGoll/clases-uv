\documentclass[10pt]{article}
\usepackage[spanish]{babel}
\usepackage{graphicx}
\usepackage{amssymb}
\usepackage{epstopdf}
\usepackage{enumitem}
\usepackage{multicol,multirow}
\DeclareGraphicsRule{.tif}{png}{.png}{`convert #1 `dirname #1`/`basename #1 .tif`.png}

% For a visual definition of these parameters, see
\textwidth = 6.5 in
\textheight = 9 in
\oddsidemargin = 0.0 in
\evensidemargin = 0.0 in
\topmargin = 0.0 in             
\headheight = 0.0 in            
\headsep = 0.0 in
            
\parskip = 0.2in                % vertical space between paragraphs
% Delete the % in the following line if you don't want to have the first line of every paragraph indented
%\parindent = 0.0in

\begin{document}
    \begin{center}
		{\Large Certamen 3, Programaci\'on II} \\
		\emph{\small Prof. Rodrigo Olivares} \\
		\emph{\small Ayud. Juan Carlos Tapia} \\
		\emph{\scriptsize Noviembre 30, 2015} 
	\end{center}

	\vspace*{-35pt}
	\begin{center}
		\rule{1\textwidth}{.3pt}
	\end{center}
	\vspace*{-42pt}
	\begin{center}
		\rule{1\textwidth}{2pt}
	\end{center}

	\vspace*{-15pt}

	{\small \textbf{Instrucciones}:}

	\vspace*{-15pt}

	{\scriptsize
	\begin{itemize}
		\item[-] El puntaje m\'aximo del certamen es 100\%, siendo el 60\% el m\'inimo requerido para aprobar.
		\item[-] Responda cada pregunta en la hoja indicada, agregando su nombre. Si no responde alguna pregunta, debe entregar la hoja con su nombre e indicar que \textbf{no responde}.
		\item[-] El certamen es \underline{\textbf{individual}}. Cualquier intento de copia, ser\'a sancionado con nota \textbf{1,0}.
	\end{itemize}
	
	\vspace*{-20pt}

	\begin{enumerate}

		\item \emph{20pts.} De las siguentes afirmaciones, eval\'ue con verdadero ($\mathcal{V}$) o falso ($\mathcal{F}$) las siguientes afirmaciones (\emph{1pt c/u}). \textbf{Se descontar\'a una respuesta correcta por cada respuesta incorrecta}.
		
		\begin{enumerate}
            \item \_\_\_\_\_ Un thread es un flujo de proceso que se ejecuta en memoria.
            \item \_\_\_\_\_ Un thread puede ser instanciado como objeto.
            \item \_\_\_\_\_ Para crear un thread se debe extender de una sub-clase Thread.
            \item \_\_\_\_\_ Para iniciar un thread se utiliza el m\'etodo \emph{run}. 
            \item \_\_\_\_\_ Para iniciar un thread se debe sobre-escribir el m\'etodo \emph{start}.
            \item \_\_\_\_\_ En el ciclo de vida de una hebra, el estado \emph{New} es el encargado de crear e instanciar una thread.
            \item \_\_\_\_\_ En el ciclo de vida de una hebra, el estado \emph{Dead} es invocado generalmente por el m\'etodo \emph{stop}.
            \item \_\_\_\_\_ Un recurso compartido puede ser una clase.
            \item \_\_\_\_\_ Un recurso compartido siempre debe estar sincronizado.
            \item \_\_\_\_\_ La API \emph{Swing} sustituye a API AWT.
            \item \_\_\_\_\_ AWT incopora los \emph{JComponents} de \emph{Swing}.
            \item \_\_\_\_\_ \emph{JPane} es habitualmente utilizada para crear la ventana principal.
            \item \_\_\_\_\_ Para crear la ventana (usando \emph{Swing}) se de implementar la clase \emph{JFrame}.
            \item \_\_\_\_\_ El m\'etodo \emph{getPaneContent} de la clase \emph{JFrame} obtiene el panel principal.
            \item \_\_\_\_\_ El m\'etodo \emph{add} de la clase \emph{JFrame} permite agregar componentes al panel.
            \item \_\_\_\_\_ Para agregar funcionalidad a un JButton se debe crear y agregar una clase que implemente un ActionEvent
            \item \_\_\_\_\_ Para agregar funcionalidad a una JList se debe sobreescribir el m\'etodo actionList(ActionPerformed)
            \item \_\_\_\_\_ El componente JFileChooser es un componente di\'alogo de selecci\'on de archivos.
            \item \_\_\_\_\_ Algunos de componentes que permiten ingresar texto son: JPasswordField, JTextField y JTextArea.
            \item \_\_\_\_\_ Algunos de componentes que permiten manipular eventos/acciones son: JList, JButton y JTextField.
        \end{enumerate}

		\newpage

		\item \emph{40pts.} Simular el proceso de giro y dep\'osito de dinero de una cuenta corriente. La idea es controlar el ingreso y egreso de dinero, de tal manera que no sea factible girar dinero si la cuenta est\'a en valor cero 0 negativo y adem\'as no sea factible mantener m\'as de \$100.000 (cien mil pesos) en la cuenta. Cada transaci\'on debe esperar un n\'umero aletorio de segundos (entre 1 y 5) y el monto a depositar o girar tambi\'en deber\'a ser aleatorio entre \$1.000 (mil) y \$10.000 (diez mill). Considere como \textbf{regi\'on cr\'itica} la clase que gestiona la cuenta corriente (el saldo), por lo cual debe utilizar \textbf{bloqueos/sincronizaci\'on de thread}.

		\newpage

		\item \emph{40pts.} Construya una aplicaci\'on Java con interfaz gr\'afica que pemita realizar transformaciones de sistemas n\'umeros (binario, octal y hexadecimal) de un valor entero ingresado en un campo de texto y muestre el resultado en otro campo de texto no editable. Para la transformaci\'on utilice los m\'etodos: \emph{toBinaryString}(), \emph{toOctalString}() y \emph{toHexString}() de la clase \emph{Integer}, para transformar a binario, ocatl y hexadecimal, respectivamente.

	\end{enumerate}}
\end{document} 
