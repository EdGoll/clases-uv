\documentclass{article}

\usepackage[spanish]{babel}
\usepackage[latin1]{inputenc}
\usepackage{colortbl}
\usepackage{amsmath}

\begin{document}

	\centerline{\sc \large Gu\'ia 2 de ejercicios: Programaci\'on 2}
	\centerline{\sc \normalsize Escuela de Ingenier\'ia Civil Inform\'atica}
	\centerline{\sc \normalsize  Universidad de Valpara\'iso}

	\vspace{1pc}

	\begin{enumerate}
	    \item Desarrolle un programa (clases independientes) para generar un arreglo de 100 elementos con n\'umeros aleatorios y los muestre en pantalla. Luego, ordene de menor a mayor y vuelva a
mostrar. Finalmente, ordene de mayor a menor y muestre nuevamente. Para ordenar el arreglo, puede utilizar el m\'etodo \emph{bubble-sort} que reciba como par\'ametro de entrada el tipo de orden: mayor a menor o menor a mayor. Recuerde implementar todo el comportamiento de forma independientes.
        \item Desarrolle un programa que permita calcular el determinante de una matriz $3 \times 3$, llenados de forma rand\'omica. En general, para el c\'alculo de un determinante de $3 \times 3$, de la forma: \\
        $A = \left[
        \begin{array}{ccc}
            a & b & c \\
			d & e & f \\
			g & h & i \\
		\end{array}
        \right]$,
        $detA = aei + bfg + cdh - afh - bdi - ceg$.
        \item Escribir un programa que genere n\'umeros aleatorios entre 1 y 100 y construya su histograma con las frecuencias de cada n\'umero en la secuencia. Por ejemplo, la secuencia: 1, 1, 20, 1, 2, 20, 3, 3, 3, 4, 4, 4, 33, 3 generar\'ia la siguiente salida: 1:***, 2:*, 3:****, 4:***, 20:**, 33:*
		\item Desarrolle un programa (clases independientes) para que calcule uni\'on $\cup$, intersecci\'on $\cap$, pertenencia $\in$, subconjunto $\subseteq$, diferencia $-$ y el conjunto vac\'io $\emptyset$, de dos vectores din\'amicos de enteros $A$ y $B$. El programa debe permitir la inicializaci\'on de los vectores desde el teclado o de forma rand\'omica. Implementar todo el comportamiento, de forma independientes.
\end{enumerate}
\end{document}
