\documentclass[10pt]{article}
\usepackage[spanish]{babel}
\usepackage{graphicx}
\usepackage{amssymb}
\usepackage{epstopdf}
\usepackage{enumitem}
\usepackage{multicol,multirow}
\usepackage{ulem}

\usepackage{algorithm,algorithmic}

\renewcommand{\algorithmicrequire}{\textbf{entrada:}}
\renewcommand{\algorithmicensure}{\textbf{salida:}}
\renewcommand{\algorithmicend}{\textbf{fin}}
\renewcommand{\algorithmicif}{\textbf{si}}
\renewcommand{\algorithmicthen}{\textbf{entonces}}
\renewcommand{\algorithmicelse}{\textbf{si no}}
\renewcommand{\algorithmicelsif}{\algorithmicelse,\ \algorithmicif}
\renewcommand{\algorithmicendif}{\algorithmicend\ \algorithmicif}
\renewcommand{\algorithmicfor}{\textbf{Para}}
\renewcommand{\algorithmicforall}{\textbf{Para Todo}}
\renewcommand{\algorithmicdo}{\textbf{Repetir}}
\renewcommand{\algorithmicendfor}{\algorithmicend\ \algorithmicfor}
\renewcommand{\algorithmicwhile}{\textbf{Mientras}}
\renewcommand{\algorithmicendwhile}{\algorithmicend\ \algorithmicwhile}
\renewcommand{\algorithmicloop}{\textbf{Repetir}}
\renewcommand{\algorithmicendloop}{\algorithmicend\ \algorithmicloop}
\renewcommand{\algorithmicrepeat}{\textbf{Repetir}}
\renewcommand{\algorithmicuntil}{\textbf{Hasta que}}
\renewcommand{\algorithmicprint}{\textbf{Escribir}} 
\renewcommand{\algorithmicreturn}{\textbf{retornar}} 
\renewcommand{\algorithmictrue}{\textbf{Verdadero}} 
\renewcommand{\algorithmicfalse}{\textbf{Falso}}

\DeclareGraphicsRule{.tif}{png}{.png}{`convert #1 `dirname #1`/`basename #1 .tif`.png}

% For a visual definition of these parameters, see
\textwidth = 6.5 in
\textheight = 9 in
\oddsidemargin = 0.0 in
\evensidemargin = 0.0 in
\topmargin = 0.0 in             
\headheight = 0.0 in            
\headsep = 0.0 in
            
\parskip = 0.2in                % vertical space between paragraphs
% Delete the % in the following line if you don't want to have the first line of every paragraph indented
%\parindent = 0.0in

\begin{document}
    \begin{center}
		{\Large Certamen 2, Programaci\'on II} \\
		\emph{\small Prof. Rodrigo Olivares} \\
		\emph{\small Ayud. Juan Carlos Tapia} \\
		\emph{\scriptsize Mayo 26, 2016} 
	\end{center}

	\vspace*{-35pt}
	\begin{center}
		\rule{1\textwidth}{.3pt}
	\end{center}
	\vspace*{-42pt}
	\begin{center}
		\rule{1\textwidth}{2pt}
	\end{center}

	\vspace*{-15pt}

	{\small \textbf{Instrucciones}:}

	\vspace*{-15pt}

	{\scriptsize
	\begin{itemize}
		\item[-] El puntaje m\'aximo del certamen es 100\%, siendo el 60\% el m\'inimo requerido para aprobar.
		\item[-] Responda cada pregunta en la hoja indicada, agregando su nombre. Si no responde alguna pregunta, debe entregar la hoja con su nombre e indicar que \textbf{no responde}.
		\item[-] El certamen es \underline{\textbf{individual}}. Cualquier intento de copia, ser\'a sancionado con nota \textbf{1,0}.
	\end{itemize}
	
	\vspace*{-20pt}

	\begin{enumerate}

		\item \emph{30pts.} De las siguentes afirmaciones, encierre en un c\'irculo la o las alternativas correctas.
		
		\begin{multicols}{2}

			\begin{enumerate}[label=(\alph*)]
				\item[i.] \textbf{Un constructor}:
				\item Siempre existe.
				\item Puede no ser incluirlo por el desarrollador.
				\item Al ser private s\'olo instancia objetos dentro de la clase.
				\item Al ser protected, no puede ser sobrecargado.
				\item Debe siempre tener el mismo nombre de la clase.
			\end{enumerate}

			\begin{enumerate}[label=(\alph*)]
				\item[ii.] \textbf{La clase Vector}:
				\item Es un List.
				\item Es un ArrayList.
				\item Es synchronized.
				\item Es no synchronized.
				\item Es un nodo.
			\end{enumerate}

			\begin{enumerate}[label=(\alph*)]
				\item[iii.] \textbf{Paso de par\'ametros}:
				\item Puede ser por valor.
				\item Puede ser por omisi\'on.
				\item Puede ser por referencia.
				\item Puede ser por convensi\'on.
				\item Puede ser por default.
			\end{enumerate}

			\begin{enumerate}[label=(\alph*)]
				\item[iv.] \textbf{La instancia this}:
				\item Invoca al constructor de una clase padre.
				\item Invoca al garbage collection.
				\item Referencia al construtor de la clase.
				\item Referencia a los atributos de la clase.
				\item Referencia a los m\'etodo de la clase.
			\end{enumerate}

			\begin{enumerate}[label=(\alph*)]
				\item[v.] \textbf{Una clase abstracta}:
				\item Pose\'e s\'olo m\'etodos abstractos.
				\item \xout{Alguno de sus m\'etodos son abstractos.}
				\item Instancia objetos abstractos.
				\item Permite extender una interfaz.
				\item Permite implementar una clase padre.
			\end{enumerate}

            \begin{enumerate}[label=(\alph*)]
				\item[vi.] \textbf{Respecto a las interfaces interfaz}:
				\item Su constructor es creado en compilaci\'on.
				\item Sus m\'etodos definidos deben ser public.
				\item Sus m\'etodos son abstract.
				\item Se implementa.
				\item Se extiende.
			\end{enumerate}

			\begin{enumerate}[label=(\alph*)]
				\item[vii.] \textbf{Un TDA Bean}.
				\item S\'olo tiene atributos.
				\item S\'olo tiene m\'etodos.
				\item No debe incluir l\'ogica.
				\item El constructor debe ser siempre incluido.
				\item Es posible agregar m\'todos como equals() y toString().
			\end{enumerate}

			\begin{enumerate}[label=(\alph*)]
				\item[viii.] \textbf{Sobre la herencia}:
				\item En java se realiza con la palabra reservada extends.
				\item En java se realiza con la palabra reservada impelements.
				\item Todas las clases heredan de la clase Object.
				\item La clase padre hereda el comportamiento de la clase hija.
				\item La clase hija hereda el comportamiento de la clase padre.
			\end{enumerate}

			\begin{enumerate}[label=(\alph*)]
				\item[ix.] \textbf{En relaci\'on a la manipulaci\'on de archivos}.
				\item Se lee un archivo con la instancia FileWriter.
				\item Se lee un archivo con la instancia FileReader.
				\item Se leen s\'olo archivos con delimitar y de largo fijo.
				\item StringTokenizer se usa para archivos con delimitador.
				\item Se requiere de la clase Scanner para la lectura.
			\end{enumerate}

			\begin{enumerate}[label=(\alph*)]
				\item[x.] \textbf{En relaci\'on a la manipulaci\'on de archivos}.
				\item Se escribe un archivo con la instancia FileWriter.
				\item Se escribe un archivo con la instancia FileReader.
				\item No es factible agregar contenido a un archivo existente.
				\item Es necesario utilizar la clase InputStreamReader.
				\item FileWriter("f.txt", false) agrega el contenido al final.
			\end{enumerate}
		
		\end{multicols}
		
		\newpage

		\item \emph{70pts.} Como Ingeniero Civil en Inform\'atica, la escuela le ha pedido que desarrolle un sistema que permita:

        \begin{enumerate}
            \item Cargar las notas de los alumnos de dos asignaturas. Para ello, debe leer 3 dataset: 
            \begin{itemize}
                \item[-] \emph{alumnos.csv}: Identificar, Apellido Paterno, Apellido Materno, Nombre(s).
                \item[-] \emph{asignatura1.csv}: Identificar del alumno, Nota Quiz 1, Nota Quiz 2, Nota Quiz 3, Nota Tarea 1, Nota Tarea 2, Nota Certamen 1, Nota Certamen 2 y Nota Certamen 3
                \item[-] \emph{asignatura2.csv}: Identificar del alumno, Nota Quiz 1, Nota Quiz 2, Nota Quiz 3, Nota Tarea 1, Nota Tarea 2, Nota Certamen 1, Nota Certamen 2 y Nota Certamen 3
            \end{itemize}
            \item Calcular el promedio de cada alumno, por asignatura, de la siguiente forma:
            \begin{multicols}{3}
                $$PQ = \frac{\displaystyle\sum_{i=1}^{3}NQ_{i}}{3}$$
                $$PT = \frac{\displaystyle\sum_{i=1}^{2}NT_{i}}{2}$$
                $$PC = \frac{\displaystyle\sum_{i=1}^{3}NC_{i}}{3}$$
            \end{multicols}
            Para calcular el promedio final:
                \floatname{algorithm}{Pseudo-c\'odigo}
                \begin{algorithm}[!ht]
                	    \caption{Promedio\_Asignatura}
                    \begin{algorithmic}%[1]
                        \IF{$PC \geqslant 4.0$} 
                            \RETURN $PC * 0.7 + NQ * 0.15 + NT * 0.15$
                        \ELSIF{$PC \geqslant 3.5$}
                            \RETURN $PC * 0.8 + NQ * 0.10 + NT * 0.10$
                        \ELSIF{$PC \geqslant 3.0$}
                            \RETURN $PC * 0.9 + NQ * 0.05 + NT * 0.05$
                        \ELSE
                            \RETURN $PC$
                        \ENDIF
                    \end{algorithmic}
                \end{algorithm}	
            \item Por \'ultimo, almacene en un cuarto archivo, denominado \emph{consolidado.csv} -delimitado por caracter- la siguiente informaci\'on: Id del alumno, Nota Final Asignatura 1, Nota Final Asignatura 2.
        \end{enumerate}
	\end{enumerate}}
	\begin{table}[!ht]
       {\scriptsize
        \begin{center}
             \begin{tabular}{|p{3.5cm}|p{3.5cm}|p{3.5cm}|p{3.5cm}|}\hline
                \multicolumn{4}{|c|}{\textbf{\textquestiondown C\'omo ser\'e evaluado en la pregunta 2?} } \\ \hline
                \multicolumn{1}{|c|}{\textbf{T\'opico}} & 
                \multicolumn{1}{c|}{\textbf{Logrado}} & 
                \multicolumn{1}{c|}{\textbf{Medianamente logrado}} & 
                \multicolumn{1}{c|}{\textbf{No logrado}} \\ \hline
                Manipulaci\'on de archivo. & 
                \emph{25pts} Lee correctamente los archivos, los mapea a entidad y escribe en el archivo. & 
                \emph{13pts} Realiza dos de las tres acciones del punto anterior. & 
                \emph{ 0pts} No realiza la acciones del punto anterior. \\ \hline
                TDA Lista. & 
                \emph{15pts} Crea la clase TDA Lista e implementa todos los m\'etodos. & 
                \emph{ 7pts} Crea la clase TDA Lista e implementa algunos m\'etodos. & 
                \emph{ 0pts} No crea la clase TDA Lista. \\ \hline
                TDA Bean / Entidad & 
                \emph{10pts} Crea la clase entidad para alumno y asignatura. & 
                \emph{5pts} Crea la clase entidad para alumno o la asignatura (no ambas). & 
                \emph{0pts} No crea las clases entidad. \\ \hline                
                Clase principal y m\'etodo main. &
                \emph{5pts} Crea la clase principal en un archivo independiente con el m\'etodo main. & 
                \emph{ 3pts} Crea el m\'etodo main en la misma clase. & 
                \emph{ 0pts} No crea el m\'etodo main. \\ \hline
                Paradigma Orientaci\'on a Objetos  & 
                \emph{15pts} Resuelve el problema utilizando el POO. & 
                \emph{ 7pts} Utiliza parte del POO para resolver el problema. & 
                \emph{ 0pts} No utiliza el POO para dar soluci\'on al problema.\\ \hline
                Total m\'aximo puntaje pregunta 2 & 
                \emph{70pts} & 
                \emph{35pts} & 
                \emph{ 0pts} \\ \hline
            \end{tabular}
        \end{center}}
     \end{table}
\end{document} 
