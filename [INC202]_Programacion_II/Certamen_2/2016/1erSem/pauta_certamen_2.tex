\documentclass[10pt]{article}
\usepackage[spanish]{babel}
\usepackage{graphicx}
\usepackage{amssymb}
\usepackage{epstopdf}
\usepackage{enumitem}
\usepackage{multicol,multirow}
\usepackage{ulem}
\usepackage{tikz}
\usepackage{algorithm,algorithmic}

\renewcommand{\algorithmicrequire}{\textbf{entrada:}}
\renewcommand{\algorithmicensure}{\textbf{salida:}}
\renewcommand{\algorithmicend}{\textbf{fin}}
\renewcommand{\algorithmicif}{\textbf{si}}
\renewcommand{\algorithmicthen}{\textbf{entonces}}
\renewcommand{\algorithmicelse}{\textbf{si no}}
\renewcommand{\algorithmicelsif}{\algorithmicelse,\ \algorithmicif}
\renewcommand{\algorithmicendif}{\algorithmicend\ \algorithmicif}
\renewcommand{\algorithmicfor}{\textbf{Para}}
\renewcommand{\algorithmicforall}{\textbf{Para Todo}}
\renewcommand{\algorithmicdo}{\textbf{Repetir}}
\renewcommand{\algorithmicendfor}{\algorithmicend\ \algorithmicfor}
\renewcommand{\algorithmicwhile}{\textbf{Mientras}}
\renewcommand{\algorithmicendwhile}{\algorithmicend\ \algorithmicwhile}
\renewcommand{\algorithmicloop}{\textbf{Repetir}}
\renewcommand{\algorithmicendloop}{\algorithmicend\ \algorithmicloop}
\renewcommand{\algorithmicrepeat}{\textbf{Repetir}}
\renewcommand{\algorithmicuntil}{\textbf{Hasta que}}
\renewcommand{\algorithmicprint}{\textbf{Escribir}} 
\renewcommand{\algorithmicreturn}{\textbf{retornar}} 
\renewcommand{\algorithmictrue}{\textbf{Verdadero}} 
\renewcommand{\algorithmicfalse}{\textbf{Falso}}

\DeclareGraphicsRule{.tif}{png}{.png}{`convert #1 `dirname #1`/`basename #1 .tif`.png}
\newcommand*\circled[1]{\tikz[baseline=(char.base)]{\node[shape=circle,blue,draw,inner sep=.5pt] (char) {#1};}}
% For a visual definition of these parameters, see
\textwidth = 6.5 in
\textheight = 9 in
\oddsidemargin = 0.0 in
\evensidemargin = 0.0 in
\topmargin = 0.0 in             
\headheight = 0.0 in            
\headsep = 0.0 in
            
\parskip = 0.2in                % vertical space between paragraphs
% Delete the % in the following line if you don't want to have the first line of every paragraph indented
%\parindent = 0.0in

\begin{document}
    \begin{center}
		{\Large Pauta Certamen 2, Programaci\'on II} \\
		\emph{\small Prof. Rodrigo Olivares} \\
		\emph{\small Ayud. Juan Carlos Tapia} \\
		\emph{\scriptsize Mayo 26, 2016} 
	\end{center}

	\vspace*{-35pt}
	\begin{center}
		\rule{1\textwidth}{.3pt}
	\end{center}
	\vspace*{-42pt}
	\begin{center}
		\rule{1\textwidth}{2pt}
	\end{center}

	\vspace*{-15pt}

	{\small \textbf{Instrucciones}:}

	\vspace*{-15pt}

	{\scriptsize
	\begin{itemize}
		\item[-] El puntaje m\'aximo del certamen es 100\%, siendo el 60\% el m\'inimo requerido para aprobar.
		\item[-] Responda cada pregunta en la hoja indicada, agregando su nombre. Si no responde alguna pregunta, debe entregar la hoja con su nombre e indicar que \textbf{no responde}.
		\item[-] El certamen es \underline{\textbf{individual}}. Cualquier intento de copia, ser\'a sancionado con nota \textbf{1,0}.
	\end{itemize}
	
	\vspace*{-20pt}

	\begin{enumerate}

		\item \emph{30pts.} De las siguentes afirmaciones, encierre en un c\'irculo la o las alternativas correctas.
		
		\begin{multicols}{2}

			\begin{enumerate}[label=(\alph*)]
				\item[i.] \textbf{Un constructor}:
				\item[\circled{(a)}] Siempre existe.
				\item[\circled{(b)}] Puede no ser incluirlo por el desarrollador.
				\item[\circled{(c)}] Al ser private s\'olo instancia objetos dentro de la clase.
				\item[(b)] Al ser protected, no puede ser sobrecargado.
				\item[\circled{(e)}] Debe siempre tener el mismo nombre de la clase.
			\end{enumerate}

			\begin{enumerate}[label=(\alph*)]
				\item[ii.] \textbf{La clase Vector}:
				\item[\circled{(a)}] Es un List.
				\item[(b)] Es un ArrayList.
				\item[\circled{(c)}] Es synchronized.
				\item[(d)] Es no synchronized.
				\item[(e)] Es un nodo.
			\end{enumerate}

			\begin{enumerate}[label=(\alph*)]
				\item[iii.] \textbf{Paso de par\'ametros}:
				\item[\circled{(a)}] Puede ser por valor.
				\item[(b)] Puede ser por omisi\'on.
				\item[\circled{(c)}] Puede ser por referencia.
				\item[(b)] Puede ser por convensi\'on.
				\item[(c)] Puede ser por default.
			\end{enumerate}

			\begin{enumerate}[label=(\alph*)]
				\item[iv.] \textbf{La instancia this}:
				\item[(a)] Invoca al constructor de una clase padre.
				\item[(b)] Invoca al garbage collection.
				\item[(c)] Referencia al construtor de la clase.
				\item[(d)] Referencia a los atributos de la clase.
				\item[(e)] Referencia a los m\'etodo de la clase.
			\end{enumerate}

			\begin{enumerate}[label=(\alph*)]
				\item[v.] \textbf{Una clase abstracta}:
				\item[(a)] Pose\'e s\'olo m\'etodos abstractos.
				\item[\circled{\xout{(b)}}] \xout{Alguno de sus m\'etodos son abstractos.}
				\item[(c)] Instancia objetos abstractos.
				\item[(d)] Permite extender una interfaz.
				\item[(e)] Permite implementar una clase padre.
			\end{enumerate}

            \begin{enumerate}[label=(\alph*)]
				\item[vi.] \textbf{Respecto a las interfaces}:
				\item[(a)] Su constructor es creado en compilaci\'on.
				\item[\circled{(b)}] Sus m\'etodos definidos deben ser public.
				\item[\circled{(c)}] Sus m\'etodos son abstract.
				\item[\circled{(d)}] Se implementa.
				\item[(e)] Se extiende.
			\end{enumerate}

			\begin{enumerate}[label=(\alph*)]
				\item[vii.] \textbf{Un TDA Bean}.
				\item[(a)] S\'olo tiene atributos.
				\item[(b)] S\'olo tiene m\'etodos.
				\item[\circled{(c)}] No debe incluir l\'ogica.
				\item[(d)] El constructor debe ser siempre incluido.
				\item[\circled{(e)}] Es posible agregar m\'etodos como equals() y toString().
			\end{enumerate}

			\begin{enumerate}[label=(\alph*)]
				\item[viii.] \textbf{Sobre la herencia}:
				\item[\circled{(a)}] En java se realiza con la palabra reservada extends.
				\item[\circled{(b)}] En java se realiza con la palabra reservada impelements.
				\item[\circled{(c)}] Todas las clases heredan de la clase Object.
				\item[(d)] La clase padre hereda el comportamiento de la clase hija.
				\item[\circled{(e)}] La clase hija hereda el comportamiento de la clase padre.
			\end{enumerate}

			\begin{enumerate}[label=(\alph*)]
				\item[ix.] \textbf{En relaci\'on a la manipulaci\'on de archivos}.
				\item[(a)] Se lee un archivo con la instancia FileWriter.
				\item[\circled{(b)}] Se lee un archivo con la instancia FileReader.
				\item[(c)] Se leen s\'olo archivos con delimitar y de largo fijo.
				\item[\circled{(d)}] StringTokenizer se usa para archivos con delimitador.
				\item[(e)] Se requiere de la clase Scanner para la lectura.
			\end{enumerate}

			\begin{enumerate}[label=(\alph*)]
				\item[x.] \textbf{En relaci\'on a la manipulaci\'on de archivos}.
				\item[\circled{(a)}] Se escribe en un archivo con la instancia FileWriter.
				\item[(b)] Se escribe en un archivo con la instancia FileReader.
				\item[(c)] No es factible agregar contenido a un archivo existente.
				\item[(d)] Es necesario utilizar la clase InputStreamReader.
				\item[(e)] FileWriter("f.txt", false) agrega el contenido al final.
			\end{enumerate}
		
		\end{multicols}
		
		\newpage

		\item \emph{70pts.} Como Ingeniero Civil en Inform\'atica, la escuela le ha pedido que desarrolle un sistema que permita:

        \begin{enumerate}
            \item Cargar las notas de los alumnos de dos asignaturas. Para ello, debe leer 3 dataset: 
            \begin{itemize}
                \item[-] \emph{alumnos.csv}: Identificar, Apellido Paterno, Apellido Materno, Nombre(s).
                \item[-] \emph{asignatura1.csv}: Identificar del alumno, Nota Quiz 1, Nota Quiz 2, Nota Quiz 3, Nota Tarea 1, Nota Tarea 2, Nota Certamen 1, Nota Certamen 2 y Nota Certamen 3
                \item[-] \emph{asignatura2.csv}: Identificar del alumno, Nota Quiz 1, Nota Quiz 2, Nota Quiz 3, Nota Tarea 1, Nota Tarea 2, Nota Certamen 1, Nota Certamen 2 y Nota Certamen 3
            \end{itemize}
            \item Calcular el promedio de cada alumno, por asignatura, de la siguiente forma:
            \begin{multicols}{3}
                $$PQ = \frac{\displaystyle\sum_{i=1}^{3}NQ_{i}}{3}$$
                $$PT = \frac{\displaystyle\sum_{i=1}^{2}NT_{i}}{2}$$
                $$PC = \frac{\displaystyle\sum_{i=1}^{3}NC_{i}}{3}$$
            \end{multicols}
            Para calcular el promedio final:
                \floatname{algorithm}{Pseudo-c\'odigo}
                \begin{algorithm}[!ht]
                	    \caption{Promedio\_Asignatura}
                    \begin{algorithmic}%[1]
                        \IF{$PC \geqslant 4.0$} 
                            \RETURN $PC * 0.7 + NQ * 0.15 + NT * 0.15$
                        \ELSIF{$PC \geqslant 3.5$}
                            \RETURN $PC * 0.8 + NQ * 0.10 + NT * 0.10$
                        \ELSIF{$PC \geqslant 3.0$}
                            \RETURN $PC * 0.9 + NQ * 0.05 + NT * 0.05$
                        \ELSE
                            \RETURN $PC$
                        \ENDIF
                    \end{algorithmic}
                \end{algorithm}	
            \item Por \'ultimo, almacene en un cuarto archivo, denominado \emph{consolidado.csv} -delimitado por caracter- la siguiente informaci\'on: Id del alumno, Nota Final Asignatura 1, Nota Final Asignatura 2.
        \end{enumerate}
	\end{enumerate}}
	\begin{table}[!ht]
       {\scriptsize
        \begin{center}
             \begin{tabular}{|p{3.5cm}|p{3.5cm}|p{3.5cm}|p{3.5cm}|}\hline
                \multicolumn{4}{|c|}{\textbf{\textquestiondown C\'omo ser\'e evaluado en la pregunta 2?} } \\ \hline
                \multicolumn{1}{|c|}{\textbf{T\'opico}} & 
                \multicolumn{1}{c|}{\textbf{Logrado}} & 
                \multicolumn{1}{c|}{\textbf{Medianamente logrado}} & 
                \multicolumn{1}{c|}{\textbf{No logrado}} \\ \hline
                Manipulaci\'on de archivo. & 
                \emph{25pts} Lee correctamente los archivos, los mapea a entidad y escribe en el archivo. & 
                \emph{13pts} Realiza dos de las tres acciones del punto anterior. & 
                \emph{ 0pts} No realiza la acciones del punto anterior. \\ \hline
                TDA Lista. & 
                \emph{15pts} Crea la clase TDA Lista e implementa todos los m\'etodos. & 
                \emph{7pts} Crea la clase TDA Lista e implementa algunos m\'etodos. & 
                \emph{0pts} No crea la clase TDA Lista. \\ \hline
                TDA Bean / Entidad & 
                \emph{10pts} Crea la clase entidad para alumno y asignatura. & 
                \emph{5pts} Crea la clase entidad para alumno o la asignatura (no ambas). & 
                \emph{0pts} No crea las clases entidad. \\ \hline                
                Clase principal y m\'etodo main. &
                \emph{5pts} Crea la clase principal en un archivo independiente con el m\'etodo main. & 
                \emph{3pts} Crea el m\'etodo main en la misma clase. & 
                \emph{0pts} No crea el m\'etodo main. \\ \hline
                Paradigma Orientaci\'on a Objetos  & 
                \emph{15pts} Resuelve el problema utilizando el POO. & 
                \emph{7pts} Utiliza parte del POO para resolver el problema. & 
                \emph{0pts} No utiliza el POO para dar soluci\'on al problema.\\ \hline
                Total m\'aximo puntaje pregunta 2 & 
                \emph{70pts} & 
                \emph{35pts} & 
                \emph{0pts} \\ \hline
            \end{tabular}
        \end{center}}
     \end{table}
     \newpage
\begin{verbatim}
public class Principal {

    public static void main(String[] args) {
        ListaAlumnoNotaImp lan = new ListaAlumnoNotaImp();
        lan.calcularPromedios();
    }
}

public class Alumno {

    private int id;
    private String apellidoPaterno;
    private String apellidoMaterno;
    private String nombres;

    public int getId() {
        return id;
    }

    public void setId(int id) {
        this.id = id;
    }

    public String getApellidoPaterno() {
        return apellidoPaterno;
    }

    public void setApellidoPaterno(String apellidoPaterno) {
        this.apellidoPaterno = apellidoPaterno;
    }

    public String getApellidoMaterno() {
        return apellidoMaterno;
    }

    public void setApellidoMaterno(String apellidoMaterno) {
        this.apellidoMaterno = apellidoMaterno;
    }

    public String getNombres() {
        return nombres;
    }

    public void setNombres(String nombres) {
        this.nombres = nombres;
    }
}

public class NotaAlumno {

    private int idAlumno;
    private float notaQuiz1;
    private float notaQuiz2;
    private float notaQuiz3;
    private float notaTarea1;
    private float notaTarea2;
    private float notaCertamen1;
    private float notaCertamen2;
    private float notaCertamen3;

    public int getIdAlumno() {
        return idAlumno;
    }

    public void setIdAlumno(int idAlumno) {
        this.idAlumno = idAlumno;
    }

    public float getNotaQuiz1() {
        return notaQuiz1;
    }

    public void setNotaQuiz1(float notaQuiz1) {
        this.notaQuiz1 = notaQuiz1;
    }

    public float getNotaQuiz2() {
        return notaQuiz2;
    }

    public void setNotaQuiz2(float notaQuiz2) {
        this.notaQuiz2 = notaQuiz2;
    }

    public float getNotaQuiz3() {
        return notaQuiz3;
    }

    public void setNotaQuiz3(float notaQuiz3) {
        this.notaQuiz3 = notaQuiz3;
    }

    public float getNotaTarea1() {
        return notaTarea1;
    }

    public void setNotaTarea1(float notaTarea1) {
        this.notaTarea1 = notaTarea1;
    }

    public float getNotaTarea2() {
        return notaTarea2;
    }

    public void setNotaTarea2(float notaTarea2) {
        this.notaTarea2 = notaTarea2;
    }

    public float getNotaCertamen1() {
        return notaCertamen1;
    }

    public void setNotaCertamen1(float notaCertamen1) {
        this.notaCertamen1 = notaCertamen1;
    }

    public float getNotaCertamen2() {
        return notaCertamen2;
    }

    public void setNotaCertamen2(float notaCertamen2) {
        this.notaCertamen2 = notaCertamen2;
    }

    public float getNotaCertamen3() {
        return notaCertamen3;
    }

    public void setNotaCertamen3(float notaCertamen3) {
        this.notaCertamen3 = notaCertamen3;
    }

    public float getPromedioQuiz() {
        return (notaQuiz1 + notaQuiz2 + notaQuiz3) / 3;
    }

    public float getPromediTarea() {
        return (notaTarea1 + notaTarea2) / 2;
    }

    public float getPromedioCertamen() {
        return (notaCertamen1 + notaCertamen2 + notaCertamen3) / 3;
    }

    public float getPromedioFinal() {
        if (getPromedioCertamen() >= 4) {
            return (getPromedioCertamen() * 0.7f + 
                   getPromedioQuiz() * 0.15f + 
                   getPromediTarea() * 0.15f);
        } else if (getPromedioCertamen() >= 3.5) {
            return (getPromedioCertamen() * 0.8f + 
                   getPromedioQuiz() * 0.1f + 
                   getPromediTarea() * 0.1f);
        } else if (getPromedioCertamen() >= 3) {
            return (getPromedioCertamen() * 0.9f + 
                   getPromedioQuiz() * 0.05f + 
                   getPromediTarea() * 0.05f);
        } else {
            return getPromedioCertamen();
        }
    }
}

import java.util.ArrayList;
import java.util.List;
import java.util.StringTokenizer;

public class ListaAlumnoNotaImp {

    private List<NotaAlumno> notasAlumnosAsignatura1 = null;
    private List<NotaAlumno> notasAlumnosAsignatura2 = null;
    private List<Alumno> alumnos = null; 
    private FuenteDeDatos fd = null;
    private List<String> lineas = null;
    private StringTokenizer stringTokenizer = null;
    private NotaAlumno notaAlumno = null;
    private Alumno alumno = null;

    public ListaAlumnoNotaImp() {
        notasAlumnosAsignatura1 = new ArrayList<NotaAlumno>();
        notasAlumnosAsignatura2 = new ArrayList<NotaAlumno>();
        alumnos = new ArrayList<Alumno>();
        fd = new FuenteDeDatos();
        cargarAlumnos();
        cargarNotasAlumnos("asignatura1.csv", notasAlumnosAsignatura1);
        cargarNotasAlumnos("asignatura2.csv", notasAlumnosAsignatura2);
    }

    private void cargarAlumnos() {
        lineas = fd.leerArchivo("alumnos.csv");
        for (String linea : lineas) {
            stringTokenizer = new StringTokenizer(linea, ";");
            if (stringTokenizer.hasMoreElements()) {
                alumno = new Alumno();
                alumno.setId(Integer.parseInt(stringTokenizer.nextToken()));
                alumno.setApellidoPaterno(stringTokenizer.nextToken());
                alumno.setApellidoMaterno(stringTokenizer.nextToken());
                alumno.setNombres(stringTokenizer.nextToken());
                alumnos.add(alumno);
            }
        }
    }

    private void cargarNotasAlumnos(String nombreArchivo, List<NotaAlumno> lista) {
        lineas = fd.leerArchivo(nombreArchivo);
        for (String linea : lineas) {
            stringTokenizer = new StringTokenizer(linea, ";");
            if (stringTokenizer.hasMoreElements()) {
                notaAlumno = new NotaAlumno();
                notaAlumno.setIdAlumno(Integer.parseInt(stringTokenizer.nextToken()));
                notaAlumno.setNotaQuiz1(Float.parseFloat(stringTokenizer.nextToken().replace(",", ".")));
                notaAlumno.setNotaQuiz2(Float.parseFloat(stringTokenizer.nextToken().replace(",", ".")));
                notaAlumno.setNotaQuiz3(Float.parseFloat(stringTokenizer.nextToken().replace(",", ".")));
                notaAlumno.setNotaTarea1(Float.parseFloat(stringTokenizer.nextToken().replace(",", ".")));
                notaAlumno.setNotaTarea2(Float.parseFloat(stringTokenizer.nextToken().replace(",", ".")));
                notaAlumno.setNotaCertamen1(Float.parseFloat(stringTokenizer.nextToken().replace(",", ".")));
                notaAlumno.setNotaCertamen2(Float.parseFloat(stringTokenizer.nextToken().replace(",", ".")));
                notaAlumno.setNotaCertamen3(Float.parseFloat(stringTokenizer.nextToken().replace(",", ".")));
                lista.add(notaAlumno);
            }
        }
    }

    public void calcularPromedios() {
        lineas = new ArrayList<String>();
        String linea;
        for (Alumno alumnoAux : alumnos) {
            linea = alumnoAux.getId() + ";";
            for (NotaAlumno notaAlumnoAux : notasAlumnosAsignatura1) {
                if (notaAlumnoAux.getIdAlumno() == alumnoAux.getId()) {
                    linea += String.format("%f", notaAlumnoAux.getPromedioFinal()) + ";";
                    break;
                }
            }
            for (NotaAlumno notaAlumnoAux : notasAlumnosAsignatura2) {
                if (notaAlumnoAux.getIdAlumno() == alumnoAux.getId()) {
                    linea += String.format("%f", notaAlumnoAux.getPromedioFinal());
                    break;
                }
            }
            lineas.add(linea);
        }
        fd.escribirArchivo("consolidado.csv", lineas);
    }
}

import java.io.BufferedReader;
import java.io.File;
import java.io.FileReader;
import java.io.FileWriter;
import java.io.IOException;
import java.io.PrintWriter;
import java.util.ArrayList;
import java.util.List;

public class FuenteDeDatos {

    public List<String> leerArchivo(String nombreArchivo) {
        File archivo;
        FileReader fileReader = null;
        List<String> lineas = null;
        BufferedReader bufferedReader;

        try {
            archivo = new File(nombreArchivo);
            lineas = new ArrayList<String>();
            String linea;
            fileReader = new FileReader(archivo);
            bufferedReader = new BufferedReader(fileReader);
            while ((linea = bufferedReader.readLine()) != null) {
                lineas.add(linea);
            }
        } catch (IOException e) {
            System.out.println(e);
        } finally {
            try {
                if (fileReader != null) {
                    fileReader.close();
                }
            } catch (IOException e) {
                System.out.println(e);
            }
        }
        return lineas;
    }

    public void escribirArchivo(String nombreArchivo, List<String> lineas) {
        FileWriter archivo;
        PrintWriter printWriter = null;
        try {
            archivo = new FileWriter(nombreArchivo);
            printWriter = new PrintWriter(archivo);
            for (String linea : lineas) {
                printWriter.println(linea);
            }
        } catch (IOException e) {
            System.out.println(e);
        } finally {
            printWriter.close();
        }
    }
}
\end{verbatim}
\end{document} 
