\documentclass[10pt]{article}
\usepackage[spanish]{babel}
\usepackage{graphicx}
\usepackage{amssymb}
\usepackage{epstopdf}
\usepackage{enumitem}
\usepackage{multicol,multirow}
\DeclareGraphicsRule{.tif}{png}{.png}{`convert #1 `dirname #1`/`basename #1 .tif`.png}

% For a visual definition of these parameters, see
\textwidth = 6.5 in
\textheight = 9 in
\oddsidemargin = 0.0 in
\evensidemargin = 0.0 in
\topmargin = 0.0 in             
\headheight = 0.0 in            
\headsep = 0.0 in
            
\parskip = 0.2in                % vertical space between paragraphs
% Delete the % in the following line if you don't want to have the first line of every paragraph indented
%\parindent = 0.0in

\begin{document}
    \begin{center}
		{\Large Certamen 2, Programaci\'on II} \\
		\emph{\small Prof. Rodrigo Olivares} \\
		\emph{\small Ayud. Juan Carlos Tapia} \\
		\emph{\scriptsize Noviembre 3, 2015} 
	\end{center}

	\vspace*{-35pt}
	\begin{center}
		\rule{1\textwidth}{.3pt}
	\end{center}
	\vspace*{-42pt}
	\begin{center}
		\rule{1\textwidth}{2pt}
	\end{center}

	\vspace*{-15pt}

	{\small \textbf{Instrucciones}:}

	\vspace*{-15pt}

	{\scriptsize
	\begin{itemize}
		\item[-] El puntaje m\'aximo del certamen es 100\%, siendo el 60\% el m\'inimo requerido para aprobar.
		\item[-] Responda cada pregunta en la hoja indicada, agregando su nombre. Si no responde alguna pregunta, debe entregar la hoja con su nombre e indicar que \textbf{no responde}.
		\item[-] El certamen es \underline{\textbf{individual}}. Cualquier intento de copia, ser\'a sancionado con nota \textbf{1,0}.
	\end{itemize}
	
	\vspace*{-20pt}

	\begin{enumerate}

		\item \emph{30pts.} De las siguentes afirmaciones, encierre en un c\'irculo la o las alternativas correctas.
		
		\begin{multicols}{2}

			\begin{enumerate}[label=(\alph*)]
				\item[i.] Un constructor:
				\item No necesariamente debe ir en una clase.
				\item Al ser private s\'olo instancia objetos dentro de la clase.
				\item No puede ser sobrecargado.
				\item Si no se declara, se crea uno en tiempo de compilaci\'on
				\item Si no se declara, se crea uno en tiempo de ejecuci\'on
			\end{enumerate}

			\begin{enumerate}[label=(\alph*)]
				\item[ii.] La lectura de datos de la entrada est\'andar:
				\item Puede realizarse con la clase FileReader.
				\item Puede realizarse con la clase Scanner.
				\item S\'olo puede ser de tipo String.
				\item Permite leer los argumentos del m\'etodo \emph{main}.
				\item Ninguna de las anteriores
			\end{enumerate}

			\begin{enumerate}[label=(\alph*)]
				\item[iii.] En relaci\'on a los arreglos:
				\item No almacenan los elementos de tipo char.
				\item Pueden almacenar tipos de datos primitivos.
				\item Pueden almacenar objetos.
				\item Se instancian s\'olo con el operador new.
				\item Su dimensi\'on puede ser determinada en ejecuci\'on.
			\end{enumerate}

			\begin{enumerate}[label=(\alph*)]
				\item[iv.] Una clase abstracta:
				\item Puede instanciar objetos.
				\item Pose\'e s\'olo m\'etodos abstractos.
				\item Pose\'e al menos un m\'etodo abstracto.
				\item No permite extender una interfaz.
				\item Ninguna de las anteriores.
			\end{enumerate}

			\begin{enumerate}[label=(\alph*)]
				\item[v.] La interfaz List.
				\item Puede ser implementada por la clase ArrayVector.
				\item Puede instanciar objetos.
				\item Pose\'e s\'olo m\'etodos abstractos.
				\item Puede ser extendida en una clase hija.
				\item No pose\'e atributos.
			\end{enumerate}

			\begin{enumerate}[label=(\alph*)]
				\item[vi.] Con respecto al paso de par\'ametros:
				\item Puede ser por valor.
				\item Puede ser por omisi\'on.
				\item Puede ser por referencia.
				\item Puede ser por convensi\'on.
				\item Puede ser por default.
			\end{enumerate}

			\begin{enumerate}[label=(\alph*)]
				\item[vii.] El relaci\'on a la auto-referencia:
				\item Cumple la misma funci\'on que el operador \emph{super}.
				\item Permite referenciar el construtor de la clase.
				\item Permite referenciar los atributos de la clase.
				\item Permite referenciar los m\'etodo de la clase.
				\item Ninguna de las anteriores.
			\end{enumerate}

			\begin{enumerate}[label=(\alph*)]
				\item[viii.] Sobre la herencia:
				\item Se realiza con la palabra reservada extends.
				\item Se realiza con la palabra reservada include.
				\item Todas las clases heredan de la clase Object.
				\item La clase padre hereda el comportamiento de la clase hija.
				\item La clase hija hereda el comportamiento de la clase padre.
			\end{enumerate}

			\begin{enumerate}[label=(\alph*)]
				\item[xi.] Respecto a la herencia m\'ultiple:
				\item No existe en Java.
				\item Existe en Java.
				\item Se puede emular con clases abstractas.
				\item Se puede emular con interfaces.
				\item Se puede emular con clases est\'aticas.
			\end{enumerate}

			\begin{enumerate}[label=(\alph*)]
				\item[x.] En relaci\'on a la manipulaci\'on de archivos.
				\item Se lee un archivo con la instancia FileWriter.
				\item Se lee un archivo con la instancia FileReader.
				\item No es factible agregar contenido a un archivo existente.
				\item StringTokenizer se usa para archivos con delimitador.
				\item Ninguna de las anteriores.
			\end{enumerate}
		
		\end{multicols}

		\newpage

		\item \emph{30pts.} Desarrolle un programa en Java que permita llenar un array con 10 nombres de personas, ingresados desde la entrada est\'andar. A partir de ese arreglo, construya un nuevo arreglo que almacene el largo de cada nombre. Por \'ultimo, muestre el nombre y su largo, desde los arreglos ya creados.
		
		\newpage

		\item \emph{40pts.} El BancoPais le ha solicitado implementar un sistema de seguridad de transferencias bancar\'ias, utilizando un ``digipass''. Este dispositivo trabaja con una combinaci\'on aleatoria de 3 pares de n\'umeros (entre el 10 y el 99, inclusive) de un total de 50 posibles valores, no necesariamente distintos. Estos n\'umeros deben estar almacenados en una fuente de datos permanente, \'unico por cliente. Dise\~ne un programa en Java que gestione una transferencia bancaria, incluyendo el monto a transferir e ingresando un c\'odigo de ``digipass'' y lo compare con cualquier combinaci\'on de los valores almacenados. Debe informar al usuario (por la salida est\'andar) si la transferencia se realiz\'o correcta o incorrectamente, informando para este caso, los errores comentidos.

	\end{enumerate}}
\end{document} 
