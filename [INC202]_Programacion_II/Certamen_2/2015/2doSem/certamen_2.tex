\documentclass[10pt]{article}
\usepackage[spanish]{babel}
\usepackage{graphicx}
\usepackage{amssymb}
\usepackage{epstopdf}
\usepackage{enumitem}
\usepackage{multicol,multirow}
\DeclareGraphicsRule{.tif}{png}{.png}{`convert #1 `dirname #1`/`basename #1 .tif`.png}

% For a visual definition of these parameters, see
\textwidth = 6.5 in
\textheight = 9 in
\oddsidemargin = 0.0 in
\evensidemargin = 0.0 in
\topmargin = 0.0 in             
\headheight = 0.0 in            
\headsep = 0.0 in
            
\parskip = 0.2in                % vertical space between paragraphs
% Delete the % in the following line if you don't want to have the first line of every paragraph indented
%\parindent = 0.0in

\begin{document}
    \begin{center}
        {\Large Certamen 2, Programaci\'on II} \\
        \emph{\small Prof. Rodrigo Olivares} \\
        \emph{\small Ayud. Diego Agull\'o} \\
        \emph{\scriptsize Mayo 26, 2015} 
    \end{center}
    \vspace*{-35pt}
    \begin{center}
        \rule{1\textwidth}{.3pt}
    \end{center}
    \vspace*{-42pt}
    \begin{center}
        \rule{1\textwidth}{2pt}
    \end{center}

    \vspace*{-15pt}
    {\small \textbf{Instrucciones}:}
    \vspace*{-15pt}

    {\scriptsize
    \begin{itemize}
        \item[-] El puntaje m\'aximo del certamen es 100\%, siendo el 60\% el m\'inimo requerido para aprobar.
        \item[-] Responda las preguntas en un \'unico archivo, agregando el n\'umero de la pregunta, su nombre y RUT. Si no responde alguna pregunta, debe indicar en el mismo archivo que \textbf{no responde}. El nombre del archivo debe tener la forma $<<$\emph{apellido\_nombre.ext}$>>$ y debe ser subido al aula virtual.
        \item[-] El certamen es \underline{\textbf{individual}}. Cualquier intento de copia, ser\'a sancionado con nota \textbf{1,0}.
    \end{itemize}
    
    \vspace*{-25pt}

    \begin{enumerate}

        \item \emph{30pts (6pts c/u).} \textbf{Comente} las siguientes declaraciones.
    
        \begin{enumerate}[label=(\alph*)]
            \item La herencia es un mecanismo que permite construir clases tipo Padre-Hija(s). Esta construcci\'on es llevada a cabo sin la necesidad de conocer el comportamiento de la clase Padre.
            \item En la implementaci\'on de una clase abstracta, la sub-clase que la extiende debe manipular todos los m\'etodos, sean o no abstractos.
            \item La el uso de interfaces permite simular la herencia m\'ultiple.  
            \item Una lista es un tipo de dato abstracto gen\'erico, ideal para gestionar colecciones de datos.
            \item La entidad es un compomente fundamental en la programaci\'on orientada a objeto.
        \end{enumerate}

        \item \emph{70pts.} Considere 3 dataset: \emph{regiones.txt}, \emph{provincias.txt} y \emph{comunas.txt}: La informaci\'on contenida es la siguiente:
        \begin{enumerate}
            \item[-] \emph{regiones.txt}: identificador y nombre de la regi\'on.
            \item[-] \emph{provincias.txt}: identificador y nombre de la provincia, adem\'as del identificador de la regi\'on a la que pertenece.
            \item[-] \emph{comunas.txt}: identificador y nombre de la comuna, adem\'as del identificador de la provincia a la que pertenece.
            \item[] De acuerdo a esto, debe:
        \begin{enumerate}
            \item[\emph{20pts}] Construir las entidades que permitan mapear los dataset. Utilice \textbf{herencia} para ''heredar'' el compartamiento com\'un (ver figura \ref{fig:diagrama}).
            \item[\emph{40pts}] Construya un clase que:
            \begin{enumerate}[label=(\alph*)]
                \item[\emph{30pts}] Desarrolle los m\'etodos de lectura de los dataset.
                \item[\emph{10pts}] Desarrolle los m\'etodos necesarios para ''buscar'' la informaci\'on en las listas de objetos.
            \end{enumerate}
            \item[\emph{10pts}] Construya la clase y el m\'etodo principal para la ejecuci\'on del programa.
        \end{enumerate}
         \end{enumerate}
    \end{enumerate}

    \begin{figure}[htbp!]
        \begin{center}
            \fbox{\fbox{\includegraphics[scale=.5]{images/diagrama.png}}}
            \caption{{\scriptsize Diagrama de clase/entidades}}\label{fig:diagrama}
        \end{center}    
    \end{figure}
        \newpage
    Para la pregunta \emph{2.-}, considere el uso de la siguiente clase:
        \begin{multicols}{2}
            \begin{verbatim}
import java.io.BufferedReader;
import java.io.File;
import java.io.FileReader;
import java.io.FileWriter;
import java.io.IOException;
import java.io.PrintWriter;
import java.util.ArrayList;
import java.util.List;

public class LecturaEscritura {

    public List<String> leer(String nombreArchivo) {

        File archivo;
        FileReader fr = null;
        List<String> lineas = null;

        try {
            archivo = new File(nombreArchivo);
            lineas = new ArrayList<String>();
            String linea;
            fr = new FileReader(archivo);
            BufferedReader br = new BufferedReader(fr);

            while ((linea = br.readLine()) != null) {
                lineas.add(linea);
            }
        } catch (IOException e) {
            System.out.println(e);
        } finally {
            try {
                if (fileReader != null) {
                    fileReader.close();
                }
            } catch (IOException e) {
                System.out.println(e);
            }
        }
        return lineas;
    }

    public void escribir(String nombreArchivo, 
                         List<String> lineas) {

        FileWriter archivo;
        PrintWriter printWriter = null;

        try {
            archivo = new FileWriter(nombreArchivo, true);
            printWriter = new PrintWriter(archivo);

            for (String linea : lineas) {
                printWriter.println(linea);
            }
        } catch (IOException e) {
            System.out.println(e);
        } finally {
            printWriter.close();
        }
    }
}
            \end{verbatim}
        \end{multicols}
        \vspace*{-20pt}
        \begin{center}
            \textbf{\textquestiondown C\'omo ser\'e evaluado en la pregunta 1?} \linebreak
            \begin{tabular}{|p{2cm}|p{4cm}|p{4cm}|p{4cm}|}\hline
                \multicolumn{1}{|c|}{\textbf{T\'opico}} & 
                \multicolumn{1}{c|}{\textbf{Logrado}} & 
                \multicolumn{1}{c|}{\textbf{Medianamente logrado}} & 
                \multicolumn{1}{c|}{\textbf{No logrado}} \\ \hline
                Herencia & 
                \emph{6pts} Comenta satisfactoriamente el mecanismo de herencia en la relaci\'on \textbf{es-un}. & 
                \emph{3pts} Comenta parcialmente el mecanismo de herencia en la relaci\'on \textbf{es-un}, dejando dudas respecto a la jerarqu\'ia Padre-Hijo. & 
                \emph{1pts} Comenta err\'oneamente el mecanismo de herencia en la relacion \textbf{es-un}.\\ \hline
                Clase abstracta & 
                \emph{6pts} Comenta satisfactoriamente el concepto de clase abstracta. & 
                \emph{3pts} Comenta parcialmente el concepto de clase abstracta, dejando dudas respecto a la manipulaci\'on de sus m\'etodos. & 
                \emph{1pts} Comenta err\'oneamente el concepto de clase abstracta.\\ \hline
                Interfaces & 
                \emph{6pts} Comenta satisfactoriamente la ''simulaci\'on'' de herencia m\'ultiple. & 
                \emph{3pts} Comenta parcialmente la herencia m\'ultiple, no utilizando interfaces. & 
                \emph{1pts} Comenta err\'oneamente la ''simulaci\'on'' de herencia m\'ultiple. \\ \hline
                TDA Lista & 
                \emph{6pts} Comenta satisfactoriamente el uso de TDA Listas para la colecci\'on de objetos. & 
                \emph{3pts} Comenta parcialmente el uso de TDA Listas para la colecci\'on de objetos, dejando dudas respecto a la manipulaci\'on de \'estos. & 
                \emph{1pts} Comenta err\'oneamente el uso de TDA Listas. \\ \hline
                TDA Bean & 
                \emph{6pts} Comenta satisfactoriamente el uso de TDA Beans como principal componente de la POO. & 
                \emph{3pts} Comenta parcialmente el uso de TDA Beans como principal componente de la POO, dejando dudas respecto a la utilidad de \'estos. & 
                \emph{1pts}  Comenta err\'oneamente el uso de TDA Beans. \\ \hline
                Total m\'aximo puntaje pregunta 1 & 
                \emph{30pts} & 
                \emph{15pts} & 
                \emph{5pts} \\ \hline
            \end{tabular}
        \end{center}
        \vspace*{-20pt}
        \begin{center}
            \textbf{\textquestiondown C\'omo ser\'e evaluado en la pregunta 2?} \linebreak
            \begin{tabular}{|p{2cm}|p{4cm}|p{4cm}|p{4cm}|}\hline
                \multicolumn{1}{|c|}{\textbf{T\'opico}} & 
                \multicolumn{1}{c|}{\textbf{Logrado}} & 
                \multicolumn{1}{c|}{\textbf{Medianamente logrado}} & 
                \multicolumn{1}{c|}{\textbf{No logrado}} \\ \hline
                Construir entidades & 
                \emph{20pts} Aplica en forma correcta la herencia con el desarrollo de las entidades. & 
                \emph{12pts} No aplica de forma correcta la herencia, pero si construye las entidades. & 
                \emph{6pts} No aplica de forma correcta la herencia y no construye las entidades. \\ \hline
                Construir clase UbicacionImpl & 
                \emph{40pts} Construye satisfactoriamente la clase e implementa todos los m\'etodos de lectura de los dataset y b\'usqueda de informaci\'on en las listas de objetos. & 
                \emph{25pts} Construye la clase con lectura parcial de los dataset, con s\'olo algunos m\'etodos de b\'usqueda de informaci\'on en las listas de objetos. & 
                \emph{12pts} No construye la clase. \\ \hline
                Construir clase principal & 
                \emph{10pts} Construye satisfactoriamente la clase principal y el m\'etodo est\'atico main, con las instancias de los objetos y la llamada a sus m\'etodos. & 
                \emph{5pts} Construye la clase principal con el m\'etodo est\'atico main, pero no realiza correctamente las instancias de los objetos y las llamadas a los m\'etodos. & 
                \emph{0pts} No construye la clase principal. \\ \hline
                Total m\'aximo puntaje pregunta 2 & 
                \emph{70pts} & 
                \emph{42pts} & 
                \emph{18pts} \\ \hline
            \end{tabular}
        \end{center}

    }
\end{document} 
