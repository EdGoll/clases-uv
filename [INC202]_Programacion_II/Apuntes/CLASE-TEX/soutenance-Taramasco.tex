
%\newcommand{\pd}[2]{\frac{\partial #1}{\partial #2}}
%\def\abs{\operatorname{abs}}
%\def\argmax{\operatornamewithlimits{arg\,max}}
\def\argmin{\operatornamewithlimits{argmin}}

%\def\diag{\operatorname{Diag}}
%\newcommand{\eqRef}[1]{(\ref{#1})}
\documentclass[11pt%,trans
]{beamer}

\usepackage{movie15}
\usepackage{hyperref}
\usepackage{ragged2e}
\usepackage{hyperref}
\include{formatAndDefspres}

\hypersetup{pdfpagemode=FullScreen} % makes your presentation go automatically to full screen
\usepackage{tikz}
%\usepackage[french]{babel}
\usepackage{animate}
 \setbeamertemplate{note page}[plain]
\usetheme{Frankfurt}
%\usetheme{Warsaw}
\usepackage{times}
\usefonttheme{structurebold}
\usepackage[francais]{babel}
\usepackage[utf8]{inputenc}
%\usepackage{tabularx}
%\usepackage[french]{babel}
%\usepackage{pgf,pgfarrows,pgfnodes,pgfautomata,pgfheaps}
%\usepackage{amsmath,amssymb}
\usepackage{multicol}
\usepackage{graphicx}
\usepackage{epstopdf}
\usepackage{fancyhdr}
\usepackage{hyperref}
\usepackage{listings}


%\setbeamercolor{frametitle}{bg=macouleur}
\newcommand{\Link}{\text{L}}
\newcommand{\argmax}{\operatornamewithlimits{argmax}}
 \setbeamertemplate{itemize item}[square]
 \setbeamertemplate{itemize subitem}[circle]    
\newcommand{\TB}[1]{\textcolor{blue}{#1}}
\setbeamertemplate{navigation symbols}{}
 \setbeamercolor{section in head/foot}{parent=palette primary}
\definecolor{progressbar@bgblue}{rgb}{0.75,0.75,0.95} % use structure theme to change
 \definecolor{progressbar@fgblue}{rgb}{0.1,0.1,0.1} % use structure theme to change 
 \setbeamercolor*{palette primary}{fg=progressbar@fgblue,bg=progressbar@bgblue}
 \setbeamercolor*{palette secundary}{fg=progressbar@fgblue,bg=progressbar@bgblue}
 \setbeamercolor*{palette tertiary}{fg=progressbar@fgblue,bg=progressbar@bgblue}
 \setbeamercolor{subsection in sidebar}{parent=palette primary}
 \setbeamercolor{section in head/foot}{parent=palette primary!80!blue}
\definecolor{fondtitre}{rgb}{0.90,0.43,0.09}  % vert fonce
\definecolor{coultitre}{rgb}{0.21,0.55,0.05}  % marron
%%%je règle ici l'ensemble de mes couleurs


\definecolor{fondtitre}{rgb}{0.75,0.65,0.95} % use structure theme to change
 \definecolor{coultitre}{rgb}{0.2,0.2,0.3} % use structure theme to change

\setbeamercolor{structure}{fg=coultitre, bg=fondtitre!10}% box
\setbeamercolor{frametitle}{fg=coultitre,bg=fondtitre!45}% titre slide
 \setbeamercolor{section in head/foot}{fg=black,bg=fondtitre}%partie supérieure slide5
\setbeamercovered{dynamic}
\setbeamersize{text margin left=2em,text margin right=2em}

\newcommand{\citex}[1]{{\scriptsize \em\cite{#1}}}
\newcommand{\MISSING}[1]{{\bf [\TB{{\sc A venir: --} \footnotesize{#1}}]}}

\newcommand{\LW}{\linewidth}
\newcommand{\BC}{\begin{column}}
\newcommand{\EC}{\end{column}}
\newcommand{\BB}{\begin{block}}
\newcommand{\EB}{\end{block}}
\newcommand{\BK}[2]{\BB{#1}#2\EB}

%PARA VER LOS COMENTARIOS TENGO QUE DESCOMENTAR ESTO
%\newcommand{\FTR}[1]{\noindent{[{\scriptsize\em \TR{-}}]}}


\newcommand{\COLS}[1]{\begin{columns}#1\end{columns}}
\newcommand{\GFX}[2]{\begin{center}\includegraphics[width=#1\LW]{#2}\end{center}}
\newcommand{\GFXT}[2]{\begin{figure}[!t]\begin{center}\includegraphics[width=#1\LW]{#2}\end{center}\end{figure}}

\newcommand{\GFXw}[3]{\begin{center}\includegraphics[width=#1\LW,height=#2]{#3}\end{center}}
\newcommand{\GFXW}[2]{\begin{center}\includegraphics[width=#1\LW]{#2}\end{center}}
\newcommand{\GFXh}[3]{\begin{center}\includegraphics[height=#1, width = #2]{#3}\end{center}}
\newcommand{\GFXhc}[4]{\begin{center}\includegraphics[height=#1, width = #2]{#3}\caption{#4}\end{center}}

\newcommand{\GFXH}[2]{\begin{center}\includegraphics[height=#1]{#2}\end{center}}
\newcommand{\GFXc}[3]{\begin{center}\includegraphics[ width = #1\LW]{#2}\caption{#3}\end{center}}
\newcommand{\framex}[2]{\frame{\frametitle{#1}#2}}


\newcommand{\NOTES}[1]{\note[itemize]{\vspace{0.5cm}{ \tiny{#1}}}}

%crée une colonne de taille #1 et contenant #2
%typiquement: \CK{.5\LW}{salut!}
\newcommand{\CK}[2]{\BC{#1}#2\EC}

%crée une colonne de taille #2\LW contenant #4, entourée à gauche d'une colonne #1\LW et à droite d'une colonne #3\LW
%typiquement: \CX{.1}{.8}{.1}{colonne centrée}
\newcommand{\CX}[4]{\begin{columns}\CK{#1\LW}{}\CK{#2\LW}{#4}\CK{#3\LW}{}\end{columns}}
\newcommand{\centercolumn}[4]{\begin{columns}\CK{#1\LW}{}\CK{#2\LW}{#4}\CK{#3\LW}{}\end{columns}}

%crée une paire de colonnes de taille #1\LW et #2\LW et contenant #3 et #4:
%typiquement: \balancecolumns{.4}{.6}{gauche}{droite}
\newcommand{\balancecolumns}[4]{\begin{columns}\CK{#1\LW}{#3}\CK{#2\LW}{#4}\end{columns}}
\newcommand{\balancecolumnst}[4]{\begin{columns}[t]\CK{#1\LW}{#3}\CK{#2\LW}{#4}\end{columns}}

\newcommand{\ITZ}[1]{#1}
\newcommand{\IT}[2]{\uncover<#1->{\BK{}{#2}}}
\newcommand{\ITT}[3]{\uncover<#1->{\BK{#2}{#3}}}

\newcommand{\SM}[1]{\CX{.01}{.97}{.01}{\BK{}{\footnotesize \em #1}}}

%\pgfdeclareimage[height=1.2cm]{logo}{image/banniere}
%\logo{\pgfuseimage{logo}}

\pgfdeclareimage[height=3cm]{agro}{image/agro}
\pgfdeclareimage[height=3cm]{ifris}{image/logo_IFRIS}

\setbeamertemplate{navigation symbols}{}

\newcommand{\FTR}[1]{\noindent{[{\scriptsize\em \TR{#1}}]}}
\newcommand{\Aset}{\mathcal{A}}
\newcommand{\Cset}{\mathcal{C}}
\newcommand{\hyperA}[1]{\mathfrak{A}_{#1}}
\newcommand{\hyperC}[1]{\mathfrak{C}_{#1}}
\newcommand{\hyperE}[1]{\mathfrak{E}_{#1}}
\newcommand{\hypere}{\mathfrak{e}}
\newcommand{\Aproj}[1]{{#1}^{\Aset{}}}
\newcommand{\Cproj}[1]{{#1}^{\Cset{}}}
\newcommand{\W}{\mathbf{W}}
\newcommand{\w}{\mathbf{w}}
\newcommand{\hw}{\mathbf{\hat{w}}}
\newcommand{\s}{\mathbf{s}}
\newcommand{\B}{\mathfrak{B}}
\newcommand{\M}{\mathbf{M}}
\newcommand{\m}{\mathbf{m}}
\newcommand{\TT}[1]{\texttt{#1}}
\newcommand{\VS}{\vspace{2.3cm}}
\newcommand{\vs}{\vspace{.2cm}}
\newcommand{\FNS}{\footnotesize}
\newcommand{\ITX}[1]{\begin{itemize}#1\end{itemize}}
\newcommand{\ENX}[1]{\begin{enumerate}#1\end{enumerate}}
\newcommand{\RA}[1]{$\Rightarrow$ {#1}}
\newcommand{\ra}[1]{$\rightarrow$ {#1}}
\newcommand{\U}[1]{\underline{#1}}
\newcommand{\QM}{{\Large\TT{[?]}}}
\newcommand{\EX}[1]{\begin{quotation}{\em {ex:} \footnotesize\em #1}\end{quotation}}
\newcommand{\TR}[1]{\textcolor{red}{#1}}
\newcommand{\rationale}[1]{\ra{\TR{\em #1}}}
 

\title
{%\includegraphics[height=0.8cm]{image/Logo_AgroParisTech.png}\\
%\vspace{1cm}
{\vspace{0.1cm}
\footnotesize{{%\hspace{-0.8cm}
{\large Programación II}}}}
%\vspace{0.2cm}\scriptsize{{Vers une interface d'observation des grands corpus textuels pour équiper les SHS
%\\}}
%\\
%\vspace{0.2cm}
{\scriptsize{ }}
}
%{Modéliser la dynamique des communautés épistémiques}


\author{
{\Large \TB{Carla {\sc Taramasco}}}\\
%\phantom{sdq}
{\tiny Profesora e Investigadora} \\
{\tiny DECOM \& CNRS} \\
{\footnotesize{mail : alumnos\_uv@yahoo.cl}} \\
%\vspace{-0.2cm}\tiny{                        }\\
}
%{\scriptsize Co-dirige par : Dominique {\sc Bicout }}}
{\em 
%\vspace{0.5cm}
%\pgfuseimage{ifris}
}
%\date{{\scriptsize 14 de agosto 2012}\\ \TB{}}

%\pgfpagesuselayout{3 on 1 with notes}[a4paper,border shrink=5mm]
\begin{document}
\addtobeamertemplate{footline}{\hfill \insertframenumber/\inserttotalframenumber}

\frame{\titlepage}
%%
%DEFINICIONES
%API : application programming interface de java. un numero importante de clases que forman parte del propio lebguaje
%Applets : esu na aplicacion especial qe se ejecuta dentro de un navegador al cargar una pagina HTML desde un servidor Web.  Un applet se descarga desde el servidor y no requiere instalacion en el computador donde esta el navegador. 
%Servlet es una aplicacion sin interface grafica que se ejecuta en un servidor internet. 
%IDE : integrated development environment ambiente de desarrollo por ejemplo eclipse
%Los threads (a veces llamados, procesos ligeros), son básicamente pequeños procesos o piezas independientes de un gran proceso. Al estar los threads contruidos en el lenguaje, son más fáciles de usar y más robustos que sus homólogos en C o C++.

\graphicspath{{./pics/}}



\section{Organización}
\subsection{Organización}
\begin{frame}{Organización del curso}

\ITZ{	
\uncover<1->{\ITT{1}{General}{\begin{itemize}
\uncover<1->

{\item Asistencia libre}  
{\item Horario de atención alumnos : Jueves de 14h30 a 15h30.}
{\item 1 bloque de ayudantía por semana (ayudante por definir).}
\end {itemize}}}

\uncover<2->{\ITT{2}{Calendario}{\begin{itemize}
\uncover<2->

{\item Fechas sin clases : 
\begin{itemize} \item {del 15 al 20 de marzo.}   
 \item {del 10 de mayo al 10 de junio.} 
 \end{itemize} }
{\item Cierre del semestre el 19 de julio.}
{\item 17 clases por recuperar. Horario para clases recuperativas :   miércoles en la tarde?}
 \end{itemize}}}}

 \end{frame}
 
\subsection{Organización}
\begin{frame}{Organización del curso}
\ITZ{	
\uncover<1->{\ITT{1}{Notas}{\begin{itemize}
\uncover<1->

{\item 2 certámenes (Cada certamen representa el 30\% de la nota del ramo).}  
{\item 6 trabajos en clase (en computador). Cada trabajo tiene una ponderación igual, en total serán el 30\% de la nota total del ramo (cada trabajo seria = 1/6*0.30).}
{\item Trabajo final en grupo (generar 10 grupos) trabajan como programadores para el curso de Metodología y Diseño esto representara el 10\% de la nota. Esta nota agrupara dos evaluaciones (interna del ramo y externa de los diseñadores que evaluarán el trabajo).}
{\item Resumen de ponderación : certámenes (60\%), nota de trabajos (30\%), nota de trabajo final (10\%).}
 \end{itemize}}}}
 \end{frame}

\subsection{Organización}
\begin{frame}{Organización del curso}
\ITZ{	
\uncover<1->{\ITT{1}{Calendario de certámenes y trabajos}{\begin{itemize}
\uncover<1->
{\item 1er Certamen : 24 de abril}  
{\item 2do Certamen : 29 de junio}  
{\item Prueba Recuperativa : 4 de julio}
{\item Prueba Especial : 8 de julio}
{\item Entrega trabajo : 14 de mayo}
{\item Entrega trabajo : 28 de junio}


{\item  Trabajos \begin{itemize} \item {27 de marzo} \item {10 y 17 de abril} \item{8 de mayo} \item {5 y 19 de junio} \end {itemize}
}
\end {itemize}}}}
 \end{frame}
 
 
\section{Paradigmas}
\subsection{Paradigmas}
\begin{frame}{Plan de la sección Paradigmas de Programación}
\ITZ{	
\uncover<1->{\ITT{1}{}{\begin{itemize}
\uncover<1->
{\item Definición de paradigma} 
{\item Principales Paradigmas}
\begin {itemize}
\item {Imperativo}
\item {Declarativo}
\item {Estructurado}
\item{Orientado a Objetos}
\item {Funcional}
\item{Logico}
\end {itemize}
\end{itemize}}}}
\end{frame}

\section{POO, Java}
\subsection{Introduction}
\begin{frame}{Plan de la Introducción a la POO y a Java}
\ITZ{	
\uncover<1->{\ITT{1}{}{\begin{itemize}
\uncover<1->
{\item Introducción a la Programación Orientada a Objetos.} 
{\footnotesize{\begin {itemize}
\item {Objetos y clases}
\item {Comportamiento y atributos}
{\item Características asociadas a la POO.} 
\begin{itemize}
\item {Abstracción}
\item {Encapsulamiento}
\item {Ocultamiento}
\item {Herencia}
\item {Polimorfismo}
\end {itemize}
\end {itemize}}}
{\item Introducción a la Programación en Java}
{\footnotesize{\begin {itemize}
\item {Inicios de Java}
\item {Ventajas de Java}
\item{Programar en Java}
\end {itemize}}}
{\item Ejercicios de Introducción a Java}
\end{itemize}}}}
\end{frame}

\section{Fundamentos}
\subsection{Fundamentos}
\begin{frame}{Plan de la sección Fundamentos de Java}
\ITZ{	
\uncover<1->{\ITT{1}{}{ \footnotesize{\begin{itemize}
\uncover<1->
{\item Enunciados y expresiones} 
{\item Variables y tipos de datos} 
{\item Comentarios} 
{\item Literales} 
{\item Expresiones y Operadores} 
{\item Ejercicios Fundamentos}
\end{itemize}}}}}
\end{frame}

\section{Objetos}
\subsection{Objetos}
\begin{frame}{Plan de la sección Objetos}
\ITZ{	
\uncover<1->{\ITT{1}{}{\begin{itemize}
\uncover<1->
\item {Objetos}
{\footnotesize{\begin {itemize}
\item {Crear nuevos Objetos}
\item {Llamar a Métodos}
\item {Biblioteca de clase Java}
\item {Ejercicio de Objetos}
\end {itemize}}}
\end{itemize}}}}
\end{frame}

\section{Condiciones, ciclos}
\subsection{Arreglos, condiciones y ciclos}
\begin{frame}{Plan Arreglos, condiciones y ciclos}
\ITZ{	
\uncover<1->{\ITT{1}{}{\begin{itemize}
\uncover<1->
\item {Arreglos}
{\footnotesize{\begin {itemize}
\item {Declarar variables de arreglo}
\item {Crear objetos de arreglo}
\item {Acceder a los elementos de un arreglo}
\item {Cambiar elementos de un arreglo}
\end{itemize}}}
\item {Arreglos multidimensionales}
\item {ArrayList}
\item{Condicionales}
{\footnotesize{\begin {itemize}
\item{if}
\item{switch}
\end{itemize}}}
\item {Ciclos}
{\footnotesize{\begin {itemize}
\item{for}
\item{while y do}
\end{itemize}}}
{\item Ejercicio de Arreglos, condicionales y ciclos}
\end{itemize}}}}
\end{frame}

\section{Clases, Objetos y Metodos}
\subsection{Clases, Objetos y Metodos}
\begin{frame}{Clases, Objetos y Metodos}
\ITZ{	
\uncover<1->{\ITT{1}{}{\begin{itemize}
\uncover<1->
\item {Clases y objetos}
{\footnotesize{\begin {itemize}
\item {Definir Clases}
\item {Constructores}
\item {Definición de objetos}
\end {itemize}}}
\item {Métodos}
{\footnotesize{\begin {itemize}
\item {Reglas de escritura de métodos}
\item {Atributos y metodos de clase}
\item {Sobrecarga de métodos }
\item {Intercambio de información con métodos}
\item {Recursividad}
\end {itemize}}}
\item {Aplicaciones Java}
{\footnotesize{\begin {itemize}
\item {Multiples Clases}
\item {Encapsulamiento}
\item {Paquetes}
\item {Ejercicio Clases, Objetos y Metodos}
\end {itemize}}}
\end{itemize}}}}
\end{frame}

\section{Herencia}
\subsection{Herencia}
\begin{frame}{Herencia}
\ITZ{	
\uncover<1->{\ITT{1}{}{\begin{itemize}
\uncover<1->
\item {Herencia}
{\footnotesize{\begin {itemize}
\item {Acceso a una clase derivada}
\item {Construcción de objetos derivados}
\item {Redefinición y sobrecarga de metodos }
\item {Polimorfismo}
\item {Super clase object}
\item {Miembros protected}
\item {Clases y metodos de finalización}
\item {Clases abstractas}
\item {Clases Interfaces}
\end {itemize}}}
\end{itemize}}}}
\end{frame}

\section{Programación Gráfica}
\subsection{Programación Gráfica}
\begin{frame}{Programación Gráfica}
\ITZ{	
\uncover<1->{\ITT{1}{}{\begin{itemize}
\uncover<1->
\item {Ventanas}
\item {Componentes: botón, campos de texto, listas, etiquetas, combobox }
\item {Dinámica de Componentes}
\item{Eventos}
\item {Dibujar}
\end{itemize}}}}
\end{frame}


\section{Java Avanzado}
\subsection{Java Avanzado}
\begin{frame}{Java Avanzado}
\ITZ{	
\uncover<1->{\ITT{1}{}{\begin{itemize}
\uncover<1->
\item {Manejo de excepciones}
{\footnotesize{\begin {itemize}
\item {try}
\item {catch}
\item {throws}
\item {finally}
\end {itemize}}}

\item {Applets}
{\footnotesize{\begin {itemize}
\item {Insertar applets}
\item {Pasar parámetros}
\item {Gráficos, fuentes y color}
\item {Animación sencilla}
\end {itemize}}}
\end{itemize}}}}
\end{frame}



%%
%DEFINICIONES
%API : application programming interface de java. un numero importante de clases que forman parte del propio lebguaje
%Applets : es una aplicacion especial qe se ejecuta dentro de un navegador al cargar una pagina HTML desde un servidor Web.  Un applet se descarga desde el servidor y no requiere instalacion en el computador donde esta el navegador. 
%Servlet es una aplicacion sin interface grafica que se ejecuta en un servidor internet. 
%IDE : integrated development environment ambiente de desarrollo por ejemplo eclipse
%Los threads (a veces llamados, procesos ligeros), son básicamente pequeños procesos o piezas independientes de un gran proceso. Al estar los threads contruidos en el lenguaje, son más fáciles de usar y más robustos que sus homólogos en C o C++.
%Un thread es un flujo secuencial de control dentro de un programa.
%appletviewer interprete de applets
\graphicspath{{./pics/}}
 
\section{Paradigmas}
\subsection{Paradigmas}
\begin{frame}{Plan de la sección Paradigmas de Programación}
\ITZ{	
\uncover<1->{\ITT{1}{}{\begin{itemize}
\uncover<1->
{\item Definición de paradigma} 
{\item Principales Paradigmas}
\begin {itemize}
\item {Imperativo}
\item {Declarativo}
\item {Estructurado}
\item{Orientado a Objetos}
\item {Funcional}
\item{Logico}
\end {itemize}
\end{itemize}}}}
\end{frame}
 
 \subsection{Introducción a la Programación en Java}
\begin{frame}{Bibliografia}
\ITZ{	
\uncover<1->{\ITT{1}{}{\scriptsize{\begin{itemize}
\uncover<1->
\item {Aprendiendo Java en 21 dias. Laura Lemay and Charles Perkins. Edición : Prentice Hall Hispanoamericana SA, 1996}
\item {Java como programar. P Deital and H Deitel. Edición : Pearson Prentice Hall, 2008}
\item {Java como programar. P Deital and H Deitel. Edición : Pearson Prentice Hall, 2004}
\item {Aprenda Java como si estuviera en primero. Javier Garcia et all. Universidad de Navarra, 2000}
\item {Guia de iniciación al lenguaje Java, Universidad de Burgos, 1999}
\item {Aprendiendo Java y programación a objetos. Gustavo Perez, 2008}
\item {Apuntes de programación. Universidad de Cergy Pointoise, 2010}
\item {Curso Java de Everis, 2006}
\item {Programación con Java. Profesor : Carlos Alberto Román Zamitiz. Universidad Nacional Autonoma de Mexico : http://profesores.fi-b.unam.mx/carlos/java/indice.html}
\item {Documentación JAVA : http://www.oracle.com/technetwork/java/javase/documentation/index.html}
\end{itemize}
}}}}
\end{frame}

\subsection{Paradigmas}
\begin{frame}{Paradigmas de Programación}
\ITZ{	
\uncover<1->{\ITT{1}{}{
\uncover<1->
{Un paradigma de programación es un modelo básico de diseño y desarrollo de programas, que permite producir programas con directrices específicas, tales como: estructura modular, fuerte cohesión, alta rentabilidad, etc.}}}}
 \end{frame}

\subsection{Paradigmas}
\begin{frame}{Plan de la sección Paradigmas de Programación}
\ITZ{	
\uncover<1->{\ITT{1}{Paradigma Imperativo}{
\uncover<1-> {
Describe la programación como una secuencia instrucciones o comandos que cambian el estado de un programa. El código máquina en general está basado en el paradigma imperativo. Su contrario es el paradigma declarativo. Los programas imperativos son un conjunto de instrucciones que le indican al computador cómo realizar una tarea. 
Los primeros lenguajes imperativos fueron los lenguajes de máquina. }}}}
 \end{frame}
  %perl , pascal, fortran
  
 \subsection{Paradigmas}
\begin{frame}{Paradigmas de Programación}
\ITZ{	
\uncover<1->{\ITT{1}{Paradigma Declarativo}{
\uncover<1-> {
 No se basa en el cómo se hace algo (cómo se logra un objetivo paso a paso), sino que describe (declara) cómo es algo. \\
Se enfoca en el desarrollo de programas especificando o "declarando" un conjunto de condiciones, proposiciones, afirmaciones, restricciones, ecuaciones o transformaciones que describen el problema y detallan su solución. La solución es obtenida mediante mecanismos internos de control, sin especificar exactamente cómo encontrarla (tan sólo se le indica a la computadora que es lo que se desea obtener o que es lo que se está buscando).
 }}}}
 \end{frame}
 
%En la programación imperativa se describe paso a paso un conjunto de instrucciones que deben ejecutarse para variar el estado del programa y hallar la solución, es decir, un algoritmo en el que se describen los pasos necesarios para solucionar el problema.
%En la programación declarativa las sentencias que se utilizan lo que hacen es describir el problema que se quiere solucionar, pero no las instrucciones necesarias para solucionarlo. Esto último se realizará mediante mecanismos internos de inferencia de información a partir de la descripción realizada.
 %prolog
 
 \subsection{Paradigmas}
\begin{frame}{Paradigmas de Programación}
\ITZ{	
\uncover<1->{\ITT{1}{Paradigma Estructurado}{
\uncover<1-> {
 La programación se divide en bloques (procedimientos y funciones) que pueden o no comunicarse entre sí. Además la programación se controla con secuencia, selección e iteración.  Su principal ventaja es la estructura clara lo que entrega una mejor comprensión de la programación.}}}}
 \end{frame}
%c
 %Permite reutilizar código programado y otorga una mejor compresión de la programación.
 
  \subsection{Paradigmas}
\begin{frame}{Paradigmas de Programación}
\ITZ{	
\uncover<1->{\ITT{1}{Paradigma Orientado a Objetos}{
\uncover<1-> {
Está basado en el uso de objetos: estructuras de datos que contienen atributos y métodos con sus interacciones. La idea es  encapsular estados o propiedades y operaciones o comportamientos en objetos que se comunican entre si. Su principal ventaja es la reutilización de códigos y su facilidad para pensar soluciones a determinados problemas.}}}}
 \end{frame}
 
%ta abstraction, encapsulation, messaging, modularity, polymorphism, and inheritance. 
 
\subsection{Paradigmas}
\begin{frame}{Paradigmas de Programación}
\ITZ{	
\uncover<1->{\ITT{1}{Paradigma Funcional}{
\uncover<1-> {
Este paradigma concibe a la computación como la evaluación de funciones y evita declarar y cambiar datos. En otras palabras, hace hincapié en la aplicación de las funciones y composición entre ellas, en contraste con el estilo de programación imperativa, más que en los cambios de estados y la ejecución secuencial de comandos.}}}}
 \end{frame}
 % ocamel
\subsection{Paradigmas}
\begin{frame}{Paradigmas de Programación}
\ITZ{	
\uncover<1->{\ITT{1}{Paradigma Lógico}{
\uncover<1-> {
Se basa en la definición de reglas lógicas para luego, a través de un motor de inferencias lógicas, responder preguntas planteadas al sistema y así resolver los problemas.}}}
%prolog

\uncover<2->{\ITT{1}{Otros Paradigmas y subparadigmas}{
\uncover<2-> {
Paradigma orientado al sujeto, paradigma reflectante, programación basada en reglas, paradigma basado en restricciones, programación basada en prototipos, etc. 
}}}}
 \end{frame}
 
 
\section{POO y Java}
\subsection{Introduction}
\begin{frame}{Plan de la Introducción a la POO y a Java}
\ITZ{	
\uncover<1->{\ITT{1}{}{\begin{itemize}
\uncover<1->
{\item Introducción a la Programación Orientada a Objetos.} 
{\footnotesize{\begin {itemize}
\item {Objetos y clases}
\item {Comportamiento y atributos}
{\item Características asociadas a la POO.} 
\begin{itemize}
\item {Abstracción}
\item {Encapsulamiento}
\item {Ocultamiento}
\item {Herencia}
\item {Polimorfismo}
\end {itemize}
\end {itemize}}}
{\item Introducción a la Programación en Java}
{\footnotesize{\begin {itemize}
\item {Inicios de Java}
\item {Ventajas de Java}
\item{Programar en Java}
\end {itemize}}}
{\item Ejercicios de Introducción a Java}
\end{itemize}}}}
\end{frame}

\subsection{Introducción a la Programación Orientada a Objetos}
\begin{frame}{Definiciones}
\ITZ{	
\uncover<1->{\ITT{1}{POO}{
\uncover<1->
La \textbf{POO} se basa en la dividir el programa en pequeñas unidades lógicas de código. A estas pequeñas unidades lógicas de código se les llama objetos. Los objetos son unidades independientes que se comunican entre ellos.\\
La \textbf{POO}  es que proporciona conceptos y herramientas con las cuales se modela y representa el mundo real tan fielmente como sea posible.
}}}
\end{frame}

\subsection{Introducción a la Programación Orientada a Objetos}
\begin{frame}{Definiciones}
\ITZ{	
\uncover<1->{\ITT{1}{Que es un objeto?}{
\uncover<1->{
Cualquier cosa que vemos a nuestro alrededor.  Ej : auto.
Componentes de los objetos :
\begin{itemize}
\item {características (marca, modelo, color, etc ) }
\item {comportamiento (frenar, acelerar, retroceder, encender, etc).}
\end{itemize}}}}}
\end{frame}

%Ejemplo: libros, computadora, teléfono celular, árbol o auto. \\
%No necesitamos ser expertos en mecánica para saber que un auto está compuesto internamente por varios componentes: ruedas, motor, caja de cambio, etc, el conjunto de esos componentes hace funcionar el auto. Cada componente puede ser muy complicado y ser fabricado por distintas empresas. Si quisiéramos armar un auto no necesitamos saber el funcionamiento de cada objeto sino mas bien como interctua con los otros.\\


\subsection{Introducción a la Programación Orientada a Objetos}
\begin{frame}{Definiciones}
\uncover<1->{\ITT{1}{Objetos y clases}{
\uncover<1->{
Los programas en POO están construidos a base de objetos con características (atributos o variables) y comportamiento (métodos) específicos y que pueden comunicarse entre si.}}}
\balancecolumns{.45}{.55}
{
\uncover<1->{\begin{center}\emph{Clase} \end{center}}
\uncover<1->{\GFXH{3cm}{ClaseAuto.pdf}}}{
\uncover<1->{{\begin{center}\emph{Clase y Objeto} \end{center}}
\uncover<1->{\GFXH{4.2cm}{ClaseObAuto.pdf}}}}
\end{frame}


\subsection{Introducción a la Programación Orientada a Objetos}
\begin{frame}{Definiciones}
\ITZ{	
\uncover<1->{\ITT{1}{Objetos y clases}{\begin {itemize}
\uncover<1->

\item {\textbf{Clases:} Modelo para múltiples objetos con características similares. Las clases comprenden todas las características de una serie particular de objetos. En POO se definen clases como un modelo abstracto de un objeto. En programación estructurada seria un tipo de dato (en C seria struct o typedef).}
\item {\textbf{Instancias:} representación concreta de un objeto. Al definir una clase se pueden crear muchas instancias de la misma y cada instancia puede tener diversas características mientras se comporte y reconozca como objeto de la clase. En programación estructurada seria una variable}

%objetos : no necesitas saber como funciona cada componente sino mas bien como interactuan entre si para poder ensamblarlos.
%arbol y los tipos de arbol
{\footnotesize\item {\textit {Ejemplo}: Clase Button.}}
\end {itemize}}}}
\end{frame}

% define las características de un botón (etiqueta, tamaño, apariencia, etc) y el comportamiento (necesita 1 clic o 2, cambia de color, etc). Los botones tendrán las mismas características generales definidas por la clase pero con apariencia y comportamiento diferente. }

%\subsection{Introducción a la Programación Orientada a Objetos}
%\begin{frame}{Definiciones}
%\ITZ{	
%\uncover<1->{\ITT{1}{Objetos y clases}{
%\uncover<1->
%
%\textbf{Biblioteca de clases:} Java posee un conjunto de clases (biblioteca de clases) implementada tanto para las funciones básicas como también para gráficos, red, etc. En muchos casos solo necesitas crear una clase que utilice la biblioteca de clases estándar.\\
%\textbf{Importante:} Crear el conjunto adecuado de clases para lograr lo que el programa necesita cumplir. %Los objetos se crean y descartan como sea necesario.\\
%
%}}}
%\end{frame}


\subsection{Introducción a la Programación Orientada a Objetos}
\begin{frame}{Definiciones}
\ITZ{	
\uncover<1->{\ITT{1}{Atributos y Comportamiento}{
\uncover<1->

\textbf{Atributos:} Características que diferencian a un objeto de otro y determinan la apariencia, estado u otras cualidades de ese objeto. \\
Los atributos se definen como variables de hecho podrían ser las variables globales del objeto completo. Cada instancia de una clase puede tener diferentes valores para sus variables, a cada variable se le llama una variable de instancia. La clase define el tipo de atributo y cada instancia guarda su propio valor para ese atributo.\\
}}}
%Las variables de instancia definen los atributos de un objeto, al cambiar el valor de una variable se cambia el atributo de ese objeto.
%Se pueden definir variables de instancia cuyos valores se almacenan en la instancia y de clase (estas se aplican a la clase y a todas sus instancias) y sus valores se almacenan en la clase.}}}
\end{frame}

\subsection{Introducción a la Programación Orientada a Objetos}
\begin{frame}{Definiciones}
\ITZ{	
\uncover<1->{\ITT{1}{Atributos y Comportamiento}{
\uncover<1-> 
{Los \textbf{métodos} son funciones definidas dentro de una clase que operan en las instancias de esas clases. Los objetos se comunican entre si mediante el uso de métodos, una clase puede llamar al método de otra clase.\\
Se pueden definir métodos de instancia (que operan en una instancia de la clase) y los métodos de clase que operan sobre la clase.}
%\textbf{Comportamiento:} determina qué instancias de esa clase requieren cambiar su estado interno o cuando esa instancia es llamada para realizar algo por otra clase u objeto. Para definir el comportamiento de un objeto se crean métodos (similar a las funciones) dentro de una clase. \\%Java no cuenta con métodos definidas fuera de la clase. 
}}}
\end{frame}


\subsection{ Características asociadas a la POO}
\begin{frame}{Características asociadas a la POO}
\ITZ{	
\uncover<1->{\ITT{1}{Abstracción}{
\uncover<1->
La abstracción consiste en captar las características esenciales y el comportamiento de un objeto. \\
Ej: ¿Qué características podemos abstraer de los autos?¿Qué características semejantes tienen todos los autos?. Todos tendrán una marca, un modelo, puertas, ventanas, etc. Y en cuanto a su comportamiento todos los automóviles podrán acelerar, frenar, retroceder, etc.\\
En POO el concepto de clase es la representación y el mecanismo por el cual se gestionan las abstracciones.\\
En Java las clase se definen:
\footnotesize{
\lstinputlisting{./codigosEx/Auto.java}}
}}}
\end{frame}


\subsection{ Características asociadas a la POO}
\begin{frame}{Características asociadas a la POO}
\ITZ{	
\uncover<1->{\ITT{1}{Encapsulamiento}{
\uncover<1->
El encapsulamiento consiste en unir en la clase las características y comportamientos, esto es, las variables y métodos. Es tener todo esto es una sola entidad. En los lenguajes estructurados esto era imposible. \\
La utilidad del encapsulamiento va por la facilidad para manejar la complejidad, ya que tendremos clases como cajas negras donde sólo se conoce el comportamiento pero no los detalles internos.%, y esto es conveniente porque lo que nos interesará será conocer qué hace la clase pero no será necesario saber cómo lo hace.
}}}
\end{frame}



\subsection{ Características asociadas a la POO}
\begin{frame}{Características asociadas a la POO}
\ITZ{	
\uncover<1->{\ITT{1}{Ocultamiento}{
\uncover<1->{
Es la capacidad de ocultar los detalles internos del comportamiento de una clase y exponer sólo los detalles que sean necesarios para el resto del sistema, cada tipo de objeto expone una interfaz a otros objetos que especifica cómo pueden interactuar con los objetos de la clase. \\
 \begin{itemize}
\item {Restringir el uso de la clase porque habrá cierto comportamiento privado de la clase que no podrá ser accedido por otras clases.} \item {Controlar el uso de la clase porque se darán ciertos mecanismos para modificar el estado de una clase}\\ % y es en estos mecanismos dónde se validarán que algunas condiciones se cumplan. 
\end{itemize}
En Java el ocultamiento se logra usando las palabras reservadas: {\textit {public, private}} y {\textit {protected}} delante de las variables y métodos.}
}}}
\end{frame}


\subsection{ Características asociadas a la POO}
\begin{frame}{Características asociadas a la POO}
\ITZ{	
\uncover<1->{\ITT{1}{Herencia}{
\uncover<1->

Organización jerárquica de las clases. \\
Cada clase tiene una superclase y puede tener una o mas subclases. Las subclases heredan todos los métodos y variables de las superclases, esto significa que  las subclases, aparte de los atributos y métodos propios, tienen incorporados los atributos y métodos heredados de la superclases. De esta manera se crea una jerarquía de clases.\\ %La herencia es uno de los conceptos mas importante de la programación orientada a objetos. \\ % son una combinación de métodos y variables de las clases superiores.\\
En Java la herencia se logra usando la palabra reservada: {\textit  {extends}}. En la parte superior de la jerarquía de clase Java esta la clase \textit{Object} todas las clases heredan de esta superclase, cada clase hacia abajo agrega mas información. %Cada clase puede tener solo una superclase.\\
}}}
\end{frame}

\subsection{ Características asociadas a la POO}
\begin{frame}{Características asociadas a la POO}
\ITZ{	
\uncover<1->{\ITT{1}{Crear una jerarquía entre clases}{
\uncover<1->

\textbf{Subclasificación:} Creación de una nueva clase que heredara de otra clase en la jerarquía. Al usar subclasificación, solo se necesita definir las diferencias entre la clase creada y las super clases de esta. Si se crea una clase sin indicar la super clase en la primera linea, java automáticamente supone que esta heredando de la clase \textit{Object}\\
\textbf{Jerarquía de herencia:} se deben desarrollar las clases en una jerarquía en la cual se separe la información común para múltiples clases de la información particular.\\
\textit{Ejemplo: } Crear una jerarquía para una clase Auto, una clase Moto y una clase Bicicleta.
}}}
\end{frame}


\subsection{ Características asociadas a la POO}
\begin{frame}{Jerarquia de herencia}
\GFXH{5cm}{JerarquiaVehiculo.pdf}
\end{frame}


\subsection{ Características asociadas a la POO}
\begin{frame}{Jerarquia}
\GFXH{5cm}{JerarquiaVehiculoV2.pdf}
\end{frame}


\subsection{ Características asociadas a la POO}
\begin{frame}{Características asociadas a la POO}
\ITZ{	
\uncover<1->{\ITT{1}{Polimorfismo}{
\uncover<1->

Literalmente significa "cualidad de tener mas de una forma". En POO es la propiedad que le permite a métodos con el mismo nombre implementar distintas funcionalidades según las clases donde se apliquen. Es decir, métodos diferentes, asociados a objetos distintos, pueden compartir el mismo nombre. Al llamarlos se utilizará el comportamiento correspondiente al objeto que se esté usando. }}}%O dicho de otro modo, las referencias y las colecciones de objetos pueden contener objetos de diferentes tipos, y la invocación de un comportamiento en una referencia producirá el comportamiento correcto para el tipo real del objeto referenciado.}}}
\end{frame}
%metodos con = nombre implementan distintas funcionalidades segun las clases donde se apliquen.
\subsection{Características asociadas a la POO}
\begin{frame}{Características asociadas a la POO}
\ITZ{	
\uncover<1->{\ITT{1}{Envio de mensajes}{
\uncover<1->
Un objeto es inútil si está aislado. Los objetos de un programa interactúan y se comunican entre ellos por medio de mensajes. Cuando un objeto A quiere que un objeto B ejecute una de sus funciones (métodos de B), el objeto A manda un mensaje al objeto B. Los mensajes son invocaciones a los métodos de los objetos.
Por ejemplo, si objeto miAuto debe acelerar.  
\begin{itemize} 
\item {El objeto al cual se manda el mensaje (miAuto).}
\item {El método que debe ejecutar (acelerar()).}
\item {Los parámetros que necesita ese método (10).}
\end{itemize}

Estas tres partes del mensaje (objeto destinatario, método y parámetros) son suficiente información para que el objeto que recibe el mensaje ejecute el método.
}}}
\end{frame}

\subsection{Introducción a la Programación en Java}
\begin{frame}{Inicios de Java}
\ITZ{	
\uncover<1->{\ITT{1}{}{\begin {itemize}
\uncover<1->
\item {\textbf{Desarrollado por : } Sun Microsystems en 1991}
\item {\textbf{Objetivo Inicial y actual:} }
{\footnotesize{\begin {itemize}
\item {Desarrollar un lenguaje para crear software pequeños, rápidos, eficientes y portátiles para diversos dispositivos de hardware (telefonos celulares, radiolocalizadores y asistentes digitales personales)}
\item {Ser el nexo universal que conecte a los usuarios con la información que este situada en el computador local, en un servidor Web o en una base de datos. }%Así permite entre otras funcionalidades,  agregar contenido dinamico, interactividad y animaciones en paginas Web, mejorar funcionalidad de los servidores Web.}
\end {itemize}}}
\item {\textbf{Slogan:}  \textit{Write Once, Run Everywhere}}
\end {itemize}}}}
\end{frame}


\subsection{Introducción a la Programación en Java}
\begin{frame}{Ventajas de Java : independencia de plataformas}
\ITZ{	
\uncover<1->{\ITT{1}{}{\begin {itemize}
\uncover<1->
\item{\textbf{Independencia de plataforma}, tanto a nivel del código fuente como del binario} \\
\textit {Diferencia entre código fuente y binario?}
%Los fuentes pertenecen al código fuente del programa que te querés bajar, es decir, el código puro sin compilar, en el/los lenguaje/s en el/los cual/es fue desarrollado
%Los binarios es el programa ya compilado para "X" arquitectura o sistema operativo. Entre los binarios por lo general vas a encontrar los archivos "ejecutables" del programa.
%Un ejemplo rápido para que veas mejor la diferencia es el siguiente:
%Vos tenés un programa hecho en lenguaje C que lo único que hace es sacar por pantalla un mensaje que dice "Hello World". (clásico)
%Lo que vos tenés hasta ahí es el código fuente (source) del programa.
%Luego si querés usar ese programa en windows lo vas a tener que compilar (con un compilador C para windows) y de este modo te genera un archivo binario el cual vas a poder ejecutar (en este caso)
%Si vos querés usar el mismo programa en Linux, todo lo que tenés que hacer es agarrar el código fuente y compilarlo con un compilador de C en Linux, y vas a obtener exactamente lo mismo, un archivo binario que también lo vas a poder ejecutar.
%El tema de los compiladores se extiende a todos los sistemas operativos como así también existen compiladores para las diferentes arquitecturas. Pero no tiene sentido que entre en esos temas.
%En pocas palabras, si te bajas el código fuente lo vas a tener que compilar, la ventaja de esto es que podés modificar el código fuente del programa (y así modificar el programa en si mismo) como así también al compilarlo podes customizar parámetros para aprovechar al máximo la arquitectura de tu equipo.
%Si te bajas el binario para el sistema operativo que tenés y la arquitectura que tenés la ventaja es que vas a poder ejecutar directamente el programa sin necesidad de compilar absolutamente nada. (viene precompilado). Lo bajás y lo ejecutas...
{\footnotesize{\begin{itemize}
\item{Independencia en Código Fuente :  los tipos primitivos de datos de Java tienen tamaño consistentes en todas las plataformas de desarrollo. Las bibliotecas de Java facilitan la escritura del código, que puede desplazar se plataforma a plataforma}
\item{Independencia en Binario :  los archivos binarios (bytecodes) pueden ejecutarse en distintas plataformas sin necesidad de volver a compilar la fuente.}\\
\textit {bytecodes : conjunto de instrucciones más abstracto que el código máquina (código intermedio), pero no son especificas a un procesador.}
%El bytecode es un código intermedio más abstracto que el código máquina. Habitualmente es tratado como un fichero binario que contiene un programa ejecutable similar a un módulo objeto, que es un fichero binario producido por el compilador cuyo contenido es el código objeto o código máquina .
\end{itemize}}}
\end {itemize}}}}
\end{frame}


\subsection{Introducción a la Programación en Java}
\begin{frame}{Ventajas de Java : Orientado a Objetos}
\ITZ{	
\uncover<1->{\ITT{1}{}{\begin {itemize}
\uncover<1->
\item{Fomenta la reutilización y extensión del código.}
\item {Permite crear sistemas más complejos.}
\item {Relacionar el sistema al mundo real.}
\item {Facilita la creación de programas visuales.}
\item {Elimina redundancia a traves de la herencia y polimorfismo}
%permite la modificacion de datos heredados
%\item {Construcción de prototipos.}
\item {Agiliza el desarrollo de software.}
\item {Facilita el trabajo en equipo.}
\item {Facilita el mantenimiento del software.}
\item {Recolección de basura}
%Recolección de basura: la recolección de basura o garbage collector es la técnica por la cual el entorno de objetos se encarga de destruir automáticamente, y por tanto desvincular la memoria asociada, los objetos que hayan quedado sin ninguna referencia a ellos. Esto significa que el programador no debe preocuparse por la asignación o liberación de memoria, ya que el entorno la asignará al crear un nuevo objeto y la liberará cuando nadie lo esté usando. En la mayoría de los lenguajes híbridos que se extendieron para soportar el Paradigma de Programación Orientada a Objetos como C++ u Object Pascal, esta característica no existe y la memoria debe desasignarse manualmente.
%Programas flexibles y modulares , reutilización del código}}\\
\end {itemize}}}}
\end{frame}


\subsection{Introducción a la Programación en Java}
\begin{frame}{Ventajas de Java : Simplicidad}
\ITZ{	
\uncover<1->{\ITT{1}{}{\begin {itemize}
\uncover<1->
\item{En Java no hay punteros}
\item {Las cadenas y los arreglos son objetos reales}
\item {La administración de la memoria es automatica}
%el programador no debe liberar memoria esto lo hace JVM mediante el recolector de basura lanzando un thread (hilos de codigo paralelos) de forma automatica
\end {itemize}}}}
\end{frame}


\subsection{Introducción a la Programación en Java}
\begin{frame}{Programar en Java}
\ITZ{	
\uncover<1->{ \ITT{1}{El ambiente de desarrollo Java}{\begin{itemize}
\uncover<1->

\item {Un compilador : genera los ficheros compilados en bytecode (en vez de generar código de maquina) (extensión \textbf{ *.class}) a partir del código fuente (extensión \textbf{ *.java}). El bytecode son instrucciones  independientes de la plataforma.} 
\item  {Un interprete Java denominado \textit {Java Virtual Machine (JVM)}  : que interpreta el bytecode (código neutro) convirtiendo a código particular de la CPU utilizada, lo que permite ejecutar el programa. JVM es una aplicación que simula una computadora pero oculta el sistema operativo y el hardware subyacente. JVM es un programa nativo ejecutable en una plataforma especifica.}
\item Java tiene la característica de ser al mismo tiempo compilado e interpretado.
\end {itemize}}}}
% se pueden obtener distintos compiladores o imterpretadores, recomendamos Kaffe como JVM para proyectos avanzados o de alto rendimiento, recomendamos Jikes como compilador de ByteCodes para Java, Jikes no es interpretado como el compilador proporcionado en el JDK de Sun Microsystems.
\end{frame}

\subsection{Introducción a la Programación Orientada a Objetos}
\begin{frame}{Compilación y Ejecución}
\GFXH{5cm}{compilacionEjecucion.pdf}
{\textit {Figura extraída de : $http://profesores.fi\-b.unam.mx/carlos/java/java\_basico1\_1.html$}}
\end{frame}

\newcounter{saveenumi}

\subsection{Introducción a la Programación en Java}
\begin{frame}{Programar en Java}
\ITZ{	
\uncover<1-> { \ITT{1}{Kit de desarrollo Java (JDK) 1-2}{
\uncover<1-> {\footnotesize{
Contiene las herramientas y el conjunto programas y bibliotecas (Interfaces de programación de aplicaciones : APIs)  necesarias para desarrollar, compilar y ejecutar programas en Java. 
\begin{enumerate}
\item {\textbf{javac }: Es el compilador de Java. Se encarga de convertir el código fuente escrito en Java  a bytecode. Recibe como argumento todos los archivos de codigo fuente (con extension .java). Este comando no es parte de Java Runtine Environment (JRE) dado que JRE esta destinado unicamente a ejecutar código binario, no permite compilar. Ej: {\textit{javac Auto.java}}}
\item {\textbf{java }: Es el intérprete de Java. Ejecuta el bytecode a partir de los archivos con extension .class. Recibe como argumento el nombre del binario ejecutable en formato bytecode sin la extension de archivo .class que identifica de manera visual un binario java. Este comando es parte de JRE y JDK. Ej: {\textit{java Auto}}}
\end{enumerate}}
% que ofrece: tipos de datos básicos, capacidades de entrada y salida y otras funciones de utilidad.
 \setcounter{saveenumi}{\theenumi}
%{Un gran numero de clases como por ejemplo para soporte de red, ejecución remota, seguridad, etc protocolos comunes de Internet y funciones para interfaz de usuario}
}}}}
\end{frame}


\subsection{Introducción a la Programación en Java}
\begin{frame}{Programar en Java}
\ITZ{	
\uncover<1-> { \ITT{1}{Kit de desarrollo Java (JDK) 2-2}{
\uncover<1-> {\footnotesize{
 \begin{enumerate}
 \setcounter{enumi}{\thesaveenumi}
% que ofrece: tipos de datos básicos, capacidades de entrada y salida y otras funciones de utilidad.
\item {\textbf{jar }: Herramienta para trabajar con archivos JAR. Permite empaquetar las clases y archivos de Java para fabricar un único archivo contenedor de las aplicaciones, multimedia y
gráficos. Es decir, comprimir el proyecto en un solo archivo de tipo JAR.  Este comando es parte solo de JDK.Ex: para crear un JAR  {\textit{jar cf Auto.jar Auto*}}, para extraer un JAR {\textit{jar xf jar-file}}}
\item {\textbf{javadoc}: Crea documentación en formato HTML a partir de el código fuente y los comentarios.Ej: {\textit{javadoc Auto.java}}}
\item {\textbf{jdb}: Debugger permite detener la ejecución del programa en un punto deseado, lo que ayuda la detección y corrección de errores. Ej: se debe compilar {\textit{javadoc Auto.java}} y luego {\textit{ jdb Auto}} } 
% primero compilar con javac -g Autot.java 
% jdb Autot
%stop in Autot.main()
%run
%step para ejecutar linea a linea
%stop in o stop at colocan breakpoint en una determinada linea
%locals para ver las variables locales
%list numero de linea para consultar comando
%cont se ejecuta el programa y se sale de jdb
% print y dump nos permiten acceder a campos de un objeto (print t1)
\end {enumerate}
%{Un gran numero de clases como por ejemplo para soporte de red, ejecución remota, seguridad, etc protocolos comunes de Internet y funciones para interfaz de usuario}
}}}}}
\end{frame}

\subsection{Introducción a la Programación en Java}
\begin{frame}{Programar en Java}
\ITZ{	
\uncover<1-> { \ITT{1}{Fases para el desarrollo de un programa}{\begin{enumerate}
\uncover<1->
 \footnotesize{
\item {\textbf{Creación:} Usar un editor para escribir el código fuente. Guardarlo con extensión {\textbf {.java}}. Se pueden también usar entornos de desarrollo (IDEs)}
\item {\textbf{Compilación:} se usa el comando {\textbf {javac}}. Ej compilación: {\textit {javac HelloWorld.java}}. El resultado de esta fase es un archivo {\textbf {.class}}
}
\item{\textbf{Cargar en memoria:} El cargador de memoria toma los archivos {\textbf {.class }}}
\item{\textbf{Verificación del bytecode:} Verifica que los bytecode sean validos y no violen restricciones de seguridad.}
\item{\textbf{Ejecución:} La JVM ejecuta los bytecode usando el comando {\textbf {java}}. Las JVM usan una combinación de interpretación y de {\textit compilación justo a tiempo (JIT).} La JVM analiza el código buscando las partes que se ejecutan con frecuencia, para traducirlas al lenguaje de maquina del computador, así cuando encuentra nuevamente el código lo ejecuta en lenguaje de maquina que es mas rápido. Así lo programas en java pasan por dos fases de compilación una para traducir el código fuente a bytecode y la otra para traducir el bytecode a lenguaje de maquina. Ej ejecución: {\textit {java HelloWorld}}}
}
%{Un gran numero de clases como por ejemplo para soporte de red, ejecución remota, seguridad, etc protocolos comunes de Internet y funciones para interfaz de usuario}
\end{enumerate}}}}
\end{frame}

\section{Fundamentos}
\subsection{Fundamentos}
\begin{frame}{Plan de la sección Fundamentos de Java}
\ITZ{	
\uncover<1->{\ITT{1}{}{ \footnotesize{\begin{itemize}
\uncover<1->
%{\item Enunciados y expresiones} 
{\item Variables y tipos de datos} 
{\item Definiciones basicas}
{\item Comentarios} 
{\item Literales} 
{\item Expresiones y Operadores} 
{\item Ejercicios Fundamentos}
\end{itemize}}}}}
\end{frame}

\subsection{Fundamentos}
\begin{frame}{Mi primer programa}
\lstinputlisting{./codigosEx/Hello.java}

\end{frame}

\subsection{Fundamentos}
\begin{frame}{Variables y tipos de datos}
\ITZ{	
\uncover<1->{\ITT{1}{Identificadores}{
\uncover<1-> {Los nombres de las clases, métodos y variables deben :
\begin{itemize}
\item {Empezar con una letra, subrayado (\_) o dólar (\$), no puede tener espacios ni comenzar por números. Ex : {\textit {\_var1}}, {\textit {varx}}, {\textit {MAX\_NUM}}, {\textit{\$var}}}
\item {Después del primer carácter pueden usarse números.}
\item {No puede coincidir con las palabras claves o reservadas.}
\item {No hay un límite en el número de caracteres que pueden tener los identificadores. }
\end{itemize}
}}}}
\end{frame}

	
\subsection{Fundamentos}
\begin{frame}{Variables y tipos de datos}
{\textbf {Palabras Claves de Java}}
\GFXH{4.5cm}{palabrasClaves.pdf}
\tiny{
{\textbf{true, false, null} son palabras reservadas, por lo cual, tampoco pueden usarse como identificadores. }\\
{\textit {Tabla de : $http://profesores.fi-b.unam.mx/carlos/java/java\_basico2\_3.html$}}}
\end{frame}

\subsection{Fundamentos}
\begin{frame}{Variables y tipos de datos}
\ITZ{	
\uncover<1->{\ITT{1}{}{
\uncover<1-> {Las variables deben tener :
\begin{itemize}
\item {Un tipo : que describa el tipo de dato que contiene}
\item {Un identificador: para referir al dato que contiene}
\item {Un valor}
\end{itemize}
}}}}
\end{frame}

\subsection{Fundamentos}
\begin{frame}{Variables y tipos de datos}
\ITZ{	
\uncover<1->{\ITT{1}{Tipos de Variables}{
\uncover<1-> {Java tiene 3 tipos de variables: 
\begin{itemize}
\item {Variables de instancia: se usan para definir atributos o estados de un objeto en particular.}
\item {Variables de clase: similares a las variables de instancia, la unica diferencia es que sus valores se aplican a todas las instancias de la clase.}
\item {Variables locales: se utilizan y se declaran dentro de las definiciones de metodo o bloques, Ej:  contadores de indice de ciclos, temporales, etc. Al terminar la ejecución del bloque o metodo la variable deja de existir.}
\end{itemize}
{\textit{Java no tiene variables globales.}}
}}}}
\end{frame}

\subsection{Fundamentos}
\begin{frame}{Clases de Variables}
\ITZ{	
\uncover<1->{\ITT{1}{Declaración de Variables}{
\uncover<1-> {Para usar una variable primero hay que declararla con el tipo y el identificador (o nombre) de la variable.
\lstinputlisting{./codigosEx/variables.java}
\begin{itemize}
\item {\textit{String:} es una clase que esta dentro de las bibliotecas de clases de Java.}
\item {\textit{boolean:} es un tipo de dato real que puede ser true o false, los booleanos no son números en Java.}
 \end{itemize}
}}}}
\end{frame}

\subsection{Fundamentos}
\begin{frame}{Definiciones básicas Java}
\textbf{Definir una clase}\\
\lstinputlisting{./codigosEx/Auto.java}
\tiny \textit{public}: la clase es accesible desde métodos de cualquier clase. Atributo que permite acceder a la clase desde clases e interfaces que estén en otros paquetes. 
\end{frame}

\subsection{Fundamentos}
\begin{frame}{Definiciones básicas Java}
\textbf{Definir variables}\\
\lstinputlisting{./codigosEx/Autov.java}
{\textit{Si bien, la declaración de variables puede ir en cualquier parte, habitualmente se declaran al inicio metodo o clase.}}
% solo puede haber 1 clase publica por fichero fuente
\end{frame}

\subsection{Fundamentos}
\begin{frame}{Definiciones básicas Java}
\textbf{Definir variables}\\
Es posible {\textbf{encadenar}} nombres de variables del mismo tipo
\lstinputlisting{./codigosEx/Autoven.java}
\end{frame}

\subsection{Fundamentos}
\begin{frame}{Definiciones básicas Java}
\textbf{Definir variables}\\
También se le puede dar un {\textbf{valor inicial}} a las variables
\lstinputlisting{./codigosEx/Autovenv.java}
\end{frame}

\subsection{Fundamentos}
\begin{frame}{Sintaxis básicas de Java}
\ITZ{	
\uncover<1->{\ITT{1}{}{\begin{itemize}
\uncover<1->
\item {\textbf{Comentario : }  Hay tres tipos de comentarios en Java
\begin{itemize} 
\item {// comentario de una linea}
\item {/*comentario de mas de una linea */} 
\item {/**comentarios para javadoc */ }
\end{itemize}}
\item {\textbf{Sentencia: } Es una linea de código terminada con punto y coma ";".}
\item {\textbf{Bloque: } Es un conjunto de sentencias agrupadas entre llaves "\{\}". Los bloques pueden ser anidados}
\end{itemize}
}}}
\end{frame}

\subsection{Fundamentos}
\begin{frame}{Definiciones básicas Java}
\textbf{Definir un método}: Arrancar el auto
 \footnotesize{
\lstinputlisting{./codigosEx/Autovm.java}}
\end{frame}

\subsection{Fundamentos}
\begin{frame}{Rappels !}
\begin{itemize}
\item {Los archivos deben llamarse igual a la clase definida public con extensión .java}
\item {La indentación no es importante para el compilador Java. Sin embargo la indentación facilita la lectura y comprensión.}
\item {Java es sensible a las mayúsculas y minúsculas por lo cual, a1 es distinto de A1.}
\end{itemize}
\end{frame}

\subsection{Fundamentos}
\begin{frame}{Rappels !}
\ITZ{	
\uncover<1->{\ITT{1}{Por Convención los nombre de : }{\begin{itemize}
\uncover<1->
\item {Las clases comienzan con una letra mayúscula : {\textit{ClasePrueba}}}
\item {Los métodos comienzan con minúscula : {\textit{abrirPuerta}} }
\item {Las variables empiezan por una letra minúscula : {\textit{nombreDato}} }
\item {Si el nombres de la clase, metodo o varible esta compuesto por mas de una palabra, las palabras se ponen juntas y la inicial de la segunda, tercera....palabra se escribe con mayúsculas }
\item {Las constantes se escriben en mayúsculas. Si tiene varias palabras se separan con $"_"$ : {\textit{MUN\_MAX}} }
\end {itemize}}}}
\end{frame}

\subsection{Fundamentos}
\begin{frame}{Definiciones básicas Java}
\textbf{Definir un nuevo método}: Mostrar atributos
 \scriptsize{
\lstinputlisting{./codigosEx/Autovm2.java}}
\end{frame}

\subsection{Fundamentos}
\begin{frame}{Definiciones básicas Java}
Para hacer algo con la clase \textit{Auto} se deberá crear una aplicación que la utilice o agregarle un método \textit{main}. Todas las aplicaciones deben tener el método main()
\footnotesize{
\lstinputlisting{./codigosEx/Automain.java}}
\tiny  \textit{static}: atributo de la clase no del objeto. Almacena el mismo valor para todos los objetos de la clase. No es necesario crear un objeto para acceder al atributo. No se puede llamar a métodos no estáticos desde la misma clase (se necesita un objeto)

\end{frame}

\subsection{Fundamentos}
\begin{frame}{Salida del programa}
\GFXH{2cm}{./pics/salidaAuto.pdf}
\end{frame}


\subsection{Fundamentos}
\begin{frame}{Explicación del código}
\ITZ{	
\uncover<1->{\ITT{1}{}{\begin{itemize}
\uncover<1->
\item {La linea : \textit{Auto a = new Auto();} crea una nueva instancia de la clase Auto y guarda una referencia de ella en a. Por lo general, no se opera directamente con las clases sino que se crean objetos y luego se llama a los metodos en esos objetos.}
\item {Las lineas : \textit{a.marca = "Toyota Yaris";} y \textit{a.color="rojo";} son las variables de instancia para el objeto autot la crea una nueva instancia de la clase Autot y guarda una referencia de ella en a. }
\item {Las lineas : \textit{a.showAtr();} y \textit{a.startMotor();} llama a los metodos \textit{showAtri}  y \textit{statrMotot} desde los objetos.
}
\end{itemize}
}}}
\end{frame}

\subsection{Fundamentos}
\begin{frame}{Explicación del código}
\ITZ{	
\uncover<1->{\ITT{1}{}{\begin{itemize}
\uncover<1->
\item {La linea : \textit{Auto a = new Auto();} crea una nueva instancia de la clase Auto y guarda una referencia de ella en a. Por lo general, no se opera directamente con las clases sino que se crean objetos y luego se llama a los metodos en esos objetos.}
\item {Las lineas : \textit{a.marca = "Toyota Yaris";} y \textit{a.color="rojo";} son las variables de instancia para el objeto autot la crea una nueva instancia de la clase Autot y guarda una referencia de ella en a. }
\item {Las lineas : \textit{a.showAtr();} y \textit{a.startMotor();} llama a los metodos \textit{showAtri}  y \textit{statrMotot} desde los objetos.
}
\end{itemize}
}}}
\end{frame}

\subsection{Fundamentos}
\begin{frame}{Tipos de variables}
\ITZ{	
\uncover<1->{\ITT{1}{Tipos enteros}{
\uncover<1->{
\begin{itemize}
\item {byte : 8 bits $: -128 a 127$}
\item {short : 8 bits  $: -32,768 a 32,767$}
\item {int : 32 bits $: -2,147,483,648 a 2,147,483,647$ (en valores decimales : float)}
\item {long : 64 bits $: -9223372036854775808 a 9223372036854775808$ (en valores decimales : double)}
\end{itemize}
Ademas de los tipo enteros y decimales, Java tiene {\textit{char}} para caracteres individuales (16 bits) y boolean para true y false.
}}}}
\end{frame}

\subsection{Fundamentos}
\begin{frame}{Variables}
\ITZ{	
\uncover<1->{\ITT{1}{Asignación de variables}{
\uncover<1->{
\lstinputlisting{./codigosEx/variables.java}}}}
\uncover<2->{\ITT{2}{Expresiones y operadores}{\begin{itemize}
\uncover<2->{
\item {Expresiones : enunciados que regresan un valor.}
\item {Operadores : símbolos que se utilizan en expresiones.}
\end{itemize}
}}}}
\end{frame}

\subsection{Fundamentos}
\begin{frame}{Operadores}
\ITZ{	
\uncover<1->{\ITT{1}{Aritméticos}{\begin{itemize}
\uncover<1->{
\item {$+$ : Suma}
\item {$ - $: Resta}
\item {$*$ : Multiplicación}
\item {$/$ : División}
\item {$\%$ : Modulo}
\end{itemize}
}}}}
\end{frame}


\subsection{Fundamentos}
\begin{frame}{Preguntas y ejercicios del curso}
\ITZ{	
\uncover<1->{\ITT{1}{}{\begin{itemize}
\uncover<1->
\item {Java es un lenguaje compilado o interpretado?}
\item {Explique las etapas de producción de un programa, precisando los comandos necesarios}
\item {Escriba un programa que utilice 3 variables x,y,z de tipo int y 3 variables a,b,c de tipo float. Efectue los siguientes cálculos : 
\begin{enumerate}
\item {$z=x+y$}
\item {$z=x-y$}
\item {$z=x*y$}
\item {$z=x/y$}
\item {$c=a-b$}
\item {$c=a/b$}
\end{enumerate}
}
\item Prueba el programa con diferentes valores de a,b,d,e.
\item Prueba con valores $a<b$, que observa en la division?
\end{itemize}
}}}
\end{frame}

\subsection{Fundamentos}
\begin{frame}{Variables}
\ITZ{	
\uncover<1->{\ITT{1}{Asignación de variables}{
\uncover<1->{
\tiny {\lstinputlisting{./codigosEx/AritTest.java}}}
{\textit{$+$ concatena la cadena de caracteres en el System.out.println}}}}}
\end{frame}


\subsection{Fundamentos}
\begin{frame}{Operadores}
\ITZ{	
\uncover<1->{\ITT{1}{Asignación}{\begin{itemize}
\uncover<1->{
\item {$x+=y$ : $x=x+y$}
\item {$x-=y $: $x=x-y$}
\item {$x*=y$ : $x=x*y$}
\item {$x/=y$ : $x=x/y$}
\item {$x\%$ : $x=x\%y$}
\end{itemize}
}}}}
\end{frame}

\subsection{Fundamentos}
\begin{frame}{Incrementos y decrementos}
\ITZ{	
\uncover<1->{\ITT{1}{Asignación}{
\uncover<1->{
{ \footnotesize{\begin{itemize}
\item {$y=x++$}
\item {$y=++x$}
\end{itemize}}
Cual es la diferencia?}
\tiny {\lstinputlisting{./codigosEx/IncreTest.java}
}}}}}
\end{frame}

\subsection{Fundamentos}
\begin{frame}{Operadores}
\ITZ{	
\uncover<1->{\ITT{1}{Comparación}{
\uncover<1->{
\begin{itemize}
\item {$==$ : igual}
\item {$ != $: diferente}
\item {$ < $: menor que }
\item {$ <= $: menor o igual que}
\item {$ > $: mayor que }
\item {$ >= $: mayor o igual que}
\end{itemize}
}}}}
\end{frame}

\subsection{Fundamentos}
\begin{frame}{Operadores}
\ITZ{	
\uncover<1->{\ITT{1}{Lógicos}{
\uncover<1->{
\begin{itemize}
\item {$ |$ o $||$ : OR.  Usando solo un operador $|$ se evaluarán siempre ambas expresiones, en cambio, usando los dos $||$ si el lado izquierdo de la expresion es verdadero la expresion será verdadera, nunca evaluará el lado derecho.}
\item {$ \&$ o $ \&\& $: AND. Usando solo un operador $ \& $ se evaluarán siempre ambas expresiones, en cambio,  usando los dos $\&\& $ si el lado izquierdo de la expresion es falso la expresion será falsa, nunca evaluará el lado derecho.}
\item {$\wedge$: XOR }
\item {$ ! $: NOT }
\end{itemize}
}}}}
\end{frame}


%\section{Objetos}
%\subsection{Objetos}
%\begin{frame}{Plan de la sección Objetos}
%\ITZ{	
%\uncover<1->{\ITT{1}{}{\begin{itemize}
%\uncover<1->
%\item {Objetos}
%{\footnotesize{\begin {itemize}
%\item {Crear nuevos Objetos}
%\item {Llamar a Metodos}
%\item {Biblioteca de clase Java}
%\item {Ejercicio de Objetos}
%\end {itemize}}}
%\end{itemize}}}}
%\end{frame}




%\graphicspath{{./pics/}}

\section{Objetos}
\subsection{Objetos}
\begin{frame}{Plan de la sección Objetos}
\ITZ{	
\uncover<1->{\ITT{1}{}{\begin{itemize}
\uncover<1->
\item {Objetos}
{\footnotesize{\begin {itemize}
\item {Crear nuevos Objetos}
\item {Llamar a Metodos}
\item {Biblioteca de clase Java}
\item {Ejercicio de Objetos}
\end {itemize}}}
\end{itemize}}}}
\end{frame}

\subsection{Objetos}
\begin{frame}{Crear objetos}
\ITZ{	
\uncover<1->{\ITT{1}{}{
\uncover<1->{Para crear instancias de clases se debe usar el operador {\textit{new}} seguido de parentesis. Crea una nueva instancia de la clase y se le asigna memoria. Para esto se llama al metodo constructor de la clase.
\begin {itemize}
\item {String str = new String()}
\item {Random r = new Random ()}
\item {Auto a = new Auto()}
\end {itemize}
}}}}
\end{frame}

\subsection{Objetos}
\begin{frame}{Acceso a las variables de instancia y de clase}
\ITZ{	
\uncover<1->{\ITT{1}{Rappel!}{
\uncover<1->
{Las variables de instancia y de clase se comportan de la misma forma que las locales pero para acceder a ellas se debe usar un punto. Así al lado izquierdo de la notación queda el objeto y al lado derecho la variable. 
\begin {itemize}
\item {myVar.val;}
\item {a.marca;}
\end {itemize}
}}}
\uncover<2->{\ITT{2}{Rappel 2!}{
\uncover<2->
{Para cambiar el valor a una variable : 
\begin {itemize}
\item {myVar.val = 80;}
\item {a.marca = "Ford";}
\end {itemize}
}}}}
\end{frame}

\subsection{Objetos}
\begin{frame}{Ejercicios Iniciales}
\ITZ{	
\uncover<1->{\ITT{1}{}{
\uncover<1->{

\begin {itemize}
 \item {Escriba una programa que nos muestre la fecha de hoy. Para eso use la clase Date del paquete java.util (\textit{import java.util.Date;})}
\item {Escriba un programa que declare 3 notas y calcule el promedio de estas.}
\item {Escriba una programa que calcule el area de un rectángulo, declarando 2 variables enteras largo y ancho. Si uno de los dos lados es inferior o igual a 0 el programa deberá mostrar un mensaje de "error".}
\item {Escriba una programa que compare dos enteros a y b y señale si a es menor, mayor o igual a b}
\end {itemize}
}}}}
\end{frame}

\subsection{Objetos}
\begin{frame}{Ejercicio resueltos}
\ITZ{	
\uncover<1->{\ITT{1}{}{
\uncover<1->{
\lstinputlisting{./codigosEx/Fecha.java}
}}}}
\end{frame}
\subsection{Objetos}
\begin{frame}{Ejercicios resueltos}
\ITZ{	
\uncover<1->{\ITT{1}{}{
\uncover<1->{
\lstinputlisting{./codigosEx/Notas2.java}
}}}}
\end{frame}

\subsection{Objetos}
\begin{frame}{Ejercicios resueltos}
\ITZ{	
\uncover<1->{\ITT{1}{}{
\uncover<1->{
\tiny{\lstinputlisting{./codigosEx/Notas.java}}
}}}}
\end{frame}

\subsection{Objetos}
\begin{frame}{Ejercicios resueltos}
\ITZ{	
\uncover<1->{\ITT{1}{}{
\uncover<1->{
\tiny{\lstinputlisting{./codigosEx/Area.java}}
}}}}
\end{frame}

\subsection{Objetos}
\begin{frame}{Ejercicios resueltos}
\ITZ{	
\uncover<1->{\ITT{1}{}{
\uncover<1->{
\tiny{\lstinputlisting{./codigosEx/Comparar.java}}
}}}}
\end{frame}


\subsection{Objetos}
\begin{frame}{Llamar a Metodos}
\ITZ{	
\uncover<1->{\ITT{1}{}{
\uncover<1->
{Para llamar a metodos en los objetos es similar al acceso de variables, tambien usa el punto. El objeto que llama al metodo esta en el lado izquierdo y el metodo con sus argumentos al lado derecho} 
\begin {itemize}
\item {myObj.method(arg1,arg2);}
\item {a.mostrarAtr();}
\end {itemize}
}}
\uncover<2->{\ITT{2}{}{
\uncover<2->{\footnotesize{
{Algunos métodos de la clase String: 
\begin {itemize}
\item {length() : nos da el largo de la cadena}
\item {charAt(?) : el carácter que esta en la posición "?"}
\item {substring(??,??) : el substring entre la posición ?? y ??}
\item {indexOf('?') : el indice del caracter '?' o del inicio de la cadena "?????"}
\item {toUpperCase() : la cadena en mayúscula}
\end {itemize}
}}}}}}
\end{frame}

\subsection{Objetos}
\begin{frame}{Ejercicio resuelto String}
\ITZ{	
\uncover<1->{\ITT{1}{}{
\uncover<1->{
\tiny{\lstinputlisting{./codigosEx/TestString.java}}
}}}}
\end{frame}

\subsection{Objetos}
\begin{frame}{Ejercicio resuelto de String}
\ITZ{	
\uncover<1->{\ITT{1}{}{
\uncover<1->{
\tiny{\lstinputlisting{./codigosEx/TestString2.java}}
}}}}
\end{frame}

\subsection{Objetos}
\begin{frame}{Referencias a objetos}
\ITZ{	
\uncover<1->{\ITT{1}{}{
\uncover<1->{
Que pasa con pt2 después de cambiar las variables de instancia de pt1?}
\tiny{\lstinputlisting{./codigosEx/TestReferencias.java}}
}}} \end{frame}

\subsection{Objetos}
\begin{frame}{Referencias a objetos}
\ITZ{	
\uncover<1->{\ITT{1}{}{
\uncover<1->{
Cuando se asigna el valor pt1 a pt2, se crea una referencia de pt2 al mismo objeto al cual se refiere pt1. Si cambia el objeto al que se refiere pt2 tambien modificará el objeto al que apunta pt1, ya que ambos se refieren al mismo objeto. Es decir, se crea un objeto punto que referencia a 2 variables (pt1 y pt2).
}}}}
\uncover<1->{\begin{center}\emph{Referencias a objetos} \end{center}}
\uncover<1->{\GFXH{3cm}{referencias.pdf}}
\end{frame}

\subsection{Objetos}
\begin{frame}{Referencias a objetos}
\ITZ{	
\uncover<1->{\ITT{1}{}{
\uncover<1->{\footnotesize{
Que pasa con pt2 y b después de cambiar las variables pt1 y a?}
\tiny{\lstinputlisting{./codigosEx/TestReferencias1.java}}
{\tiny{Rappel: Solo los tipos primitivos y String se manejan por valor, es decir, al realizar str2=str1 o b=a se genera una referencia a un mismo objeto pero esta referencia se rompe y se generan 2 objetos al cambiar el valor de alguna de las variables (str2="otro punto") o crear una nueva variable (str2=new String().}
}}}}} \end{frame}

\subsection{Objetos}
\begin{frame}{Referencias a objetos}
\ITZ{	
\uncover<1->{\ITT{1}{}{
\uncover<1->{
Que pasa con pt2 después de cambiar las variables de instancia de pt1?}
\tiny{\lstinputlisting{./codigosEx/TestReferencias2.java}}
}}} \end{frame}

\subsection{Objetos}
\begin{frame}{Conversión de objetos y tipos primitivos}
\ITZ{	
\uncover<1->{\ITT{1}{}{
\uncover<1->{
Para cambiar un valor de un tipo a otro se usa el mecanismo de conversión llamado "forzar". El resultado es una nueva referencia o valor por lo cual no afecta al objeto o valor original.
Las reglas de conversion dicen relación con los tipos de datos de java. java posee tipo primitivos (int, float, boolean) y tipos objeto (String, Point, Window), por lo cual hay 3 formas de conversion:
\begin {itemize}
\item {Forzar entre tipos primitivos : de int a float a boolean.}
\item{Forzar entre tipos de objeto : de una instancia de una clase a una instancia de otro clase.}
\item {Convertir tipos primitivos a objetos y después extraer el valor de dichos objetos.}
\end{itemize}
}}}} \end{frame}

\subsection{Objetos}
\begin{frame}{Forzar tipos primitivos}
\ITZ{	
\uncover<1->{\ITT{1}{}{
\uncover<1->{
\begin {itemize}
\item {Los valores booleanos no pueden forzarse a otro tipo.}
\item {Si forzamos a un tipo mas grande que el valor original puede tratarse de manera automática dado que no perderá información. \\ Ejemplo:  int i = 5;\\
double d = i;}
\item {Para convertir de un valor de tipo grande a uno pequeño se debe usar el forzado explicito : 
\begin{itemize}
\item{ (tipo de dato) valor}
\item { (int) x}
\item { (int) (x/y)}
 \end{itemize}
 } 
{\footnotesize{Rappel : la precedencia del forzado es mas alta que la aritmética}}
 \end{itemize}
}}}} \end{frame}


\subsection{Objetos}
\begin{frame}{Forzar Objetos}
\ITZ{	
\uncover<1->{\ITT{1}{}{
\uncover<1->{
\begin {itemize}
\item{Al forzar un objeto el tipo de dato no cambia, sólo cambia la manera en que el compilador va a tratar a dicho objeto.}
\item {Solo se puede forzar instancia de una clase a otra si estas están relacionadas por la herencia. Se puede forzar un objeto solo a cierta distancia de la sub o super clase de la clase a la que pertenece, no a cualquier clase.}
\item {Se pueden usar instancias de la subclase en cualquier superclase (un objeto de la subclase es también un objeto de la superclase por lo cual puede ser tratado como instancia de la superclase).}
\item {Forzar un objeto a una superclase de ese objeto (se perderá la información proporcionada por la subclase) se debe usar un forzado explícito : 
\begin{itemize}
\item {(nombre clase) objeto}
\item {a1=(Moto) a;}
\end{itemize}}
\end{itemize}
}}}}
\end{frame}

\subsection{Objetos}
\begin{frame}{Forzar Objetos}
\ITZ{	
\uncover<1->{\ITT{1}{}{
\uncover<1->{
Habitualmente los métodos son declarados para ser genéricos, posiblemente devolviendo o aceptando un tipo Object. Si se necesita acceder a un parámetro por un tipo especifico puede forzarse.
\tiny{\lstinputlisting{./codigosEx/TestConvertir.java}}
}}}}
\end{frame}

\subsection{Objetos}
\begin{frame}{Forzar Objetos}
\ITZ{	
\uncover<1->{\ITT{1}{}{
\uncover<1->{
Para verificar si una referencia a un objeto es una instancia de cierta clase o de su padre se usa el operador : {\textbf{instanceof}}. Este operador tiene un objeto a la izquierda y el nombre de una clase a derecha. La expresion regresa {\textbf{true}} o {\textbf{false}} dependiendo si el objeto es una instancia de la clase nombrada o de alguna de sus subclases.
\tiny{\lstinputlisting{./codigosEx/TestConvertir2.java}}
}}}}
\end{frame}


\subsection{Objetos}
\begin{frame}{Forzar tipos primitivos a Objetos}
\ITZ{	
\uncover<1->{\ITT{1}{}{
\uncover<1->{
No es posible forzar de tipos de datos primitivos a objetos ni vice versa. Sin embargo el paquete java.lang incluye varias clases especiales que corresponden a tipos de datos primitivos. \\ {\textbf{Integer}} para {\textbf{ints}}, {\textbf{Float}} para {\textbf{floats}}, {\textbf{Boolean}} para {\textbf{boolenos}}, etc.\\
Para utilizar los métodos de estas clases puede crear objetos equivalentes a todos los tipos de datos primitivos mediante el uso de {\textbf{new}}.\\
{\textit{Integer intObjeto = new Integer(23);}}\\
Esto crea una instancia de la clase {\textbf{Integer}} con el valor 23, este valor puede ser tratado como objeto.
}}}}
\end {frame}

\subsection{Objetos}
\begin{frame}{Forzar Objetos a tipo de dato primitivos}
\ITZ{	
\uncover<1->{\ITT{1}{}{
\uncover<1->{
Existen metodos para regresar los valores a tipos primitivos. Por ejemplo : {\textbf{intValues()}}, que extrae un tipo primitivo int de un objeto Integer.\\
{\textit{int elEntero =  intObjeto.intValues();}}
}}}}
\end {frame}

\subsection{Objetos}
\begin{frame}{Ejemplo resumen : conversión de tipo de datos (objeto y tipos primitivos)}
\ITZ{	
\uncover<1->{\ITT{1}{}{
\uncover<1->{
\tiny{\lstinputlisting{./codigosEx/TestConvertir3.java}}
}}}}
\end{frame}

\subsection{Objetos}
\begin{frame}{Comparar Objetos}
\ITZ{	
\uncover<1->{\ITT{1}{}{
\uncover<1->{
La mayoría de los operadores funciona solo en tipos primitivos no en objetos excepto los operadores de igualdad {\textbf{==}} y {\textbf{!=}}. \\
Estos operadores prueban si los dos operandos se refieren al mismo objeto, es decir, los operadores de igualdad no evaluarán si los valores de dos objetos son iguales sino mas bien, si referencian al mismo objeto.
\tiny{\lstinputlisting{./codigosEx/TestIgual.java}}
}}}}
\end{frame}


\subsection{Objetos}
\begin{frame}{}
\ITZ{	
\uncover<1->{\ITT{1}{Determinar la clase de un objeto}{
\uncover<1->{{\footnotesize{
{\textit{String nombre = obj.getClass().getName();}}\\
El método {\textit{getClass()}} da como resultado un objeto {\textit{Class}} que posee un método llamado {\textit{getName()}} que regresa la cadena de caracteres con el nombre de la clase.}}}}}
\uncover<2->{\ITT{2}{Algunos paquetes de Java}{
\uncover<2->{\footnotesize{
Cada biblioteca de java ofrece un conjunto de clases disponible.
{\begin{itemize}
\item {{\textit{java.lang :}} clases que se aplican al lenguaje mismo entre ellas la clase {\textit{Object, System,String}}. También contiene las clases especiales para tipos primitivos  {\textit {Integer, Float, Character}}, etc.}
\item {{\textit{java.util :}} clases utilitarias como {\textit{Date}} y clases de colección de datos como {\textit{Vector}}.} 
\item {{\textit{java.io: }} clases de entrada y salida, para escribir y leer flujos de datos y manejar archivos.} 
\item {{\textit{java.net : }} clases para soporte de red, clases como {\textit{URL}}.}
\item {{\textit{java.awt : }} clases para trabajar con una interfaz gráfica de usuario y procesar imagenes, incluye las clases {\textit{Window,Menu, Button,Font, Image,}} entre otras.}
\end{itemize}}}}}}}
\end{frame}

%\section{Arreglos, condiciones y ciclos}
\subsection{Arreglos, condicionales y ciclos}
\begin{frame}{Plan Arreglos, condiciones y ciclos}
\ITZ{	
\uncover<1->{\ITT{1}{}{\begin{itemize}
\uncover<1->
\item {Arreglos}
{\footnotesize{\begin {itemize}
\item {Declarar variables de arreglo}
\item {Crear objetos de arreglo}
\item {Acceder a los elementos de un arreglo}
\item {Cambiar elementos de un arreglo}
\end{itemize}}}
\item {Arreglos multidimensionales}
\item{Condicionales}
{\footnotesize{\begin {itemize}
\item{if}
\item{switch}
\end{itemize}}}
\item {Ciclos}
{\footnotesize{\begin {itemize}
\item{for}
\item{while y do}
\end{itemize}}}
{\item Ejercicio de Arreglos, condicionales y ciclos}
\end{itemize}}}}
\end{frame}

\subsection{Arreglos, condicionales y ciclos}
\begin{frame}{Arreglos}
\ITZ{	
\uncover<1->{\ITT{1}{}{
\uncover<1->
{Los arreglos son una forma de almacenar una lista de elementos, cada espacio del arreglo guarda un elemento individual. Los arreglos pueden tener cualquier tipo de valor (tipos primitivos u objetos) pero no puede almacenar distintos tipos en un mismo arreglo.\\
Para crear un arreglo : 
\begin{itemize}
\item {Declarar una variable para guardar el arreglo.}
\item {Crear un nuevo objeto de arreglo y asígnelo a la variable de arreglo.}
\item {Guardar los valores en el arreglo.}
\end {itemize}}}}}
\end{frame}

\subsection{Arreglos, condicionales y ciclos}
\begin{frame}{Arreglos}
\ITZ{	
\uncover<1->{\ITT{1}{Declarar variables de arreglo}{
\uncover<1->
{Las variables de arreglo indican el tipo de objeto que el arreglo contendrá y el nombre del arreglo. Los corchetes vacíos pueden ponerse después del tipo de dato o después del nombre del arreglo indistintamente.
\begin{itemize}
\item {String palabras[];}
\item {Point hist[];}
\item {int temps[];}
\item {float [] promedio;}
\item {String [] palabras;}
\item {Point [] hist;}
\end {itemize}}}}}
\end{frame}


\subsection{Arreglos, condicionales y ciclos}
\begin{frame}{Arreglos}
\ITZ{	
\uncover<1->{\ITT{1}{Crear objetos de arreglo}{
\uncover<1->{
\begin{itemize}
\item {Usar new :   {\textit{String [] nombre = new String[10]; }}\\ Esta linea crea un arreglo de Strings con 10 espacios. Siempre que se crea un arreglo con new se debe indicar de que tamaño será (numero de espacios o casillas). \\
El arreglo se inicializará con : 
\begin {itemize}
\item { 0 para arreglos numéricos}
\item {false para booleanos}
\item { $'/0'$  para arreglos de caracter}
\item {null para objetos}
\end{itemize}
}
\item {Inicializar de manera directa el contenido del arreglo : poniendo entre llaves los elementos del arreglo : \\
 {\textit{String [] nombres =  \{"pedro", "rodrigo","carlo","andres"\} }}\\
Cada elemento dentro de las llaves debe ser del mismo tipo y coincidir con el tipo de variable que contiene el arreglo.}
\end {itemize}}}}}
\end{frame}

\subsection{Arreglos, condicionales y ciclos}
\begin{frame}{Arreglos}
\ITZ{	
\uncover<1->{\ITT{1}{Acceso a los elementos del arreglo}{
\uncover<1->{
Para acceder a un elemento del arreglo se usan los subindices:\\
 {\textit{String [] arr = new String [10]; }}\\
 {\textit{arr [9] = "test"; }}\\
 {\textit{int len = arr.length; }} \\
 La ultima linea de código permite ver la longitud del arreglo, este atributo esta disponible para todos los objetos arreglo sin importar el tipo.\\
En java no puede asignar un valor a una casilla del arreglo fuera de las fronteras de este. Los subindices se inician en 0. }}}}
\end{frame}

\subsection{Arreglos, condicionales y ciclos}
\begin{frame}{Arreglos}
\ITZ{	
\uncover<1->{\ITT{1}{Cambiar elementos del arreglo}{
\uncover<1->{
Para asignar elementos a una casilla del arreglo :\\
 {\textit{ arr [1]=10; }}\\
 {\textit{promedio [9] = 9.7; }}\\
 {\textit{texto [4] = "test"; }} \\
 {\textit{cadenas [10] = cadenas[1]; }} \\
Al igual que con los objetos, un arreglo de objetos en java consiste en un arreglo de referencias a dichos objetos. Cuando asigna un valor a una casilla en un arreglo, crea un referencia a ese objeto. Cuando desplaza valores dentro de un arreglo solo se reasigna la referencia, {\textbf{no }} se copia el valor de una casilla a otra. En cambio los arreglos de tipos primitivos {\textbf{si}} copian los valores de una casilla a otra.
}}}}\end{frame}


\subsection{Arreglos, condicionales y ciclos}
\begin{frame}{Arreglos}
\ITZ{	
\uncover<1->{\ITT{1}{Arreglos Multidimensionales}{
\uncover<1->{
Java no soporta los arreglos multidimensionales. Sin embargo, se pueden declarar un arreglo de arreglos y acceder a el de la siguiente manera : \\
 {\textit{ int coods [][]= new int [12][12]; }}\\
 {\textit{ coords [0][0]=2;}}\\
}}}}\end{frame}

\subsection{Arreglos, condicionales y ciclos}
\begin{frame}{Condicionales} 
\ITZ{	
\uncover<1->{\ITT{1}{if}{
\uncover<1->{
El condicional  {\textbf{ if }} permite ejecutar partes del código basandose en una simple prueba. Contiene la palabra clave {\textbf{ if }} seguida de una prueba booleana y de un enunciado a ejecutar si la prueba es verdadera. Una palabra opcional {\textbf{ else}} ofrece el enunciado a ejecutarse si la prueba es falsa.
\tiny{\lstinputlisting{./codigosEx/estado.java}}
}}}}\end{frame}

\subsection{Arreglos, condicionales y ciclos}
\begin{frame}{Operador condicional}
\ITZ{	
\uncover<1->{\ITT{1}{?}{
\uncover<1->{
El operador condicional  {\textbf{ ? }} es un operador ternario solo tiene tres términos.  \\
Es útil para condicionales cortos o sencillos.\\
 {\textit{ test ? trueResult : falseResult; }}\\
 {\textit{ int smaller = $x<y$ ? x:y;}}\\
 }}}}\end{frame}

\subsection{Arreglos, condicionales y ciclos}
\begin{frame}{Condicional switch}
\ITZ{	
\uncover<1->{\ITT{1}{}{
\uncover<1->{
Para evitar {\textbf{if}} muy largos se usa el {\textbf{switch o case}} que permite agrupar pruebas y acciones en un solo enunciado.
\tiny{\lstinputlisting{./codigosEx/switch.java}}
{\footnotesize{Si no existen coincidencias en ninguno de los casos y el {\textit{default}} no existe el {\textit{switch}} se completa sin hacer nada.}}
}}}}\end{frame}

\subsection{Arreglos, condicionales y ciclos}
\begin{frame}{Condicional switch}
\ITZ{	
\uncover<1->{\ITT{1}{Limitaciones y break}{
\uncover<1->{
Una limitación del condicional {\textbf{switch}} es que las pruebas y los valores solo pueden ser de tipos primitivos específicamente {\textbf{int}}. \\ No puede usar tipos primitivos mas grandes ({\textbf{long,float}}), cadenas u objetos, tampoco puede probar otra relación que la igualdad.\\
Si se encuentra una coincidencia se ejecuta el enunciado de esta y todos los inferiores hasta encontrar un {\textbf{break}} o hasta el final del {\textbf{switch}}. Si solo se quiere ejecutar el enunciado de la prueba se debe poner un {\textbf{break}} después de cada linea.
\tiny{\lstinputlisting{./codigosEx/switch2.java}}
}}}}\end{frame}

\subsection{Arreglos, condicionales y ciclos}
\begin{frame}{Ciclos for}
\ITZ{	
\uncover<1->{\ITT{1}{for}{
\uncover<1->{
El ciclo {\textbf{for}} repite una declaración o bloque de enunciados un numero de veces hasta que una condición se cumple. \\
 {\textit{ for (inicialización; test; incremento)\{\\
 enunciados \}}}\\
 \begin{itemize}
 \item {inicialización : inicializa el principio del ciclo (int i = 0). Las variables que se declaran en esta parte del ciclo son locales al ciclo.}
 \item {test :  prueba que ocurre después de cada vuelta del ciclo, esta debe ser una expresión booleana o una función que regresa un valor booleano (i $<$10). Si la prueba es verdadera el ciclo se ejecuta sino detiene su ejecución. }
 \item {incremento :  expresión o llamada a función, por lo general se usa para cambiar el valor del indice del ciclo.} 
 \end{itemize} 
}}}}\end{frame}


\subsection{Arreglos, condicionales y ciclos}
\begin{frame}{Ejercicio} 
\ITZ{	
\uncover<1->{\ITT{1}{}{
\uncover<1->{
 \begin{itemize}
 \item {Escriba un programa que construya un arreglo de tamaño 10 y lo llene con números enteros entre 0 y 100 generados aleatoreamente.} \item {Recorra el arreglo y sume su contenido. Imprima cada valor y la suma total en pantalla. Para generar números aleatorios use el metodo random de la clase Math que esta en el paquete java.lang.Math}
 \end{itemize} 
}}}}\end{frame}


\subsection{Arreglos, condicionales y ciclos}
\begin{frame}{Ejercicio resuelto} 
\ITZ{	
\uncover<1->{\ITT{1}{}{
\uncover<1->{
\tiny{\lstinputlisting{./codigosEx/EjArregloSum.java}}
}}}}\end{frame}

\subsection{Arreglos, condicionales y ciclos}
\begin{frame}{Ejemplo discutido en clases} 
\ITZ{	
\uncover<1->{\ITT{1}{}{
\uncover<1->{
\tiny{\lstinputlisting{./codigosEx/EjemploCaracter.java}}
}}}}\end{frame}

\subsection{Arreglos, condicionales y ciclos}
\begin{frame}{Ejercicios} 
\ITZ{	
\uncover<1->{\ITT{1}{Mostrar las notas de un alumno}{\footnotesize{\begin {itemize}
\uncover<1->{
\item {Cree una clase "estudiante", declare un arreglo de notas de tipo int y una constante con la cantidad de notas totales del estudiante ( {\textit{ static final int NB\_NOTAS = 10; 
}})}
\item {Cree un método "mostrar" que imprima en pantalla las notas de un alumno.}
\item {Cree un método "llenar" que llene el arreglo con notas aleatorias entre 0 y 10.}
\item {Cree un método "promedio" que calcule el promedio de notas del alumno.}
\item {En el método principal : Cree un objeto del tipo "estudiante", cree el arreglo declarado como atributo de la clase y (usando el objeto recién creado) llame a los métodos llenar, mostrar y promedio.}
\end{itemize}
}}}}}\end{frame}


\subsection{Arreglos, condicionales y ciclos}
\begin{frame}{Ejercicio resuelto} 
\ITZ{	
\uncover<1->{\ITT{1}{}{
\uncover<1->{
\tiny{\lstinputlisting{./codigosEx/Estudiante0.java}}
}}}}\end{frame}

\subsection{Arreglos, condicionales y ciclos}
\begin{frame}{Ejercicio resuelto continuación} 
\ITZ{	
\uncover<1->{\ITT{1}{}{
\uncover<1->{
\tiny{\lstinputlisting{./codigosEx/Estudiante1.java}}
}}}}\end{frame}

\subsection{Arreglos, condicionales y ciclos}
\begin{frame}{Lectura argumentos} 
\ITZ{	
\uncover<1->{\ITT{1}{Usando arreglos}{
\uncover<1->{
Como podemos leer una lista de argumentos ingresados al momento de ejecutar el código? }
\footnotesize{\lstinputlisting{./codigosEx/LecturaArgumentos.java}}
}}}\end{frame}

\subsection{Arreglos, condicionales y ciclos}
\begin{frame}{Lectura argumentos} 
\ITZ{	
\uncover<1->{\ITT{1}{Usando arreglos}{\begin{itemize}
\uncover<1->{
\item {Desarrolle un programa que lea un conjunto de notas ingresadas como argumentos al ejecutar el programa. Use el método parseInt de la clase Integer para pasar de carácter a enteros y parseFloat de la clase Float para flotantes.}
\item {Cree un método que ingrese las notas a un arreglo.}
\item {Cree un método que calcule el promedio de notas y lo imprima en pantalla.}
\end{itemize}
}}}}\end{frame}

\subsection{Arreglos, condicionales y ciclos}
\begin{frame}{Ejercicio resuelto} 
\ITZ{	
\uncover<1->{\ITT{1}{}{
\uncover<1->{
\tiny{\lstinputlisting{./codigosEx/AdjuntarNotas.java}}
}}}}\end{frame}

\subsection{Arreglos, condicionales y ciclos}
\begin{frame}{Lectura de datos} 
\ITZ{	
\uncover<1->{\ITT{1}{Usando la clase Scanner}{
\uncover<1->{Como podemos leer datos desde el teclado?}
\tiny{\lstinputlisting{./codigosEx/LecturaScanner.java}}
}}}\end{frame}

\subsection{Arreglos, condicionales y ciclos}
\begin{frame}{Lectura de datos} 
\ITZ{	
\uncover<1->{\ITT{1}{Usando la clase Scanner}{
\uncover<1->{Para leer datos float, long, short, double, String, debemos usar : . \begin{itemize}
\item {nextFloat()}
\item {nextShort()}
\item {nextDouble()}
\item {next() y nextLine()}}
\end{itemize}
}}}\end{frame}


%La clase Scanner permite crear objetos capaces de leer información desde una fuente de datos que puede ser un archivo, una cadena de caracteres, el teclado, etc. Los objetos de esta clase, serán los que utilizaremos para pedir los datos que se requieran para dar solución a un problema.

\subsection{Arreglos, condicionales y ciclos}
\begin{frame}{Ejercicios } 
\ITZ{	
\uncover<1->{\ITT{1}{Trabajar con arreglos y metodos}{\begin {itemize}
\uncover<1->{
\item {Cree una clase "EjArreglos"}
\item {Construya 4 arreglos de tamaño N=10 y uno de tamaño definido por el usuario.}
\item {Elija si va a trabajar con metodos static o creara objetos.}
\item {Metodos para llenar cada arreglo: 

\begin {enumerate}
\item {Arr1 : con elementos enteros en orden creciente entre $[$0, n$[$. }
\item {Arr2 : con elementos enteros en orden creciente a partir de 5.}
\item {Arr3 : con elementos enteros en orden decreciente de $[$n,1$]$.}
\item {Arr4 : con elementos enteros generados aleatoriamente entre $[$0,10$]$.}
\item {Arr5 : con elementos enteros ingresados por el usuario como argumentos o por teclado.}
\end{enumerate}}
\end{itemize}
}}}}\end{frame}

\subsection{Arreglos, condicionales y ciclos}
\begin{frame}{Ejercicios continuación} 
\ITZ{	
\uncover<1->{\ITT{1}{Trabajar con arreglos y metodos}{\footnotesize{\begin {itemize}
\uncover<1->{
\item {Métodos a aplicar a cada arreglo : 
\begin {enumerate}
\item {crear método "mostrar" que imprima en pantalla los elementos del arreglo en pantalla.}
\item {crear método "mostrarMayorMenor" que el numero de elementos mayor que 4 o menor que 2.}
\item {crear método "mostrarNueve" que calcule e imprima en pantalla el numero de elementos de valor 9 del arreglo. }
\item {crear método "mostrarSuma" que calcule e imprima la suma de todos los elementos del arreglo.}
\item {crear método "mostrarMayor" que imprima el mayor valor del arreglo.}
\item {crear método "mostrarPromedio" que calcule e que imprima el promedio de los elementos del arreglo.}
\item {Genere un método llamado "operaciones" que llame a los 6 métodos auxiliares creados. entregando el arreglo como parametro.}
\end{enumerate}}
\end{itemize}
}}}}}\end{frame}

\subsection{Arreglos, condicionales y ciclos}
\begin{frame}{Ejercicio resuelto} 
\ITZ{	
\uncover<1->{\ITT{1}{}{
\uncover<1->{
\tiny{\lstinputlisting{./codigosEx/EjArreglos0.java}}
}}}}\end{frame}

\subsection{Arreglos, condicionales y ciclos}
\begin{frame}{Ejercicio resuelto continuación} 
\ITZ{	
\uncover<1->{\ITT{1}{}{
\uncover<1->{
\tiny{\lstinputlisting{./codigosEx/EjArreglos1.java}}
}}}}\end{frame}

\subsection{Arreglos, condicionales y ciclos}
\begin{frame}{Ejercicio resuelto continuación} 
\ITZ{	
\uncover<1->{\ITT{1}{}{
\uncover<1->{
\tiny{\lstinputlisting{./codigosEx/EjArreglos2.java}}
}}}}\end{frame}

\subsection{Arreglos, condicionales y ciclos}
\begin{frame}{Ejercicio resuelto continuación} 
\ITZ{	
\uncover<1->{\ITT{1}{}{
\uncover<1->{
\tiny{\lstinputlisting{./codigosEx/EjArreglos3.java}}
}}}}\end{frame}

\subsection{Arreglos, condicionales y ciclos}
\begin{frame}{Ejercicio resuelto continuación} 
\ITZ{	
\uncover<1->{\ITT{1}{}{
\uncover<1->{
\tiny{\lstinputlisting{./codigosEx/EjArreglos4.java}}
}}}}\end{frame}

\subsection{Arreglos, condicionales y ciclos}
\begin{frame}{Ciclos}
\ITZ{	
\uncover<1->{\ITT{1}{While y do}{
\uncover<1->{
El ciclo {\textbf{while y do}} permiten ejecutar un bloque de código de manera repetida hasta encontrar una condición especifica.
\begin {itemize}
\item  {Se ejecuta el bloque de código hasta que la condición sea verdadera. Si la condición es falsa el ciclo nunca se ejecutara.{\tiny{\lstinputlisting{./codigosEx/while.java}}}
} 
\item{ Este ciclo prueba la condición después de haberse ejecutado una vez. {\tiny{\lstinputlisting{./codigosEx/dowhile.java}}}

}
 \end{itemize} 
}}}}\end{frame}

\subsection{Arreglos, condicionales y ciclos}
\begin{frame}{Ciclos}
\ITZ{	
\uncover<1->{\ITT{1}{Como salir de los ciclos?}{
\uncover<1->{
Todos los ciclos se terminan cuando la condición se cumple, si desea salir de manera anticipada  y detener por completo el ciclo actual debe usar \textbf{break}.
{\tiny{\lstinputlisting{./codigosEx/break1.java}}}
Inserte este código dentro de una clase y pruebe el resutado.
{\tiny{\lstinputlisting{./codigosEx/break.java}}}
}}}}\end{frame}

\subsection{Arreglos, condicionales y ciclos}
\begin{frame}{Ciclos}
\ITZ{	
\uncover<1->{\ITT{1}{Como salir de los ciclos?}{
\uncover<1->{
Todos los ciclos se terminan cuando la condición se cumple, si desea salir de manera anticipada  e iniciar la siguiente iteración (reiniciar el ciclo) debe usar \textbf{continue}. Para los ciclos \textbf{ while} y  \textbf{do} significa que la ejecución del bloque se inicia de nuevo. Para el ciclo  \textbf{for} el incremento se evalúa y después el bloque se ejecuta. 
{\tiny{\lstinputlisting{./codigosEx/continue1.java}}}
Inserte este código dentro de una clase y pruebe el resultado.
{\tiny{\lstinputlisting{./codigosEx/continue.java}}}
}}}}\end{frame}

\subsection{Arreglos, condicionales y ciclos}
\begin{frame}{Ciclos}
\ITZ{	
\uncover<1->{\ITT{1}{Como salir de los ciclos?}{
\uncover<1->{
Se pueden usar etiquetas para salir de ciclos aninadados o salir de más de un ciclo al mismo tiempo.
{\tiny{\lstinputlisting{./codigosEx/breaketiqueta.java}}}
}}}}\end{frame}


\subsection{Arreglos, condicionales y ciclos}
\begin{frame}{Ejercicios lectura, condicionales y ciclos}
\ITZ{	
\uncover<1->{\ITT{1}{}{
\uncover<1->{
\begin {itemize}
\item{ Escriba un programa que calcule la raíz cuadrada de valores ingresados por el usuario. Pregunte al usuario cuantos valores ingresará. Solo debe considerar valores positivos.}
\item {Escriba un programa de facturación con descuento. El usuario debe ingresar el precio sin impuestos, y el programa calculará los impuestos (fijo a 19\%) y hará el descuento si corresponde. 
\begin {itemize}
\item {0\% si el monto es inferior a 1000}
\item {1\% si el monto es superior o igual a 1000 e inferior a 3000}
\item {3\% si el monto es superior o igual a 3000 e inferior a 5000}
\item {5\% si el monto es superior o igual a 5000}
\end{itemize}}
\item {Escriba un programa que calcule el área de un rectángulo.  El usuario debe ingresar los valores de x1,x2,y1,y2.}
\item {Escriba un programa que identifique si un caracter es vocal o consonante. Cree 20 caracteres a partir de números aleatorios. Imprima en pantalla el caracter y su tipo.}
\end{itemize}

}}}}\end{frame}

\subsection{Arreglos, condicionales y ciclos}
\begin{frame}{Ejercicios lectura, condicionales y ciclos}
\ITZ{	
\uncover<1->{\ITT{1}{}{
\uncover<1->{
{\tiny{\lstinputlisting{./codigosEx/Raices.java}}}
}}}}\end{frame}

\subsection{Arreglos, condicionales y ciclos}
\begin{frame}{Ejercicios lectura, condicionales y ciclos}
\ITZ{	
\uncover<1->{\ITT{1}{}{
\uncover<1->{
{\tiny{\lstinputlisting{./codigosEx/Factura.java}}}
}}}}\end{frame}


\subsection{Arreglos, condicionales y ciclos}
\begin{frame}{Ejercicios lectura, condicionales y ciclos}
\ITZ{	
\uncover<1->{\ITT{1}{}{
\uncover<1->{
{\tiny{\lstinputlisting{./codigosEx/Rectangulo.java}}}
}}}}\end{frame}

\subsection{Arreglos, condicionales y ciclos}
\begin{frame}{Ejercicios lectura, condicionales y ciclos}
\ITZ{	
\uncover<1->{\ITT{1}{}{
\uncover<1->{
{\tiny{\lstinputlisting{./codigosEx/Consonante.java}}}
}}}}\end{frame}


%\subsection{Recapitulativo Fechas}
%\begin{frame}{Próximas notas y clases recuperativas: }
%\ITZ{	
%\uncover<1->{\ITT{1}{}{
%\uncover<1->{ 
%\begin{itemize}
%\item {martes 24 abril a las 15h30 - 18h30 : Segundo trabajo en clases con nota.}
%\item {martes 8 mayo a las 15h30 - 18h30 1er Certamen.}
%\end{itemize}}}}
%\uncover<2->{\ITT{2}{}{
%\uncover<2->{ 
%Total de clases a recuperar (nuevo calculo que incluye la semana de recuperativas y la semana mechona): 10 clases.\\
%Se ha recuperado 1 clase en el mes de marzo.\\
%Se recuperaran 4 clases en la fecha de ejercicio y certamen (24 de abril y 8 de mayo).\\
%Quedan por recuperar : 5 clases (que serán fijados entre los días de ejercicio y certamen).}}}
%}
%\end{frame}

%98747251 matias

\graphicspath{{./pics/}}
 
\section{ArrayList}
\subsection{ArrayList}
\begin{frame}{ArrayList}
\ITZ{	
\uncover<1->{\ITT{1}{}{\begin{itemize}
\uncover<1->
\item {Generalidades}
\item {Operaciones usuales}
\item {Ejercicios}
\end{itemize}}}}
\end{frame}

\subsection{ArrayList}
\begin{frame}{Generalidades de ArrayList}
\ITZ{	
\uncover<1->{\ITT{1}{}{
\uncover<1->{
La clase ArrayList ofrece funcionalidades de acceso rápido comparables a las de un arreglo de objetos. Esta clase es mas flexible que los arreglos de objeto, su tamaño (numero de elementos) puede variar durante la ejecución. \\
Para que el acceso directo a los elementos de un rango dado sea posible es necesario que los objetos (o mas bien sus referencias) estén contiguos en la memoria (como para los arreglos).\\ Para usar ArrayList se debe importar el paquete :  \emph{java.util}. 
}}}}
\end{frame}



\subsection{ArrayList}
\begin{frame}{Operadores usuales de ArrayList}
\ITZ{	
\uncover<1->{\ITT{1}{}{
\uncover<1->{
\begin{itemize}
\item {Construcción : Un vector dinamico puede ser construido vacío o a partir de un conjunto de datos. \begin {itemize} \item{\emph{ArrayList v1 = new ArrayList ();}} \item{\emph{ArrayList v2 = new ArrayList(c);}} \end{itemize}}
\item{Agregar un elemento : \begin {itemize} \item {Agregar un elemento al final del vector usando \emph{ add(elem);}} \item{Agregar un elemento en una posición $i$ dada \emph{add(i,elem);}} \end{itemize}}
\item {Suprimir un elemento: \begin {itemize} \item {Suprimir un elemento en la posición $i$ \emph{remove(i);}} \item {Suprimir un rango consecutivo de elementos \emph{removeRange(n,p)}} \item {suprimir todo \emph {removeAll()}}\end{itemize}
El método remove retorna el objeto o rango de objetos eliminados (tipo \emph{Object}). Si no elimina el objeto (por que no lo encuentra) retorna \emph{false}.
}
\end {itemize}}}}}
\end{frame}

\subsection{ArrayList}
\begin{frame}{Operadores usuales de ArrayList}
\ITZ{	
\uncover<1->{\ITT{1}{Acceso o modificación de los elementos}{
\uncover<1->{
\begin {itemize}
\item {Acceder a los elementos usando get(i). Para acceder a todos los elementos del vector se puede usar : \\
\emph{public static void mostrar (ArrayList v) \{ \\
for (int i =0; i$<$v.size(); i++) \\
System.out.println(v.get(i));\}}}
\item {Modificar los elementos usando set(i). Para modificar todos los elementos de un vector se puede usar : \\
\emph{public static void modificar (ArrayList v) \{ \\
for (int i =0; i$<$v.size(); i++) \\
if (condicion) set (i,null);\}}}
\end{itemize}
}}}}
\end{frame}

\subsection{ArrayList}
\begin{frame}{Ejercicios}
\ITZ{	
\uncover<1->{\ITT{1}{}{
\uncover<1->{
Escriba un programa que cree un vector dinamico que contenga 10 objetos de tipo entero. \\ Luego elimine los objetos de la posición 3 y 5. \\Verifique el tamaño del vector al inicio y al final usando el metodo \emph{size()}. \\Modifique los valores de la posición 2 y 6 \\ Imprima los valores iniciales y finales.
}}}}\end{frame}

\subsection{ArrayList}
\begin{frame}{Ejercicios}
\ITZ{	
\uncover<1->{\ITT{1}{}{
\uncover<1->{
{\tiny{\lstinputlisting{./codigosEx/ArrayL.java}}}
}}}}\end{frame}


%\graphicspath{{./pics/}}


\section{Clases, Objetos y Metodos}
\subsection{Clases, Objetos y Metodos}
\begin{frame}{Clases, Objetos y Metodos}
\ITZ{	
\uncover<1->{\ITT{1}{}{\begin{itemize}
\uncover<1->
\item {Clases y objetos}
{\footnotesize{\begin {itemize}
\item {Definir Clases}
\item {Constructores}
\item {Definición de objetos}
\end {itemize}}}
\item {Métodos}
{\footnotesize{\begin {itemize}
\item {Reglas de escritura de métodos}
\item {Atributos y metodos de clase}
\item {Sobrecarga de métodos }
\item {Intercambio de información con métodos}
\item {Recursividad}
\end {itemize}}}
\item {Aplicaciones Java}
{\footnotesize{\begin {itemize}
\item {Multiples Clases}
\item {Encapsulamiento}
\item {Paquetes}
\item {Ejercicio Clases, Objetos y Metodos}
\end {itemize}}}
\end{itemize}}}}
\end{frame}

\subsection{Clases y objetos}
\begin{frame}{Recordatorio : Definir Clases}
\ITZ{	 
\uncover<1->{\ITT{1}{Definición de Clases}{
\uncover<1->
{La definición general de clases que hemos usado : 
\lstinputlisting{./codigosEx/Auto.java}
}}}}
\end{frame}

\subsection{Clases y objetos}
\begin{frame}{Recordatorio : Definir Clases}
\ITZ{	 
\uncover<1->{\ITT{1}{Variables de instancia}{
\uncover<1->
\tiny{{\lstinputlisting{./codigosEx/Autov2.java}}
La variable privada no será accesible fuera de la clase}}}
\uncover<2->{\ITT{1}{Variables de clase}{
\uncover<2->
\tiny{{\lstinputlisting{./codigosEx/Autov2C.java}}
Las variables de clase son globales para todas las instancias de clase. Para definir una variable de clase se usa la palabra clave {\textbf{static}}}}}}
\end{frame}

\subsection{Clases y objetos}
\begin{frame}{Recordatorio : Definir Clases}
\ITZ{	 
\uncover<1->{\ITT{1}{Variables}{
\uncover<1->{
{\tiny{\lstinputlisting{./codigosEx/VaIns.java}}}}}}}
\uncover<1->{\begin{center}\emph{Variables de Instancia} \end{center}}
\uncover<1->{\GFXH{3cm}{referenciasI.pdf}}
\end{frame}

\subsection{Clases y objetos}
\begin{frame}{Recordatorio : Definir Clases}
\ITZ{	 
\uncover<1->{\ITT{1}{Variables}{
\uncover<1->{
{\tiny{\lstinputlisting{./codigosEx/VaClas.java}}}}}}}
\uncover<1->{\begin{center}\emph{Variables de Clase}}
\uncover<1->{\GFXH{3cm}{referenciasC.pdf}} \end{center}
\end{frame}

\subsection{Clases y objetos}
\begin{frame}{Recordatorio : Definir Clases}
\ITZ{	 
\uncover<1->{\ITT{1}{Constantes}{
\uncover<1->
\tiny{{\lstinputlisting{./codigosEx/Const.java}}
Las constantes son útiles para definir valores comunes a todos los métodos de un objeto.}}}}
\end{frame}

\subsection{Clases y objetos}
\begin{frame}{Recordatorio :  Clases, objetos y métodos}
\ITZ{	 
\uncover<1->{\ITT{1}{}{
\uncover<1->
{Cual es la diferencia entre : 
\begin{itemize}
\item {Auto a ;}
\item {int a;}
\end {itemize}
La primera no reserva un espacio en memoria para un objeto del tipo Auto sino que una referencia a un objeto del tipo Auto. El lugar de la memoria sera asignado al realizar {\textit{new Auto();}}. Esto ultimo crea un lugar de para el objeto de tipo Auto y entrega como resultado su referencia. Por ende,  {\textit{a=new Auto();}} crea un objeto del tipo Auto y pone su referencia en a.}}}}
\end{frame}

\subsection{Métodos}
\begin{frame}{Recordatorio : Definir Métodos}
\ITZ{	 
\uncover<1->{\ITT{1}{Métodos}{
\uncover<1->
Si el método es publico será accesible de cualquier programa. Si el método no retorna valor se debe usar {\textbf{void}}. 
{\tiny{\lstinputlisting{./codigosEx/Autovm.java}}}
De la misma manera que tenemos variables de clase y de instancia existen métodos de clase y de instancia. \\Para definir métodos de clase se debe usar la palabra {\textbf{static}} (al igual que en las variables de clase, los métodos de clase están disponibles para cualquier instancia de la clase).}}}
\end{frame}

\subsection{Clases, objetos y métodos}
\begin{frame}{Recordatorio :  Clases, objetos y métodos}
\ITZ{	 
\uncover<1->{\ITT{1}{}{
\uncover<1->{Supongamos que tenemos una clase Punto con dos instancias {\textit{Punto a,b;}}.\\
 Le asignamos los valores (3,5) al punto 1 y (2,0) al punto 2 ({\textit{pt1.x=3;pt1.y=5;}}).}}}
\uncover<1->{\begin{center}\emph{Referencias a objetos} \end{center}}
\uncover<1->{\GFXH{3cm}{referencias2.pdf}}}
\end{frame}

\subsection{Clases, objetos y métodos}
\begin{frame}{Recordatorio :  Clases, objetos y métodos}
Que pasa si hacemos : {\textit{pt1=pt2;}}?. \\
%\begin{center}\emph{Referencias a objetos} \end{center}
\GFXH{3cm}{referencias3.pdf}
Lo que ocurre es que copia en pt1 la referencia al contenido de pt2, así pt1 y pt2 referencian al mismo objeto y no a dos objetos que tienen el mismo valor. En este caso Java con el recolector de basura libera el espacio en memoria del otro objeto.
\end{frame}

\subsection{Clases, objetos y métodos}
\begin{frame}{Paso de argumentos a los métodos}
\ITZ{	 
\uncover<1->{\ITT{1}{}{
\uncover<1->{
{\footnotesize{Cuando llama a un método con parámetros de objetos, las variables que pasa al cuerpo del método lo hacen por {\textit{referencias}}, lo que significa que las instrucciones aplicadas a esos objetos dentro del método afectara los objetos originales también (esto incluye arreglos). Recuerde que los tipos de dato primitivos son pasados por {\textit{valor}}, por lo cual, si fueron modificados al interior de un método, esta modificación no se mantiene de  un método a otro.}}
\tiny{\lstinputlisting{./codigosEx/PasoRef.java}}}}}}
\end{frame}

\subsection{Clases, objetos y métodos}
\begin{frame}{Recordatorio: Intercambio de información entre los métodos}
\ITZ{	 
\uncover<1->{\ITT{1}{}{
\uncover<1->{
\begin{itemize}
\item {Por valor : el método recibe una copia del valor del argumento; el método trabaja sobre esa copia que podrá modificar sin que esto tenga incidencia al valor efectivo del argumento.}
\item {Por referencia : el método recibe la referencia del argumento con la cual trabaja directamente, por lo cual, el método podrá modificar el valor efectivo del argumento.}
\end{itemize}}}}}\end{frame}


\subsection{Aplicaciones Java}
\begin{frame}{Autoreferencia : La palabra clave this}
\ITZ{	 
\uncover<1->{\ITT{1}{}{
\uncover<1->{
Para hacer una referencia a la instancia actual de una clase usamos la palabra clave this. Así, la palabra clave {\textit{this}} se usa para referirse al objeto actual o a las variables de instancia de este objeto. No se usa this en los metodos declarados como {\textit{static}}.
{\tiny{\lstinputlisting{./codigosEx/this.java}}}
Se puede omitir la palabra clave this para las variables de instancia, esto depende de la existencia o no de variables con el mismo nombre en el ámbito local.
}}}}
\end{frame}

\subsection{Aplicaciones Java}
\begin{frame}{Nombre de variables y la palabra clave this}
\ITZ{	 
\uncover<1->{\ITT{1}{}{
\uncover<1->{
Java busca la definición de las variables primero en el ámbito actual, después en el exterior hasta la definición del método actual. Si esa variable no es local, entonces java busca una definición de ella como instancia de la clase actual y por ultimo de una super clase. En estos casos se usa por ejemplo :   {\textit{this.test }} para referirse a la variable de instancia y solo {\textit{test}} para referirse a la variable local.
\tiny{\lstinputlisting{./codigosEx/ehis2.java}}}}}}
\end{frame}

\subsection{Aplicaciones Java}
\begin{frame}{Sobrecarga : Métodos con el mismo nombre y argumentos diferentes}
\ITZ{	 
\uncover<1->{\ITT{1}{}{
\uncover<1->{
En java se pueden crear métodos con el mismo nombre pero con definiciones diferentes. No hay necesidad de métodos completamente diferentes que realizan en esencia lo mismo, simplemente se comportaran distinto en base a la entrada.\\ Al llamar a un método, si hay múltiples declaraciones de este, java hará coincidir su nombre, numero y tipo de argumentos (no considera el tipo de dato de retorno)para seleccionar que definición de método utilizar. \\
Si se quiere realizar múltiples declaraciones de  métodos en la clase, lo único que se debe hacer es crear definiciones distintas  todas con el mismo nombre, pero con diferentes listas de parámetros (es necesario que cada lista de parámetros sea única). En la sobrecarga de métodos java no diferencia una variable definida como {\textit{int a}} o {\textit{final int a}}.}}}}
\end{frame}

\subsection{Aplicaciones Java}
\begin{frame}{Sobrecarga}
\ITZ{	 
\uncover<1->{\ITT{1}{}{
\uncover<1->{
\tiny{\lstinputlisting{./codigosEx/Rectangulo0.java}}
\tiny{\lstinputlisting{./codigosEx/Rectangulo1.java}}
}}}}
\end{frame}

\subsection{Aplicaciones Java}
\begin{frame}{Sobrecarga}
\ITZ{	 
\uncover<1->{\ITT{1}{}{
\uncover<1->{
\tiny{\lstinputlisting{./codigosEx/Rectangulo2.java}}
\tiny{\lstinputlisting{./codigosEx/Rectangulo3.java}}
}}}}
\end{frame}


\subsection{Aplicaciones Java}
\begin{frame}{Sobrecarga}
\ITZ{	 
\uncover<1->{\ITT{1}{}{
\uncover<1->{
\tiny{\lstinputlisting{./codigosEx/Rectangulo4.java}}
\tiny{\lstinputlisting{./codigosEx/Rectangulo5.java}}
}}}}
\end{frame}

\subsection{Aplicaciones Java}
\begin{frame}{Programa con varias Clases}
\ITZ{	 
\uncover<1->{\ITT{1}{}{
\uncover<1->{
\begin{itemize}
\item{Un archivo fuente por clase:  para ejecutar un programa compuesto de varias clases en archivos separados es necesario compilar todas las clases y luego ejecutar la clase que tenga el método {\textit{main}} (que probablemente llamaremos  {\textit{public class Main())}}. Las clases deben estar en la misma carpeta o darle la ruta para que pueda acceder a ellas.}
\item {Un archivo fuente para todas las clases : para esto solo, una clase que será la que contenga el método {\textit{main}} debe ser {\textit{public}}. La maquina virtual accede a la clase publica y por ende, al método {\textit{main}}. Las otras clases pueden estar dentro de la clase {\textit{public}} o fuera de ella ya sea al inicio o al final del archivo.}
\end{itemize}
}}}}
\end{frame}

\subsection{Aplicaciones Java}
\begin{frame}{Constructor}
\ITZ{	 
\uncover<1->{\ITT{1}{}{
\uncover<1->{
El constructor asigna memoria para el objeto y automatiza el proceso de inicialización de este. Un constructor será un método que \begin{itemize} \item{tiene el mismo nombre de que la clase}, \item{puede tener argumentos} \item{no tiene un valor de retorno(return).} \end{itemize}
{\tiny{\lstinputlisting{./codigosEx/Punto.java}}}
Cuando la clase tiene un constructor debemos crear los objetos llamando al constructor. En este caso: {\textit{Punto a = Punto (1,2);}}}}}}
\end{frame}

\subsection{Aplicaciones Java}
\begin{frame}{Constructor - ejercicio}
\ITZ{	 
\uncover<1->{\ITT{1}{}{
\uncover<1->{
Crear una clase Punto. Crear 3 constructores \begin{itemize}  \item {uno que no reciba argumentos y que inicialice {\textit{x}} e {\textit{y}} en 0} \item{uno que defina {\textit{x=y=abs}} y reciba como argumento (int abs)} \item{uno que reciba (int abs, int ord)}\end{itemize}
}}}
\uncover<2->{\ITT{1}{}{
\uncover<2->{
Cree un método para mostrar en pantalla y tres métodos para desplazar para desplazar el punto \begin {itemize} \item{ el primero recibe 2 enteros }\item{ el segundo recibe 1 enteros }\item{ el tercero recibe 1 short } \end{itemize} Pruebe que pasa si llama al método desplazar con una variable declarada como byte. 
\\Use esta clase Punto en el ejercicio anterior de construcción del rectángulo.
}}}}\end{frame}

\subsection{Aplicaciones Java}
\begin{frame}{Constructor - ejercicio}
\ITZ{	 
\uncover<1->{\ITT{1}{}{
\uncover<1->{
{\tiny{\lstinputlisting{./codigosEx/PuntoT0.java}
}}}}}}
\end{frame}

\subsection{Aplicaciones Java}
\begin{frame}{Constructor - ejercicio}
\ITZ{	 
\uncover<1->{\ITT{1}{}{
\uncover<1->{
{\tiny{\lstinputlisting{./codigosEx/PuntoT1.java}}}
}}}}
\end{frame}

\subsection{Aplicaciones Java}
\begin{frame}{Constructor - ejercicio}
\ITZ{	 
\uncover<1->{\ITT{1}{}{
\uncover<1->{
{\tiny{\lstinputlisting{./codigosEx/PuntoT2.java}}}
}}}}
\end{frame}

\subsection{Aplicaciones Java}
\begin{frame}{Reglas de los Constructores}
\ITZ{	 
\uncover<1->{\ITT{1}{}{
\uncover<1->{
\begin{itemize}
\item {Los constructores no retornan ningún valor. Por lo cual, no debe figurar nada antes del nombre del constructor {\textit{public void Punto()}}}.
\item {Podemos declarar una clase sin constructor. En este caso, para instanciar los objetos se usa : {\textit{Punto a = Punto ();}}, como si existiera un constructor por defecto sin argumentos. Desde que la clase posee al menos un constructor este constructor {\textit{por defecto}} no podrá ser utilizado.}
%En el caso que el constructor no tenga argumentos no podemos distinguirlo del constructor por defecto {\textit{Punto a = Punto ();}}, es decir, con esta instrucción podemos entender que Punto no tiene constructor o que tiene un constructor sin argumentos. }
\item {Un constructor no puede ser llamado directamente desde otro método. \\ Por ejemplo : {\textit{Punto a = Punto (2,3);}}\\
{\textit{a.Punto (4,6);}} }
%\item {Un constructor puede ser llamado por otro constructor de la misma clase, para esto se usa la palabra {\textit{super}}}
\end{itemize}
}}}}\end{frame}


\subsection{Aplicaciones Java}
\begin{frame}{Construcción e inicialización de un objeto}
\ITZ{	 
\uncover<1->{\ITT{1}{}{
\uncover<1->{
La creación de un objeto conlleva : \begin{itemize}
\item {Inicialización por defecto de todos los campos del objeto (null).}.
\item {Inicialización explícita al declarar los campos de un objeto.}
\item {Ejecución de las instrucciones del constructor.}
\end{itemize}
{\tiny{\lstinputlisting{./codigosEx/EjClass.java}}}
Si ejecutamos la instrucción : {\textit{A a = new A();}} que ocurre?\\
{\footnotesize{Si los campos del objeto son constantes, es decir, definidos con la palabra clave {\textit{final}}, estos deberán ser inicializados a mas tardar por el constructor y evidentemente no podan ser modificados posteriormente.}}
{\footnotesize{A diferencia de los objetos las variables locales no son inicializadas de manera implicita. Toda variable local debe ser inicializada antes de ser usada.}}
}}}}\end{frame}


\subsection{Aplicaciones Java}
\begin{frame}{Llamar a un constructor dentro de otro constructor}
\ITZ{	 
\uncover<1->{\ITT{1}{}{
\uncover<1->{
No se puede llamar directamente a un constructor ({\textit{a.Point(2,3)}}). Pero si se puede llamar a un constructor al interior de otro constructor de la misma clase. Para esto se usara la palabra clave {\textit{this}}. 
{\tiny{\lstinputlisting{./codigosEx/Constructor.java}}}
}}}}\end{frame}

\subsection{Aplicaciones Java}
\begin{frame}{Recursividad}
\ITZ{	 
\uncover<1->{\ITT{1}{}{
\uncover<1->{
En Java es posible usar la recursividad ya sea en los metodos de instancia o en los metodos de clase ({\textit{static}})..
\begin{itemize}
\item {Directa : un método tiene, en su definición, una llamada a si mismo.}
\item {Cruzada : el llamado de un método lleva el llamado de otro método que llama al método inicial.}
\end{itemize}
Cree un programa recursivo que calcule el factorial de un numero.
}}}}\end{frame}

\subsection{Aplicaciones Java}
\begin{frame}{Recursividad- ejercicio}
\ITZ{	 
\uncover<1->{\ITT{1}{}{
\uncover<1->{
{\tiny{\lstinputlisting{./codigosEx/CalcularFac.java}}}
}}}}
\end{frame}

\subsection{Aplicaciones Java}
\begin{frame}{Recursividad- ejercicio}
\ITZ{	 
\uncover<1->{\ITT{1}{}{
\uncover<1->{
{\tiny{\lstinputlisting{./codigosEx/CalcularFac2.java}}}
}}}}
\end{frame}

\subsection{Aplicaciones Java}
\begin{frame}{Atributos private}
\ITZ{	 
\uncover<1->{\ITT{1}{}{
\uncover<1->{
La  encapsulación consiste en el ocultar los datos miembros de una clase, de modo que solo será posible modificarlos mediante métodos definidos dentro de la misma clase. De esta forma, los detalles de implementación permanecen "ocultos" a las personas que usan las clases. \\La encapsulación es una de las principales ventajas que proporciona la programación orientada a objetos. \\
Java implementa la encapsulación de datos usando la palabra clave : {\textit{private}}. 
\\El uso de la encapsulación de datos no es obligatoria pero es recomendada.
}}}}
\end{frame}


\subsection{Aplicaciones Java}
\begin{frame}{Atributos private}
\ITZ{	 
\uncover<1->{\ITT{1}{}{
\uncover<1->{Cree una clase Punto con {\textit{private int x,int y;}} y con un método mostrar el punto. \\Cree una clase circulo con : {\tiny{\lstinputlisting{./codigosEx/CercleConst.java}}} Cree ademas un método mostrar los parámetros del circulo y otro que desplace los valores del punto central del circulo.}}}}
\end{frame}

\subsection{Aplicaciones Java}
\begin{frame}{Atributos private - ejercicio}
\ITZ{	 
\uncover<1->{\ITT{1}{}{
\uncover<1->{
{\tiny{\lstinputlisting{./codigosEx/PuntoTest.java}}}
}}}}
\end{frame}
\subsection{Aplicaciones Java}
\begin{frame}{Atributos private - ejercicio}
\ITZ{	 
\uncover<1->{\ITT{1}{}{
\uncover<1->{
{\tiny{\lstinputlisting{./codigosEx/Cercle.java}}}
}}}}
\end{frame}
\subsection{Aplicaciones Java}
\begin{frame}{Atributos private - ejercicio}
\ITZ{	 
\uncover<1->{\ITT{1}{}{
\uncover<1->{
{\tiny{\lstinputlisting{./codigosEx/MainCerclePuntoTest.java}}}
}}}}
\end{frame}

\subsection{Aplicaciones Java}
\begin{frame}{Atributos private}
\ITZ{	 
\uncover<1->{\ITT{1}{}{
\uncover<1->{
Dado que los atributos de la clase Punto son privados, estos no son accesibles por la clase circulo. Para evitar esto crearemos la misma clase circulo que tiene un objeto miembro del tipo Punto y a la vez definiremos métodos de acceso {\textit{getX}} y  {\textit{getY}}  de alteración  {\textit{setX}} y  {\textit{setY}}.
{\tiny{\lstinputlisting{./codigosEx/setget.java}}}
Cree los métodos \textit{set} y \textit{get} para corregir el ejemplo precedente.}}}}
\end{frame}

\subsection{Aplicaciones Java}
\begin{frame}{Uso set y get- ejercicio}
\ITZ{	 
\uncover<1->{\ITT{1}{}{
\uncover<1->{
{\tiny{\lstinputlisting{./codigosEx/PuntoTest2.java}}}
}}}}
\end{frame}
\subsection{Aplicaciones Java}
\begin{frame}{Uso set y get- - ejercicio}
\ITZ{	 
\uncover<1->{\ITT{1}{}{
\uncover<1->{
{\tiny{\lstinputlisting{./codigosEx/Cercle2.java}}}
}}}}
\end{frame}
\subsection{Aplicaciones Java}
\begin{frame}{Uso set y get- - ejercicio}
\ITZ{	 
\uncover<1->{\ITT{1}{}{
\uncover<1->{
{\tiny{\lstinputlisting{./codigosEx/MainCerclePuntoTest2.java}}}
}}}}
\end{frame}

\subsection{Aplicaciones Java}
\begin{frame}{Paquetes}
\ITZ{	 
\uncover<1->{\ITT{1}{}{
\uncover<1->{

La noción de paquete corresponde a un reagrupamiento logico bajo un identificador común de un conjunto de clases. Los paquetes Java agrupan las clases en librerías (bibliotecas). Los paquetes Java se utilizan de forma similar a como se utilizan las librerías en C++, sólo que en Java se agrupan clases y/o interfaces. \\

Los paquetes se caracterizan por un nombre que puede ser un simple identificador {\textit{MisClases}} único o una continuación de identificadores separados por puntos {\textit{Utilitarios.Matematica}}. 
Los paquetes proporcionan una forma de ocultar clases, evitando que otros programas o paquetes accedan a clases que son de uso exclusivo de una aplicación determinada.\\
 }}}}
\end{frame}

\subsection{Aplicaciones Java}
\begin{frame}{Paquetes}
\ITZ{	 
\uncover<1->{\ITT{1}{Paquetes}{
\uncover<1->{

Los paquetes se declaran utilizando la palabra reservada \textit{package} seguida del nombre del paquete. Esta sentencia debe estar al comienzo del archivo fuente. Concretamente debe ser la primera sentencia ejecutable del código Java, excluyendo, los comentarios y espacios en blanco.
Por ejemplo: \\
\textit{ package figuras;}\\
\textit{ public class Circulo \{ }\\ 
\textit{  . . \}.} 
}}}}
\end{frame}

\subsection{Aplicaciones Java}
\begin{frame}{Paquetes}
\ITZ{	 
\uncover<1->{\ITT{1}{Paquetes}{
\uncover<1->{
Para incluir nuevas clases en el paquete se debe colocar la misma sentencia al comienzo de los archivos que contengan la declaración de las clases. Cada uno de los archivos que contengan clases pertenecientes a un mismo paquete, deben incluir la misma sentencia  \textit{package}, y solamente puede haber una sentencia \textit{package} por fichero. La sentencia \textit{package} colocada el comienzo de un fichero fuente afectará a todas las clases que se declaren en ese fichero.}}}}
\end{frame}


\subsection{Aplicaciones Java}
\begin{frame}{Paquetes}
\ITZ{	 
\uncover<1->{\ITT{1}{Uso de una clase de un paquete}{
\uncover<1->{

Cuando un programa usa una clase el compilador la busca en el paquete por defecto. Para usar una clase que pertenezca a otro paquete se debe : 
\begin{itemize}
\item {usar el nombre del paquete y de la clase: \\{\textit{MisClases.Punto p = new MisClases.Punto(2,3);}}}
\item {usar la instrucción {\textit{ import}} para importar una clase : {\textit{import MisClases.Punto}} o \textit{import MisClases.*}} para importar el paquete completo. Luego se podrá usar el nombre de la clase sin especificar el nombre del paquete.}
\end{itemize}
}}}\end{frame}

\subsection{Aplicaciones Java}
\begin{frame}{Paquetes}
\ITZ{	 
\uncover<1->{\ITT{1}{Acceso a la clases y paquetes}{
\uncover<1->{
Cada clase dispone de derecho de acceso, este es definido por la palabra clave {\textit{public}}.
\begin{itemize}
\item {con la palabra clave {\textit{public}} la clase es accesible por todas las otras clases (usando import).}
\item {sin la palabra clave {\textit{public}} la clase será accesible solo por la clases del mismo paquete.}
\end{itemize}
Mientras se trabaje con el paquete por defecto, la ausencia de la palabra {\textit{public}} no tiene importancia, pero al definir un paquete la clases no definidas como publicas solo serán accesibles por la clases que pertenecen al paquete. 
}}}}\end{frame}



%\item {Un constructor puede ser llamado por otro constructor de la misma clase, para esto se usa la palabra {\textit{super}}}
%\graphicspath{{./pics/}}
 
\section{Herencia}
\subsection{Herencia}
\begin{frame}{Herencia}
\ITZ{	
\uncover<1->{\ITT{1}{}{\begin{itemize}
\uncover<1->
\item {Herencia}
{\footnotesize{\begin {itemize}
\item {Acceso a una clase derivada}
\item {Construcción de objetos derivados}
\item {Sobreescritura y sobrecarga de metodos }
\item {Polimorfismo}
\item {Super clase object}
\item {Miembros protected}
\item {Clases y metodos de finalización}
\item {Clases abstractas}
\item {Clases Interfaces}
\end {itemize}}}
\end{itemize}}}}
\end{frame}

\subsection{Herencia}
\begin{frame}{Herencia}
\ITZ{	
\uncover<1->{\ITT{1}{Definiciones}{
\uncover<1->{
La herencia es uno de los conceptos fundamentales de la POO, esta al origen de la reutilización. \\ La herencia permite definir una nueva clase llamada  {\textbf{derivada}} a partir de una clase existente llamada clase {\textbf{base}}. \\Esta nueva clase hereda automáticamente las funcionalidades de la clase base (métodos y atributos), los que podrá modificar o completar libremente. \\ Es posible crear muchas clases derivadas a partir de una clase base. }}}}% Esto permite desarrollar nuevas herramientas basandose en un cierto }
\end{frame}


\subsection{Herencia}
\begin{frame}{Herencia}
\ITZ{	
\uncover<1->{\ITT{1}{Nociones iniciales}{
\uncover<1->{
Tenemos una clase base Punto:
{\tiny{\lstinputlisting{./codigosEx/PuntoT0Herencia.java}}}
Definir una clase derivada de la clase Punto que se llame : {\textbf{PuntoColor}}. Para esto, en java usaremos la palabra clave {\textbf{extends}}. Esta clase debe manipular puntos coloreados en un plano.\\ Dentro de la clase {\textbf{PuntoColor}} cree un método llamado {\textbf{color}} encargado de definir el color del punto. \\ Cree un metodo que muestre el color del punto. \\ 
}}}}\end{frame}

\subsection{Herencia}
\begin{frame}{Herencia}
\ITZ{	
\uncover<1->{\ITT{1}{Nociones iniciales}{
\uncover<1->{
Tenemos una clase derivada de la Punto llamada PuntoColor:
{\tiny{\lstinputlisting{./codigosEx/PuntoTHerencia.java}}}
Al usar extends Punto se le especifica al compilador que la clase PuntoColor es una clase derivada de la clase Punto.
}}}}\end{frame}

\subsection{Herencia}
\begin{frame}{Herencia}
\ITZ{	
\uncover<1->{\ITT{1}{Nociones iniciales}{
\uncover<1->{
Un objeto del tipo PuntoColor puede acceder a : \begin{itemize} \item{A los métodos públicos de PuntoColor} \item{A los métodos públicos de Punto} \end{itemize}
}}}}\end{frame}

\subsection{Herencia}
\begin{frame}{Acceso de una clase derivada}
\ITZ{	
\uncover<1->{\ITT{1}{a la clase base}{
\uncover<1->{
Si bien una clase derivada hereda los miembros (atributos y métodos) de una clase base, el acceso que tenga a estos dependerá de si estos son privados o públicos.
\begin{itemize}
\item {{\textbf{Una clase derivada no accede a los miembros privados}}: Un método de la clase derivada no puede acceder a miembros privados de su clase base. }
\item {{\textbf{Una clase derivada accede a los miembros públicos}}: tal como si hubiesen sido declarados en la clase la misma clase derivada. (Ejemplo precedente).}
\end{itemize}
}}}}\end{frame}

\subsection{Herencia}
\begin{frame}{Acceso de una clase derivada}
\ITZ{	
\uncover<1->{\ITT{1}{a la clase base}{
\uncover<1->{
\begin {itemize}
\item {Las clases de base pueden llamarse también superclases. De la misma manera que las clases derivadas puedes llamarse subclases.}
\item {Si trabajamos con múltiples clases (cada una en archivos fuentes separados), se debe compilar primero la superclase y luego las subclases.}
\item {El acceso a las clases no publicas será restringido si no pertenecen al mismo paquete. Una clase sin la palabra clave {\textit{public}} será accesible solo por las clases del mismo paquete.
}\end{itemize}}}}}\end{frame}

\subsection{Herencia}
\begin{frame}{Acceso de una clase derivada}
\ITZ{	
\uncover<1->{\ITT{1}{a los miembros de la clase base}{
\uncover<1->{
En resumen: 
\begin{itemize}
\item {Los miembros públicos de la superclase quedan como miembros públicos para la subclase. Por esa razón, en el ejemplo precedente, pudimos aplicar el método inicializar a un objeto de tipo PuntoColor.}
\end{itemize}
Modifique el método mostrar de la clase PuntoColor para que ademas muestre el valor de x e y usando : {\textit{System.out.println("punto de coordenadas : "+x+" " + y);}}. \\ Que ocurre?
}}}}
\end{frame}

\subsection{Herencia}
\begin{frame}{Acceso de una clase derivada}
\ITZ{	
\uncover<1->{\ITT{1}{a los miembros de la clase base}{
\uncover<1->{
{\tiny{\lstinputlisting{./codigosEx/Erencia.java}}}
Si usamos {\textit{this.mostrar()}}, lo que hace es aplicar el metodo mostrar al objeto (de tipo PuntoColor) que llamo al metodo {\textit{mostrarC}}.\\
Cree un nuevo metodo inicializar que se atribuya las coordenadas y el color a un punto.
{\tiny{\lstinputlisting{./codigosEx/Erencia2.java}}}}}}}
\end{frame}

\subsection{Herencia}
\begin{frame}{Construcción e inicialización}
\ITZ{	
\uncover<1->{\ITT{1}{de objetos derivados}{
\uncover<1->{
{\textbf{Llamar a constructores : }} La creación de un objeto usando {\textit{new}} llama al constructor (que tiene el mismo nombre de la clase). Si la clase no dispone de un constructor usara un pseudo-construtor por defecto.\\
{\tiny{\lstinputlisting{./codigosEx/Erencia2Const.java}}}
Si necesita inicializar ciertos campos de la superclase sera necesario disponer de las funciones de alteración (set) o recurrir al constructor de la superclase.}}}}
\end{frame}

\subsection{Herencia}
\begin{frame}{Construcción e inicialización}
\ITZ{	
\uncover<1->{\ITT{1}{de objetos derivados}{
\uncover<1->{
El constructor de PuntoColor podrá : 
\begin{itemize}
\item{Inicializar el campo color, accesible dado que es miembro de PuntoColor;}
\item {Llamar al constructor de Punto para inicializar los campos x e y}.
\end{itemize}
{\textbf{Regla:}} Si un constructor de una subclase llama a un constructor de una superclase, debe ser la primera instrucción del constructor y debe usar la palabra clave {\textbf{super}}. Solo es posible llamar al constructor de la superclase inmediatamente superior usando {\textbf{super}}.}}}}
\end{frame}

\subsection{Herencia}
\begin{frame}{Construcción e inicialización}
\ITZ{	
\uncover<1->{\ITT{1}{de objetos derivados}{
\uncover<1->{
{\tiny{\lstinputlisting{./codigosEx/Erencia2Const2.java}}}
Reemplace los metodos inicializar de la clase Punto y PuntoColor por constructores.}}}
\uncover<2->{\ITT{1}{}{
\uncover<2->{
Para inicializar la parte heredada de una superclase que no tiene constructor se podrá llamar al constructor por defecto usando {\textit{super ();}}}}}}
\end{frame}

\subsection{Herencia}
\begin{frame}{Construcción e inicialización}
\ITZ{	
\uncover<1->{\ITT{1}{Que pasa si una de las clases no tiene constructor?}{
\uncover<1->{
\begin{itemize}
\item{La clase base no tiene constructor : en estos casos se llama al constructor por defecto de la clase base usando{\textit{super ();}}. Asi la parte heredada de la superclase sera inicializada.
{\tiny{\lstinputlisting{./codigosEx/Erencia3.java}}}}
\end{itemize}}}}}
\end{frame}

\subsection{Herencia}
\begin{frame}{Construcción e inicialización}
\ITZ{	
\uncover<1->{\ITT{1}{Que pasa si una de las clases no tiene constructor?}{
\uncover<1->{
\begin{itemize}
\item {La clase de derivada no tiene constructor :  en estos casos la clase base debe : \begin{enumerate} \item{tener un constructor publico sin argumentos o } \item { no tener constructor} 
\end{enumerate}
{\tiny{\lstinputlisting{./codigosEx/Erencia4.java}}}}
\end{itemize}}}}}
\end{frame}

\subsection{Herencia}
\begin{frame}{Ejercicio de constructores y herencia}
\ITZ{	
\uncover<1->{\ITT{1}{}{
\uncover<1->{
{\tiny{\lstinputlisting{./codigosEx/PuntoHerenciaConstr.java}}}
}}}}\end{frame}

\subsection{Herencia}
\begin{frame}{Ejercicio : Constructores y Herencia}
\ITZ{	
\uncover<1->{\ITT{1}{}{
\uncover<1->{
Cree una superclase Fruta con : tipoFruta (string) y precio (float). Cree el constructor de dicha clase y un método mostrar que muestre los 2 atributos. \\
Cree una subclase Manzana con : rebanada (int), tipoFruta (string) y precio (float). Cree el constructor de dicha clase y un metodo mostrar que despliegue los valores.}}}}\end{frame}


\subsection{Herencia}
\begin{frame}{Ejercicio : Constructores, Herencia y Encapsulamiento}
\ITZ{	
\uncover<1->{\ITT{1}{}{
\uncover<1->{
{\tiny{\lstinputlisting{./codigosEx/HerenciaFruta.java}}}
}}}}\end{frame}

\subsection{Herencia}
\begin{frame}{Ejercicio : Constructores y Herencia}
\ITZ{	
\uncover<1->{\ITT{1}{}{
\uncover<1->{
{\tiny{\lstinputlisting{./codigosEx/HerenciaFrutaPrivate.java}}}
}}}}\end{frame}


\subsection{Herencia}
\begin{frame}{Ejercicio : Constructores, Herencia y Encapsulamiento}
\ITZ{	
\uncover<1->{\ITT{1}{}{
\uncover<1->{
Cree una superclase Persona con : rut (string) y nombre (string). Cree el constructor de dicha clase .\\
Cree una subclase Estudiante con : rol (string), rut (string) y nombre (string). Cree el constructor de dicha clase. \\ Cree un metodo principal que llame a las clases y muestre en pantalla sus atributos.}}}}\end{frame}

\subsection{Herencia}
\begin{frame}{Ejercicio de constructores y herencia}
\ITZ{	
\uncover<1->{\ITT{1}{}{
\uncover<1->{
{\tiny{\lstinputlisting{./codigosEx/EstuHerencia/Persona.java}}}
}}}}\end{frame}

\subsection{Herencia}
\begin{frame}{Ejercicio de constructores y herencia}
\ITZ{	
\uncover<1->{\ITT{1}{}{
\uncover<1->{
{\tiny{\lstinputlisting{./codigosEx/EstuHerencia/Estudiante.java}}}
}}}}\end{frame}

\subsection{Herencia}
\begin{frame}{Inicalización de un objeto derivado}
\ITZ{	
\uncover<1->{\ITT{1}{Recordatorio : Creación de un objeto}{
\uncover<1->{
\begin{itemize}
\item{asignación de memoria.}
\item{inicialización por defecto de los campos.}
\item{inicialización explicita de los campos.}
%\item{ejecución de las instrucciones de un constructor.}
\end{itemize}}}}}
\end{frame}

\subsection{Herencia}
\begin{frame}{Inicialización de un objeto derivado}
\ITZ{	
\uncover<1->{\ITT{1}{Creación de un objeto derivado}{
\uncover<1->{
\begin{itemize}
\item{asignación de memoria, considerando los campos heredados y los propios.}
\item{inicialización por defecto de los campos, considerando los campos heredados y los propios}
\item{inicialización explicita de los campos, considerando los campos heredados y los propios por lo que se ejecuta los bloques de inicialización de la clase base.}
%\item{ejecución de las instrucciones del constructor de la clase base.}
\item{inicialización explicita de los campos de la clase derivada.}
%\item{ejecución de las instrucciones de un constructor de la clase derivada.}
\end{itemize}}}}}
\end{frame}

\subsection{Herencia}
\begin{frame}{Derivaciones sucesivas}
\ITZ{	
\uncover<1->{\ITT{1}{}{
\uncover<1->{
\begin{itemize}
\item{De una clase base pueden derivar varias clases diferentes.}
\item{La noción de clase de base y clase derivada es relativa dado que la clase derivada puede a la vez servir de clase base para otras clases.}
\end{itemize}}}}}
\end{frame}

\subsection{Herencia}
\begin{frame}{Herencia}
\GFXH{5cm}{JerarquiaVehiculoV3.pdf}
\end{frame}

\subsection{Herencia}
\begin{frame}{Polimorfismo}
\ITZ{	
\uncover<1->{\ITT{1}{}{
\uncover<1->{
El polimorfismo sugiere múltiples formas. El polimorfismo es un concepto muy importante dentro de POO. En java es la habilidad de una variable por referencia de cambiar su comportamiento en función de que instancia de objeto posee. Esto permite tratar de la misma manera, como objetos de la súperclase, a múltiples objetos de la subclase, seleccionando en cada caso los métodos apropiados. El polimorfismo se puede establecer mediante la sobrecarga, sobre-escritura y la ligadura dinámica.}}}}
\end{frame}

\subsection{Herencia}
\begin{frame}{Sobreescritura y sobrecarga de miembros}
\ITZ{	
\uncover<1->{\ITT{1}{}{
\uncover<1->{\begin{itemize}
\item{Una clase derivada podrá sobrecargar un método de la clase base. Estos nuevos métodos serán usados solo por la clase derivada y sus descendientes.}
\item {Una clase derivada podrá sobreescribir (redefinir) un método de la clase base. En este caso, no solo serán métodos con el mismo nombre como en la sobrecarga sino también con los mismo argumentos de entrada (cantidad y tipo) y el mismo tipo de valor de retorno.}
\end{itemize}
La sobrecarga permite acumular métodos con el mismo nombre la Sobreescritura permite sustituir un método con otro.
}}}}\end{frame}

\subsection{Herencia}
\begin{frame}{Sobreescritura de métodos}
\ITZ{	
\uncover<1->{\ITT{1}{}{
\uncover<1->{
Un objeto de una clase derivada puede acceder a todos los miembros públicos de la clase de base.
{\tiny{\lstinputlisting{./codigosEx/PuntoEx.java}}}
Si llamamos p.mostrar() o pc.mostrar().
Obtendremos las coordenadas del punto $p$ o del punto $pc$ respectivamente. Pero no obtendremos el color del punto color. Por esto, hemos creado anteriormente dos métodos con distinto nombre.}}}}
\end{frame}

\subsection{Herencia}
\begin{frame}{Sobreescritura de métodos}
\ITZ{	
\uncover<1->{\ITT{1}{}{
\uncover<1->{
En java es posible sobreescribir en la clase derivada un método de la clase base con los mismos argumentos de entrada y el mismo tipo de retorno. Dentro de la clase derivada se llamará al nuevo método {\textit{sobrescrito}} que esta dentro de ella ocultando de alguna manera el método de la clase base.
{\tiny{\lstinputlisting{./codigosEx/PuntoEx2.java}}} 

}}}}\end{frame}

\subsection{Herencia}
\begin{frame}{Sobreescritura de métodos}
\ITZ{	
\uncover<1->{\ITT{1}{}{
\uncover<1->{
El código anterior provoca una llamada recursiva al método mostrar de la clase PuntoColor y no llama al método mostrar de la clase Punto. Para eso es necesario especificar que se quiere llamar al método mostrar de la clase base. 
{\tiny{\lstinputlisting{./codigosEx/PuntoEx3.java}}}

}}}}\end{frame}

\subsection{Herencia}
\begin{frame}{Sobrecarga y Sobreescritura de métodos}
\ITZ{	
\uncover<1->{\ITT{1}{}{
\uncover<1->{
Puede usarse sobrecarga de métodos de la clase base en la clase derivada.
{\tiny{\lstinputlisting{./codigosEx/PuntoEx0.java}}} 
}}}}\end{frame}

\subsection{Herencia}
\begin{frame}{Restricciones en la Sobreescritura} 
\ITZ{	
\uncover<1->{\ITT{1}{}{
\uncover<1->{
\begin{itemize}
\item {debe tener los mismo argumentos (tipo y cantidad) de entrada y el mismo tipo de retorno}
{\tiny{\lstinputlisting{./codigosEx/PuntoEx01.java}}} 
\item {no debe disminuir los derecho de acceso del método} {\tiny{\lstinputlisting{./codigosEx/PuntoEx02.java}}} 
\item { puede aumentar los permisos de acceso.}
{\tiny{\lstinputlisting{./codigosEx/PuntoEx03.java}}}
\end{itemize}
}}}}\end{frame}

\subsection{Herencia}
\begin{frame}{Restricciones en la Sobreescritura} 
\ITZ{	
\uncover<1->{\ITT{1}{}{
\uncover<1->{
Un método de clase (static) no puede ser sobrescrito en una clase derivada dado que el tipo de objeto que llama al método permite elegir entre el método de la clase base y el método de la clase derivada. \\
Recuerde que un método de clase puede ser llamado sin un objeto, por lo cual, la elección del método no es posible. 
}}}}\end{frame}

\subsection{Herencia}
\begin{frame}{Duplicación de campos} 
\ITZ{	
\uncover<1->{\ITT{1}{}{
\uncover<1->{
Una clase derivada puede definir un campo con el mismo nombre que un campo de la clase base.
{\tiny{\lstinputlisting{./codigosEx/Main.java}}}
}}}}\end{frame}

\subsection{Polimorfismo}
\begin{frame}{Polimorfismo} 
\ITZ{	
\uncover<1->{\ITT{1}{}{
\uncover<1->{
Permite manipular los objetos sin conocer su tipo. \\Por ejemplo, se puede crear un arreglo de objetos unos de tipo Punto y otros PuntoColor y llamar al método mostrar para cada objeto del arreglo. Cada objeto actuará en función de su tipo. Esto es inducido por la herencia. \\Un objeto PuntoColor es también un Punto por lo cual puede ser tratado como un Punto (esta relación no es reciproca).
}}}}\end{frame}


\subsection{Polimorfismo}
\begin{frame}{Polimorfismo} 
\ITZ{	
\uncover<1->{\ITT{1}{}{
\uncover<1->{
{\tiny{\lstinputlisting{./codigosEx/MainPoli.java}}}}}}}\end{frame}

\subsection{Polimorfismo}
\begin{frame}{Polimorfismo} 
Punto p;
\GFXH{5cm}{puntoP.pdf}
p=new PuntoColor(4,6,2);
\end{frame}

\subsection{Polimorfismo}
\begin{frame}{Polimorfismo} 
\GFXH{5cm}{puntoP1.pdf}
\end{frame}

\subsection{Polimorfismo}
\begin{frame}{Polimorfismo} 
\ITZ{	
\uncover<1->{\ITT{1}{}{
\uncover<1->{
Java permite asignar a una variable objeto no solo la referencia del tipo correspondiente sino también la referencia a un objeto de un tipo derivado. Encontramos así, una compatibilidad por asignación entre un tipo clase derivada y un tipo de clase base.\\
Si ejecutamos : \\
 {\textbf{Punto p = new Punto(3,5);\\
p.mostrar();\\
p=new PuntoColor(4,5,2);\\
p.mostrar();}}\\
La variable p es del tipo Punto mientras que el objeto referenciado por p es del tipo PuntoColor. La variable p llamara a mostrar de la clase PuntoColor.}}}}\end{frame}

\subsection{Polimorfismo}
\begin{frame}{Resumen : Polimorfismo} 
\ITZ{	
\uncover<1->{\ITT{1}{}{
\uncover<1->{
\begin{itemize}
\item{compatibilidad: existe una conversion implicita de una referencia a un objeto de una clase derivada a una referencia de un objeto de una clase base (ascendente).}
\item {ligado dinámico de métodos}
\end{itemize}
El polimorfismo permite obtener un comportamiento adecuado de cada tipo de objeto, sin la necesidad de conocer su tipo.
}}}}\end{frame}

\subsection{Polimorfismo}
\begin{frame}{Ejercicio : Polimorfismo} 
\ITZ{	
\uncover<1->{\ITT{1}{}{
\uncover<1->{
Haga un arreglo que contenga 2 objetos del tipo Punto y dos objetos del tipo PuntoColor.
Muestre los valores en pantalla.
}}}}\end{frame}

\subsection{Polimorfismo}
\begin{frame}{Ejercicio : Polimorfismo}  
\ITZ{	
\uncover<1->{\ITT{1}{}{
\uncover<1->{
{\tiny{\lstinputlisting{./codigosEx/MainPolimorf.java}}}
}}}}\end{frame}

\subsection{Polimorfismo}
\begin{frame}{Conversion explicita} 
\ITZ{	
\uncover<1->{\ITT{1}{}{
\uncover<1->{
Los objetos derivados son compatibles con los objetos de clases ascendentes, pero no ocurre lo mismo inversamente.
{\tiny{\lstinputlisting{./codigosEx/test1.java}}}
}}}}\end{frame}

%\subsection{Polimorfismo}
%\begin{frame}{La palabra super} 
%\ITZ{	
%\uncover<1->{\ITT{1}{}{
%\uncover<1->{
%{\tiny{\lstinputlisting{./codigosEx/Super.java}}}
%Super llama al metodo de la clase ascendente. Sin embargo a.f() llama al metodo f de B como lo prevée el prolimorfismo. 
%}}}}\end{frame}

\subsection{Super clase Object}
\begin{frame}{La super clase Object} 
\ITZ{	
\uncover<1->{\ITT{1}{}{
\uncover<1->{
De la clase Object derivan todas las clases simples. Por ejemplo, la clase Punto :
 {\textbf{class Punto }} es lo mismo que decir :  {\textbf{ class Punto extends Object }}.
Una variable de tipo Object puede ser usada para referenciar un objeto de cualquier tipo :
 {\tiny{\lstinputlisting{./codigosEx/Obj.java}}}
 Esto es conveniente si necesitamos trasmitir a un método una referencia sin conocer su tipo. Evidentemente, cualquier se quiera aplicar un método particular de un objeto referenciado por una variable de tipo Object será necesario hacer una conversión explícita.
}}}}\end{frame}

\subsection{Super clase Object}
\begin{frame}{La super clase Object} 
\ITZ{	
\uncover<1->{\ITT{1}{}{
\uncover<1->{
Si la clase Punto tiene un método desplazar.
 {\tiny{\lstinputlisting{./codigosEx/Obj2.java}}}
 Si bien, el objeto referenciado por $o$ es de un tipo que contiene un método $f$, es necesario que dicho método exista en la clase Object.
 }}}}\end{frame}

\subsection{Super clase Object}
\begin{frame}{Metodos de la clase Object} 
\ITZ{	
\uncover<1->{\ITT{1}{toString}{
\uncover<1->{
Este método de la clase Object entrega una cadena de caracteres con :
\begin{itemize}
\item {el nombre de la clase}
\item{la dirección del objeto en hexadecimal precedido de $@$}
\end{itemize}
 {\tiny{\lstinputlisting{./codigosEx/ClaseToString.java}}}
 }}}}\end{frame}
 
 
\subsection{Super clase Object}
\begin{frame}{Metodos de la clase Object} 
\ITZ{	
\uncover<1->{\ITT{1}{toString}{
\uncover<1->{
\begin{itemize}
\item {El nombre de la clase: nombre de la clase del objeto que llama a toString.}
\item {Puede ser llamada automaticamente en el caso de necesitar una conversion implicita a cadena de caracteres. }
\end{itemize}
 }}}}\end{frame}

\subsection{Super clase Object}
\begin{frame}{Metodos de la clase Object} 
\ITZ{	
\uncover<1->{\ITT{1}{equals}{
\uncover<1->{
Este metodo compara las direcciones de dos objetos.\\
 {\textit{Object o1= new Punto(1,3);}}\\
 {\textit{Object o1= new Punto(1,3);}}\\
 {\textit{o1.equals(o2);}}\\
 
 Será false.
 
 }}}}\end{frame}


\subsection{Herencia}
\begin{frame}{Miembros protegidos} 
\ITZ{	
\uncover<1->{\ITT{1}{protected}{
\uncover<1->{
Tipo de acceso protected actua en :
\begin{itemize}
\item{empaquetamiento de clases}\item{clases derivadas}
\end{itemize}
Un miembro protegido es accesible a las clases del mismo paquete así que a las clases derivadas (que pueden pertenecer o no al mismo paquete).\\
Considere :\\ 
class A $\{$\\
protected int n;\} 
 }}}}\end{frame}
 
 \subsection{Herencia}
\begin{frame}{Herencia}
\GFXH{5cm}{ABCD.pdf} 
\begin{itemize}
\item{$B$ accede a $n$ de $A$}
\item{$D$ accede a $n$ de $B$ o de $A$}
\item{$C$ no accede a $n$ de $B$ (excepto si $B$ y $C$ están en el mismo paquete) }
\end{itemize}
\end{frame} 

\subsection{Herencia}
\begin{frame}{Clases y metodos finales} 
\ITZ{	
\uncover<1->{\ITT{1}{final}{
\uncover<1->{
Si se aplica la palabra clave final a variables locales o campos de una clase, esta prohibe la modificación de su valor. Esta palabra puede también aplicarse a una clase o método pero con una significación totalmente diferente.
\begin{itemize}
\item{Un método declarado final no puede ser sobreescrito en una clase derivada.}
\item {Una clase declarada final no puede puede ser derivada.}
\end{itemize} 
 }}}}\end{frame}
\subsection{Herencia}
\begin{frame}{Interfaces}
\ITZ{	
\uncover<1->{\ITT{1}{}{
\uncover<1->{
Un interface es una colección de declaraciones de métodos (sin definirlos)  y también puede incluir constantes.
Una interface define las cabeceras de un cierto numero de métodos. \begin {itemize} \item {Una clase podrá implementar diversas interfaces} \item{la noción de interface se superpone a la derivación} \item{las interfaces puedes derivarse} \item {es posible usar variables de tipo interface} \end{itemize} 
 }}}}\end{frame}
 
 \subsection{Herencia}
\begin{frame}{Definir una interface}
\ITZ{	
\uncover<1->{\ITT{1}{interfaces}{
\uncover<1->{
Una interface puede tener los mismo permisos que una clase.  Por esencia las interfaces son publicas y sus métodos son abstractos. 
{\tiny{\lstinputlisting{./codigosEx/Interfez.java}}}
}}}}\end{frame}
 
 \subsection{Herencia}
\begin{frame}{Implementar una interface}
\ITZ{	
\uncover<1->{\ITT{1}{interfaces}{
\uncover<1->{
Al definir una clase se puede precisar que implementa una interface dada usando la palabra clave $implements$. Una clase puede implementar diferentes interfaces.
{\tiny{\lstinputlisting{./codigosEx/Interfez2.java}}}
Realizar una programa que tenga una interface Afichar con un metodo mostrar. Dos clases Entero y Flotante implementando esta interface. Crear un arreglo heterogéneo de referencias de tipo Afichar que se llenaran con instancias de objetos Flotante y Entero.
}}}}\end{frame}
 
\subsection{Herencia}
\begin{frame}{Ejercicio}
\ITZ{	
\uncover<1->{\ITT{1}{}{
\uncover<1->{
{\tiny{\lstinputlisting{./codigosEx/Tabla2.java}}}
 }}}}\end{frame}

\subsection{Herencia}
\begin{frame}{Clases abstractas }
\ITZ{	
\uncover<1->{\ITT{1}{Clases abstractas}{
\uncover<1->{
Una clase abstracta es una clase que no permite instanciar objetos. Una clase abstracta permite definir en una clase base las funcionalidades comunes a todas las clases derivadas, si consideramos que una clase abstracta no implementa ningún método y ningún campo eso será una interface. \\Una clase abstracta puede servir solo como clase base, se usa como interface de las clases que harán conversión hacia arriba (Podemos describir el upcasting como la acción de declarar una variable de una clase base (abstracta en la mayoría de los casos), pero instanciando una implementación de la misma (comportamiento polifórmico).  }}}}\end{frame}
 
 
 
 \subsection{Herencia}
\begin{frame}{Clases abstractas }
\ITZ{	
\uncover<1->{\ITT{1}{Clases abstractas}{
\uncover<1->{
En una clase abstracta se pueden encontrar métodos y campos que heredan todas las clases derivadas. \\
$abstract$  $class$  $A$\\
Será posible declarar : 
$A$ $a; $, pero no sera posible $a=new$ $A();$
Sin embargo se puede derivar A e instanciar un objeto de la clase derivada.\\
$class$  $B$ $extends$ $A$\\
$A$ $a$ $=$ $new$ $B();$
 }}}}\end{frame}
 


\subsection{Herencia}
\begin{frame}{Métodos abstractos }
\ITZ{	
\uncover<1->{\ITT{1}{Métodos abstract}{
\uncover<1->{No se realiza la definición del método sino que solo que entrega la lista de argumentos de entrada y el tipo de valor de salida. Ejemplo : $ public $ $abstract$ $void$ $g$ $(int$ $n);$ }}}}\end{frame}

\subsection{Herencia}
\begin{frame}{Reglas de abstract}
\ITZ{	
\uncover<1->{\ITT{1}{abstract}{
\uncover<1->{
\begin{enumerate}
\item {Cuando una clase tiene uno o varios métodos abstractos, la clase es abstracta aunque no sea indicado explícitamente en su declaración.} \item{Un método abstracto debe obligatoriamente ser declarado publico, dado que se espera que sea sobrescrito en una clase derivada} \item{En la cabecera de la declaración de un método abstracto los nombres de los argumentos deben figurar} \item{Una clase derivada de una clase abstracta no esta obligada a sobreescribir todos los métodos abstractos (puede no sobreescribirlos)} \item{Una clase derivada de una clase abstracta puede ser definida abstracta} \end{enumerate} }}}}\end{frame}

\subsection{Herencia}
\begin{frame}{Interes de abstract}
\ITZ{	
\uncover<1->{\ITT{1}{abstract}{
\uncover<1->{
Las clases abstractas facilitan la concepción OO dado que en una clase abstracta se pueden poner todas las funcionalidades que se desea disponer para todas las clases descendientes. Esto puede ser a través de : \begin{itemize} \item{implementando completamente los métodos y campos cuando son comunes a todos los descendientes} \item{como interface de métodos abstractos}
\end{itemize} }}}}\end{frame}
\subsection{Herencia}
\begin{frame}{Ejercicio}
\ITZ{	
\uncover<1->{\ITT{1}{}{
\uncover<1->{
Declare una clase abstracta Afichar con un metodo abstracto mostrar. Dos clases Entero y Flotante derivan de esta clase. El metodo main usa un arreglo heterogéneo de objetos de tipo Afichar que completa instanciando objetos de tipo Entero y Flotante.}
 }}}\end{frame}

\subsection{Herencia}
\begin{frame}{Ejercicio}
\ITZ{	
\uncover<1->{\ITT{1}{}{
\uncover<1->{
{\tiny{\lstinputlisting{./codigosEx/Tabla.java}}}
 }}}}\end{frame}


\subsection{Herencia}
\begin{frame}{Clase abstract e interfaces}
\ITZ{	
\uncover<1->{\ITT{1}{Resumen}{
\uncover<1->{\begin{itemize} 
\item {Clase Abstracta : \begin{itemize} \item{Contiene tanto métodos ejecutables como métodos abstractos} \item{Una clase puede extender solo una clase}\item{Puede tener variables de instancia, constructores y cualquier tipo de visibilidad (public, private, protected)} \end{itemize}}

\item{Interface: \begin{itemize} \item{No contiene código de implementación}\item{Una clase puede implementar $n$ numero de interfaces} \item {No puede tener variables de instancia, constructores y solo puede tener métodos publicos o package}\end{itemize}}
\end{itemize}}}}}\end{frame}

%\graphicspath{{./pics/}}
 
\section{Programación Gráfica}
\subsection{Programación Gráfica}
\begin{frame}{Programación Gráfica}
\ITZ{	
\uncover<1->{\ITT{1}{}{\begin{itemize}
\uncover<1->
\item {Ventana}
\item {Clic en una ventana}
\item {Primer componente : un botón}
\item {Componentes}
\item {Primer dibujo}
\item {Manejo de dimensiones}
\end{itemize}}}}
\end{frame}

\subsection{Programación Gráfica}
\begin{frame}{Programación Gráfica}
\ITZ{	
\uncover<1->{\ITT{1}{}{
\uncover<1->{
Para crear una ventana gráfica disponemos en el paquete llamado \emph{javax.swing}, de una clase clase \emph{JFrame} que posee un constructor sin argumentos.

\emph {JFrame ven = new JFrame(); }\\

Esto crea un objeto del tipo JFrame dejando su referencia en ven. 
\begin{itemize}
\item{Para hacer visible esta ventana : \emph {ven.setVisible(true); }}
\item {Para darle el tamaño a la ventana :  \emph {ven.setSize(300,150); }, altura de 150 pixeles y largo de 300.}
\item{Para mostrar un texto en la barra de titulo : \emph {ven.setTitle("Mi primera ventana"); }}
\end{itemize}
Cree un programa que cree una ventana, la visualice, le de un tamaño y le agregue un titulo.
}}}}
\end{frame}

\subsection{Programación Gráfica}
\begin{frame}{Programación Gráfica}
\ITZ{	
\uncover<1->{\ITT{1}{}{
\uncover<1->{
{\tiny{\lstinputlisting{./codigosEx/Grafica.java}}}
}}}}
\end{frame}

\subsection{Programación Gráfica}
\begin{frame}{Programación Gráfica}
\ITZ{	
\uncover<1->{\ITT{1}{Ventana}{
\uncover<1->{
El usuario podrá  
\begin{itemize} \item{cambiar el tamaño de la ventana} \item{desplazar la ventana} \item{reducirla} \item{maximizarla}\end{itemize}
}}}}\end{frame}

\subsection{Programación Gráfica}
\begin{frame}{Programación Gráfica}
\ITZ{	
\uncover<1->{\ITT{1}{Parar el programa}{
\uncover<1->{
Una vez que el método main a llegado a su final la ventana gráfica sigue abierta. Dado que un programa java puede contar con diversos procesos independientes llamados \emph{threads}. \\
Aqui el \emph{threads} principal corresponde al metodo main y un \emph{threads} utilizador lanza la ventana grafica. Cuando ha terminado el metodo main solo el  \emph{threads} principal se interrumpe.\\
Cerrar la ventana grafica no finalizara el \emph{threads}. Mas adelante veremos como finalizar este \emph{threads}.\\
Use : \emph{ CTL C} para finalizar.
}}}}\end{frame}

\subsection{Programación Gráfica}
\begin{frame}{Programación Gráfica}
\ITZ{	
\uncover<1->{\ITT{1}{Creación de una ventana personalizada}{
\uncover<1->{
Anteriormente hemos creado un objeto JFrame y utilizado sus funcionalidades.\\
Para  personalizar la ventana y asociarle campos o funcionalidades suplementarias. Para eso será necesario definir una clase derivada de JFrame y crear un objeto de este nuevo tipo. \\
Transforme el programa desarrollado anteriormente (sin agregar aun nuevas funcionalidades) creando una clase derivada de JFrame.
}}}}\end{frame}

\subsection{Programación Gráfica}
\begin{frame}{Programación Gráfica}
\ITZ{	
\uncover<1->{\ITT{1}{a la clase base}{
\uncover<1->{
{\tiny{\lstinputlisting{./codigosEx/Grafica1.java}}}
}}}}\end{frame}

\subsection{Programación Gráfica}
\begin{frame}{Programación Gráfica}
\ITZ{	
\uncover<1->{\ITT{1}{Acciones sobre las características de una ventana}{
\uncover<1->{
Naturalmente será posible cambiar los atributos de la ventana por ejemplo : a partir de datos ingresados por el usuario.
Otros métodos : 
\begin {enumerate} \item{Posición de la ventana y sus dimensiones : \emph{setBounds(10,40,300,200);} con esta instrucción la esquina superior izquierda de la ventana esta en el pixel $10,40$ y sus dimensiones son $300*200$.}
\item{Modificar el color de fondo \emph{setBackground (Color.red) (import java.awt.Color;)}}
\item{Obtener el tamaño actual \emph{getsize()}}
\end{enumerate}
}}}}\end{frame}


\subsection{Programación Gráfica}
\begin{frame}{Programación Gráfica}
\ITZ{	
\uncover<1->{\ITT{1}{Acciones sobre las características de una ventana}{
\uncover<1->{
Realice un programa que cree ventanas con dimensiones y color definido por el usuario.\\
Cree una clase derivada de JFrame con un titulo y dimensiones iniciales y luego solicite al usuario que las redefina y que ingrese el color de fondo de la ventana.
}}}}\end{frame}

\subsection{Programación Gráfica}
\begin{frame}{Programación Gráfica}
\ITZ{	
\uncover<1->{\ITT{1}{Acciones sobre las características de una ventana}{
\uncover<1->{
{\tiny{\lstinputlisting{./codigosEx/Grafica2.java}}}
}}}}\end{frame}

\subsection{Programación Gráfica}
\begin{frame}{Programación Gráfica}
\ITZ{	
\uncover<1->{\ITT{1}{Acciones sobre las características de una ventana}{
\uncover<1->{
{\tiny{\lstinputlisting{./codigosEx/Grafica3.java}}}
}}}}\end{frame}

\subsection{Programación Gráfica}
\begin{frame}{Programación Gráfica}
\ITZ{	
\uncover<1->{\ITT{1}{Acciones sobre las características de una ventana}{
\uncover<1->{
{\tiny{\lstinputlisting{./codigosEx/Grafica31.java}}}
}}}}\end{frame}

\subsection{Programación Gráfica}
\begin{frame}{Programación Gráfica}
\ITZ{	
\uncover<1->{\ITT{1}{Cerrar ventanas}{
\uncover<1->{
Si el usuario cierra la ventana gráfica simplemente la deja invisible, semejante a decir \emph{setVisible(false)}. Podríamos usar el método \emph{setDefaultCloseOperation} con alguno de los argumentos siguientes :
\begin{itemize}
\item{\emph{$DO\_NOTHING\_ON\_CLOSE$}: no hacer nada.}
\item{\emph{$HIDE\_ON\_CLOSE$}: ocultar la ventana (por defecto).}
\item{\emph{$DISPOSE\_ON\_CLOSE$}: destruir el objeto ventana.}
\item{\emph{$EXIT\_ON\_CLOSE$} : sale de la aplicación usando el metodo system exit.}
\end{itemize}
}}}}\end{frame}

\subsection{Programación Gráfica}
\begin{frame}{Programación Gráfica}
\ITZ{	
\uncover<1->{\ITT{1}{Clic en una ventana}{
\uncover<1->{
La programación gráfica se basa en eventos que son creados por componentes que se introducen en la ventana gráfica como por ejemplo : menú, botones, ....\\
En java todos los eventos tienen una fuente, es decir, un objeto botón, menú, ventana, ...otro. Por ahora será la ventana principal.\\
Para tratar un evento se le asocia a la fuente un objeto, la clase implementa una interfaz particular que corresponde a una categoría de eventos.  Decimos así que este objeto es un escuchador o \emph{listener} de esa categoría de eventos. Los escuchadores son interfaces de java. %Cada método propuesto por la interfaz corresponde a un evento de la categoría. \\
}}}}\end{frame}


\subsection{Programación Gráfica}
\begin{frame}{Programación Gráfica}
\ITZ{	
\uncover<1->{\ITT{1}{Clic en una ventana}{
\uncover<1->{
Por ejemplo : Existe una categoría \emph{evento mouse} que se puede tratar con un objeto de esa clase implementando la interfaz \emph{MouseListener}. Esta tiene cinco métodos : 
\begin{enumerate}
\item{\emph{mousePressed}}
\item{\emph{mouseReleased}}
\item{\emph{mouseEntered}}
\item{\emph{mouseExited}}
\item{\emph{mouseClicked}}
\end{enumerate}
}}}}\end{frame}

\subsection{Programación Gráfica}
\begin{frame}{Programación Gráfica}
\ITZ{	
\uncover<1->{\ITT{1}{Clic en una ventana}{
\uncover<1->{
Una clase susceptible de instanciar un objeto escuchador de los diferentes eventos deberá corresponder al esquema siguiente :
{\tiny{\lstinputlisting{./codigosEx/Grafica4.java}}}
Dado que a interfaz \emph{MouseListener} tiene 5 métodos será necesario redefinir todos los métodos aunque se dejen vacíos.
}}}}\end{frame}

\subsection{Programación Gráfica}
\begin{frame}{Programación Gráfica}
\ITZ{	
\uncover<1->{\ITT{1}{Click en una ventana}{
\uncover<1->{
Para tratar un click en la ventana, será necesario redefinir no vacío el metodo \emph{mouseCliked}. A cada método se le asociar un objeto del tipo \emph{MouseEvent} y a la clase se le asocia un objeto del tipo \emph{MouseListener} usando \emph{addMouseListener(objetListener)}, \emph{objetListener} es un objeto de una clase de tipo \emph{MouseListener}
Podemos incluir este objeto en el constructor de la siguiente manera :
{\tiny{\lstinputlisting{./codigosEx/Ventana0.java}}}
%{\tiny{\lstinputlisting{./codigosEx/Grafica5.java}}}
}}}}\end{frame}

\subsection{Programación Gráfica}
\begin{frame}{Programación Gráfica}
\ITZ{	
\uncover<1->{\ITT{1}{Click en una ventana}{
\uncover<1->{
Haga un programa que escriba un mensaje en consola cada vez que el usuario haga click en la ventana. \\ Para esto debe importar el paquete \emph{java.awt.event} que gestionara los eventos. \\
El método \emph{mouseClicked} debe ser publico dado que una clase no puede restringir el acceso de métodos ya implementados.
}}}}\end{frame}


\subsection{Programación Gráfica}
\begin{frame}{Programación Gráfica}
\ITZ{	
\uncover<1->{\ITT{1}{Click en una ventana}{
\uncover<1->{
{\tiny{\lstinputlisting{./codigosEx/Ventana.java}}}
}}}}\end{frame}

\subsection{Programación Gráfica}
\begin{frame}{Programación Gráfica}
\ITZ{	
\uncover<1->{\ITT{1}{Usar la información asociada al evento}{
\uncover<1->{
El argumento del método \emph{mouseClicked} es un objeto de tipo : \emph{MouseEvent}. Esta clase corresponde a la categoria de eventos manejados por la interfaz \emph{MouseListener}. \\
Java crea un objeto de esa clase automaticamente luego de un click y lo envia a escuchador deseado. Este contiene diversas informaciones, en particular las coordenadas del \emph{mouse} que son accesibles a traves de los metodos \emph{getX} y \emph{getY}.\\
Adapte el programa precedente desplegando las coordenadas del \emph{mouse} al hacer click en la ventana.
}}}}\end{frame}

\subsection{Programación Gráfica}
\begin{frame}{Programación Gráfica}
\ITZ{	
\uncover<1->{\ITT{1}{Click en una ventana}{
\uncover<1->{
{\tiny{\lstinputlisting{./codigosEx/Ventana1.java}}}
}}}}\end{frame}

\subsection{Programación Gráfica}
\begin{frame}{Programación Gráfica}
\ITZ{	
\uncover<1->{\ITT{1}{Resumen gestión de eventos}{
\uncover<1->{
Un evento generado por una fuente es tratado por otro objeto llamado escuchador asociado previamente a la fuente. El objeto escuchador podrá ser cualquier objeto incluso un objeto fuente.\\
En particular a una categoria dada $Xxx$ se le asocia siempre un objeto escuchador de eventos del tipo $XxxEvent$  usando el metodo \emph{addXxxListener}. Cada vez que una categoria dispone de varios metodos podemos :
\begin{itemize} \item{redefinir todos los metodos de la interfaz correspondiente \emph{XxxListener} ($implements$ debe figurar en la cabeza de la clase del escuchador), ciertos metodos podrán ser definidos vacíos.}
\item{ llamar a una clase derivada de una clase adaptador \emph{XxxAdapter} y definir solo los métodos que nos interesan}
\end{itemize}
}}}}\end{frame}

\subsection{Programación Gráfica}
\begin{frame}{Programación Gráfica}
\ITZ{	
\uncover<1->{\ITT{1}{Primer componente  : un botón }{
\uncover<1->{
Para crear un botón usamos el constructor de la clase \emph{JButton} : \\
\emph{JButton miBoton;}\\
\emph {miBoton = new JButton("Mi Primer Boton");}\\
Un objeto del tipo \emph{JFrame} esta formado de : \begin{itemize} \item{una raiz} \item{un contenido} \item{un vidrio} \end{itemize}
En el contenido incluiremos los diferentes componentes. 
}}}}\end{frame}

\subsection{Programación Gráfica}
\begin{frame}{Programación Gráfica}
\ITZ{	
\uncover<1->{\ITT{1}{Primer componente : un botón }{
\uncover<1->{
El metodo \emph{getContentPane} de la clase \emph{JFrame} referencia al contenido de tipo \emph{Container}.\\
\emph{Container c = getContentPane();}\\
El metodo $add$ de la clase Container permitira agregar un componente a un objeto de ese tipo. \\ Para agregar el botón al contenido del objeto de tipo Container debemos : \\\emph{c.add(miBoton);}\\
Una manera condensada sería : \\ \emph {getContentPane().add(miBoton);}
}}}}\end{frame}

\subsection{Programación Gráfica}
\begin{frame}{Programación Gráfica}
\ITZ{	
\uncover<1->{\ITT{1}{Primer componente : un botón }{
\uncover<1->{
El constructor de la ventana con un botón sería : 
{\tiny{\lstinputlisting{./codigosEx/Boton0.java}}}
}}}}\end{frame}

\subsection{Programación Gráfica}
\begin{frame}{Programación Gráfica}
\ITZ{	
\uncover<1->{\ITT{1}{Primer componente : un botón }{
\uncover<1->{
Un botón  es visible por defecto. Al mostrar la ventana vemos que el botón esta pero ocupa todo el espacio disponible. La disposición de componentes en una ventana la organiza un gestor de disposición (\emph {Layout Manager}). \\ Existen diversos gestores (en forma de clase). Por defecto Java usa \emph{BorderLayout} con el cual, si no hay otra información, un componente ocupa toda la ventana. \\
Un gestor interesante es \emph{FlowLayout} que dispone de diversos componentes mostrandolos como texto, uno después de otro en una linea y luego en la siguiente. \\
}}}}\end{frame}

\subsection{Programación Gráfica}
\begin{frame}{Programación Gráfica}
\ITZ{	
\uncover<1->{\ITT{1}{Primer componente : un botón }{
\uncover<1->{
Para elegir el gestor simplemente se aplica el metodo \emph{setLayout} al objeto contenido de la ventana (al igual que el componente botón). \\
Para obtener el gestor : \\
\emph{getContentPane().setLayout(new Layout());}\\
Escriba un programa que cree un botón en la ventana gráfica.
}}}}\end{frame}

\subsection{Programación Gráfica}
\begin{frame}{Programación Gráfica}
\ITZ{	
\uncover<1->{\ITT{1}{Primer componente : un botón }{
\uncover<1->{
{\tiny{\lstinputlisting{./codigosEx/MainBoton.java}}}
}}}}\end{frame}

\subsection{Programación Gráfica}
\begin{frame}{Programación Gráfica}
\ITZ{	
\uncover<1->{\ITT{1}{Acción sobre un botón : un evento}{
\uncover<1->{
De las misma manera que con la ventana debemos :
\begin{itemize}
\item {Crea un escuchador que será un objeto de una clase que implemente la interfaz \emph{ActionListener}. Esta interfaz tiene 1 método en la categoría \emph{Action} llamado \emph{actionPerformed}.}
\item{Asociar el escuchador al botón por medio del metodo \emph{addActionListener}   }
\end{itemize}
Modifique el programa anterior para que muestre un mensaje cada vez que haga click sobre el botón.
}}}}\end{frame}

\subsection{Programación Gráfica}
\begin{frame}{Programación Gráfica}
\ITZ{	
\uncover<1->{\ITT{1}{Acción sobre un botón : un evento}{
\uncover<1->{
{\tiny{\lstinputlisting{./codigosEx/MainBotonClick.java}}}
}}}}\end{frame}

\subsection{Programación Gráfica}
\begin{frame}{Programación Gráfica}
\ITZ{	
\uncover<1->{\ITT{1}{Resumen : Acción en un botón }{
\uncover<1->{
\begin {itemize}
\item{Para actuar sobre un componente a partir del teclado este debe estar seleccionado. Solo un componente puede estar seleccionado a la vez. Incluso una simple acción en la barra de espacio mostrara el mensaje.}
\item{La categoria de eventos \emph{Action} tiene un solo metodo \emph{actionPerformed}.}
\end{itemize}
}}}}\end{frame}

\subsection{Programación Gráfica}
\begin{frame}{Programación Gráfica}
\ITZ{	
\uncover<1->{\ITT{1}{Gestión de multiples componentes }{
\uncover<1->{
Si queremos agregar nuevos botones usando el gestor \emph{FlowLayout}, los botones se muestran secuencialmente en el orden que se agregan. \\
En lo que respecta a la gestión de acciones sobre los botones cada acción sobre un componente puede disponer de su propio objeto escuchador.\\
Para verificar la fuente de un evento puede usar los método \emph{getSource} o \emph{getActionCommand}\\
Modifique el código anterior para que cree 2 botones que realicen la misma acción (mostrar un mensaje en pantalla)
}}}}\end{frame}
\subsection{Programación Gráfica}
\begin{frame}{Programación Gráfica}
\ITZ{	
\uncover<1->{\ITT{1}{Acciones sobre dos botones : un evento}{
\uncover<1->{
{\tiny{\lstinputlisting{./codigosEx/Main2BotonClick.java}}}
}}}}\end{frame}

\subsection{Programación Gráfica}
\begin{frame}{Programación Gráfica}
\ITZ{	
\uncover<1->{\ITT{1}{Acciones sobre dos botones : dos eventos}{
\uncover<1->{
Obtenga la fuente del evento usando \emph{getSource}
{\tiny{\lstinputlisting{./codigosEx/Main2Boton2Click.java}}}
}}}}\end{frame}

\subsection{Programación Gráfica}
\begin{frame}{Programación Gráfica}
\ITZ{	
\uncover<1->{\ITT{1}{Acciones sobre dos botones : getSource}{
\uncover<1->{
{\tiny{\lstinputlisting{./codigosEx/Main2Boton2Click1.java}}}
}}}}\end{frame}

\subsection{Programación Gráfica}
\begin{frame}{Programación Gráfica}
\ITZ{	
\uncover<1->{\ITT{1}{Acciones sobre dos botones : getSource}{
\uncover<1->{
El método \emph{getSource} nos permite identificar la fuente de un evento (para esto debemos aplicarlo a cada botón).\\
El método \emph{getActionCommand} presente solamente en la clase \emph{ActionEvent} permite obtener la cadena de caracteres asociada a la fuente de un evento. \\ Aplique este método al programa anterior para capturar la cadena de caracteres asociada a los botones 1 y 2. Por defecto la cadena de comando asociada a un botón es su etiqueta.\\
Para imponer una cadena de comando debemos usar el metodo \emph{setActionCommand} 
}}}}\end{frame}

\subsection{Programación Gráfica}
\begin{frame}{Programación Gráfica}
\ITZ{	
\uncover<1->{\ITT{1}{Acciones sobre dos botones : getActionCommand y setActionCommand}{
\uncover<1->{
{\tiny{\lstinputlisting{./codigosEx/Main2Boton2Click2.java}}}
}}}}\end{frame}


\subsection{Programación Gráfica}
\begin{frame}{Programación Gráfica}
\ITZ{	
\uncover<1->{\ITT{1}{Clase escuchador diferente de la ventana}{
\uncover<1->{
Escuchador diferente de la ventana : \begin{itemize} \item{una clase escuchador por botón} \item {una clase escuchador para todos los botones}
\end{itemize}
}}}}\end{frame}

\subsection{Programación Gráfica}
\begin{frame}{Programación Gráfica}
\ITZ{	
\uncover<1->{\ITT{1}{Una clase escuchador por botón}{
\uncover<1->{
{\tiny{\lstinputlisting{./codigosEx/Main2Escuchador.java}}}
}}}}\end{frame}

\subsection{Programación Gráfica}
\begin{frame}{Programación Gráfica}
\ITZ{	
\uncover<1->{\ITT{1}{Una clase escuchador por botón}{
\uncover<1->{
{\tiny{\lstinputlisting{./codigosEx/Main2Escuchador1.java}}}
}}}}\end{frame}

\subsection{Programación Gráfica}
\begin{frame}{Programación Gráfica}
\ITZ{	
\uncover<1->{\ITT{1}{Una clase escuchador para los dos botones}{
\uncover<1->{
Podemos disponer un solo método \emph{actionPerformed} común a los dos botones. Para identificar el botón de la acción podemos usar \emph{getActionCommand}.
}}}}\end{frame}

\subsection{Programación Gráfica}
\begin{frame}{Programación Gráfica}
\ITZ{	
\uncover<1->{\ITT{1}{Una clase escuchador para los botones}{
\uncover<1->{
{\tiny{\lstinputlisting{./codigosEx/Main1Escuchador.java}}}
}}}}\end{frame}

\subsection{Programación Gráfica}
\begin{frame}{Programación Gráfica}
\ITZ{	
\uncover<1->{\ITT{1}{Dinámica de componentes}{
\uncover<1->{
En los ejemplos precedentes los botones fueron creados al mismo tiempo que la ventana y quedaban visibles y activos. \\
Como : \begin{itemize} \item{ crear un nuevo componente} \item{suprimir un componente} \item{desactivar un componente}\item {reactivar un componente} \end{itemize}
}}}}\end{frame}

\subsection{Programación Gráfica}
\begin{frame}{Programación Gráfica}
\ITZ{	
\uncover<1->{\ITT{1}{Dinámica de componentes}{
\uncover<1->{
\begin {enumerate}
\item {Para {\textbf{crear}} un nuevo componente sabemos que debemos usar el metodo \emph{add}, pero en el caso que la ventana ya este visible, será necesario decirle al gestor que recalcule la posición de los componentes en la ventana usando : \begin {itemize} \item{\emph{revalidate} para el componente } \item {\emph{validate} para su contenido} \end {itemize}}
\item {Para {\textbf{suprimir}} un componente usamos el metodo \emph{remove} y volvemos a llamar a \emph{validate}.}
\item {Para {\textbf{desactivar}} usaremos el metodo \emph{content.setEnabled (false)}, para {\textbf{reactivar}} usaremos  \emph{content.setEnabled (true)}, para saber si el componente  {\textbf{esta activo}} usaremos \emph{content.isEnabled ()}}
\end{enumerate}
}}}}\end{frame}

\subsection{Programación Gráfica}
\begin{frame}{Programación Gráfica}
\ITZ{	
\uncover<1->{\ITT{1}{Dinámica de componentes : Crear botones}{
\uncover<1->{
{\tiny{\lstinputlisting{./codigosEx/MainCrearBoton.java}}}
}}}}\end{frame}

\subsection{Programación Gráfica}
\begin{frame}{Programación Gráfica}
\ITZ{	
\uncover<1->{\ITT{1}{Dinámica de componentes}{
\uncover<1->{
Desarrolle un programa que muestre un numero dado por el usuario de botones.\\
El usuario al hacer click sobre un botón lo desactiva.\\
Imprima : \begin{itemize} \item {un mensaje en la consola para especificar cual botón desactivo} \item { 
 un mensaje en la consola con el estado de los botones.} \end{itemize}
}}}}\end{frame}

\subsection{Programación Gráfica}
\begin{frame}{Programación Gráfica}
\ITZ{	
\uncover<1->{\ITT{1}{Dinámica de componentes : Activar-Desactivar botones}{
\uncover<1->{
{\tiny{\lstinputlisting{./codigosEx/MainCrearBoton2.java}}}
}}}}\end{frame}

\subsection{Programación Gráfica}
\begin{frame}{Programación Gráfica}
\ITZ{	
\uncover<1->{\ITT{1}{Dinámica de componentes : Activar-Desactivar botones}{
\uncover<1->{
{\tiny{\lstinputlisting{./codigosEx/MainCrearBoton21.java}}}
}}}}\end{frame}

\subsection{Programación Gráfica}
\begin{frame}{Programación Gráfica}
\ITZ{	
\uncover<1->{\ITT{1}{Mi primer dibujo}{
\uncover<1->{
%Para lograr visualizar los dibujos debe poner los dibujos en método particular del componente llamado \emph{paintComponent}, este método será llamado automáticamente por Java cada vez que el componente deba será dibujado o redibujado.\\
%La ventana de clase JFrame (o derivada) posee un metodo de dibujo llamado \emph{paint} que no es llamado automaticamente como \emph{paintComponent}.\\
Para realizar un dibujo en una ventana lo mas recomendable será evitar dibujar directamente en la ventana (JFrame) y usar un panel, es decir, un objeto de la clase \emph{JPanel}. \\Los paneles pueden estar en un contenedor y ser a la vez contenedor de otros componentes (no como los botones que no pueden contener otros componentes). \\ Un panel es como una \emph{subventana} sin titulo no bordes, es simplemente un rectángulo del color de la ventana. \\ Un panel no puede existir de manera autonoma (como una ventana) deberá ser asociado por el metodo \emph{add} a un contenedor.}
}}}\end{frame}

\subsection{Programación Gráfica}
\begin{frame}{Programación Gráfica}
\ITZ{	
\uncover<1->{\ITT{1}{Mi primer dibujo- Crear un panel}{
\uncover<1->{
{\tiny{\lstinputlisting{./codigosEx/MainPanel.java}}}
}}}}\end{frame}



\subsection{Programación Gráfica}
\begin{frame}{Programación Gráfica}
\ITZ{	
\uncover<1->{\ITT{1}{Mi primer dibujo- Crear un panel}{
\uncover<1->{
Con el código anterior si bien creamos un panel no lo vemos ya que por defecto es del mismo color y tamaño que la ventana. Usando \emph{setBackground() cambie el color del panel para que sea visible.}
}}}}\end{frame}

\subsection{Programación Gráfica}
\begin{frame}{Programación Gráfica}
\ITZ{	
\uncover<1->{\ITT{1}{Mi primer dibujo- Crear un panel}{
\uncover<1->{
{\tiny{\lstinputlisting{./codigosEx/MainPanel0.java}}}
}}}}\end{frame}

\subsection{Programación Gráfica}
\begin{frame}{Programación Gráfica}
\ITZ{	
\uncover<1->{\ITT{1}{Mi primer dibujo - Dibujar en el panel}{
\uncover<1->{
Para obtener un dibujo permanente en un componente es necesario redefinir el metodo \emph{paintComponent} que será llamado automaticamente por Java cada vez que un componente deba ser redibujado.\\
Como debemos redefinir un metodo de la clase \emph{JPanel} debemos crear el panel como un objeto de una clase derivada de \emph{JPanel}.\\
La cabecera del metodo \emph{paintComponent} a redefinir es :\\
\emph{void paintComponent (Graphics g) }\\
La clase \emph{Graphics} encapsula todas las informaciones y métodos para dibujar sobre un componente (color de fondo, color de la linea, estilo, tipos de letra, tamaño, ...). 
}}}}\end{frame}

\subsection{Programación Gráfica}
\begin{frame}{Programación Gráfica}
\ITZ{	
\uncover<1->{\ITT{1}{Mi primer dibujo- Dibujar una linea en el panel}{
\uncover<1->{
Si queremos dibujar una linea en el panel debemos solamente llamar a metodo \emph{drawLine} usando el objeto \emph{g}.\\
\emph{g.drawLine(15,10,100,50)} \\
Esta instrucción hace una linea de punto 15,10 al punto 15$+$100, 10$+$50. \\
Las coordenadas se expresan en pixeles relativas a la esquina superior izquierda del componente.
}}}}\end{frame}

\subsection{Programación Gráfica}
\begin{frame}{Programación Gráfica}
\ITZ{	
\uncover<1->{\ITT{1}{Dibujar una linea en el panel}{
\uncover<1->{
Para trabajar con el metodo \emph{paintComponent} debemos llamarlos explicitamente de la clase ascendente \emph{JPanel}.\\
\emph{super.paintComponent(g)} \\
Desarrolle un programa que muestre una linea en un panel rojo que ocupa toda la ventana.
}}}}\end{frame}

\subsection{Programación Gráfica}
\begin{frame}{Programación Gráfica}
\ITZ{	
\uncover<1->{\ITT{1}{Mi primer dibujo - Crear un panel}{
\uncover<1->{
{\tiny{\lstinputlisting{./codigosEx/MainPanel1.java}}}
}}}}\end{frame}

\subsection{Programación Gráfica}
\begin{frame}{Programación Gráfica}
\ITZ{	
\uncover<1->{\ITT{1}{Dibujar en el panel}{
\uncover<1->{
Si se quiere dibujar y redibujar directamente en un panel se debe usar el metodo \emph{repaint} que llama al método \emph{paintComponent} actualizando el contenido del panel.\\
Dentro del gestor de contenido por defecto llamado \emph{BorderLayout} (que usa todo el espacio para insertar un componente) existe la posibilidad de dejar los componente no solo al centro sino que tambien sobre los 4 bordes de la ventana. \\Para eso solo será necesario en el metodo \emph{add} un argumento \emph {"North", "South","East" o "West" }.
}}}}\end{frame}


\subsection{Programación Gráfica}
\begin{frame}{Programación Gráfica}
\ITZ{	
\uncover<1->{\ITT{1}{Dibujar en el panel}{
\uncover<1->{
Desarrolle un programa que tenga 2 botones (uno en la parte de arriba y otro en la parte de abajo del contenido de la ventana). \\El primer boton dibujará un circulo en un panel y el segundo dibujara en el mismo panel un rectángulo que remplazara eventualmente el circulo. \\Al inicio del programa no se muestra nada en el panel. \\El panel ocupara la parte libre del contenido (el centro). 
}}}}\end{frame}

\subsection{Programación Gráfica}
\begin{frame}{Programación Gráfica}
\ITZ{	
\uncover<1->{\ITT{1}{Mi primer dibujo - Ejercicio}{
\uncover<1->{
{\tiny{\lstinputlisting{./codigosEx/MainPanelDibujo0.java}}}
}}}}\end{frame}

\subsection{Programación Gráfica}
\begin{frame}{Programación Gráfica}
\ITZ{	
\uncover<1->{\ITT{1}{Mi primer dibujo - Ejercicio}{
\uncover<1->{
{\tiny{\lstinputlisting{./codigosEx/MainPanelDibujo1.java}}}
}}}}\end{frame}

\subsection{Programación Gráfica}
\begin{frame}{Programación Gráfica}
\ITZ{	
\uncover<1->{\ITT{1}{Campos de texto}{
\uncover<1->{
Los campos de texto son una zona rectangular en la cual el usuario puede entrar o modificar texto (de una sola linea).\\
Se obtienen instanciando un objeto \emph{JTextField}. \\Su constructor debe obligatoriamente indicar un tamaño (que indica el numero de caracteres, estos caracteres dependen del tipo de letra). \\
La construcción de campos de texto en un objeto del tipo JFrame : \\
\emph{JTextField entrada1,entrada2;}\\
\emph{entrada1 = new JTextField(20);}\\
\emph{entrada2 = new JTextField("texto inicial",15);}\\
}}}}\end{frame}


\subsection{Programación Gráfica}
\begin{frame}{Programación Gráfica}
\ITZ{	
\uncover<1->{\ITT{1}{Campos de texto}{
\uncover<1->{
Para obtener la información que figura en un campo de texto usamos :\\ 
\emph{String cadena = entrada1.getText();}\\
Para definir si el campo de texto será editable usaremos : \\
\emph{entrada1.setEditable(false);}\\
\emph{entrada1.setEditable(true);}\\
Para modificar durante la ejecución usarmos (como ya lo vimos anteriormente):\\
\emph{entrada1.revalidate();}\\
}}}}\end{frame}


\subsection{Programación Gráfica}
\begin{frame}{Programación Gráfica}
\ITZ{	
\uncover<1->{\ITT{1}{Ejercicio - Campos de texto}{
\uncover<1->{
Desarrolle un programa que le proponga al usuario un campo de texto y un botón con la etiqueta "Copiar". \\Cada acción sobre el botón provoca la copia en un segundo campo de texto (no editable) del contenido del primer campo de texto.
}}}}\end{frame}

\subsection{Programación Gráfica}
\begin{frame}{Programación Gráfica}
\ITZ{	
\uncover<1->{\ITT{1}{Ejercicio - Campos de texto}{
\uncover<1->{
{\tiny{\lstinputlisting{./codigosEx/MainPanelTexto0.java}}}
}}}}\end{frame}


\subsection{Programación Gráfica}
\begin{frame}{Programación Gráfica}
\ITZ{	
\uncover<1->{\ITT{1}{Ejercicio - Campos de texto}{
\uncover<1->{
{\tiny{\lstinputlisting{./codigosEx/MainPanelTexto1.java}}}
}}}}\end{frame}

\subsection{Programación Gráfica}
\begin{frame}{Programación Gráfica}
\ITZ{	
\uncover<1->{\ITT{1}{Ejercicio - Campos de texto}{
\uncover<1->{
Desarrolle un programa que repita las acciones realizadas en un campo de texto en otro. De alguna manera un campo de texto espejo.\\
Use un objeto del tipo \emph{Document} para conservar el contenido del componente. Las modificaciones al objeto \emph{Document} generan uno de los 3 eventos de la categoria \emph{Document}. El escuchador sera \emph{DocumentListener}\\
Encabezados de metodos: \\
\emph {public void insertUpdate(DocumentEvent ev)}\\
\emph {public void removeUpdate(DocumentEvent ev)}\\
\emph {public void changedUpdate(DocumentEvent ev)}
}}}}\end{frame}

\subsection{Programación Gráfica}
\begin{frame}{Programación Gráfica}
\ITZ{	
\uncover<1->{\ITT{1}{Ejercicio - Campos de texto}{
\uncover<1->{
{\tiny{\lstinputlisting{./codigosEx/MainPanel2Texto0.java}}}
}}}}\end{frame}
\subsection{Programación Gráfica}
\begin{frame}{Programación Gráfica}
\ITZ{	
\uncover<1->{\ITT{1}{Mi primer dibujo - Ejercicio}{
\uncover<1->{
{\tiny{\lstinputlisting{./codigosEx/MainPanel2Texto1.java}}}
}}}}\end{frame}

\subsection{Programación Gráfica}
\begin{frame}{Programación Gráfica}
\ITZ{	
\uncover<1->{\ITT{1}{Listas}{
\uncover<1->{
Las listas son componente que permiten elegir uno o varios valores en una lista predefinida.
Para crear una lista se le entrega un arreglo de cadenas al constructor.\\
\emph {String [] colores =\{"rojo","verde","azul","blanco", "gris"\}
}.\\
\emph {JList lista = new JList(colores)}\\
Para seleccionar un elemento usaremos :\\ \emph{lista.setSelectedIndex(2);} \\
Selecciona el indice en la posición 2.
}}}}\end{frame}

\subsection{Programación Gráfica}
\begin{frame}{Programación Gráfica}
\ITZ{	
\uncover<1->{\ITT{1}{Tipos de Listas}{
\uncover<1->{
\begin{itemize}
%\item{Asociar el escuchador al botón por medio del metodo \emph{addActionListener}   }
\item {SINGLE\_SELECTION : selecciona un solo valor.}
\item {SINGLE\_INTERVAL\_SELECTION : selecciona un grupo de valores seguidos.}
\item {MULTIPLE\_INTERVAL\_SELECTION : selecciona una cantidad de valores. (Valor por defecto)}
\end{itemize}
Para modificar la selección usaremos : \emph{lista.setSelectionMode(SINGLE\_SELECTION)}
}}}}\end{frame}

\subsection{Programación Gráfica}
\begin{frame}{Programación Gráfica}
\ITZ{	
\uncover<1->{\ITT{1}{Listas - Barra de desplazamiento}{
\uncover<1->{
Para incluir una barra o panel de desplazamiento debemos : \\
\emph{JScrollPane barra = new JScrollPane(lista);}\\
\emph{getContentPane().add(barra);}\\
Por defecto mostrara ocho valores, si la lista tiene menos valores la barra no aparecera. Para hacerla visible sera necesario : \emph {lista.setVisibleRowCount(3);}
}}}}\end{frame}

\subsection{Programación Gráfica}
\begin{frame}{Programación Gráfica}
\ITZ{	
\uncover<1->{\ITT{1}{Listas - Acceder a la información seleccionada}{
\uncover<1->{
Para una lista con SINGLE\_SELECTION el método \emph{getSelectedValue} entrega el unico valor seleccionado (para los otros tipos de lista este método entrega solo el primer valor seleccionado). \\El resultado es del tipo \emph{Object} y no \emph{String} por lo que será necesario aplicar una conversion de tipo explicita :\\
 \emph{String cadena= (String) lista.getSelectedValue();} 
Para obtener todos los valores seleccionados usaremos :\\ \emph{getSelectedValues()}
}}}}\end{frame}

\subsection{Programación Gráfica}
\begin{frame}{Programación Gráfica}
\ITZ{	
\uncover<1->{\ITT{1}{Listas}{
\uncover<1->{
{\tiny{\lstinputlisting{./codigosEx/select.java}}}
Para conocer la posición del valor seleccionado \\: \emph{int $[]$ getSelectIndices()) }
}}}}\end{frame}


\subsection{Programación Gráfica}
\begin{frame}{Programación Gráfica}
\ITZ{	
\uncover<1->{\ITT{1}{Listas - Eventos}{
\uncover<1->{
A diferencia de otros componente las listas no generan eventos \emph{Action}. El escuchador apropiado es : \emph{ListSelectionListener} que dispone de un solo metodo:\\ \emph{public void valueChanged(ListSelectionEvent ev)}.\\
Para evitar la redundancia (que se produce en la fase de transición) en la selección usaremos : \emph{getValuesIsAdjusting}
 {\tiny{\lstinputlisting{./codigosEx/select1.java}}}
}}}}\end{frame}

\subsection{Programación Gráfica}
\begin{frame}{Programación Gráfica}
\ITZ{	
\uncover<1->{\ITT{1}{Listas - Ejercicio}{
\uncover<1->{
Desarrolle un programa que cree una lista con el nombre de 5 colores. Al seleccionar uno y muchos colores debe imprimirlos en la consola. }}}}\end{frame}

\subsection{Programación Gráfica}
\begin{frame}{Programación Gráfica}
\ITZ{	
\uncover<1->{\ITT{1}{Listas - Ejercicio}{
\uncover<1->{
{\tiny{\lstinputlisting{./codigosEx/Lista0.java}}}
}}}}\end{frame}

\subsection{Programación Gráfica}
\begin{frame}{Programación Gráfica}
\ITZ{	
\uncover<1->{\ITT{1}{Listas - Ejercicio}{
\uncover<1->{
{\tiny{\lstinputlisting{./codigosEx/Lista1.java}}}
}}}}\end{frame}

\subsection{Programación Gráfica}
\begin{frame}{Programación Gráfica}
\ITZ{	
\uncover<1->{\ITT{1}{Etiquetas }{
\uncover<1->{
Un componente del tipo \emph{JLabel} permite mostrar en un contenedor un texto no modificable por el usuario.
\emph{JLabel texto = new JLabel("Texto Inicial ")} 
Este componente no tiene borde ni color de fondo. Para modificar el texto de una etiqueta usuaremos : \emph{texto.setText("Nueva etiqueta");}
}}}}\end{frame}

\subsection{Programación Gráfica}
\begin{frame}{Programación Gráfica}
\ITZ{	
\uncover<1->{\ITT{1}{Ejercicio}{
\uncover<1->{
Desarrolle un programa en el cual muestre en permanencia el numero de click realizados por un usuario sobre un botón.}}}}\end{frame}

\subsection{Programación Gráfica}
\begin{frame}{Programación Gráfica}
\ITZ{	
\uncover<1->{\ITT{1}{Listas - Ejercicio}{
\uncover<1->{
{\tiny{\lstinputlisting{./codigosEx/BotonClick0.java}}}
}}}}\end{frame}

\subsection{Programación Gráfica}
\begin{frame}{Programación Gráfica}
\ITZ{	
\uncover<1->{\ITT{1}{Listas - Ejercicio}{
\uncover<1->{
{\tiny{\lstinputlisting{./codigosEx/BotonClick1.java}}}
}}}}\end{frame}
\subsection{Programación Gráfica}
\begin{frame}{Programación Gráfica}
\ITZ{	
\uncover<1->{\ITT{1}{Ejercicio}{
\uncover<1->{
Desarrolle un programa con dos botones uno que incremente y el otro disminuya un contador. El contador debe estar en el contenido de la ventana use \emph{JLabel} para mostrar el contador.}}}}\end{frame}


\subsection{Programación Gráfica}
\begin{frame}{Programación Gráfica}
\ITZ{	
\uncover<1->{\ITT{1}{Ejercicio}{
\uncover<1->{
{\tiny{\lstinputlisting{./codigosEx/Exemple111.java}}}
}}}}\end{frame}

\subsection{Programación Gráfica}
\begin{frame}{Programación Gráfica}
\ITZ{	
\uncover<1->{\ITT{1}{Ejercicio}{
\uncover<1->{
{\tiny{\lstinputlisting{./codigosEx/Exemple112.java}}}
}}}}\end{frame}

\subsection{Programación Gráfica}
\begin{frame}{Programación Gráfica}
\ITZ{	
\uncover<1->{\ITT{1}{Combo-box}{
\uncover<1->{
Los ComboBox están asociados a un campo de texto no editable. Cuando el componente no esta seleccionado se muestra solo el campo de texto, al seleccionar el componente se despliega la lista.\\
El usuario puede elegir un valor en la lista. Por defecto el texto asociado a un comboBox no es editable.\\
Construcción de un comboBox : \\
\emph {String [] colores =\{"rojo","verde","azul","blanco", "gris"\}
}.\\
\emph{JComboBox= combo = new JComboBox(colores);}\\
\emph{combo.setEditable(true);}\\
\emph{combo.setMaximumRowCount(4);}\\
\emph{combo.selectionIndex(2);}\\
}}}}\end{frame}


\subsection{Programación Gráfica}
\begin{frame}{Programación Gráfica}
\ITZ{	
\uncover<1->{\ITT{1}{Combo-box - acceso a los datos}{
\uncover<1->{
El  método \emph{getSelectedItem} entrega los valores seleccionados, este metodo es como el  método \emph{getSelectedValues} de las listas.\\
\emph{Object valor = combo.getSelectedItem();}\\
El método \emph{getSelectedItem();} entrega el rango de valores seleccionado.
}}}}\end{frame}


\subsection{Programación Gráfica}
\begin{frame}{Programación Gráfica}
\ITZ{	
\uncover<1->{\ITT{1}{Combo-box - eventos}{
\uncover<1->{
Los comboBox generan evento \emph{Action} al seleccionar un elemento en la lista. Ademas un comboBox genera eventos \emph{Item} en cada modificación de selección lo que será tratado con el escuchador \emph{ItemListener} que tiene un solo método: \\
\emph{public void itemStateChanged(ItemEvent e) }\\
En un comboBox simple siempre tenemos dos eventos (suprimir la selección y nueva selección) ya sea en un campo de texto o en una lista.
}}}}\end{frame}

\subsection{Programación Gráfica}
\begin{frame}{Programación Gráfica}
\ITZ{	
\uncover<1->{\ITT{1}{Combo-box - eventos}{
\uncover<1->{
Desarrolle un programa que genere un comboBox con 5 colores que sea editable en el cual podamos incluir en curso de ejecución un nuevo color.
}}}}\end{frame}

\subsection{Programación Gráfica}
\begin{frame}{Programación Gráfica}
\ITZ{	
\uncover<1->{\ITT{1}{Ejercicio}{
\uncover<1->{
{\tiny{\lstinputlisting{./codigosEx/EjemploCombo0.java}}}
}}}}\end{frame}

\subsection{Programación Gráfica}
\begin{frame}{Programación Gráfica}
\ITZ{	
\uncover<1->{\ITT{1}{Ejercicio}{
\uncover<1->{
{\tiny{\lstinputlisting{./codigosEx/EjemploCombo1.java}}}
}}}}\end{frame}

\subsection{Programación Gráfica}
\begin{frame}{Programación Gráfica}
\ITZ{	
\uncover<1->{\ITT{1}{Combo-box - evolución dinamica de una lista de un combo}{
\uncover<1->{
Los bomboBox disponen de método que permiten agregar un nuevo valor al final de la lista : \emph{combo.addItem("naranja");}, agrega un elemento al final de la lista.\\
El método \emph{addItemAt("naranja", 2);} agrega un elemento en la posición dos.\\
El método \emph{removeItem("gris");}  suprime un valor existente.
}}}}\end{frame}

\subsection{Programación Gráfica}
\begin{frame}{Programación Gráfica}
\ITZ{	
\uncover<1->{\ITT{1}{Ejercicio}{
\uncover<1->{
Modifique el programa anterior para que los valores ingresados por el usuario sean adjuntados a la lista de valores del comboBox.
Para distinguir una selección de un ingreso de datos, usaremos el método \emph{getSelectedIndex}
}}}}\end{frame}

\subsection{Programación Gráfica}
\begin{frame}{Programación Gráfica}
\ITZ{	
\uncover<1->{\ITT{1}{Ejercicio}{
\uncover<1->{
{\tiny{\lstinputlisting{./codigosEx/EjemploComboDin0.java}}}
}}}}\end{frame}

\subsection{Programación Gráfica}
\begin{frame}{Programación Gráfica}
\ITZ{	
\uncover<1->{\ITT{1}{Ejercicio}{
\uncover<1->{
{\tiny{\lstinputlisting{./codigosEx/EjemploComboDin1.java}}}
}}}}\end{frame}


\subsection{Programación Gráfica}
\begin{frame}{Programación Gráfica}
\ITZ{	
\uncover<1->{\ITT{1}{Eventos Focus}{
\uncover<1->{
Los eventos Focus son tratados por el escuchador :  \emph{FocusListener} que tiene dos métodos : \\
\emph{public void focusGained(FocusEvent ev);}\\
\emph{public void focusLost(FocusEvent ev);}\\
En general, se tratara a la vez la validación, por ejemplo: en un campo de texto, y la perdida de foco.
}}}}\end{frame}

\subsection{Programación Gráfica}
\begin{frame}{Programación Gráfica}
\ITZ{	
\uncover<1->{\ITT{1}{Ejercicio: Una aplicación completa}{
\uncover<1->{
Desarrolle un programa que permita al usuario dibujar formas (rectángulo y/o ovalo) en una ventana, sus dimensiones y el color de fondo. Use combo, campo de texto y checkBox.
\\ Las dimensiones, comunes a las diferentes formas, son ingresadas en campos de texto (los valores obtenidos de tipo String deben ser convertidos con \emph{Integer.parseInt} a enteros).\\
El color de fondo sera elegido en un combo.\\
Para seleccionar la figura puede usar una lista o un \emph{JCheckBox}: \emph{ovalo = new JCheckBox("Ovalo");}. 
}}}}\end{frame}


\subsection{Programación Gráfica}
\begin{frame}{Programación Gráfica}
\ITZ{	
\uncover<1->{\ITT{1}{Ejercicio: Una aplicación completa}{
\uncover<1->{
Se aconseja dibujar en un panel con el gestor \emph{FlowLayout} para que quede en el centro, para facilitar las cosas ponga los otros componentes en un segundo panel usando el gestor por defecto para poner los controles abajo (\emph{South}).\\
 Debe considerar los eventos \emph{Focus} para validar el contenido de los campos de texto. \\
Use los eventos acción para los campos de texto y los checkBox. Use los eventos Item para el combo.
Para la comunicación entre el objeto ventana y el objeto panel del dibujo use, y por lo cual cree, los métodos de modificación : \emph{setLargo, setAncho, setOvalo, setRectangulo}
}}}}\end{frame}

\subsection{Programación Gráfica}
\begin{frame}{Programación Gráfica}
\ITZ{	
\uncover<1->{\ITT{1}{Ejercicio}{
\uncover<1->{
{\tiny{\lstinputlisting{./codigosEx/Final0.java}}}
}}}}\end{frame}

\subsection{Programación Gráfica}
\begin{frame}{Programación Gráfica}
\ITZ{	
\uncover<1->{\ITT{1}{Ejercicio}{
\uncover<1->{
{\tiny{\lstinputlisting{./codigosEx/Final1.java}}}
}}}}\end{frame}

\subsection{Programación Gráfica}
\begin{frame}{Programación Gráfica}
\ITZ{	
\uncover<1->{\ITT{1}{Ejercicio}{
\uncover<1->{
{\tiny{\lstinputlisting{./codigosEx/Final12.java}}}
}}}}\end{frame}

\subsection{Programación Gráfica}
\begin{frame}{Programación Gráfica}
\ITZ{	
\uncover<1->{\ITT{1}{Ejercicio}{
\uncover<1->{
{\tiny{\lstinputlisting{./codigosEx/Final2.java}}}
}}}}\end{frame}

\subsection{Programación Gráfica}
\begin{frame}{Programación Gráfica}
\ITZ{	
\uncover<1->{\ITT{1}{Ejercicio}{
\uncover<1->{
{\tiny{\lstinputlisting{./codigosEx/Final3.java}}}
}}}}\end{frame}

\subsection{Programación Gráfica}
\begin{frame}{Programación Gráfica}
\ITZ{	
\uncover<1->{\ITT{1}{Ejercicio}{
\uncover<1->{
{\tiny{\lstinputlisting{./codigosEx/Final4.java}}}
}}}}\end{frame}

%%%MEJORAR PARTE INTERFAZ, CLASES Abtractas, metodos abstractos, 
%%%adaptadores
%%%Preparar una nueva clase con :  clases anonimas.
%FOCO pagina 360
% eje final 378


\end{document}
