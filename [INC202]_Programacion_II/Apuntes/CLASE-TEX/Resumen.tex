%
%DEFINICIONES
%API : application programming interface de java. un numero importante de clases que forman parte del propio lebguaje
%Applets : esu na aplicacion especial qe se ejecuta dentro de un navegador al cargar una pagina HTML desde un servidor Web.  Un applet se descarga desde el servidor y no requiere instalacion en el computador donde esta el navegador. 
%Servlet es una aplicacion sin interface grafica que se ejecuta en un servidor internet. 
%IDE : integrated development environment ambiente de desarrollo por ejemplo eclipse
%Los threads (a veces llamados, procesos ligeros), son básicamente pequeños procesos o piezas independientes de un gran proceso. Al estar los threads contruidos en el lenguaje, son más fáciles de usar y más robustos que sus homólogos en C o C++.

\graphicspath{{./pics/}}



\section{Organización}
\subsection{Organización}
\begin{frame}{Organización del curso}

\ITZ{	
\uncover<1->{\ITT{1}{General}{\begin{itemize}
\uncover<1->

{\item Asistencia libre}  
{\item Horario de atención alumnos : Jueves de 14h30 a 15h30.}
{\item 1 bloque de ayudantía por semana (ayudante por definir).}
\end {itemize}}}

\uncover<2->{\ITT{2}{Calendario}{\begin{itemize}
\uncover<2->

{\item Fechas sin clases : 
\begin{itemize} \item {del 15 al 20 de marzo.}   
 \item {del 10 de mayo al 10 de junio.} 
 \end{itemize} }
{\item Cierre del semestre el 19 de julio.}
{\item 17 clases por recuperar. Horario para clases recuperativas :   miércoles en la tarde?}
 \end{itemize}}}}

 \end{frame}
 
\subsection{Organización}
\begin{frame}{Organización del curso}
\ITZ{	
\uncover<1->{\ITT{1}{Notas}{\begin{itemize}
\uncover<1->

{\item 2 certámenes (Cada certamen representa el 30\% de la nota del ramo).}  
{\item 6 trabajos en clase (en computador). Cada trabajo tiene una ponderación igual, en total serán el 30\% de la nota total del ramo (cada trabajo seria = 1/6*0.30).}
{\item Trabajo final en grupo (generar 10 grupos) trabajan como programadores para el curso de Metodología y Diseño esto representara el 10\% de la nota. Esta nota agrupara dos evaluaciones (interna del ramo y externa de los diseñadores que evaluarán el trabajo).}
{\item Resumen de ponderación : certámenes (60\%), nota de trabajos (30\%), nota de trabajo final (10\%).}
 \end{itemize}}}}
 \end{frame}

\subsection{Organización}
\begin{frame}{Organización del curso}
\ITZ{	
\uncover<1->{\ITT{1}{Calendario de certámenes y trabajos}{\begin{itemize}
\uncover<1->
{\item 1er Certamen : 24 de abril}  
{\item 2do Certamen : 29 de junio}  
{\item Prueba Recuperativa : 4 de julio}
{\item Prueba Especial : 8 de julio}
{\item Entrega trabajo : 14 de mayo}
{\item Entrega trabajo : 28 de junio}


{\item  Trabajos \begin{itemize} \item {27 de marzo} \item {10 y 17 de abril} \item{8 de mayo} \item {5 y 19 de junio} \end {itemize}
}
\end {itemize}}}}
 \end{frame}
 
 
\section{Paradigmas}
\subsection{Paradigmas}
\begin{frame}{Plan de la sección Paradigmas de Programación}
\ITZ{	
\uncover<1->{\ITT{1}{}{\begin{itemize}
\uncover<1->
{\item Definición de paradigma} 
{\item Principales Paradigmas}
\begin {itemize}
\item {Imperativo}
\item {Declarativo}
\item {Estructurado}
\item{Orientado a Objetos}
\item {Funcional}
\item{Logico}
\end {itemize}
\end{itemize}}}}
\end{frame}

\section{POO, Java}
\subsection{Introduction}
\begin{frame}{Plan de la Introducción a la POO y a Java}
\ITZ{	
\uncover<1->{\ITT{1}{}{\begin{itemize}
\uncover<1->
{\item Introducción a la Programación Orientada a Objetos.} 
{\footnotesize{\begin {itemize}
\item {Objetos y clases}
\item {Comportamiento y atributos}
{\item Características asociadas a la POO.} 
\begin{itemize}
\item {Abstracción}
\item {Encapsulamiento}
\item {Ocultamiento}
\item {Herencia}
\item {Polimorfismo}
\end {itemize}
\end {itemize}}}
{\item Introducción a la Programación en Java}
{\footnotesize{\begin {itemize}
\item {Inicios de Java}
\item {Ventajas de Java}
\item{Programar en Java}
\end {itemize}}}
{\item Ejercicios de Introducción a Java}
\end{itemize}}}}
\end{frame}

\section{Fundamentos}
\subsection{Fundamentos}
\begin{frame}{Plan de la sección Fundamentos de Java}
\ITZ{	
\uncover<1->{\ITT{1}{}{ \footnotesize{\begin{itemize}
\uncover<1->
{\item Enunciados y expresiones} 
{\item Variables y tipos de datos} 
{\item Comentarios} 
{\item Literales} 
{\item Expresiones y Operadores} 
{\item Ejercicios Fundamentos}
\end{itemize}}}}}
\end{frame}

\section{Objetos}
\subsection{Objetos}
\begin{frame}{Plan de la sección Objetos}
\ITZ{	
\uncover<1->{\ITT{1}{}{\begin{itemize}
\uncover<1->
\item {Objetos}
{\footnotesize{\begin {itemize}
\item {Crear nuevos Objetos}
\item {Llamar a Métodos}
\item {Biblioteca de clase Java}
\item {Ejercicio de Objetos}
\end {itemize}}}
\end{itemize}}}}
\end{frame}

\section{Condiciones, ciclos}
\subsection{Arreglos, condiciones y ciclos}
\begin{frame}{Plan Arreglos, condiciones y ciclos}
\ITZ{	
\uncover<1->{\ITT{1}{}{\begin{itemize}
\uncover<1->
\item {Arreglos}
{\footnotesize{\begin {itemize}
\item {Declarar variables de arreglo}
\item {Crear objetos de arreglo}
\item {Acceder a los elementos de un arreglo}
\item {Cambiar elementos de un arreglo}
\end{itemize}}}
\item {Arreglos multidimensionales}
\item {ArrayList}
\item{Condicionales}
{\footnotesize{\begin {itemize}
\item{if}
\item{switch}
\end{itemize}}}
\item {Ciclos}
{\footnotesize{\begin {itemize}
\item{for}
\item{while y do}
\end{itemize}}}
{\item Ejercicio de Arreglos, condicionales y ciclos}
\end{itemize}}}}
\end{frame}

\section{Clases, Objetos y Metodos}
\subsection{Clases, Objetos y Metodos}
\begin{frame}{Clases, Objetos y Metodos}
\ITZ{	
\uncover<1->{\ITT{1}{}{\begin{itemize}
\uncover<1->
\item {Clases y objetos}
{\footnotesize{\begin {itemize}
\item {Definir Clases}
\item {Constructores}
\item {Definición de objetos}
\end {itemize}}}
\item {Métodos}
{\footnotesize{\begin {itemize}
\item {Reglas de escritura de métodos}
\item {Atributos y metodos de clase}
\item {Sobrecarga de métodos }
\item {Intercambio de información con métodos}
\item {Recursividad}
\end {itemize}}}
\item {Aplicaciones Java}
{\footnotesize{\begin {itemize}
\item {Multiples Clases}
\item {Encapsulamiento}
\item {Paquetes}
\item {Ejercicio Clases, Objetos y Metodos}
\end {itemize}}}
\end{itemize}}}}
\end{frame}

\section{Herencia}
\subsection{Herencia}
\begin{frame}{Herencia}
\ITZ{	
\uncover<1->{\ITT{1}{}{\begin{itemize}
\uncover<1->
\item {Herencia}
{\footnotesize{\begin {itemize}
\item {Acceso a una clase derivada}
\item {Construcción de objetos derivados}
\item {Redefinición y sobrecarga de metodos }
\item {Polimorfismo}
\item {Super clase object}
\item {Miembros protected}
\item {Clases y metodos de finalización}
\item {Clases abstractas}
\item {Clases Interfaces}
\end {itemize}}}
\end{itemize}}}}
\end{frame}

\section{Programación Gráfica}
\subsection{Programación Gráfica}
\begin{frame}{Programación Gráfica}
\ITZ{	
\uncover<1->{\ITT{1}{}{\begin{itemize}
\uncover<1->
\item {Ventanas}
\item {Componentes: botón, campos de texto, listas, etiquetas, combobox }
\item {Dinámica de Componentes}
\item{Eventos}
\item {Dibujar}
\end{itemize}}}}
\end{frame}


\section{Java Avanzado}
\subsection{Java Avanzado}
\begin{frame}{Java Avanzado}
\ITZ{	
\uncover<1->{\ITT{1}{}{\begin{itemize}
\uncover<1->
\item {Manejo de excepciones}
{\footnotesize{\begin {itemize}
\item {try}
\item {catch}
\item {throws}
\item {finally}
\end {itemize}}}

\item {Applets}
{\footnotesize{\begin {itemize}
\item {Insertar applets}
\item {Pasar parámetros}
\item {Gráficos, fuentes y color}
\item {Animación sencilla}
\end {itemize}}}
\end{itemize}}}}
\end{frame}


