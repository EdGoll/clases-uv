\documentclass{beamer}

\mode<presentation>
{
	\usetheme{CambridgeUS}
	\setbeamercovered{transparent}
}
\usepackage[spanish]{babel}
\usepackage[latin1]{inputenc}
\usepackage{color}
\usepackage{hyperref}
\usepackage{algorithm,algorithmic}
\usepackage{colortbl}
\usepackage{graphicx}

\title[\textbf{Programaci\'on 2}]{\textbf{Programaci\'on 2}}

\subtitle{Lenguaje Java - Conceptos claves (\emph{2da parte})}

\author[Eduardo Godoy]
{
	Profesor: Eduardo Godoy \\
	\vspace{0.5mm}
	\texttt{\normalsize eduardo.gl@gmail.coml} \\ 
	Material elaborado por Rodrigo Olivares \\
	\texttt{\normalsize rodrigo.olivares@uv.cl} 
}

\institute[Universidad de Valpara\'iso]
%\date{$2^{do}$ Semestre de 2016} 

%\subject{Programaci\'on 2}
%
%\AtBeginSection
%{
%	\begin{frame}<beamer>
%	\frametitle{Contenido}
%	\tableofcontents[currentsection,currentsubsection]
%	\end{frame}
%}
%
%\AtBeginSubsection
%{
%	\begin{frame}<beamer>
%	\frametitle{Contenido}
%	\tableofcontents[currentsection,currentsubsection]
%	\end{frame}
%}
%
%\beamerdefaultoverlayspecification{<+->}

\begin{document}

	\begin{frame}
		\titlepage
	\end{frame}

	\begin{frame}
		\frametitle{Contenido}
		\tableofcontents%[pausesections]
	\end{frame}


    \section{Clase String}
 		   \subsection{Definici\'on}
		  \begin{frame}
			\frametitle{Clase String}
			\framesubtitle{Definici\'on}
  			\begin{block}{}
 				\begin{itemize}
 					\item En Java la clase  String permite instanciar objetos que representan  cadenas de caracteres  NO modificables. Entonces los objetos creados a partir de la clase String, son objetos Inmutables.
\item   Todos los objetos que representen cadenas de caracteres en Java son instancias de la Clase String.
\item La clase String se puede representan como un arreglo de car\'actereres y ser tratada como tal mediante m\'etodos implementado en ella.
				 \end{itemize}
			\end{block}
\begin{block}{}
Objeto inmutable:
Su valor inicial persiste durante toda la ejecuci\'ion  del programa y su estado no puede ser cambiado.
\end{block}
		\end{frame}
		\subsection{Principales M\'etodos}

		  \begin{frame}
			\frametitle{Clase String}
			\framesubtitle{Principales M\'etodos}
			\begin{block}{}
\begin{table}[]
\centering
\caption{M\'etodos.}
\label{my-label}
\tiny
\begin{tabular}{ll}
length()   :                     & Devuelve la longitud de la cadena                                                                   \\
indexOf(caracter):             & Devuelve la posici\'on de la primera aparici\'on de car\'acter                                            \\
lastIndexOf(caracter):         & Devuelve la posici\'on de la \'ultima aparici\'on de car\'acter                                             \\
charAt(n) :                      & Devuelve el car\'acter que est\'a en la posici\'on n                                                      \\
substring(n1,n2) :               & Devuelve la subcadena  entre las posiciones n1 y n2-1                                    \\
toUpperCase()     :              & Devuelve la cadena convertida a may\'usculas                                                          \\
toLowerCase()     :              & Devuelve la cadena convertida a min\'usculas                                                          \\
equals(cad)      :             & Compara dos cadenas y devuelve true si son iguales                                                  \\
equals(cad)                   & Compara dos cadenas y devuelve true si son iguales                                                  \\
equalsIgnoreCase(cad)         & Igual que equals pero sin considerar may\'usculas y min\'usculas                                        \\
compareTo(OtroString)           & Devuelve 0, mayor a 0, menor a 0 \\
compareToIgnoreCase(OtroString) & Compara  sin considerar may\'usculas y min\'usculas.                             \\
valueOf(N)                      & Convierte el valor  a String.                       
\end{tabular}
\end{table}
			\end{block}
		\end{frame}



    \section{Paso de par\'ametros}
    
		\subsection{Referencia / Valor}

        \begin{frame}
			\frametitle{Paso de par\'ametros}
			\framesubtitle{Referencia / Valor}

			\begin{block}{}
			    {\scriptsize
                    \begin{itemize}
                        \item[-] \textbf{Valor}: \emph{el m\'etodo recibe una copia del valor del argumento; el m\'etodo trabaja sobre esa copia que podr\'a modificar su valor, sin que esto tenga incidencia al valor efectivo del argumento.}
                        \item[-] \textbf{Referencia}: \emph{el m\'etodo recibe la referencia del argumento; el m\'etodo trabaja directamente, por lo cual podr\'a modificar el valor efectivo del argumento.}
                    \end{itemize}
			    }
			\end{block}
		\end{frame}

    \section{Autoreferencia}
    
        \subsection{Palabra reservada this}

        \begin{frame}
			\frametitle{Autoreferencia}
			\framesubtitle{Palabra reservada this}

			\begin{block}{}
			    {\scriptsize
                    \begin{itemize}
                        \item[-] Para hacer una referencia a la instancia actual de una clase usamos la palabra reservada \textcolor{blue}{this}.
                        \item[-] Utilizaremos la palabra reservada \textcolor{blue}{this} para referirnos al \textbf{objeto actual} o a las v\textbf{ariables de instancia de este objeto}. 
                        \item[-] \textbf{No} utilizaremos la palabra reservada \textcolor{blue}{this} en los m\'etodos y atributos declarados como \textcolor{blue}{static}.
                    \end{itemize}
			    }
			\end{block}	
			
			\begin{block}{}
			    {\scriptsize
                    \begin{itemize}
                        \item[] \emph{Se puede omitir la palabra reservada \textcolor{blue}{this} para las variables de instancia, esto depende de la existencia o no de variables con el mismo nombre en el \'ambito local.}
                    \end{itemize}
			    }
			\end{block}
		\end{frame}

        \begin{frame}
			\frametitle{Autoreferencia}
			\framesubtitle{Palabra reservada this}

			\begin{block}{}
				{\scriptsize
				\textcolor{blue}{public class} \textbf{\emph{Prueba}} \{ \\
				\hspace{1cm} \\
				\hspace{1cm} \textcolor{blue}{private int} \textcolor{green}{test} = 10; \ \\
				\hspace{1cm} \\
				\hspace{1cm} \textcolor{blue}{public void} \textbf{printTest}() \{ \\
				\hspace{2cm} \textcolor{blue}{int} test = 20;\\
				\hspace{2cm} System.\emph{\textcolor{green}{out}}.println(\textcolor{orange}{''test''} + test);\\
                \hspace{2cm} System.\emph{\textcolor{green}{out}}.println(\textcolor{orange}{''test''} + \textcolor{blue}{this}.\textcolor{green}{test});\\
				\hspace{1cm} \} \\
				\hspace{1cm} \\
				\hspace{1cm} \textcolor{blue}{public static void} \textbf{main}(String[ ] args) \{ \\
				\hspace{2cm} Prueba p = \textcolor{blue}{new} Prueba(); \\
				\hspace{2cm} p.\emph{printTest}();\\
				\hspace{1cm} \} \\
				\}}
			\end{block}
		\end{frame}		

    \section{Atrubutos y m\'etodos miembros}
    
        \subsection{De Objeto}

        \begin{frame}
			\frametitle{Atrubutos y m\'etodos miembros}
			\framesubtitle{De Objeto}

			\begin{block}{Atributos}
			    {\scriptsize
                    Cada objeto, instancia de una clase, tiene su propia copia de los atributos miembro y clase pueden ser de tipos primitivos (\textbf{boolean}, \textbf{int}, \textbf{long}, \textbf{double}, ...) o referencias a objetos de otra clase.
			    }
			\end{block}

			\begin{block}{M\'etodos}
			    {\scriptsize
                    Los m\'etodos son subrutinas definidas dentro de una clase. Se aplican siempre a un objeto de la clase por medio del operador punto (.) y pueden incluir argumentos expl\'icitos que van entre par\'entesis, a continuaci\'en del nombre del m\'etodo.
			    }
			\end{block}	
		\end{frame}
		
        \subsection{De Clase}

        \begin{frame}
			\frametitle{Atrubutos y m\'etodos miembros}
			\framesubtitle{De Clase}

			\begin{block}{Atributos}
			    {\scriptsize
                    Una clase puede tener atributos propios de la clase y no de cada objeto. A estos atributos se les llama \textbf{atributos de clase} o atributos \emph{static}. Los atributos \emph{static} se suelen utilizar para definir constantes comunes para todos los objetos de la clase (por ejemplo \textbf{PI} en la clase \emph{Circulo}) o atributos que s\'olo tienen sentido para toda la clase (por ejemplo, un contador de objetos creados como \emph{numCirculos} en la clase \emph{Circulo}).
			    }
			\end{block}

			\begin{block}{}
			    {\scriptsize
			        \begin{itemize}
			            \item[-] Los atributos de clase se crean anteponiendo la palabra \textcolor{blue}{\textbf{static}} a su declaraci\'on. 
                        \item[-] Para llamarlas se suele utilizar el nombre de la clase (no es imprescindible), pues as\'i, su sentido queda m\'as claro. Por ejemplo, \emph{Circulo}.\textcolor{green}{\emph{numCirculos}} es un atributo de clase que cuenta el n\'umero de C\'irculos creados.
			        \end{itemize}
			    }
			\end{block}
			
		\end{frame}		
		
        \begin{frame}
			\frametitle{Atrubutos y m\'etodos miembros}
			\framesubtitle{De Clase}

			\begin{block}{M\'etodos}
			    {\scriptsize
                    \begin{itemize}
			            \item[-] Existen m\'etodos que \textbf{no} act\'uan sobre objetos concretos a trav\'es del operador punto. A estos m\'etodos se les llama \textbf{m\'etodos de clase} o \textbf{static}.
                        \item[-] Un ejemplo t\'ipico de m\'etodos \emph{static} son los m\'etodos matem\'eticos de la clase \textbf{java.lang.Math} (\emph{sin}(), \emph{cos}(), \emph{exp}(), \emph{pow}(), etc.).
                    \end{itemize}
			    }
			\end{block}

			\begin{block}{}
			    {\scriptsize
			        \begin{itemize}
			            \item[-] Los m\'etodos crean anteponiendo la palabra \textcolor{blue}{\textbf{static}} a su declaraci\'on. 
			            \item[-]Para llamarlos se suele utilizar el nombre de la clase, en vez del nombre de un objeto de la clase (por ejemplo,
\textbf{Math}.\emph{sin}(ang), para calcular el \emph{seno} de un \'angulo).
                    \end{itemize}
			    }
			\end{block}
			
		\end{frame}	
		
    \section{Packages}
    
        \subsection{Conceptos}

        \begin{frame}
			\frametitle{Packages}
			\framesubtitle{Conceptos}

			\begin{block}{}
			    {\scriptsize
                    \textbf{Un package es una agrupaci\'on de clases}.
			    }
			\end{block}

			\begin{block}{}
			    {\scriptsize
                    Para que una clase pase a formar parte de un \textbf{package} llamado \emph{packageName}, se debe agregar en ella, la sentencia:
                    \begin{itemize}
			            \item[-] \textcolor{blue}{\textbf{package}} packageName;
			        \end{itemize}
			    }
			\end{block}
			
			\begin{block}{}
			    {\scriptsize
                    Los nombres de los \emph{packages} se suelen escribir con \textbf{min\'usculas}, para distinguirlos de las clases, que empiezan por \textbf{may\'uscula}. El nombre de un package puede constar de varios nombres unidos por puntos (los propios packages de Java siguen esta norma, como por ejemplo \emph{java.awt.event}).
			    }
			\end{block}
			
			\begin{block}{}
			    {\scriptsize
                    Todas las clases que forman parte de un \textbf{package} deben estar en el mismo directorio.
			    }
			\end{block}
	
		\end{frame}			
		
		\begin{frame}
			\frametitle{Preguntas}

			\hspace{4cm}\huge{Preguntas ?}
		
		\end{frame}
	\end{document}

\usetheme{default}
\usetheme{JuanLesPins}
\usetheme{Goettingen}
\usetheme{Szeged}
\usetheme{Warsaw}

\usecolortheme{crane}

\usefonttheme{serif}
\usefonttheme{structuresmallcapsserif}