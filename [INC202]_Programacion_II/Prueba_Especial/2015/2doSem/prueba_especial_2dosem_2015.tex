\documentclass[10pt]{article}
\usepackage{graphicx}
\usepackage{amssymb}
\usepackage{epstopdf}
\usepackage{tikz}
\usepackage{enumitem}
\usepackage{multicol,multirow}
\DeclareGraphicsRule{.tif}{png}{.png}{`convert #1 `dirname #1`/`basename #1 .tif`.png}
\renewcommand{\tablename}{Tabla} 
\renewcommand{\figurename}{Figura} 
\newcommand*\circled[1]{\tikz[baseline=(char.base)]{\node[shape=circle,blue,draw,inner sep=1pt] (char) {#1};}}

% For a visual definition of these parameters, see
\textwidth = 6.5 in
\textheight = 9 in
\oddsidemargin = 0.0 in
\evensidemargin = 0.0 in
\topmargin = 0.0 in             
\headheight = 0.0 in            
\headsep = 0.0 in
            
\parskip = 0.2in                % vertical space between paragraphs
% Delete the % in the following line if you don't want to have the first line of every paragraph indented
%\parindent = 0.0in

\begin{document}
\begin{center}
    {\Large Prueba Especial, Programaci\'on II} \\
    \emph{\small Prof. Rodrigo Olivares} \\
	\emph{\small Ayud. Juan Carlos Tapia} \\
    \emph{\scriptsize Diciembre 15, 2015}
\end{center}
\vspace*{-35pt}
\begin{center}
    \rule{1\textwidth}{.3pt}
\end{center}
\vspace*{-42pt}
\begin{center}
    \rule{1\textwidth}{2pt}
\end{center}

\vspace*{-15pt}
\textbf{Instrucciones}:
\vspace*{-15pt}

\begin{itemize}
    \item[-] El puntaje m\'aximo de la prueba especial es 100\%, siendo el 60\% el m\'inimo requerido para aprobar.
	\item[-] Responda cada pregunta en el lugar indicado. No se aceptar\'an recorrecciones de pruebas respondidas con l\'apiz grafito.
	\item[-] El tiempo m\'aximo de la evaluaci\'on es de 90 minutos.
    \item[-] La prueba especial es \underline{\textbf{individual}}. Cualquier intento de copia, ser\'a sancionado con nota \textbf{1,0}.
\end{itemize}
\vspace*{-20pt}

\begin{enumerate}

    \item \emph{20pts.} De las siguentes afirmaciones, encierre en un c\'irculo la o las alternativas correctas (\emph{3pts c/u}).    

	{\footnotesize
    
    \begin{multicols}{2}

	\begin{enumerate}[label=(\alph*)]
        \item[i.] La orientaci\'on a objeto es: 
        \item[(a)] Un paradigma de programaci\'on procedural.
        \item[(b)] Un paradigma de programaci\'on estucturado.
        \item[(c)] Una herramienta de programaci\'on.
        \item[(d)] Un lenguaje de programaci\'on.
        \item[(e)] Ninguna de las anteriores.
    \end{enumerate}

    \begin{enumerate}[label=(\alph*)]
        \item[ii.] El principio de ocultamiento: 
        \item[(a)] Es una t\'ecnica que protege el estado de una entidad.
        \item[(b)] Es \'util en enfoques procedurales.
        \item[(c)] En Java, se logra con los modificadores de acceso.
        \item[(d)] Es encapsular el conocimiento de una entidad.
        \item[(e)] Ninguna de las anteriores.
    \end{enumerate}

    \begin{enumerate}[label=(\alph*)]
        \item[iii.] Una interface:
        \item[(a)] Tiene al menos un m\'etodo impementado.
        \item[(b)] Tiene todos sus m\'etodos abstractos.
        \item[(c)] Es factible de ser implementada.
        \item[(d)] Es factible de ser extendida.
        \item[(e)] Ninguna de las anteriores.
    \end{enumerate}

    \begin{enumerate}[label=(\alph*)]
        \item[vi.] La herencia m\'ultiple:
        \item[(a)] Permite heredar diverso compartimiento.
        \item[(b)] Apoya el principio ocultamiento.
        \item[(c)] Apoya el principio de encapsulamiento.
        \item[(d)] En Java se desarrolla implementado clases abstractas.
        \item[(e)] En Java se desarrolla implementado interfaces.
    \end{enumerate}

    \begin{enumerate}[label=(\alph*)]
        \item[v.] Un thread: 
		\item[(a)] Es un flujo de un proceso en memoria.        
        \item[(b)] Es un proceso que se ejecuta en memoria.
        \item[(c)] Puede ser creado como clase en Java.
        \item[(d)] Puede ser instanciado como atributo.
        \item[(e)] Ninguna de las anteriores.
    \end{enumerate}

    \begin{enumerate}[label=(\alph*)]
        \item[vi.] Para una hebra o hilo se debe:
        \item[(a)] Iniciar con el m\'etodo run.
        \item[(b)] Iniciar con el m\'etodo start.
        \item[(c)] Sobreescribir el m\'etodo run.
        \item[(d)] Sobreescribir el m\'etodo start.
        \item[(e)] Dormir (sleep) la hebra.
    \end{enumerate}

    \begin{enumerate}[label=(\alph*)]
        \item[vii.] En el ciclo de vida de una hebra, el estado: 
        \item[(a)] New crea la hebra.
        \item[(b)] Runnable ejecuta siempre la hebra.
        \item[(c)] Blocked se ejecuta, sin importar estados internos.
        \item[(d)] Dead es invocado generalmente por el m\'etodo sleep.
        \item[(e)] Yield, verifica el desempe\~no del estado Runnable.
    \end{enumerate}

    \begin{enumerate}[label=(\alph*)]
        \item[viii.] Los bloqueos de recursos compartidos se consiguen:
        \item[(a)] Package, bloqueando los accesos a las clases internas.
        \item[(b)] Clase, bloqueando m\'etodos y atributos de la clase.
        \item[(c)] Atributo, declar\'andolos como static.
        \item[(d)] Objeto, declarando los m\'etodos como synchronized.
        \item[(e)] Ninguna de las anteriores
    \end{enumerate}

    \begin{enumerate}[label=(\alph*)]
        \item[ix.] Referente a JFrame:
        \item[(a)] Habitualmente se usa para crear la ventana principal.
        \item[(b)] getPaneContent() obtiene el panel principal.
        \item[(c)] setAdd() permite agregar componentes al panel.
        \item[(d)] setSize() permite dimensionar la ventana.
        \item[(e)] Ninguna de las anteriores
    \end{enumerate}

    \begin{enumerate}[label=(\alph*)]
        \item[x.] Para realizar acciones desde un bot\'on Se requiere:
        \item[(a)] Crear una clase que implemente un ActionEvent.
        \item[(b)] Crear una clase que implemente un ActionListener.
        \item[(c)] Re-escribir el m\'etodo actionEvent(ActionPerformed).
        \item[(d)] Re-escribir el m\'etodo actionPerformed(ActionEvent).
        \item[(e)] Agregar la instancia de la clase oyente, al bot\'on.
    \end{enumerate}

	\end{multicols}
}
    \item \emph{80pts.} El departamento de inform\'atica de la universidad le ha solicitado realizar una aplicaci\'on que permita buscar y mostrar el resultado de la prueba se selecci\'on univesitaria de un postulante. Esta aplicaci\'on debe ser desarrollada en el lenguaje JAVA y con interfaz de usuario. Además considere lo siguiente:
    
    \begin{enumerate}[label=(\alph*)]
		\item[\emph{10pts}] Leer los archivos de largo fijo \emph{B\_INSCRITOS.txt} y \emph{C\_PUNTAJES.txt}. La estructura (en caracteres) es la siguiente:
		\begin{enumerate}
		    \item[-] \emph{B\_INSCRITOS.txt}: Tipo identificaci\'on (1), identificaci\'on (12), nombres (30), apellido paterno (20), apellido materno (20) y correo elect\'onico (20).
		    \item[-] \emph{C\_PUNTAJES.txt}: Tipo identificaci\'on (1), identificaci\'on (12), promedio notas (2), puntaje nota de ense\~nanza media (3), puntaje lenguaje (3), puntaje matem\'aticas (3), puntaje historia (3) y puntaje ciencias (3).
		\end{enumerate}
		\item[\emph{10pts}] Utilizar TDA Bean para manipular los archivos.
		\item[\emph{5pts}] Utilizar listas para gestionar los registos de los archivos.
		\item[\emph{15pts}] Realizar una b\'usqueda de un postulante en particular.
		\item[\emph{20pts}] Desplegar la informaci\'on en JComponents.
		\item[\emph{20pts}] \textbf{Recuerde desarrollar la aplicaci\'on bajo el paradigma de la orientaci\'on a objetos.}
	\end{enumerate}

	\end{enumerate}
\end{document} 
