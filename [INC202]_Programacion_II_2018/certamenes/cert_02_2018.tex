\documentclass{exam}

\usepackage{fullpage}
\usepackage[spanish]{babel}
\usepackage[utf8]{inputenc}
\usepackage[T1]{fontenc}
\usepackage[newcommands]{ragged2e}
\usepackage{hyperref}
\usepackage{algorithm,algorithmic}
\usepackage{colortbl}
\usepackage{graphicx}
\usepackage{multicol}
\usepackage{enumitem}
\usepackage{minted}
\usepackage[skins,minted]{tcolorbox}

\newcommand{\code}[1]{\mintinline{java}{#1}}
\newcommand{\codet}[1]{\texttt{#1}}

\setminted[java]{
  linenos=true,
  fontfamily=tt,
  fontsize=\small,
  frame=leftline,
  autogobble=True,
}

\newminted[jsmall]{java}{
  fontsize=\footnotesize
  , linenos = false
  , frame=single
  , autogobble=true
}

\newtcblisting{java}[2][]{
  minted language=java,
  enhanced, listing engine=minted,
  listing only, #1, title=#2, left=2em
}

\setminted[bash]{
  linenos=true,
  fontfamily=tt,
  fontsize=\small,
  frame=leftline
}

\newtcblisting{bash}[2][]{minted language=bash,
  enhanced, listing engine=minted,
  listing only, #1, title=#2, left=2em
}
\usepackage{
  amsmath,
  amssymb,
  eso-pic,
  float,
  graphicx,
  lmodern,
  wrapfig,
  tabularx,
  multicol,
  multirow,
  color,
  colortbl,
  lastpage,
  titlesec,
  sectsty
}


\definecolor{azul}{RGB}{33,127,190}
\sectionfont{\color{azul}}
\subsectionfont{\color{azul}}
\renewcommand{\familydefault}{\sfdefault}

\footer{}{\thepage}{}

\makeatother

\title{\LARGE\color{azul}\textbf{INC 214 Programación 2 - Certamen 2 }}
\author{\smallsize \color{gray}{Profesores: } \color{black}{\textbf{Ismael Figueroa, Eduardo Godoy}}}
\date{\normalsize \em \today}

\begin{document}

\AddToShipoutPictureBG*{%
  \AtPageUpperLeft{\raisebox{-\height}{\includegraphics[scale=.95]{base/header.png}}}}

\maketitle

\begin{multicols}{2}
  \begin{flushleft}
    \textbf{Nombre:} \\
    \vspace*{2mm}
    \textbf{Rut:} \\
    \vspace*{2mm}
    \textbf{Paralelo:}
  \end{flushleft}
  \begin{center}
    \begin{table}[H]
      \begin{tabular}{p{4cm}|p{3cm}|}
        \arrayrulecolor{gray!50}\cline{2-2} ~ & {\em {\scriptsize \color{gray!50}{Puntaje:}}}\\
         & ~ \\
         ~ & \textbf{Nota:}
        \\ & ~ \\
        \arrayrulecolor{gray!50}\cline{2-2}
      \end{tabular}
    \end{table}
  \end{center}
\end{multicols}

%\vspace*{-18mm}
\noindent
\textbf{\\Instrucciones:}
\begin{itemize}
\item[-] El puntaje máximo  es 100 puntos.
\item[-] Tiempo máximo: 180 minutos.
\item[-] El trabajo es \underline{\textbf{individual}}. Cualquier intento de copia, será sancionado según dicta el reglamento de la carrera.
\item[-] Se debe subir al aula virtual un archivo comprimido con
  el siguiente formato: \\
  \textt{Certamen\_02\_<NombreApellidoEstudiante\_rut\_paralelo>.zip}. Dentro
  deben estar todos los archivos de código fuente requeridos.}
\item[-] Todas las clases, interfaces, u otros artefactos deben
  pertenecer al paquete {\tt certamen2}.
\end{itemize}

% \noindent
% \textbf{Resultados de aprendizaje a evaluar:}
% \begin{enumerate}
% \item Adquirir el conocimiento de OO para implementar esquemas b\'asicos \\ de herencia y registros persistentes en archivos.
% \end{enumerate}
% \vspace{2mm}

% \noindent
% \textbf{Contenido:} Este certamen evalúa los siguientes temas:

% \vspace{-2mm}
% \begin{table}[H]
%   \begin{tabular}{
%     !{\color{gray!50}\vrule}l
%     !{\color{gray!50}\vrule}c
%     !{\color{gray!50}\vrule}c
%     !{\color{gray!50}\vrule}} \arrayrulecolor{gray!50} \hline
%     \multicolumn{1}{!{\color{gray!50}\vrule}c}{\multirow{2}{*}{\textbf{
%     Tema
%     }}} &
%           \multicolumn{2}{!{\color{gray!50}\vrule}c!{\color{gray!50}\vrule}}{\textbf{
%           Puntajes
%           }} \\ \arrayrulecolor{gray!50}\cline{2-3} &
%                                                       \multicolumn{1}{!{\color{gray!50}\vrule}c!{\color{gray!50}\vrule}}{\textbf{
%                                                       Total
%                                                       }} &
%                                                            \multicolumn{1}{c!{\color{gray!50}\vrule}}{\textbf{
%                                                            Obtenido
%                                                            }} \\ \arrayrulecolor{gray!50} \hline
%     Problema 1: Programación en lenguaje Java, Herencia , polimorfismo y \\ Lectura/Escritura persistente. 
%         & \multicolumn{1}{!{\color{gray!50}\vrule}c!{\color{gray!50}\vrule}}{\textbf{
%           100 pts.
%           }} & \\ \arrayrulecolor{gray!50} \hline

%   \end{tabular}
% \end{table}

% \newpage

\section*{Enunciado}

Se busca desarrollar un sistema de registros de personas
pertenencientes a la Escuela de Ingeniería Civil Informática de la
Universidad de Valparaíso. Para ello debe considerar que existen
diversos tipos de personas: académicos, alumnos, y funcionarios. 

\begin{questions}


\item \emph{15 pts.} Implemente las clases necesarias considerando los
  distintos tipos de personas. Utilice herencia según corresponda para
  evitar la duplicación de código, o de atributos, en las clases que
  defina. Además:
  
  \begin{enumerate}
  \item[-] Todas las personas tienen los campos: DNI, nombre,
    apellidos, edad y dirección.
  \item[-] Para los académicos se debe registrar su grado académico:
    licenciado, magister, o doctor.
  \item[-] Para los alumnos, se debe registrar las asignaturas que se
    encuentra cursando en el semestre.
  \item[-] Para los funcionarios, se debe registrar su oficio o profesión.
  \end{enumerate}

\item \emph{10pts.}  En un archivo llamado
  \textbf{justificacion\_modelo.txt} comente y justifique el modelo de
  clases implementado en el punto anterior, indicando cómo se evita la
  duplicación de código.

\item \emph{5pts.} Defina el archivo {\tt Archivable.java} con la
  siguiente interfaz:

  \begin{jsmall}
    public interface ArchivableCSV<E> { void guardar(String rutaArchivo); }
  \end{jsmall}

  \noindent Donde se tiene que:

  \begin{itemize}
  \item El método {\tt guardar} agrega al final del archivo que se
    encuentra en {\tt rutaArchivo} una única línea con la información
    que corresponda, separada por comas. El formato específico para
    escribir los datos debe ser el siguiente:

    \begin{itemize}
    \item TipoPersona;DNI;Nombre;Apellidos;Edad;Dirección
    \end{itemize}

    \noindent
    Además, dependiendo del tipo de persona, se deberán incluir los
    datos adicionales al final de la fila, luego del campo Dirección.
    
  \end{itemize}

\item \emph{15pts. } Modifique las clases implementadas que
  pertenezcan a la jerarquía de las personas y sus distintas variantes
  para que éstas implementen la interfaz {\tt ArchivableCSV}.

\item Implementar la clase {\tt GestionPersona} que provea lo
  siguiente:
  \begin{itemize}
    
  \item \emph{5pts.} Un arreglo dinámico de tipo {\tt
      ArrayList<Persona}{\tt $>$} como atributo.
    
  \item \emph{15pts.} El método {\tt leerDatos} para ingresar los
    datos requeridos cumpliendo con el esquema de herencia, agregando
    cada persona al arreglo. Se debe poder agregar personas de todos
    los tipos definidos.
    
  \item \emph{10pts.} El método {\tt guardarEnArchivo} que recorra el
    el arreglo dinámico de manera polimórfica, invocando al método
    {\tt guardar} de la interfaz {\tt ArchivableCSV} para agregar los
    datos al final de un archivo.
                             
  \item \emph{15pts.} Implementar el método {\tt listar} que lea el
    archivo y liste todos sus registros. El formato de salida debe ser
    el siguiente:
    
    \begin{itemize}
    \item[$\rightarrow$] Tipo: Alumno
    \item[$\rightarrow$] DNI: 11111111-1
    \item[$\rightarrow$] nombre: Juan
    \item[$\rightarrow$] Apellidos: Gonzalez Figueroa
    \item[$\rightarrow$] Dirección: Errazuriz 355 Valparaíso
    \end{itemize}
    
    \end{itemize}
  \item \emph{10pts.} Implementar la clase {\tt PersonaMain} que
    permita ejecutar el programa y los métodos asociados. El número de
    personas a crear depende del usuario.
        
\end{questions}

\begin{table}[H]
  \centering
  \begin{tabular}{
    !{\color{gray!50}\vrule}p{3.9cm}
    !{\color{gray!50}\vrule}p{3.6cm}
    !{\color{gray!50}\vrule}p{3.6cm}
    !{\color{gray!50}\vrule}p{3.6cm}
    !{\color{gray!50}\vrule}} \arrayrulecolor{gray!50} \hline
    \multicolumn{4}{!{\color{gray!50}\vrule}c!{\color{gray!50}\vrule}}{\textbf{¿Cómo  seré evaluado en este trabajo?}} \\ \arrayrulecolor{gray!50}
    \hline
    %
    \textbf{Ítem} & \textbf{Logrado} & \textbf{Suficiente} & \textbf{No Logrado}\\ \arrayrulecolor{gray!50} \hline\newline
    Sección A: Creación de Esquema de herencia de clases con atributos y m\'etodos respectivos. &
    Aplica de forma correcta: 15pts   &
    Aplica parcialmente con menos de 2 errores: 8pts  &
    Aplica de forma incorrecta con 3 errores o más: 7 - 0 pts \\ \arrayrulecolor{gray!50} \hline

    Sección A: Justificación del modelo. &
    Aplica de forma correcta: 10pts   &
    Responde  parcialmente: 5pts  &
    Responde de forma incorrecta o no responde:  0 pts \\ \arrayrulecolor{gray!50} \hline

    Sección B: Implementación  m\'etodos de \textbf{GestionPersona}: \textbf{crear, guardar, listar}. &
    Aplica de forma correcta 100\%  del puntaje asociado &
    Aplica parcialmente  2 errores o menos 65\% - 50\%  del puntaje asociado en cada m\'etodo &
    Aplica de forma incorrecta con 3 errores o más 20\% - 0\% del puntaje asociado\\ \arrayrulecolor{gray!50} \hline

    Implementar clase instanciadora \textbf{PersonaMain} con m\'etodo main y crea instancias de objetos y llamadas a \textbf{m\'etodos requeridos}. &
    Aplica de forma correcta 15pts &
    Aplica parcialmente con menos de 2 errores 8 pts &
    Aplica de forma incorrecta con 3 errores o más 7 - 0pts\\ \arrayrulecolor{gray!50} \hline

    Total de la sección &  100pts &  61 - 20 pts & 18-0 pts\\ \arrayrulecolor{gray!50} \hline
  \end{tabular}
  \label{tbl:1}
\end{table}

\vspace{-5mm} \textbf{Nota:} En caso de que el ítem no esté presente,
tiene ponderación cero.

\end{document}
