\documentclass{exam}

\usepackage[spanish]{babel}
\usepackage[utf8]{inputenc}
\usepackage[T1]{fontenc}
\usepackage[newcommands]{ragged2e}
\usepackage{hyperref}
\usepackage{algorithm,algorithmic}
\usepackage{colortbl}
\usepackage{graphicx}
\usepackage{multicol}
\usepackage{enumitem}
\usepackage{
  amsmath,
  amssymb,
  eso-pic,
  float,
  graphicx,
  lmodern,
  wrapfig,
  tabularx,
  multicol,
  multirow,
  color,
  colortbl,
  lastpage,
  titlesec,
  sectsty
}


\definecolor{azul}{RGB}{33,127,190}
\sectionfont{\color{azul}}
\subsectionfont{\color{azul}}
\renewcommand{\familydefault}{\sfdefault}

\footer{}{\thepage}{}

\makeatother

\title{\LARGE\color{azul}\textbf{INC 214 Programación 2 - Certamen 2 }}
\author{\smallsize \color{gray}{Profesores: } \color{black}{\textbf{Ismael Figueroa, Eduardo Godoy}}}
\date{\normalsize \em \today}

\begin{document}

\AddToShipoutPictureBG*{%
  \AtPageUpperLeft{\raisebox{-\height}{\includegraphics[scale=.95]{base/header.png}}}}

\maketitle

\begin{multicols}{2}
  \begin{flushleft}
    \textbf{Nombre:} \\
    \vspace*{2mm}
    \textbf{Rut:} \\
    \vspace*{2mm}
    \textbf{Paralelo:}
  \end{flushleft}
  \begin{center}
    \begin{table}[H]
      \begin{tabular}{p{4cm}|p{3cm}|}
        \arrayrulecolor{gray!50}\cline{2-2} ~ & {\em {\scriptsize \color{gray!50}{Puntaje:}}}\\
         & ~ \\
         ~ & \textbf{Nota:}
        \\ & ~ \\
        \arrayrulecolor{gray!50}\cline{2-2}
      \end{tabular}
    \end{table}
  \end{center}
\end{multicols}

%\vspace*{-18mm}
\noindent
\textbf{\\Instrucciones:}
\begin{itemize}
\item[-] El puntaje máximo  es 100 puntos.
\item[-] Tiempo máximo: 120 minutos.
\item[-] El trabajo es \underline{\textbf{individual}}. Cualquier intento de copia, será sancionado según dicta el reglamento de la carrera.
\item[-] Se debe subir al aula virtual un archivo comprimido con
  el siguiente formato: \\
  \textt{Certamen\_02\_<NombreApellidoEstudiante\_rut\_paralelo>.zip}. Dentro
  deben estar todos los archivos de código fuente requeridos.}
\end{itemize}

\noindent
\textbf{Resultados de aprendizaje a evaluar:}
\begin{enumerate}
\item Adquirir el conocimiento de OO para implementar esquemas b\'asicos \\ de herencia y registros persistentes en archivos.
\end{enumerate}
\vspace{2mm}

\noindent
\textbf{Contenido:} Este certamen evalúa los siguientes temas:

\vspace{-2mm}
\begin{table}[H]
  \begin{tabular}{
    !{\color{gray!50}\vrule}l
    !{\color{gray!50}\vrule}c
    !{\color{gray!50}\vrule}c
    !{\color{gray!50}\vrule}} \arrayrulecolor{gray!50} \hline
    \multicolumn{1}{!{\color{gray!50}\vrule}c}{\multirow{2}{*}{\textbf{
    Tema
    }}} &
          \multicolumn{2}{!{\color{gray!50}\vrule}c!{\color{gray!50}\vrule}}{\textbf{
          Puntajes
          }} \\ \arrayrulecolor{gray!50}\cline{2-3} &
                                                      \multicolumn{1}{!{\color{gray!50}\vrule}c!{\color{gray!50}\vrule}}{\textbf{
                                                      Total
                                                      }} &
                                                           \multicolumn{1}{c!{\color{gray!50}\vrule}}{\textbf{
                                                           Obtenido
                                                           }} \\ \arrayrulecolor{gray!50} \hline
    Problema 1: Programación en lenguaje Java, Herencia , polimorfismo y \\ Lectura/Escritura persistente. 
        & \multicolumn{1}{!{\color{gray!50}\vrule}c!{\color{gray!50}\vrule}}{\textbf{
          100 pts.
          }} & \\ \arrayrulecolor{gray!50} \hline

  \end{tabular}
\end{table}

\newpage

\vspace{-7mm}
\section{\textbf{Problema 1}}
\noindent
% \textbf{Plantamiento de problema: }
\begin{questions}
\item \textbf{\emph{40pts.}} Implemente un programa en lenguaje Java
  que permita instanciar objetos y almacenar del tipo \texttt{\textbf{Persona}} e
  identificar si cumple o no con la mayoria de edad, para este
  objetivo se debe implementar lo siguiente:

  \begin{enumerate}
    \item \emph{100pts.} Como Ingeniero Civil en Inform\'atica, se le ha solicitado desarrollar un sistema de registro de personas pertenecientes a la Escuela. Para eso, usted debe:
    \begin{enumerate}
        \item \emph{35pts.} Implementar las clases requeridas utilizando el concepto de Herencia bajo el siguiente esquema.
        \begin{enumerate}
          \item \emph{25pts.} Consideraciones:
            \begin{enumerate}
                \item[-] Tipo de persona (Acad\'emico, Alumno, Funcionario).
                \item[-] Los datos personales (DNI, Nombre, Apellidos, Edad, Direcci\'on.)
                \item[-] Si es Acad\'emico, se debe registrar su grado academico (licenciado, magister o Dr.).
                \item[-] Si es Alumno, se debe registrar las asignaturas que se encuentra cursando en el semestre (pueden ser m\'as de una).
                \item[-] Si es Funcionario, se debe registrar su oficio o profesión.
                \item[-] La informaci\'on solicitada, debe ser registrada en un archivo de texto plano, que deber\'a crearse, si no existe y agregar informaci\'on al final si ya fue creado.
            \end{enumerate}
          \item \emph{10pts.}  En un archivo llamado \textbf{justificacion\_modelo.txt} comenta y justifica si al problema planteado se le debe aplicar un esquema basado en herencia o implementación de interfaces.


        \end{enumerate}
        \item \emph{40pts.} Implementar una clase GestionPersona que provea lo siguiente:
        \begin{itemize}
          \item \emph{2pts.} Un Arreglo din\'amico de tipo ArrayList como atributo.
          \item \emph{16pts.} Implementar el m\'etodo crear que permita ingresar los datos requeridos para cada persona cumpliendo con el esquema de herencia, agregando cada persona al arreglo.
          \item \emph{12pts.} Implementar el m\'etodo guardar que recorra el el ArrayList y  que tenga la responsabilidad de guardar el registro dentro del archivo identificadose claramente el tipo de persona guardada.
                              El formaro del registro guardad debe ser el siguiente:
                              \begin{itemize}
                                \item TipoPersona;DNI;Nombre;Apellidos;Edad;Direcci\'on
                                \item[$\rightarrow$] Alumno;11111111-1;Juan;Gonzales Figueroa;Errazuriz 355 Valpara\'iso
                              \end{itemize}
          \item \emph{10pts.} Implementar el m\'etodo listar que lea el archivo y liste todos sus registros. El formato de salida debe ser el siguiente:
                            \begin{itemize}
                              \item[$\rightarrow$] Tipo: Alumno
                              \item[$\rightarrow$] DNI: 11111111-1
                              \item[$\rightarrow$] nombre: Juan
                              \item[$\rightarrow$] Apellidos: Gonzales Figueroa
                              \item[$\rightarrow$] Direcci\'on: Errazuriz 355 Valpara\'iso
                            \end{itemize}
        \end{itemize}
          \item \emph{25pts.} Implementar la clase PersonaMain que permita ejecutar el programa y los m\'etodos asociados. El n\'umero de personas a crear depende del usuario..
    \end{enumerate}
    \begin{itemize}
        \item[$\rightarrow$] \textbf{Recomendaci\'on}: Utilice el nombre FuenteDatos.txt para el archivo generado.
    \end{itemize}
\end{enumerate}
  \end{enumerate}
\end{questions}

\begin{table}[H]
  \centering
  \begin{tabular}{
    !{\color{gray!50}\vrule}p{3.9cm}
    !{\color{gray!50}\vrule}p{3.6cm}
    !{\color{gray!50}\vrule}p{3.6cm}
    !{\color{gray!50}\vrule}p{3.6cm}
    !{\color{gray!50}\vrule}} \arrayrulecolor{gray!50} \hline
    \multicolumn{4}{!{\color{gray!50}\vrule}c!{\color{gray!50}\vrule}}{\textbf{¿Cómo  seré evaluado en este trabajo?}} \\ \arrayrulecolor{gray!50}
    \hline
    %
    \textbf{Ítem} & \textbf{Logrado} & \textbf{Suficiente} & \textbf{No Logrado}\\ \arrayrulecolor{gray!50} \hline\newline
    Seccion a: Creación de Esquema de herencia de clases con atributos y m\'etodos respectivos. &
    Aplica de forma correcta: 25pts   &
    Aplica parcialmente con menos de 2 errores: 15pts  &
    Aplica de forma incorrecta con 3 errores o más: 10 - 0 pts \\ \arrayrulecolor{gray!50} \hline

    Seccion a: Justificación del modelo. &
    Aplica de forma correcta: 10pts   &
    Responde  parcialmente: 5pts  &
    Responde de forma incorrecta o no responde:  0 pts \\ \arrayrulecolor{gray!50} \hline

    seccion b: Implementación  m\'etodos de \textbf{GestionPersona}: \textbf{crear, guardar, listar}. &
    Aplica de forma correcta 100\%  del puntaje asociado &
    Aplica parcialmente  2 errores o menos 65\% - 50\%  del puntaje asociado en cada m\'etodo &
    Aplica de forma incorrecta con 3 errores o más 20\% - 0\% del puntaje asociado\\ \arrayrulecolor{gray!50} \hline

    Implementar clase instanciadora \textbf{PersonaMain} con m\'etodo main y crea instancias de objetos y llamadas a \textbf{m\'etodos requeridos}. &
    Aplica de forma correcta 25pts &
    Aplica parcialmente con menos de 2 errores 15 - 8 pts &
    Aplica de forma incorrecta con 3 errores o más 7 - 0pts\\ \arrayrulecolor{gray!50} \hline

    Total de la sección &  100pts &  61 - 20 pts & 18-0 pts\\ \arrayrulecolor{gray!50} \hline
  \end{tabular}
  \label{tbl:1}
\end{table}

\vspace{-5mm} \textbf{Nota:} En caso de que el ítem no esté presente,
tiene ponderación cero.

\end{document}
