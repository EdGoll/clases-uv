\documentclass{exam}

\usepackage[spanish]{babel}
\usepackage[utf8]{inputenc}
\usepackage[T1]{fontenc}
\usepackage[newcommands]{ragged2e}
\usepackage{hyperref}
\usepackage{algorithm,algorithmic}
\usepackage{colortbl}
\usepackage{graphicx}
\usepackage{multicol}
\usepackage{enumitem}
\usepackage{
  amsmath,
  amssymb,
  eso-pic,
  float,
  graphicx,
  lmodern,
  wrapfig,
  tabularx,
  multicol,
  multirow,
  color,
  colortbl,
  lastpage,
  titlesec,
  sectsty
}


\definecolor{azul}{RGB}{33,127,190}
\sectionfont{\color{azul}}
\subsectionfont{\color{azul}}
\renewcommand{\familydefault}{\sfdefault}

\footer{}{\thepage}{}

\makeatother

\title{\LARGE\color{azul}\textbf{INC 214 Programación 2 - Certamen 1 }}
\author{\smallsize \color{gray}{Profesores: } \color{black}{\textbf{Ismael Figueroa, Eduardo Godoy}}}
\date{\normalsize \em \today}

\begin{document}

\AddToShipoutPictureBG*{%
  \AtPageUpperLeft{\raisebox{-\height}{\includegraphics[scale=.95]{base/header.png}}}}

\maketitle

\begin{multicols}{2}
  \begin{flushleft}
    \textbf{Nombre:} \\
    \vspace*{2mm}
    \textbf{Rut:} \\
    \vspace*{2mm}
    \textbf{Paralelo:}
  \end{flushleft}
  \begin{center}
    \begin{table}[H]
      \begin{tabular}{p{4cm}|p{3cm}|}
        \arrayrulecolor{gray!50}\cline{2-2} ~ & {\em {\scriptsize \color{gray!50}{Puntaje:}}}\\
         & ~ \\
         ~ & \textbf{Nota:}
        \\ & ~ \\
        \arrayrulecolor{gray!50}\cline{2-2}
      \end{tabular}
    \end{table}
  \end{center}
\end{multicols}

%\vspace*{-18mm}
\noindent
\textbf{\\Instrucciones:}
\begin{itemize}
\item[-] El puntaje máximo  es 100 puntos.
\item[-] Tiempo máximo: 120 minutos.
\item[-] El trabajo es \underline{\textbf{individual}}. Cualquier intento de copia, será sancionado según dicta el reglamento de la carrera.
\item[-] Se debe subir al aula virtual un archivo comprimido con
  el siguiente formato: \\
  \textt{Certamen\_01\_<NombreApellidoEstudiante\_rut\_paralelo>.zip}. Dentro
  deben estar todos los archivos de código fuente requeridos.}
\end{itemize}

\noindent
\textbf{Resultados de aprendizaje a evaluar:}
\begin{enumerate}
\item Desarrollar algoritmos simples usando programación orientada a
  objetos y estructurada, en el lenguaje Java
\end{enumerate}
\vspace{2mm}

\noindent
\textbf{Contenido:} Este certamen evalúa los siguientes temas:

\vspace{-2mm}
\begin{table}[H]
  \begin{tabular}{
    !{\color{gray!50}\vrule}l
    !{\color{gray!50}\vrule}c
    !{\color{gray!50}\vrule}c
    !{\color{gray!50}\vrule}} \arrayrulecolor{gray!50} \hline
    \multicolumn{1}{!{\color{gray!50}\vrule}c}{\multirow{2}{*}{\textbf{
    Tema
    }}} &
          \multicolumn{2}{!{\color{gray!50}\vrule}c!{\color{gray!50}\vrule}}{\textbf{
          Puntajes
          }} \\ \arrayrulecolor{gray!50}\cline{2-3} &
                                                      \multicolumn{1}{!{\color{gray!50}\vrule}c!{\color{gray!50}\vrule}}{\textbf{
                                                      Total
                                                      }} &
                                                           \multicolumn{1}{c!{\color{gray!50}\vrule}}{\textbf{
                                                           Obtenido
                                                           }} \\ \arrayrulecolor{gray!50} \hline
    Problema 1: Programación en lenguaje Java, sintaxis básicas e instancias.
        & \multicolumn{1}{!{\color{gray!50}\vrule}c!{\color{gray!50}\vrule}}{\textbf{
          40 pts.
          }} & \\ \arrayrulecolor{gray!50} \hline
    Problema 2: Sentencias de control, arreglos dinámicos y paradigma OO.
        & \multicolumn{1}{!{\color{gray!50}\vrule}c!{\color{gray!50}\vrule}}{\textbf{
          60 pts.
          }} & \\ \arrayrulecolor{gray!50} \hline

  \end{tabular}
\end{table}

\newpage

\vspace{-7mm}
\section{\textbf{Problema 1}}
\noindent
% \textbf{Plantamiento de problema: }
\begin{questions}
\item \textbf{\emph{40pts.}} Implemente un programa en lenguaje Java
  que permita instanciar objetos del tipo \texttt{\textbf{Persona}} e
  identificar si cumple o no con la mayoria de edad, para este
  objetivo se debe implementar lo siguiente:

  \begin{enumerate}

  \item Una clase \textbf{Persona} que posea:

    \begin{enumerate}

    \item \emph{(5pts)} Características o atributos: \emph{primer
        nombre}, \emph{segundo nombre}, \emph{apellido paterno},
      \emph{apellido materno}, \emph{edad}. Debe utilizar el tipo de
      dato idóneo para cada atributo.

    \item \emph{(10pts)} Un constructor explícito, que reciba e
      inicialice todos los atributos de la clase.

    \item \emph{(10pts)} Definir su comportamiento asociado:
      \textbf{esMayorDeEdad()} para determinar lo
      requerido. Determinando si la persona es o no mayor de edad
      mostrando mensaje por pantalla.
    \end{enumerate}

    \item Una clase \texttt{\textbf{Main.java}} con método
      principal \texttt{main} encargada de:
      \begin{enumerate}

      \item \emph{(10pts)} Instanciar dos objetos \texttt{Persona}
        para los siguientes individuos:

        \begin{itemize}
        \item Pedro José Figueroa Fernandez, de 16 años
        \item Juan Antonio Godoy Guerra, de 20 años
        \end{itemize}

      \item \emph{(5pts)} Mostrar por pantalla el nombre completo de
        cada \texttt{Persona} y un mensaje que diga si es o no es
        mayor de edad.

      \end{enumerate}

    \end{enumerate}
  \end{enumerate}
\end{enumerate}
\end{questions}

\begin{table}[H]
  \centering
  \begin{tabular}{
    !{\color{gray!50}\vrule}p{3.9cm}
    !{\color{gray!50}\vrule}p{3.6cm}
    !{\color{gray!50}\vrule}p{3.6cm}
    !{\color{gray!50}\vrule}p{3.6cm}
    !{\color{gray!50}\vrule}} \arrayrulecolor{gray!50} \hline
    \multicolumn{4}{!{\color{gray!50}\vrule}c!{\color{gray!50}\vrule}}{\textbf{¿Cómo  seré evaluado en este trabajo?}} \\ \arrayrulecolor{gray!50}
    \hline
    %
    \textbf{Ítem} & \textbf{Logrado} & \textbf{Suficiente} & \textbf{No Logrado}\\ \arrayrulecolor{gray!50} \hline\newline
    Creación de Clase Persona.java con atributos. &
    Aplica de forma correcta: 5pts   &
    Aplica parcialmente con menos de 2 errores: 3pts  &
    Aplica de forma incorrecta con 3 errores o más: 0pts \\ \arrayrulecolor{gray!50} \hline

    Creación de Constructor con atributos. &
    Aplica de forma correcta: 10pts   &
    Aplica parcialmente con menos de  errores: 5pts  &
    Aplica de forma incorrecta con 3 errores o más: 0pts \\ \arrayrulecolor{gray!50} \hline

    Implementación de método \textbf{esMayorDeEdad} asociados y correcto funcionamiento. &
    Aplica de forma correcta 10pts   &
    Aplica parcialmente  2 errores o menos 5pts &
    Aplica de forma incorrecta con 3 errores o más 0pts\\ \arrayrulecolor{gray!50} \hline

    Implementar clase instanciadora \textbf{Main} con m\'etodo main y crea instancias de objetos y llamadas a \textbf{m\'etodos requeridos}. &
    Aplica de forma correcta 15pts &
    Aplica parcialmente con menos de 2 errores 8pts &
    Aplica de forma incorrecta con 3 errores o más 0pts\\ \arrayrulecolor{gray!50} \hline

    Total de la sección &  40pts & 21pts & 0pts\\ \arrayrulecolor{gray!50} \hline
  \end{tabular}
  \label{tbl:1}
\end{table}

\vspace{-5mm} \textbf{Nota:} En caso de que el ítem no esté presente,
tiene ponderación cero.



\newpage
\vspace{-7mm}
\section{\textbf{Problema 2}}
\noindent
% \textbf{Plantamiento de problema: }

\begin{questions}

  \begin{enumerate}
  \item \textbf{\emph{60pts.}} Cree un programa que permita realizar una correcta gestion de bodega de un antiguo almacen, para esto se requiere  lo siguiente:
    \begin{itemize}

    \item \emph{(10pts)} Crear la clase \textbf{Producto} que tenga
      los siguientes atributos:

      \begin{enumerate}
      \item \emph{código}: identificar de manera única al producto.
      \item \emph{nombre}: el nombre del producto.
      \item \emph{stock}: maneja la cantidad de productos del mismo
        código que actualmente se tiene en el almacen.
      \item \emph{precio}: precio del producto con iva incluido.
      \end{enumerate}

      Debe crear un constructor explícito que reciba todos los
      atributos y los inicialice correctamente.

      Todos los atributos deben ser privados. Para manipular estos
      valores debe implementar los metodos \texttt{setX} y
      \texttt{getX}, donde \texttt{X} es el nombre de cada atributo.

    \item \emph{(5pts)} Crear la clase \textbf{GestionBodega} que
      posea el atributo \texttt{listadoProductos}, un arreglo dinámico
      que almacena objetos de tipo \texttt{Producto} y llamado
      \textbf{listaProductos}.

    \item La clase \texttt{GestionBodega} debe poseer los siguientes métodos:
      \begin{enumerate}

      \item \emph{(10pts)} \texttt{crearProducto} que permita crear un
        producto específico y asignarle un stock inicial. Este método
        recibe como parámetro el precio sin IVA (19\%), por lo que
        debe calcularse el precio actualizado a ingresar en el
        producto que se está creando. Además, este método debe agregar
        el producto creado al listado de productos de
        \texttt{GestionBodega}

      \item \emph{(15 pts)} \texttt{ingresarProductosConsola} que lee
        desde el teclado la información para la creación de productos,
        con datos ingresados por el usuario. La acción de crear
        productos debe repetirse mientras el usuario lo desee. Puede
        usar alguna convención de su elección para terminar la
        lectura. PISTA: en la implementación debe invocar a
        \texttt{crearProducto}.

      \item \emph{(10pts)} \texttt{listarStock} que permita visualizar
        el código, nombre y stock de cada producto.


      \item \emph{(10pts)} \texttt{valorInventario} que calcula el
        valor del inventario actual, mostrando la suma de precio por
        stock, y muestra el total con y sin IVA.

      \end{itemize}

    \item \emph{(10pts)} Crear la clase \texttt{Main.java} con un
      método \texttt{main} que se encargue de inicializar el programa creando la instancia de la
      clase \texttt{GestionBodega} para utilizar sus métodos asociados de la siguiente forma:
      \begin{enumerate}
      \item \texttt{ingresarProductosConsola()}, para crear 1 o más
        productos por parte del usuario
      \item \texttt{listarStock()}, para mostrar los productos creados
      \item \texttt{valorInventario()}, para mostrar los precios de
        los productos listados
      \end{enumerate}

    \end{enumerate}
  \end{questions}


  \begin{table}[H]
    \centering
    \begin{tabular}{
      !{\color{gray!50}\vrule}p{3.9cm}
      !{\color{gray!50}\vrule}p{3.6cm}
      !{\color{gray!50}\vrule}p{3.6cm}
      !{\color{gray!50}\vrule}p{3.6cm}
      !{\color{gray!50}\vrule}} \arrayrulecolor{gray!50} \hline
      \multicolumn{4}{!{\color{gray!50}\vrule}c!{\color{gray!50}\vrule}}{\textbf{¿Cómo seré evaluado en este trabajo?}} \\ \arrayrulecolor{gray!50} \hline
      \textbf{Ítem} & \textbf{Logrado} & \textbf{Suficiente} & \textbf{No Logrado}\\ \arrayrulecolor{gray!50} \hline
      Creación de Clase Producto.java con atributos. &
      Aplica de forma correcta 10pts   &
      Aplica parcialmente con menos de 2 errores 5pts  &
      Aplica de forma incorrecta con 3 errores o más 0pts\\ \arrayrulecolor{gray!50} \hline

      Creación de clase \texttt{GestionBodega.java} con arreglo dinámico implementado. &
      Aplica de forma correcta 5pts   &
      Aplica parcialmente con menos de 2 errores 3pts  &
      Aplica de forma incorrecta con 3 errores o más 0pts\\ \arrayrulecolor{gray!50} \hline

     Implementación de método \textbf{crearProducto} y correcto funcionamiento. &
     Aplica de forma correcta 10pts   &
     Aplica parcialmente  2 errores o menos 5pts &
     Aplica de forma incorrecta con 3 errores o más 0pts\\ \arrayrulecolor{gray!50} \hline

     Implementación de método \textbf{ingresarProductosConsola} y correcto funcionamiento. &
     Aplica de forma correcta 10pts   &
     Aplica parcialmente  2 errores o menos 8pts &
     Aplica de forma incorrecta con 3 errores o más 0pts\\ \arrayrulecolor{gray!50} \hline

     Implementación de método \textbf{listarStock} y correcto funcionamiento. &
     Aplica de forma correcta 5pts   &
     Aplica parcialmente  2 errores o menos 2pts &
     Aplica de forma incorrecta con 3 errores o más 0pts\\ \arrayrulecolor{gray!50} \hline

     Implementación de método \textbf{valorInventario} y correcto funcionamiento. &
     Aplica de forma correcta 10pts   &
     Aplica parcialmente  2 errores o menos 5pts &
     Aplica de forma incorrecta con 3 errores o más 0pts\\ \arrayrulecolor{gray!50} \hline

     Implementar clase instanciadora \textbf{Main} con método \texttt{main} y crea instancias de objetos y llamadas a \textbf{métodos requeridos}. &
     Aplica de forma correcta 10pts &
     Aplica parcialmente con menos de 2 errores 5pts &
     Aplica de forma incorrecta con 3 errores o más 0pts\\ \arrayrulecolor{gray!50} \hline

      Total de la sección &  60pts & 30pts & 0pts\\ \arrayrulecolor{gray!50} \hline
    \end{tabular}
    \label{tbl:1}
  \end{table}
  \vspace{-5mm}
  \textbf{Nota:} En caso de que el {í}tem no est{é} presente, tiene ponderaci{ó}n cero.

\end{document}
