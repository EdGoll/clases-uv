 \documentclass{exam}
 \usepackage[spanish]{babel}
\usepackage[spanish,activeacute]{babel}
\usepackage[utf8]{inputenc}
\usepackage[T1]{fontenc}
\usepackage[newcommands]{ragged2e}
\usepackage{hyperref}
\usepackage{algorithm,algorithmic}
\usepackage{colortbl}
\usepackage{graphicx}
\usepackage{multicol}
\usepackage{enumitem}
\usepackage{
    amsmath,
    amssymb,
    eso-pic,
    float,
    graphicx,
    lmodern,
    wrapfig,
    tabularx,
    multicol,
    multirow,
    color,
    colortbl,
    lastpage,
    titlesec,
    sectsty
}

\definecolor{azul}{RGB}{33,127,190}
\sectionfont{\color{azul}}
\subsectionfont{\color{azul}}
\renewcommand{\familydefault}{\sfdefault}

\footer{}{\thepage}{}

\makeatother

\title{\LARGE\color{azul}\textbf{ICI 202 Programaci\'on 2 - Certamen 01 }}
\author{\normalsize \color{gray}{Prof.} \color{black}{\textbf{Ismael Figueroa, Eduardo Godoy}}}
\date{\normalsize \em \today}

\begin{document}

\AddToShipoutPictureBG*{%
  \AtPageUpperLeft{\raisebox{-\height}{\includegraphics[scale=.95]{base/header.png}}}}

\maketitle

\begin{multicols}{2} \begin{flushleft} \textbf{Nombre:} \\ \vspace*{2mm} \textbf{Rut:} \\ \vspace*{2mm} \textbf{Paralelo:} \end{flushleft} \begin{center} \begin{table}[H] \begin{tabular}{p{4cm}|p{3cm}|} \arrayrulecolor{gray!50}\cline{2-2} ~ & {\em {\scriptsize \color{gray!50}{Puntaje:}}} \\ & ~ \\ ~ & \textbf{Nota:} \\ & ~ \\ \arrayrulecolor{gray!50}\cline{2-2} \end{tabular} \end{table} \end{center} \end{multicols}

\vspace*{-18mm}
\noindent
\textbf{\\Instrucciones:}
\begin{itemize}
    \item[-] El puntaje m\'aximo  es 100 puntos.
    \item[-] Tiempo m\'aximo: 120 minutos.
    \item[-] El trabajo es \underline{\textbf{individual}}. Cualquier intento de copia, ser\'a sancionado seg\'un dicta el reglamento de la carrera.
    \item[-] Se debe subir al aula virtual un archivo comprimido con el siguiente formato \\ {Certamen_01\_<NombreApellidoEstudiante\_rut\_paralelo>.zip, dentro deben estar los c\'odigos fuentes requeridos.}
\end{itemize}

\noindent
\textbf{Resultados de aprendizaje a evaluar:}
\begin{enumerate}
  \item Caracter\'isticas de Orientaci\'on a objetos.
  \item Modelamiento y su implementaci\'on.
\end{enumerate}
\vspace{2mm}

\noindent
\textbf{Contenido:} Este certamen eval\'ua los siguientes temas:

\vspace{-2mm}
\begin{table}[H]
\begin{tabular}{
    !{\color{gray!50}\vrule}l
    !{\color{gray!50}\vrule}c
    !{\color{gray!50}\vrule}c
    !{\color{gray!50}\vrule}} \arrayrulecolor{gray!50} \hline
    \multicolumn{1}{!{\color{gray!50}\vrule}c}{\multirow{2}{*}{\textbf{
        Tema
    }}} &
    \multicolumn{2}{!{\color{gray!50}\vrule}c!{\color{gray!50}\vrule}}{\textbf{
        Puntajes
    }} \\ \arrayrulecolor{gray!50}\cline{2-3} &
    \multicolumn{1}{!{\color{gray!50}\vrule}c!{\color{gray!50}\vrule}}{\textbf{
        Total
    }} &
    \multicolumn{1}{c!{\color{gray!50}\vrule}}{\textbf{
        Obtenido
    }} \\ \arrayrulecolor{gray!50} \hline
    Problema 2: Progrmaci\'on en lenguaje java, sintaxis b\'asicas e instancias.
    & \multicolumn{1}{!{\color{gray!50}\vrule}c!{\color{gray!50}\vrule}}{\textbf{
        40 pts.
    }} & \\ \arrayrulecolor{gray!50} \hline
    Problema 3: Sentencias de control, arreglos di\'amicos y paradigma OO.
    & \multicolumn{1}{!{\color{gray!50}\vrule}c!{\color{gray!50}\vrule}}{\textbf{
        60 pts.
    }} & \\ \arrayrulecolor{gray!50} \hline

\end{tabular}
\end{table}

\newpage

\vspace{-7mm}
\section{\textbf{Problema 1}}
\noindent
%\textbf{Plantamiento de problema: }
\begin{questions}
  \item \textbf{\emph{40pts.}}Implemente un programa en lenguaje java que permita instanciar objeto del tipo persona e
  identificar si cumple o no con la mayoria de edad, para este objetivo se debe implementar lo siguiente:
  \begin{enumerate}
  		\item Una clase \textbf{Persona} que posea:
  		\begin{enumerate}
  			\item \emph{(10pts)} Caracter\'isticas: \emph{nombre(s)}, \emph{apellido}, \emph{edad}. Considerar el tipo de dato id\'oneo para cada uno de los atributos y definir el valor por omisi\'on acorde a cada tipo.
  			\item \emph{(15pts)} Definir su comportamiento asociado:  \textbf{esMayorDeEdad()} para determinar lo requerido. Determinando si la persona es o no mayor de edad mostrando mensaje por pantalla mediante un mensaje en pantalla.
  			\item \emph{(15pts)} Crear Clase Main.java que posea el m\'etodo main(String [] args) encargada de:
                    \begin{enumerate}
        \item \emph{(5pts)} Inicializar el programa creando la instancia de la
        clase MedicionPersona. Adema\'as se deben generar la instancia de los siguientes objetos.

              \item \emph{(5pts)} Primer Objeto:
              \begin{itemize}
                \item nombre = Pedro
                \item apellido=Figueroa
                \item edad = 16;
              \end{itemize}
              \item \emph{(5pts)} Segundo Objeto:
              \begin{itemize}
                \item nombre = Juan.
                \item apellido=Godoy.
                \item edad = 20;
              \end{itemize}
            \end{enumerate}

  			\end{enumerate}
  		\end{enumerate}
      \end{enumerate}
\end{questions}
\begin{table}[H]
\centering
\begin{tabular}{
!{\color{gray!50}\vrule}p{3.9cm}
!{\color{gray!50}\vrule}p{3.6cm}
!{\color{gray!50}\vrule}p{3.6cm}
!{\color{gray!50}\vrule}p{3.6cm}
!{\color{gray!50}\vrule}} \arrayrulecolor{gray!50} \hline
    \multicolumn{4}{!{\color{gray!50}\vrule}c!{\color{gray!50}\vrule}}{\textbf{?`C\'omo ser{\'e} evaluado en este trabajo?}} \\ \arrayrulecolor{gray!50} \hline
    \textbf{\'Item} & \textbf{Logrado} & \textbf{Suficiente} & \textbf{No Logrado}\\ \arrayrulecolor{gray!50} \hline
  \newline Creaci\'on de Clase Persona.java con atributos. &
  \newline  Aplica de forma correcta 10pts   &
  \newline Aplica parcialmente con menos de 2 errores 5pts  &
    \newline Aplica de forma incorrecta con 3 errores o m\'as 0pts\\ \arrayrulecolor{gray!50} \hline

    \newline  Implementaci\'on de m\'etodo (comportamientos) \textbf{esMayorDeEdad} asociados y correcto funcionamiento. &
    \newline  Aplica de forma correcta 15pts   &
    \newline Aplica parcialmente  2 errores o menos 8pts &
    \newline Aplica de forma incorrecta con 3 errores o m\'as 0pts\\ \arrayrulecolor{gray!50} \hline


  \newline  Implementar clase instanciadora \textbf{Main} con m\'etodo main y crea instancias de objetos y llamadas a \textbf{m\'etodos requeridos}. &
  \newline  Aplica de forma correcta 15pts &
    \newline  Aplica parcialmente con menos de 2 errores 8pts &
    \newline Aplica de forma incorrecta con 3 errores o m\'as 0pts\\ \arrayrulecolor{gray!50} \hline

Total de la secci\'on &  40pts & 21pts & 0pts\\ \arrayrulecolor{gray!50} \hline
\end{tabular}
\label{tbl:1}
\end{table}
\vspace{-5mm}
\textbf{Nota:} En caso de que el {\'i}tem no est{\'e} presente, tiene ponderaci{\'o}n cero.



\newpage
\vspace{-7mm}
\section{\textbf{Problema 2}}
\noindent
%\textbf{Plantamiento de problema: }

\begin{questions}

  \begin{enumerate}
    \item \textbf{\emph{60pts.}} Cree un programa que permita realizar una correcta gestion de bodega de un antiguo almacen, para esto se requiere  lo siguiente:
    \begin{itemize}

    \item \emph{(10pts)} Crear La clase  \textbf{Producto} que tenga los siguientes atributos:
      \begin{enumerate}
        \item c\'odigo: de tipo entero, cuya responsabilidad es identificar al producto.
    	  \item nombre: de tipo string, que contiene al nombre del producto.
    	  \item stock: de tipo entero, dato encargado de manejar la cantidad de productos del mismo nombre que actualmente se tiene en el almacen.
    	  \item precio: precio del producto con iva incluido.
    \end{enumerate}
    Adem\'as debe contener los m\'etodos get y set asociados a cada atributo.

    \item \emph{(10pts)} Crear la Clase \textbf{GestionBodega} que posea como atributo:
     un arreglo de tipo dinámico de tipo Producto y llamado \textbf{listaProductos} que permita almacenar los productos que crear\'a el usuario.

    \item \emph{(20pts)} La clase GestionBodega debe poseer los siguientes m\'etodos:
    \begin{enumerate}

      \item \emph{(10pts)} Codificar el m\'etodo \textbf{crearProducto} que permita crear un  producto específico y asignarle un stock inicial.
      Se debe considerar que al asignarle el precio al producto, el sistema debe agregarle  de forma automática el iva (19\%) sobre el precio ingresado.
      Luego de esto se procederá a guardar el resultado en el atributo precio de la clase producto. Para finalizar agregandolo al arreglo listaProductos.
      La acci\'on de crear productos debe repetirse mientras el usuario lo desee.

      \item \emph{(10pts)} Codificar el m\'etodo \textbf{listarStock} que permita visualizar el c\'odigo, nombre y stock de cada producto.


    \end{itemize}
    \item \emph{(10pts)} Crear Clase Main.java que posea el m\'etodo main(String [] args) encargada de inicializar el programa creando la instancia de la
    clase GestionBodega y utilizar sus m\'etodos asociados de la siguiente forma:
      \begin{enumerate}
        \item crearProducto() -> se crea primer producto.
        \item crearProducto() -> se crea segundo producto.
        \item listarStock() -> se listan ambos productos.
      \end{enumerate}
    \item \emph{(10pts)} Resuelve el problema utilizando el paradigma de Orientaci\'on a Objectos.

    \end{enumerate}
\end{questions}


\begin{table}[H]
\centering
\begin{tabular}{
!{\color{gray!50}\vrule}p{3.9cm}
!{\color{gray!50}\vrule}p{3.6cm}
!{\color{gray!50}\vrule}p{3.6cm}
!{\color{gray!50}\vrule}p{3.6cm}
!{\color{gray!50}\vrule}} \arrayrulecolor{gray!50} \hline
    \multicolumn{4}{!{\color{gray!50}\vrule}c!{\color{gray!50}\vrule}}{\textbf{?`C\'omo ser{\'e} evaluado en este trabajo?}} \\ \arrayrulecolor{gray!50} \hline
    \textbf{\'Item} & \textbf{Logrado} & \textbf{Suficiente} & \textbf{No Logrado}\\ \arrayrulecolor{gray!50} \hline
  \newline Creaci\'on de Clase Producto.java con atributos. &
  \newline  Aplica de forma correcta 10pts   &
  \newline Aplica parcialmente con menos de 2 errores 5pts  &
    \newline Aplica de forma incorrecta con 3 errores o m\'as 0pts\\ \arrayrulecolor{gray!50} \hline

    \newline Creaci\'on de Clase GestionBodega.java con arreglo din\'amico implementado. &
    \newline  Aplica de forma correcta 10pts   &
    \newline Aplica parcialmente con menos de 2 errores 5pts  &
      \newline Aplica de forma incorrecta con 3 errores o m\'as 0pts\\ \arrayrulecolor{gray!50} \hline

    \newline  Implementaci\'on de m\'etodo (comportamientos) \textbf{crearProducto} asociados y correcto funcionamiento. &
    \newline  Aplica de forma correcta 10pts   &
    \newline Aplica parcialmente  2 errores o menos 5pts &
    \newline Aplica de forma incorrecta con 3 errores o m\'as 0pts\\ \arrayrulecolor{gray!50} \hline

    \newline Implementaci\'on de m\'etodo (comportamientos) \textbf{listarStock} asociados y correcto funcionamiento. &
    \newline Aplica de forma correcta 10pts   &
    \newline Aplica parcialmente  2 errores o menos 5pts &
    \newline Aplica de forma incorrecta con 3 errores o m\'as 0pts\\ \arrayrulecolor{gray!50} \hline

  \newline  Implementar clase instanciadora \textbf{Main} con m\'etodo main y crea instancias de objetos y llamadas a \textbf{m\'etodos requeridos}. &
  \newline  Aplica de forma correcta 10pts &
    \newline  Aplica parcialmente con menos de 2 errores 5pts &
    \newline Aplica de forma incorrecta con 3 errores o m\'as 0pts\\ \arrayrulecolor{gray!50} \hline

    \newline  Cumple con el \textbf{paradigma orientación a objetos y compilaci\'on correcta} &
    \newline  Aplica de forma correcta 10pts &
    \newline Aplica parcialmente con con deficiencias en el paradigma 5 pto &
    \newline Aplica de forma incorrecta con x errores o m\'as 0pts\\ \arrayrulecolor{gray!50} \hline
    Total de la secci\'on &  60pts & 30pts & 0pts\\ \arrayrulecolor{gray!50} \hline
\end{tabular}
\label{tbl:1}
\end{table}
\vspace{-5mm}
\textbf{Nota:} En caso de que el {\'i}tem no est{\'e} presente, tiene ponderaci{\'o}n cero.

\end{document}
