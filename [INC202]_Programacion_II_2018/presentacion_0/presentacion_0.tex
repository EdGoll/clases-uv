\documentclass{beamer}

\mode<presentation>
{
	\usetheme{CambridgeUS}
	\setbeamercovered{transparent}
}
\usepackage[spanish]{babel}
\usepackage[latin1]{inputenc}
\usepackage{color}
\usepackage{hyperref}
\usepackage{algorithm,algorithmic}
\usepackage{soul}

\title[\textbf{Programaci\'on 2}]{\textbf{Programaci\'on 2}}

\subtitle{Descripci\'on del curso}

%\author[Eduardo Godoy]
{
	Profesores: Ismael Figueroa / Eduardo Godoy \\
	%Material elaborado por: Rodrigo Olivares \\
	%\vspace{0.5mm}
	%Mg. en Ingenier\'ia Inform\'atica \\
	\vspace{0.5mm}
	\texttt{\small ismael.figueroa@pucv.cl}
	\texttt{\small eduardo.gl@gmail.com}
}

\institute[Universidad de Valpara\'iso]

%\date{$2^{do}$ Semestre de 2016}

\subject{Descripci\'on del curso}

%\AtBeginSection
%{
%	\begin{frame}<beamer>
%	\frametitle{Contenido}
%	\tableofcontents[currentsection,currentsubsection]
%	\end{frame}
%}
%
%\AtBeginSubsection
%{
%	\begin{frame}<beamer>
%	\frametitle{Contenido}
%	\tableofcontents[currentsection,currentsubsection]
%	\end{frame}
%}
%
%\beamerdefaultoverlayspecification{<+->}

\begin{document}

	\begin{frame}
		\titlepage
	\end{frame}

	\begin{frame}
		\frametitle{Contenido}
		\tableofcontents%[pausesections]
	\end{frame}

	\section{Antecedentes de la asignatura}

		\subsection{Descripci\'on}

		\begin{frame}
			\frametitle{Antecedentes de la asignatura}
			\framesubtitle{Descripci\'on}

			\begin{itemize}
				\item Nombre: \textbf{Programaci\'on 2}
				\item Horario:
				\begin{itemize}
					\item \textbf{Lunes: 18:00 - 19:30 hrs.}
					\item \textbf{Lunes: 19:45 - 21:15 hrs.}
					\item \textbf{Miercoles: 18:00 - 19:30 hrs.}
				\end{itemize}
				\item Ayudant\'ia: pendiente
			\end{itemize}
		\end{frame}

		\subsection{Unidades tem\'aticas}

		\begin{frame}
			\frametitle{Antecedentes de la asignatura}
			\framesubtitle{Unidades tem\'aticas}

			\begin{itemize}
				\item Unidad I: \textbf{Programaci\'on Orientado a Objeto}
				\begin{itemize}
					\item Paradigmas de programaci\'on.
					\item Lenguajes de programaci\'on.
					\item Paradigma Orientado a Objeto.
					\item Modelamiento Orientado a Objeto.
					\item Principales caracter\'isticas:
\begin{itemize}
						\item {Abstracc\'ion}.
						\item {Encapsulamiento}.
						\item {Ocultamiento}.
						\item{ Herencia}.
						\item {Polimofismo}.
\end{itemize}
				\end{itemize}
			\end{itemize}
		\end{frame}

		\begin{frame}
			\frametitle{Antecedentes de la asignatura}
			\framesubtitle{Unidades tem\'aticas}

			\begin{itemize}
				\item Unidad II: \textbf{Lenguaje de programaci\'on JAVA}
				\begin{itemize}
					\item Ambiente  de desarrollo.
					\item Fundamentos de JAVA (sintaxis,variables, tipos de datos, expresiones y operadores)
					\item Arreglos, condicionales y bucles (ciclos)
					\item Encapsulamiento (clases, m\'etodos y objetos)
					\item Herencia y polimorfismo
					\item Manejo de excepciones
					\item Interfaces gr\'aficas
					\item Entradas y salidas de datos.
					\item Manejo de Memoria en Java.
				\end{itemize}
			\end{itemize}
		\end{frame}

		\begin{frame}
			\frametitle{Antecedentes de la asignatura}
			\framesubtitle{Unidades tem\'aticas}

			\begin{itemize}
				\item Unidad III: \textbf{Patrones de dise\~no}
				\begin{itemize}
					\item Principios
					\item Descripci\'on
					\item Clasificaci\'on
					\item Framewok
					\item Aplicaci\'on
				\end{itemize}
			\end{itemize}
		\end{frame}

	\section{Sistema de evaluaci\'on}

		\subsection{Evaluaciones}

		\begin{frame}
			\frametitle{Sistema de evaluaci\'on}
			\framesubtitle{Evaluaciones}

			\begin{itemize}
				\item Escala de evaluaci\'on: seg\'un lo dicta el reglamento de la Escuela.
				\item Cert\'amenes: \textbf{2} .
				\begin{itemize}
					\item Son evaluadas en horario de clases.
					\item Toda inasistencia debe ser justificada.
				\end{itemize}
				\item Controles: \textbf{3}.
				\begin{itemize}
					\item Ser\'an evaluados en horario de clases o ayudant\'ia.
					\item Toda inasistencia debe ser justificada. De no ser as\'i se calificar\'a con nota \textbf{1,0}.
				\end{itemize}
				\item Proyecto: \textbf{3} Modo incremental.
				\begin{itemize}
				    \item Ser\'an evaluados por el ayudante.
					\item Informe t\'ecnico
					\item C\'odigo fuente.
					\item Reprotes de estados de avance. (se seleccionar\'a un integrante al azar, el cual deberá exponer el estado del proyecto al curso).
				\end{itemize}
			\end{itemize}
		\end{frame}

		\begin{frame}
			\frametitle{Sistema de evaluaci\'on}
			\framesubtitle{Evaluaciones}

			\begin{itemize}
				\item \textbf{Certamen recuperativo}: Considera todo el contenido de la asignatura y se evalur\'a al final de la misma. Es obligatoria a quien justifique una inasistencia a un certamen.
				\item \textbf{Prueba especial}: Obligatoria para quienes tengan \textbf{promedio de certamenes} entre \color{red}{3.5} \color{black}{y} \color{violet}{3.9}.
				\item En caso de rendir la prueba especial:
				\begin{itemize}
    				\item Si la nota de la prueba especial $\geq$ \textbf{4,0}, entonces la nota final es \textbf{4,0}
	    			\item Si la nota de la prueba especial $<$ \textbf{4,0}, entonces se mantiene la nota final.
				\end{itemize}
			\end{itemize}
		\end{frame}

		\subsection{M\'etodo de evaluaci\'on}

		\begin{frame}
			\frametitle{Sistema de evaluaci\'on}
			\framesubtitle{M\'etodo de evaluaci\'on}

			\begin{block}{}
				\begin{center}
					\begin{itemize}
						%\textbf{\color{cyan}{Notas de certamen}} & \textbf{\color{cyan}{Notas de controles}} & \textbf{\color{cyan}{Notas de tareas}} \\ & & \\
  							%$NC = \frac{\left(\displaystyle\sum_{i=1}^{3}{C_{i}}\right)}{3}$ &
							%$NQ = \frac{\left(\displaystyle\sum_{i=1}^{5}{Q_{i}}\right)}{5}$ &
  							%$NT = \frac{\left(\displaystyle\sum_{i=1}^{3}{T_{i}}\right)}{3}$ \\
  							\item$NC ={\left(\displaystyle(C1 * 0.4) +(C2 * 0.6)\right)}$ \\
							\item$NQ = \frac{\left(\displaystyle\sum_{i=1}^{3}{Q_{i}}\right)}{3}$ \\
  							\item$NT = (\displaystyle\frac{\left(\displaystyle\sum_{i=1}^{3}{T_{i}}\right)}{3})*0.6 +   (\displaystyle\frac{\left(\displaystyle\sum_{i=1}^{3}{INT_{i}}\right)}{3})*0.4$
					\end{itemize}
				\end{center}
donde:
				\begin{itemize}

					\item[] $C_{i}$ es el certamen en proceso.
					\item[] $Q_{i}$ es el control en proceso.
					\item[] $T_{i}$ es la tarea en proceso.
					\item[] $INT_{i}$ es la interrogaci\'on de avance de tarea.
				\end{itemize}
				%\begin{center}
				%	\begin{tabular}{c}
				%		\textbf{\color{red}{Notas de certamen}} \\
  				%			$NF = NC * 0.75 + NQ * 0.15 + NT * 0.10$\\
				%	\end{tabular}
				%\end{center}
			\end{block}
		\end{frame}

\subsection{M\'etodo de evaluaci\'on}

\begin{frame}
			\frametitle{Sistema de evaluaci\'on}
			\framesubtitle{M\'etodo de evaluaci\'on}
	\begin{block}{}
		\begin{center}
					\begin{tabular}{c}
						\textbf{\color{red}{Notas de certamen}} \\
  							$NF = NC * 0.70 + NQ * 0.15 + NT * 0.15$\\
					\end{tabular}
		\end{center}
	\end{block}
\end{frame}

		\subsection{Cronograma de evaluaci\'on}

		\begin{frame}
			\frametitle{Sistema de evaluaci\'on}
			\framesubtitle{Cronograma de evaluaci\'on}

			\textbf{Fechas de las evaluaciones}

			\begin{itemize}
				\item Certamen 1: \textbf{Miercoles 13 de Septiembre}
				\item Certamen 2: \textbf{Miercoles 22 de Noviembre}
				%\item Certamen 3: \textbf{Jueves 29 de Junio}
				\item Controles: \textbf{Durante el semestre.}
				\item Tarea 1: \textbf{Miercoles 6 de Septiembre}
				\item Tarea 2: \textbf{Miercoles 11 de Octubre}
				\item Tarea 3: \textbf{Miercoles 15 de Noviembre}
				\item Certamen recuperativo: \textbf{Por definir}
				\item Prueba Especial: \textbf{Entre el 11 y 15 de Diciembre}
			\end{itemize}
		\end{frame}

	\section{Consideraciones}

		\subsection{Software}

		\begin{frame}
			\frametitle{Consideraciones}
			\framesubtitle{Ambiente de Desarrollo.}

			\textbf{Entornos de desarrollo integrados (IDE):}

			\begin{itemize}
				\item Eclipse  \url{http://www.eclipse.org}
				\item NetBeans \url{http://www.netbeans.org}
				\item IntelliJ IDEA  \url{http://www.jetbrains.com/idea/download/}
				\item SublimeText   \url{http://www.sublimetext.com/3}
				\item Notepad++  \url{http://notepad-plus-plus.org/}
			\end{itemize}

		\end{frame}

		\begin{frame}
			\frametitle{Consideraciones}
			\framesubtitle{Ambiente de Desarrollo}

			\textbf{Kit para Desarrollo en Java:}

			\begin{itemize}
				\item Java Development Kit (jdk), disponible en \url{http://developers.redhat.com/products/openjdk/download/}.
			\end{itemize}
		\end{frame}

	\section{Bibliograf\'ia}

		\begin{frame}
			\frametitle{Bibliograf\'ia}

			\begin{thebibliography}{10}
				\beamertemplatebookbibitems
				\bibitem{java611}[Programaci\'on en Java 6] Luis Joyanes Aguilar. \newblock \emph{Editorial McGraw-Hill}, 2011
				\bibitem{poo2007}[Programaci\'on orientada a objetos con Java] Barnes, David \& Kolling, Michael. \newblock \emph{2nd. Edition, Prentice-Hall}, 2007
				\bibitem{poo2007}[Fundamentos de Java] Schildt Herbert. \newblock \emph{2nd. Edition, McGraw-Hill}, 2007.

\bibitem{poo2006}[SCJP Sun Certified Programmer for Java 6 Study Guide]\\ Kathy Sierra . \newblock \emph{ McGraw-Hill}, 2006.
			\end{thebibliography}
		\end{frame}

		\begin{frame}
			\frametitle{Preguntas}

			\hspace{4cm}\huge{Preguntas ?}

		\end{frame}
	\end{document}

\usetheme{default}
\usetheme{JuanLesPins}
\usetheme{Goettingen}
\usetheme{Szeged}
\usetheme{Warsaw}

\usecolortheme{crane}

\usefonttheme{serif}
\usefonttheme{structuresmallcapsserif}
