\documentclass{beamer}

\mode<presentation>
{
	\usetheme{CambridgeUS}
	\setbeamercovered{transparent}
}

\usepackage[T1]{fontenc}
\usepackage[utf8]{inputenc}
\usepackage[spanish]{babel}
\usepackage{color}
\usepackage{hyperref}
\usepackage{algorithm,algorithmic}
\usepackage{soul}

\title[\textbf{Programaci\'on 2}]{\textbf{Programaci\'on 2}}

\subtitle{Descripci\'on del curso}

\author[IF-EG]
{Profesores:\\
	Ismael Figueroa -  \texttt{\small ifigueroap@gmail.com} \\
	\vspace{0.5mm}
	 \\
	 Eduardo Godoy - \texttt{\small eduardo.gl@gmail.com} \\
}

\institute[Universidad de Valpara\'iso]

\subject{Descripci�n del curso}

%\AtBeginSection
%{
%	\begin{frame}<beamer>
%	\frametitle{Contenido}
%	\tableofcontents[currentsection,currentsubsection]
%	\end{frame}
%}
%
%\AtBeginSubsection
%{
%	\begin{frame}<beamer>
%	\frametitle{Contenido}
%	\tableofcontents[currentsection,currentsubsection]
%	\end{frame}
%}
%
%\beamerdefaultoverlayspecification{<+->}

\begin{document}

	\begin{frame}
		\titlepage
	\end{frame}

	\begin{frame}
		\frametitle{Contenido}
		\tableofcontents%[pausesections]
	\end{frame}

	\section{Antecedentes de la asignatura}

		\subsection{Descripci\'on}

		\begin{frame}
			\frametitle{Antecedentes de la asignatura}
			\framesubtitle{Descripci\'on}

			\begin{itemize}
				\item Nombre: \textbf{Programaci\'on 2}
				\item Horario:
				\begin{itemize}
					\item \textbf{Lunes: 18:00 - 19:30 hrs.}
					\item \textbf{Lunes: 19:45 - 21:15 hrs.}
					\item \textbf{Martes: 18:00 - 19:30 hrs.}
				\end{itemize}
				\item Ayudant\'ia: pendiente
			\end{itemize}
		\end{frame}

		\subsection{Unidades tem\'aticas}

		\begin{frame}
			\frametitle{Antecedentes de la asignatura}
			\framesubtitle{Unidades tem\'aticas}

			\begin{itemize}
				\item Unidad I: \textbf{Programaci\'on Orientada a Objetos}
				\begin{itemize}
					\item Paradigmas de programaci\'on.
					\item Lenguajes de programaci\'on.
					\item Paradigma Orientado a Objeto.
					\item Modelamiento Orientado a Objeto.
					\item Principales caracter\'isticas:
\begin{itemize}
						\item {Abstracc\'ion}.
						\item {Encapsulamiento}.
						\item {Ocultamiento}.
						\item{ Herencia}.
						\item {Polimofismo}.
\end{itemize}
				\end{itemize}
			\end{itemize}
		\end{frame}

		\begin{frame}
			\frametitle{Antecedentes de la asignatura}
			\framesubtitle{Unidades tem\'aticas}

			\begin{itemize}
				\item Unidad II: \textbf{Lenguaje de programaci\'on JAVA}
				\begin{itemize}
					\item Ambiente  de desarrollo.
					\item Fundamentos de JAVA (sintaxis,variables, tipos de datos, expresiones,manejo de excepciones y operadores)
					\item Arreglos, condicionales y bucles (ciclos)
					\item Encapsulamiento (clases, m\'etodos y objetos)
					\item Modelamiento
					\item Herencia y polimorfismo
				\end{itemize}
			\end{itemize}
		\end{frame}

		\begin{frame}
			\frametitle{Antecedentes de la asignatura}
			\framesubtitle{Unidades tem\'aticas}

			\begin{itemize}
				\item Unidad III: \textbf{Threads y Swing}
				\begin{itemize}
					\item Java Threads.
				  \item Dise\~no de GUI con Java Swing.
				\end{itemize}
			\end{itemize}
		\end{frame}

	\section{Sistema de evaluaci\'on}

		\subsection{Evaluaciones}

		\begin{frame}
			\frametitle{Sistema de evaluaci\'on}
			\framesubtitle{Evaluaciones}

			\begin{itemize}
				\item Escala de evaluaci\'on: seg\'un lo dicta el reglamento de la Escuela.
				\item Cert\'amenes: \textbf{3} .
				\begin{itemize}
					\item Son evaluados en horario de clases.
					\item Toda inasistencia debe ser justificada.
				\end{itemize}
				\item Trabajos en clases: \textbf{15}.
				\begin{itemize}
					\item Ser\'an promediados.
					\item Ser\'an evaluados en horario de clases o ayudant\'ia.
					\item Toda inasistencia debe ser justificada. De no ser as\'i se calificar\'a con nota \textbf{1,0}.
				\end{itemize}
				\item Tareas: \textbf{3} .
				\begin{itemize}
				   \item Ser\'an grupos de 4-5 personas.
					\item Informe t\'ecnico (Se entregar\'a pauta del contenido).
					\item C\'odigo fuente.
				\end{itemize}
			\end{itemize}
		\end{frame}

		\begin{frame}
			\frametitle{Sistema de evaluaci\'on}
			\framesubtitle{Evaluaciones}
			\begin{itemize}
				\item \textbf{Certamen recuperativo}: Considera todo el contenido de la asignatura y se evalur\'a al final de la misma. Es obligatoria a quien justifique una inasistencia a un certamen.
				\item \textbf{Prueba especial}: Obligatoria para quienes tengan  entre \color{red}{3.5} \color{black}{y} \color{red}{3.9} \color{black}{en a lo menos uno de los tres items anteriores (Certamenes, Ejercicios o Tareas).}
				\item  \color{black}{En caso de rendir la prueba especial:}
				\begin{itemize}
    				\item Si la nota de la prueba especial $\geq$ \textbf{4,0}, entonces la nota final es \textbf{4,0}
	    			\item Si la nota de la prueba especial $<$ \textbf{4,0}, entonces se mantiene la nota final.
				\end{itemize}
			\end{itemize}
		\end{frame}

		\newpage
		\begin{frame}
			\frametitle{Sistema de evaluaci\'on}
			\framesubtitle{M\'etodo de evaluaci\'on}
			\begin{block}{}
				\begin{center}
					\begin{itemize}
  							\item$NC =(\displaystyle\frac{\left(\displaystyle\sum_{i=1}^{3}{C_{i}}\right)}{3})*0.6$ \\
								\item$NTC = (\displaystyle\frac{\left(\displaystyle\sum_{i=1}^{15}{E_{i}}\right)}{15})*0.2 $\\
  							\item$NT = (\displaystyle\frac{\left(\displaystyle\sum_{i=1}^{3}{T_{i}}\right)}{3})*0.2 $\\
								\item$NF = NC + NTC + NT  $\\
					\end{itemize}
				\end{center}
		\end{block}
	\end{frame}
\newpage

			\begin{frame}
				\begin{center}

donde:
				\begin{itemize}

					\item[] $NC_{i}$ : Es la Nota Certamen.
					\item[] $NTC_{i}$ : Es la Nota de Ejercicio en Clases.
					\item[] $NT_{i}$ : Es la Nota Tarea.
				\end{itemize}
			\end{center}
		\end{frame}
				%\begin{center}
				%	\begin{tabular}{c}
				%		\textbf{\color{red}{Notas de certamen}} \\
  				%			$NF = NC * 0.75 + NQ * 0.15 + NT * 0.10$\\
				%	\end{tabular}
				%\end{center}
		%	\end{block}
		%\end{frame}



		\subsection{Cronograma de evaluaci\'on}


		\begin{frame}
			\frametitle{Evaluaciones}
			\framesubtitle{Cronograma de evaluaci\'on}

			\begin{itemize}
				\item Certamen 1: \textbf{Lunes 30 de Abril}
				\item Certamen 2: \textbf{Lunes 4 de Junio}
				\item Certamen 3: \textbf{Lunes 25 de Junio}
				\item Ejercicios acumulativos: \textbf{Durante el semestre en clases.}
				\item Tarea 1: \textbf{Martes 23 de Abril}
				\item Tarea 2: \textbf{Martes 29 de Mayo}
				\item Tarea 3: \textbf{Martes 10 de Julio}
				\item Certamen recuperativo: \textbf{Martes 10 de Julio}
				\item Prueba Especial: \textbf{Semana desde el 17 hasta 21 de Julio (aprox)}
			\end{itemize}

		\end{frame}
		\subsection{Cronograma de evaluaci\'on}


		\begin{frame}
			\frametitle{Evaluaciones}
			\framesubtitle{Cronograma de evaluaci\'on}

			\begin{itemize}
				\item Certamen 1: \textbf{Lunes 30 de Abril}
				\item Certamen 2: \textbf{Lunes 4 de Junio}
				\item Certamen 3: \textbf{Lunes 25 de Junio}
				\item Ejercicios acumulativos: \textbf{Durante el semestre en clases.}
				\item Tarea 1: \textbf{Martes 23 de Abril}
				\item Tarea 2: \textbf{Martes 29 de Mayo}
				\item Tarea 3: \textbf{Martes 10 de Julio}
				\item Certamen recuperativo: \textbf{Martes 10 de Julio}
				\item Prueba Especial: \textbf{Semana desde el 17 hasta 21 de Julio (aprox)}
			\end{itemize}

		\end{frame}

		\begin{frame}
			\begin{table}[!ht]
				 {\scriptsize
					\begin{center}
							 \begin{tabular}{|p{1cm}|p{3cm}|p{5cm}|}\hline
									\multicolumn{3}{|c|}{\textbf{Trabajos en Clases} } \\ \hline
									\multicolumn{1}{|c|}{\textbf{Número}} &
									\multicolumn{1}{|c|}{\textbf{Fecha}} &
									\multicolumn{1}{c|}{\textbf{T\'opico}}  \\ \hline
									  1 & 27-03-2018 & Sintaxis Java 1 \\ \hline
										2 & 10-04-2018 & Sintaxis Java 2.1 \\ \hline
										3 & 10-04-2018 & Sintaxis Java 2.2 \\ \hline
										4 & 16-04-2018 & Sintaxis Java 3.1\\ \hline
										5 & 16-04-2018 & Sintaxis Java 3.2\\ \hline
										6 & 17-04-2018 & Sintaxis Java 4 \\ \hline
										7 & 24-04-2018 & Arreglos y String \\ \hline
										8 & 14-05-2018 & Modelamiento \\ \hline
										9 & 14-05-2018 & Codificaci\'on del Modelo \\ \hline
									 10 & 28-05-2018 & Collections: List \\ \hline
									 11 & 28-05-2018 & Collections: Map \\ \hline
									 12 & 11-06-2018 & Archivos 1.1 \\ \hline
									 13 & 11-06-2018 & Archivos 1.2 \\ \hline
									 14 & 18-06-2018 & Threads: Productor Consumidor \\ \hline
									 15 & 18-06-2018 & Threads: Map-Reduce \\ \hline
							\end{tabular}
					\end{center}}
			 \end{table}
		\end{frame}

	\section{Consideraciones}

		\subsection{Software}

		\begin{frame}
			\frametitle{Consideraciones}
			\framesubtitle{Ambiente de Desarrollo.}

			\textbf{Entornos de desarrollo integrados (IDE):}

			\begin{itemize}
				\item Eclipse  \url{http://www.eclipse.org}
				\item NetBeans \url{http://www.netbeans.org}
				\item IntelliJ IDEA  \url{http://www.jetbrains.com/idea/download/}
				\item SublimeText   \url{http://www.sublimetext.com/3}
				\item Notepad++  \url{http://notepad-plus-plus.org/}
			\end{itemize}

		\end{frame}

		\begin{frame}
			\frametitle{Consideraciones}
			\framesubtitle{Ambiente de Desarrollo}

			\textbf{Kit para Desarrollo en Java:}

			\begin{itemize}
				\item Java Development Kit (JDK), disponible en \url{http://developers.redhat.com/products/openjdk/download/}.
			\end{itemize}
		\end{frame}

	\section{Bibliograf\'ia}

		\begin{frame}
			\frametitle{Bibliograf\'ia}

			\begin{thebibliography}{10}
				\beamertemplatebookbibitems
				\bibitem{java611}[Programaci\'on en Java 6] Luis Joyanes Aguilar. \newblock \emph{Editorial McGraw-Hill}, 2011
				\bibitem{poo2007}[Programaci\'on orientada a objetos con Java] Barnes, David \& Kolling, Michael. \newblock \emph{2nd. Edition, Prentice-Hall}, 2007
				\bibitem{poo2007}[Fundamentos de Java] Schildt Herbert. \newblock \emph{2nd. Edition, McGraw-Hill}, 2007.

\bibitem{poo2006}[SCJP Sun Certified Programmer for Java 6 Study Guide]\\ Kathy Sierra . \newblock \emph{ McGraw-Hill}, 2006.
			\end{thebibliography}
		\end{frame}

		\begin{frame}
			\frametitle{Preguntas}

			\hspace{4cm}\huge{Preguntas ?}

		\end{frame}
	\end{document}

\usetheme{default}
\usetheme{JuanLesPins}
\usetheme{Goettingen}
\usetheme{Szeged}
\usetheme{Warsaw}

\usecolortheme{crane}

\usefonttheme{serif}
\usefonttheme{structuresmallcapsserif}
