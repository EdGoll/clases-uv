\documentclass{beamer}

\mode<presentation>
{
  \usetheme{CambridgeUS}
  % \setbeamercovered{transparent}
}

\usepackage[T1]{fontenc}
\usepackage[utf8]{inputenc}
\usepackage[spanish]{babel}
\usepackage{color}
\usepackage{hyperref}
\usepackage{algorithm,algorithmic}
\usepackage{colortbl}
\usepackage{graphicx}
\usepackage{multicol}
\usepackage{enumitem}
\setitemize{itemsep=1.2em,%
  label=\usebeamerfont*{itemize item}%
  \usebeamercolor[fg]{itemize item}%
  \usebeamertemplate{itemize item}
}

\usepackage{minted}
\usepackage[skins,minted]{tcolorbox}

\newcommand{\code}[1]{\mintinline{java}{#1}}
\newcommand{\codet}[1]{\texttt{#1}}

\setminted[java]{
  linenos=true,
  fontfamily=tt,
  fontsize=\small,
  frame=leftline,
  autogobble=True,
}

\newminted[jsmall]{java}{
  fontsize=\footnotesize
  , linenos = false
  , frame=single
  , autogobble=true
}

\newtcblisting{java}[2][]{
  minted language=java,
  enhanced, listing engine=minted,
  listing only, #1, title=#2, left=2em
}

\setminted[bash]{
  linenos=true,
  fontfamily=tt,
  fontsize=\small,
  frame=leftline
}

\newtcblisting{bash}[2][]{minted language=bash,
  enhanced, listing engine=minted,
  listing only, #1, title=#2, left=2em
}

\title[\textbf{Programación 2}]{\textbf{Programación 2}}
\subtitle{Introducción al Lenguaje Java\\Programación Estructurada}

\author[IF-EG]
{Profesores:\\
  Ismael Figueroa -  \texttt{\small ifigueroap@gmail.com} \\
  \vspace{0.5mm} \\
  Eduardo Godoy - \texttt{\small eduardo.gl@gmail.com}
}

\institute[Universidad de Valparaíso]

\date{}

\begin{document}

\begin{frame}
  \titlepage
\end{frame}

% \begin{frame}
%   \frametitle{Contenido}
%   \tableofcontents%[pausesections]
% \end{frame} 

\section{El Lenguaje de Programación Java}

\subsection{Programación Estructurada}

\begin{frame}
  \frametitle{Programación Estructurada}

  \begin{exampleblock}{}
    La Programación Estructurada es un paradigma de programación que
    busca mejorar la claridad, calidad, y mantenibilidad de los
    programas, mediante el uso exclusivo de \emph{estructuras de
      control}
  \end{exampleblock}

  \begin{exampleblock}{}
    Las estructuras de control permiten manipular el \emph{flujo de
      ejecución del programa}, fomentando la reutilización de código y
    minimizando el código duplicado.
  \end{exampleblock}
  
\end{frame}

\begin{frame}
  \frametitle{Estructuras de Control}

  \begin{small}
  \begin{itemize}
    
  \item \textbf{Estructuras Secuenciales}: en Java por defecto los programas
    son secuenciales
    
  \item \textbf{Estructuras de Selección}: permiten ejecutar segmentos de
    código en base al cumplimiento (o no) de condiciones que dependen
    del estado del programa
    
  \item \textbf{Estructuras de Iteración}: permiten ejecutar repetidamente un
    segmento de código, en base a condiciones lógicas, o recorriendo
    los elementos de una colección

    
  \item \textbf{Estructuras de Salto}: permiten romper de---forma
    controlada---con el flujo dentro de otras estructuras, como las de
    iteración
    
  \item \textbf{Excepciones}: permiten declarar un flujo alternativo
    que se ejecuta solo cuando el programa encuentra un problema
    
  \end{itemize}
  \end{small}

\end{frame}

\subsection{Estructuras de Selección}

\begin{frame}
  \frametitle{Estructura \code{if}}

  \begin{itemize}
  \item Permite ejecutar condicionalmente un bloque de código
    
  \item Un \code{if} sencillo solo tiene un bloque que se ejecuta si
    la condición es \code{true}
    
  \item Un \code{if} compuesto tiene una cláusula \code{else} con un
    bloque que se ejecuta solo si la condición es \code{false}
    
  \item Pueden anidarse arbitrariamente varios \code{if} en las
    cláusulas \code{else}. Esto se conoce como \code{else-if}
    
  \end{itemize}
\end{frame}

\begin{frame}[fragile]
  \frametitle{Estructura \code{if}}
\begin{jsmall*}{fontsize=\scriptsize}
    void compararN(int n) {
      // if sencillo
      if(n < 45) {
        System.out.println("n: " + n + " es menor que 45");
      }

      // if con alternativa
      if(n < 3) {
        System.out.println("n: " + n + " es menor que 3");
      } else {
        System.out.println("n: " + n + " NO es menor que 3");
      }

      // if anidados
      if(n < 3) {
        System.out.println("n: " + n + " es menor que 3");
      } else if (n < 5) {
        System.out.println("n: " + n + " es menor que 5");
      } else {
        System.out.println("n: " + n + " es mayor que 3 y que 5");
      }
    }
\end{jsmall*}
\end{frame}      

% \begin{columns}
%   \begin{column}{0.4\textwidth}
%     \begin{small}
%       \begin{itemize}
%       \item
%       \end{itemize}
%     \end{small}
%   \end{column}
%    %   
%   \begin{column}{0.6\textwidth}
% \begin{jsmall}
% \end{jsmall}
%   \end{column}
% \end{columns}

\begin{frame}
  \frametitle{Otro Ejemplo de \code{if}}

  Se desea escribir el método \code{clasificar} que retorna una
  letra según el valor numérico del parámetro \code{valor}, con el
  siguiente criterio:
  
  \begin{itemize}
  \item \code{A}, para un valor entre 100 y 91 (inclusive).
  \item \code{B}, para un valor entre 90 y 81 (inclusive).
  \item \code{C}, para un valor entre 80 y 71 (inclusive).
  \item \code{F}, si no es ninguno de los anteriores.
  \end{itemize}

\end{frame}

\begin{frame}[fragile]
  \frametitle{Método \code{clasificar}}
\begin{jsmall}
    char clasificar(int valor) {
      char clase;
      
      if(valor >= 91 && valor <= 100) {
        clase = 'A';
      } else if (valor >= 81 && valor <= 90) {
        clase = 'B';
      } else if (valor >= 71 && valor <= 80) {
        clase = 'C';
      } else {
        clase = 'F';
      }
      
      return clase;
    }    
\end{jsmall}  
\end{frame}

\begin{frame}
  \frametitle{Estructura \code{switch}}

  Cuando se tienen muchos casos distintos el uso de
  \code{if-else-if} puede volverse difícil de leer. Java provee la
  estructura \code{switch}, diseñada para manejar muchos casos
  posibles:
  
  \begin{itemize}  
  \item Permite seleccionar en base a distintos \emph{casos}, donde
    cada caso es un valor constante
    
  \item Cada caso debe ser único, y el valor asociado debe ser del
    mismo tipo que el valor analizado por \code{switch}

  \item \code{switch} tiene un comportamiento en cascada, que debe
    romperse usando la instrucción \code{break}
    
  \end{itemize}
\end{frame}

\begin{frame}
  \frametitle{Ejemplo de \code{switch}}

  Ahora se desea escribir el método \code{desclasificar} que toma
  como parámetro una calificación en forma de letra, y que retorna el
  valor mínimo que generó esa clasificación. Si la letra es 'F', debe
  retornar 0.

\end{frame}

\begin{frame}[fragile]
  \frametitle{Método \code{desclasificar}}
\begin{jsmall}
    int desclasificar(char clase) {
      int valor;
      switch(valor) {
        case 'A': valor = 91;
        case 'B': valor = 81;
        case 'C': valor = 71;
        case 'F': valor = 0;
      }

      return valor;
    }    
\end{jsmall}
\end{frame}

\begin{frame}[fragile]
  \frametitle{Método \code{desclasificar}}

  ¿Qué se imprime por pantalla en el siguiente programa?
  
\begin{jsmall}
    System.out.println(desclasificar('A'));
\end{jsmall}
  \onslide<2->{Sorprendentemente imprime 0! Esto se debe al
    comportamiento en cascada de \code{switch}}
\end{frame}

\begin{frame}[fragile]
  \frametitle{Método \code{desclasificar} correcto}
\begin{jsmall}
    int desclasificar(char clase) {
      int valor;
      switch(valor) {
        case 'A': valor = 91; break;
        case 'B': valor = 81; break;
        case 'C': valor = 71; break;
        case 'F': valor = 0;  break;
      }

      return valor;
    }    
\end{jsmall}
\end{frame}

\begin{frame}[fragile]
  \frametitle{Método \code{desclasificar}}

  ¿Qué se imprime por pantalla en el siguiente programa?
  
\begin{jsmall}
    System.out.println(desclasificar('Z'));
\end{jsmall}
  \onslide<2->{Esto es problemático pues se retorna una variable no
    inicializada}
\end{frame}

\begin{frame}[fragile]
  \frametitle{Método \code{desclasificar} más correcto}
  La cláusula \code{default} permite manejar los valores que no caen
  en ninguno de los casos del \code{switch}
  
\begin{jsmall}
    int desclasificar(char clase) {
      int valor;
      switch(valor) {
        case 'A': valor = 91; break;
        case 'B': valor = 81; break;
        case 'C': valor = 71; break;
        case 'F': valor = 0;  break;
        default : valor = 0;
      }

      return valor;
    }    
\end{jsmall}
\end{frame}

\subsection{Estructuras de Iteración}

\begin{frame}
  \frametitle{Estructuras de Iteración}

  Java provee tres estructuras de iteración:
  
  \begin{itemize}
  \item \codet{while}
  \item \codet{do-while}
  \item \codet{for}
  \end{itemize}

  Si bien las 3 son equivalentes, pueden hacer lo mismo, dependiendo
  del problema algunas pueden ser más cómodas de usar que otras.
  
\end{frame}

\begin{frame}
  \frametitle{Ejemplo para iteración}

  \begin{block}{}
    Se desea escribir el método \code{sumaIntervalo}, que dados los
    valores \code{n} y \code{m} de tipo \code{int}, y asumiendo que
    siempre \code{n <= m}, se quiere calcular la suma de todos los
    enteros comprendidos entre \code{n} y \code{m}, ambos inclusive.
  \end{block}

  \begin{exampleblock}{}
    Por ejemplo, \code{sumaIntervalo(3,5)} debe retornar \code{12}  
  \end{exampleblock}

\end{frame}

\begin{frame}[fragile]
  \frametitle{Estructura \texttt{while}}

  \begin{columns}
    \begin{column}{0.4\textwidth}
      \begin{itemize}
      \item Repite el bloque mientras la condición lógica, un
        booleano, sea verdadera
        
      \item Si la condición es falsa la primera vez, nunca se
        ejecuta el bloque
        
      \item Dentro del bloque, debemos eventualmente hacer cambios
        para que la condición se vuelva falsa

      \end{itemize}      
    \end{column}
    % 
    \begin{column}{0.6\textwidth}
\begin{jsmall}
        int sumaIntervalo(int n, int m) {
          int suma = 0;
          int i = n;

          while(i <= m) {
            suma += i;
            i = i + 1;            
          }

          return suma;
        }
\end{jsmall}
    \end{column}
  \end{columns}

\end{frame}

\begin{frame}[fragile]
  \frametitle{Estructura \codet{do-while}}

  \begin{columns}
    \begin{column}{0.4\textwidth}
      \begin{itemize}
        
      \item Repite el bloque mientras la condición lógica, un
        booleano, sea verdadera
        
      \item El bloque siempre se evalúa la menos una vez, porque se
        revisa la condición luego de ejecutar el bloque
        
      \item También debemos preocuparnos de hacer falsa la condición
        para salir de la iteración

      \end{itemize}      
    \end{column}
    % 
    \begin{column}{0.6\textwidth}
\begin{jsmall}
        int sumaIntervalo(int n, int m) {
          int suma = 0;
          int i = n;

          do {
            suma += i;
            i = i + 1;
          } while (i <= m);          

          return suma;
        }
\end{jsmall}
    \end{column}
  \end{columns}
\end{frame}

\begin{frame}[fragile]
  \frametitle{Estructura \codet{for}}

  \begin{columns}
    \begin{column}{0.4\textwidth}
      \begin{itemize}
        
      \item Combina los pasos de inicialización de variables de
        iteración, condición lógica, y actualización de las variables
        de iteración
        
      \item Es una forma muy compacta de escribir iteraciones, aunque
        es equivalente en su poder a \codet{while} y \codet{do-while}

      \end{itemize}      
    \end{column}
    % 
    \begin{column}{0.6\textwidth}
\begin{jsmall}
        int sumaIntervalo(int n, int m) {
          int suma = 0;

          for(int i = n; i <= m; i = i + 1) {
            suma += i;
          }
          
          return suma;
        }
\end{jsmall}
    \end{column}
  \end{columns}

\end{frame}

\begin{frame}[fragile]
  \frametitle{Comportamiento de \codet{for}}

  \begin{itemize}
  \item La \textbf{inicialización} es una sentencia que se ejecuta \emph{solo una vez}, justo antes de evaluar la condición de término

  \item La \textbf{condición de término} es una expresión booleana,
    que determina si se ejecuta el bloque o no. Se ejecuta justo antes
    de ejecutar el bloque. El bloque del \codet{for} se ejecuta solo
    cuando la condición de término es verdadera.    

  \item La \textbf{actualización de variables de iteración} se ejecuta
    justo después de la ejecución del bloque del \codet{for}, si es
    que fue ejecutado. Se debe usar para, eventualmente, hacer falsa
    la condición de término es una expresión invocada en cada iteración del bucle.

  \item Algunas de estos elementos pueden ser expresiones vacías, pero
    el \codet{for} debe tener como mínimo la forma \codet{for( ; ; ) { ... }}
  \end{itemize}
\end{frame}

\subsection{Estructuras de Salto}

\begin{frame}
  \frametitle{Estructuras de Salto}
  
  Java provee cuatro estructuras de salto:
  \begin{itemize}

  \item\code{return}: retorna el control del flujo al método
    invocador de otro método. Permite el uso de "subrutinas"
    
  \item \code{break}: en el contexto de \code{switch} impide el
    efecto cascada. También se usa en iteraciones para
    romper el ciclo
    
  \item \code{continue}: se usa en iteraciones para saltarse la
    vuelta actual del loop, y continuar en la siguiente vuelta

  \end{itemize}
\end{frame}

\begin{frame}
  \frametitle{Ejemplo \code{break} en iteración}

  \begin{block}{}
    Se busca implementar el método \code{esPrimo}, que dado un entero
    positivo \code{n}, imprime por pantalla \code{"Es primo!"} o
    \code{"No es primo!"} según corresponda
  \end{block}
  
\end{frame}


\begin{frame}[fragile]
  \frametitle{Ejemplo \codet{break} en iteración}
\begin{jsmall}
void esPrimo(int n) {
  int i;
  for(i = 2; i < n; i = i + 1) {
    if(n % i == 0) {
      System.out.println("No es primo!");
    } else {
      System.out.println("Es primo!");
    }
  }
}
\end{jsmall}

  \only<2>{Esto no está bien porque imprimirá muchas veces ambos
    mensajes!}
\end{frame}

\begin{frame}[fragile]
  \frametitle{Ejemplo \codet{break} en iteración}
\begin{jsmall}
void esPrimo(int n) {
  int i;
  for(i = 2; i < n; i = i + 1) {
    if(n % i == 0) {
      System.out.println("No es primo!");
      break; // salimos del loop        
    }
  }

  if(i == n) {
    System.out.println("Es primo!");
  }    
}
\end{jsmall}

  Solo uno de los mensajes se imprimirá, y solo una vez!
\end{frame}

\begin{frame}
  \frametitle{Ejemplo \codet{continue}}

  \begin{block}{}
    Se busca implementar el método \code{sumaIntervaloPares} que,
    similarmente a \code{sumaIntervalo}, suma los valores entre dos
    enteros \code{n} y \code{m}, pero solo considerando los elementos
    pares.
  \end{block}
  
\end{frame}

\begin{frame}[fragile]
  \frametitle{Método \code{sumaIntervaloPares}}
\begin{jsmall}
int sumaIntervaloPares(int n, int m) {
  int suma = 0;
  for(int i = n; i <= m; i = i + 1) {
      if(i % 2 != 0) {
         /* i es impar, no hago nada */
      } else {
         suma += i;
      }
  }
}
\end{jsmall}

  Este código si bien funciona, es un poco extraño porque tiene una
  condición donde no se ejecuta nada. Más bien, la condición \emph{se
    usa para saltarse completamente esta vuelta del loop}
  
\end{frame}

\begin{frame}[fragile]
  \frametitle{Método \code{sumaIntervaloPares}}
\begin{jsmall}
int sumaIntervaloPares(int n, int m) {
  int suma = 0;
  for(int i = n; i <= m; i = i + 1) {
    if(i % 2 != 0) {
      continue; // i es impar, me salto esta vuelta
    }
    
    // si se ejecuto el continue, esta linea no se alcanza a ejecutar
    suma += i;
    
  }
}
\end{jsmall}

  Usando \code{continue} saltamos a la siguiente vuelta de la
  iteración. En el caso de \code{for}, igual se ejecuta la
  actualización de variables de iteración!
  
\end{frame}

\begin{frame}[fragile]
  \frametitle{Otro ejemplo de \code{continue}}

  Escriba el metodo \code{primeros100} que imprime por pantalla los 100 primeros números enteros que no son
  divisibles por 3:

  \begin{jsmall}
 void primeros100() {
   int cantidad = 0;
   for(int i = 0; cantidad < 100; i = i + 1) {
     if(i % 3 == 0) {
       continue;
     }

     System.out.println(i);

   }
 }
 \end{jsmall}
\end{frame}

\subsection{Excepciones}

\begin{frame}
  \frametitle{Excepciones}

  \begin{block}{}
    Una \textbf{excepción} es problema que aparece durante la
    ejecución de un programa, y que impide que se pueda continuar con
    la ejecución normal del mismo
  \end{block}

  \begin{block}{}
    Ejemplos de problemas excepcionales son:

    \begin{itemize}
    \item Una división por 0      
    \item Se está tratando de abrir un archivo pero este no existe     
    \item Se trata de abrir una conexión de red, pero no hay
      conectividad      
    \item \textbf{Se invoca un método o atributo en un objeto que es nulo}
    \end{itemize}
  \end{block}
  
\end{frame}

\begin{frame}
  \frametitle{Excepciones}

  \begin{block}{}
    Un programa \textbf{robusto} es aquel que logra sortear con éxito
    los problemas excepcionales. El mecanismo de excepciones nos
    permite manejar flujos de control alternativos en caso de
    encontrar estos problemas
  \end{block}
  
\end{frame}

\begin{frame}[fragile]
  \frametitle{Ejemplo de Excepciones}

\begin{jsmall}
void imprimirLargo(String s) {
    System.out.println(s.length());
}
\end{jsmall}

  \only<2>{Este método va a fallar si \code{s == null}. Veremos la
    famosa \codet{java.lang.NullPointerException}, conocida también
    como \codet{NPE}}
  
\end{frame}

\begin{frame}[fragile]
  \frametitle{Una versión más robusta}

\begin{jsmall}
void imprimirLargo(String s) {
    try {
      System.out.println(s.length());
    } catch(NullPointerException npe) {
      System.out.println(0);
    }
}
\end{jsmall}

  \only<2>{En esta versión se considera que el largo de un string nulo es 0}
  
\end{frame}

\begin{frame}[fragile]
  \frametitle{Estructura \codet{try-catch}}
  \begin{columns}
    \begin{column}{0.5\textwidth}
      \begin{footnotesize}
      \begin{itemize}
        
      \item El código del bloque \code{try} es el flujo normal
        de ejecución
        
      \item Si en la ejecución del \code{try} ocurre una excepción, se
        aborta la ejecución y \emph{se lanza la excepción}
        
      \item El \code{catch} siempre acompaña a un \code{try}. Se dice
        que un \code{catch} \emph{captura una excepción lanzada en el
          \code{try} asociado}       
      \end{itemize}
      \end{footnotesize}
    \end{column}
    %
    \begin{column}{0.5\textwidth}
      \begin{footnotesize}
      \begin{itemize}    
      \item El \code{catch} solo captura excepciones de un tipo
        específico
        
      \item Un \code{try} puede tener muchos \code{catch}
        
      \item El tipo \code{Exception} es el más general y subsume las
        demás excepciones        
      \end{itemize}
      \end{footnotesize}     
    \end{column}
  \end{columns}
  
\end{frame}

\begin{frame}[fragile]
  \frametitle{Múltiples \code{catch}}

\begin{jsmall}
void imprimirLargo(String s) {
    try {      
      System.out.println(s.length());
    } catch(NullPointerException npe) {
      System.out.println(0);
    } catch(Exception ex) {
      System.out.println("Error: " + ex.getMessage());
    }
}
\end{jsmall}

  \only<2>{En esta versión se considera que el largo de un string nulo es 0}
  
\end{frame}

\begin{frame}
  \frametitle{Excepciones Chequeadas y No-Chequeadas}
  Java define dos tipos distintos de excepciones:

  \begin{itemize}
    
  \item \textbf{Excepciones no chequeadas (Unchecked exceptions)}

  \item \textbf{Excepciones chequeadas (Checked exceptions)}
  \end{itemize}
\end{frame}

\begin{frame}
  \frametitle{Excepciones No-Chequeadas}
    
  \begin{block}{}
    Son problemas que aparecen por errores en la programación, y en
    general no pueden ser anticipadas. Por tanto se detectan solo
    durante la ejecución. \emph{Pueden ocurrir en cualquier parte de
      cualquier programa}
  \end{block}

  \begin{block}{}
    Ejemplos:

    \begin{itemize}
    \item \codet{java.lang.NullPointerException}: se intenta acceder
      un objeto nulo
    \item \codet{java.lang.ArithmeticException}: por ejemplo división
      por 0
    \end{itemize}
  \end{block}

\end{frame}

\begin{frame}
  \frametitle{Excepciones Chequeadas}
    
  \begin{block}{}
    Son problemas que se debe considerar obligatoriamente para una
    programación robusta. Es decir, son problemas a los que es
    obligatorio anticiparse, y el compilador detectará y nos obligará
    a hacer algo al respecto.
  \end{block}

  \begin{block}{}
    Ejemplos:

    \begin{itemize}
    \item \codet{java.io.FileNotFoundException}: si abrimos un
      archivo, debemos considerar qué hacer si no se encuentra
    \item \codet{java.io.IOException}: representa problemas de
      entrada/salida que deben considerarse
    \end{itemize}
  \end{block}

\end{frame}

\begin{frame}[fragile]
  \frametitle{Propagación de Excepciones}

  \begin{columns}
    \begin{column}{0.4\textwidth}
     \begin{small}
       Para manejar las excepciones chequeadas, tenemos dos opciones:
  \begin{itemize}

  \item Introducir un bloque \codet{try-catch}, y manejar la excepción
    localmente, o bien...
    
  \item \emph{Propagar} la excepción al código que nos está
    invocando usando la cláusula \code{throws}
  \end{itemize}
      \end{small}      
    \end{column}
    %
    \begin{column}{0.6\textwidth}
\begin{jsmall}
  void leerArchivos(String path)
       throws FileNotFoundException {
    /* ... codigo que abre archivos ... */
  }
\end{jsmall}
      \begin{small}
      \begin{itemize}

      \item La excepción declarada pasa a ser parte de \emph{la firma
          del método}, junto al tipo de retorno y parámetros de
        entrada
      
      \item Cualquier método que invoque \code{leerArchivos} deberá
        decidir nuevamente qué hacer con la posible excepción

      \end{itemize}
      \end{small}
      
    \end{column}
  \end{columns}  
\end{frame}


% \begin{frame}
%   \frametitle{Ejercicios Propuestos}  
% \end{frame}

%%%%%%%%%%%%%%%%%%% 

% \section{Estructura de clases en JAVA}

% \begin{frame}
%   \frametitle{Lenguajes de programación}
%   \framesubtitle{Estructura de clases en JAVA}

%   \begin{block}{}
%     \textcolor{blue}{public class} \textbf{Auto} \{ \\
%     \hspace{1cm} ... \\
%     \hspace{1cm} \emph{variables} \\
%     \hspace{1cm} ... \\
%     \hspace{1cm} \emph{métodos} \\
%     \hspace{1cm} ... \\
%     \}
%   \end{block}
% \end{frame}

% \begin{frame}
%   \frametitle{Lenguajes de programación}
%   \framesubtitle{Estructura de clases en JAVA - Primer programa}

%   \begin{block}{}
%     \textcolor{blue}{public class} \textbf{\emph{HelloWorld}} \{ \\
%     \hspace{1cm} \\
%     \hspace{1cm} \textcolor{blue}{public static void} \textbf{main}(String[ ] args) \{ \\
%     \hspace{2cm} System.\emph{\textcolor{green}{out}}.println(\textcolor{orange}{'' Hello, World! ''}); \\
%     \hspace{1cm} \} \\
%     \hspace{1cm} \\
%     \}
%   \end{block}
% \end{frame}

% \begin{frame}
%   \frametitle{Lenguajes de programación}
%   \framesubtitle{Estructura de clases en JAVA - Definición de variables}

%   \begin{block}{}
%     \textcolor{blue}{public class} \textbf{\emph{Auto}} \{ \\
%     \hspace{1cm} \\
%     \hspace{1cm} \textcolor{blue}{private} String \textcolor{green}{color}; \ \\
%     \hspace{1cm} String \textcolor{green}{marca}; \ \\
%     \hspace{1cm} \textcolor{blue}{boolean} \textcolor{green}{estado}; \ \\
%     \hspace{1cm} ... \\
%     \}
%   \end{block}
%   \begin{block}{}
%     {\scriptsize
%     \emph{Si bien, la declaración de variables puede ir en cualquier parte, habitualmente se declaran al inicio método o clase.}
%   }
%   \end{block}
% \end{frame}

% \begin{frame}
%   \frametitle{Lenguajes de programación}
%   \framesubtitle{Estructura de clases en JAVA - Definición de variables}

%   \begin{block}{}
%     {\scriptsize
%     \emph{Es posible encadenar nombres de variables del mismo tipo}
%   }
%   \end{block}

%   \begin{block}{}
%     \textcolor{blue}{public class} \textbf{\emph{Auto}} \{ \\
%     \hspace{1cm} \\
%     \hspace{1cm} \textcolor{blue}{private} String \textcolor{green}{color}, \textcolor{green}{marca}; \ \\
%     \hspace{1cm} \textcolor{blue}{boolean} \textcolor{green}{estado}; \ \\
%     \hspace{1cm} \textcolor{blue}{int} \textcolor{green}{x}, \textcolor{green}{y}, \textcolor{green}{z}; \ \\
%     \hspace{1cm} ... \\
%     \}
%   \end{block}

% \end{frame}

% \begin{frame}
%   \frametitle{Lenguajes de programación}
%   \framesubtitle{Estructura de clases en JAVA - Declaración de variables}

%   \begin{block}{}
%     \textcolor{blue}{public class} \textbf{\emph{Auto}} \{ \\
%     \hspace{1cm} \\
%     \hspace{1cm} \textcolor{blue}{private} String \textcolor{green}{color} = \textcolor{orange}{''rojo''}, \textcolor{green}{marca} = \textcolor{orange}{''chevrolet''}; \ \\
%     \hspace{1cm} \textcolor{blue}{boolean} \textcolor{green}{estado} = \textcolor{blue}{false}; \ \\
%     \hspace{1cm} \textcolor{blue}{int} \textcolor{green}{x} = 1, \textcolor{green}{y} = 0, \textcolor{green}{z} = -1; \ \\
%     \hspace{1cm} ... \\
%     \}
%   \end{block}

% \end{frame}

% \begin{frame}
%   \frametitle{Lenguajes de programación}
%   \framesubtitle{Estructura de clases en JAVA - Comentarios}

%   \begin{block}{}
%     \textcolor{gray}{// Comentario en una línea} \\
%     \textcolor{gray}{/* Comentario en más} \\
%     \textcolor{gray}{ de una línea */} \\
%     \textcolor{gray}{/** Comentario para} \\
%     \textcolor{gray}{ javadoc */} \\
%     \textcolor{blue}{public class} \textbf{\emph{Auto}} \{ \\
%     \hspace{1cm} ... \\
%     \}
%   \end{block}

% \end{frame}

% \begin{frame}
%   \frametitle{Lenguajes de programación}
%   \framesubtitle{Estructura de clases en JAVA - Método}

%   \begin{block}{Método: encenderMotor}
%     {\scriptsize
%     \textcolor{blue}{public class} \textbf{\emph{Auto}} \{ \\
%     \hspace{1cm} \\
%     \hspace{1cm} \textcolor{blue}{private} String \textcolor{green}{color}; \ \\
%     \hspace{1cm} \textcolor{blue}{private} String \textcolor{green}{marca}; \ \\
%     \hspace{1cm} \textcolor{blue}{private} \textcolor{blue}{boolean} \textcolor{green}{estado}; \ \\
%     \hspace{1cm} \\
%     \hspace{1cm} \textcolor{blue}{public void} \textbf{encenderMotor}() \{ \\
%     \hspace{2cm} \textcolor{blue}{if} (\textcolor{green}{estado} == \textcolor{blue}{true}) \{ \\
%     \hspace{3cm} System.\emph{\textcolor{green}{out}}.println(\textcolor{orange}{''Auto encendido''}); \\
%     \hspace{2cm} \} \textcolor{blue}{else} \{ \\
%     \hspace{3cm} \textcolor{green}{estado} = \textcolor{blue}{true}; \\
%     \hspace{3cm} System.\emph{\textcolor{green}{out}}.println(\textcolor{orange}{''OK, se encendio ''}); \\
%     \hspace{2cm} \} \\
%     \hspace{1cm} \} \\
%     \}}
%   \end{block}

% \end{frame}

% \begin{frame}
%   \frametitle{Lenguajes de programación}
%   \framesubtitle{Estructura de clases en JAVA - Método}

%   \begin{block}{Método: mostrarInfo}
%     {\scriptsize
%     \textcolor{blue}{public class} \textbf{\emph{Auto}} \{ \\
%     \hspace{1cm} \\
%     \hspace{1cm} \textcolor{blue}{public} String \textcolor{green}{color}; \ \\
%     \hspace{1cm} \textcolor{blue}{public} String \textcolor{green}{marca}; \ \\
%     \hspace{1cm} \textcolor{blue}{private} \textcolor{blue}{boolean} \textcolor{green}{estado}; \ \\
%     \hspace{1cm} \\
%     \hspace{1cm} \textcolor{blue}{public void} \textbf{encenderMotor}() \{ ... \} \\
%     \hspace{1cm} \\
%     \hspace{1cm} \textcolor{blue}{public void} \textbf{mostrarInfo}() \{ \\
%     \hspace{2cm} System.\emph{\textcolor{green}{out}}.println(\textcolor{orange}{''Este auto es un ''} + \textcolor{green}{marca} + \textcolor{orange}{'' de color ''} + \textcolor{green}{color}); \\
%     \hspace{1cm} \} \\
%     \}}
%   \end{block}

% \end{frame}

% \begin{frame}
%   \frametitle{Lenguajes de programación}
%   \framesubtitle{Estructura de clases en JAVA - Método}

%   \begin{block}{Método: main}
%     {\tiny
%     \textcolor{blue}{public class} \textbf{\emph{Auto}} \{ \\
%     \hspace{1cm} \\
%     \hspace{1cm} \textcolor{blue}{private} String \textcolor{green}{color}; \ \\
%     \hspace{1cm} \textcolor{blue}{private} String \textcolor{green}{marca}; \ \\
%     \hspace{1cm} \textcolor{blue}{private} \textcolor{blue}{boolean} \textcolor{green}{estado}; \ \\
%     \hspace{1cm} \\
%     \hspace{1cm} \textcolor{blue}{public void} \textbf{encenderMotor}() \{ ... \} \\
%     \hspace{1cm} \\
%     \hspace{1cm} \textcolor{blue}{public void} \textbf{mostrarInfo}() \{ ... \} \\
%     \hspace{1cm} \\
%     \hspace{1cm} \textcolor{blue}{public static void} \textbf{main}(String[ ] args) \{ \\
%     \hspace{2cm} Auto a = \textcolor{blue}{new} Auto ( ); \\
%     \hspace{2cm} a.\textcolor{green}{marca} = \textcolor{orange}{''Chevrolet Sail''}; \\
%     \hspace{2cm} a.\textcolor{green}{color} = \textcolor{orange}{''Gris''}; \\
%     \hspace{2cm} System.\emph{\textcolor{green}{out}}.println(\textcolor{orange}{''Llamando a showAtr() ''}); \\
%     \hspace{2cm} a.mostrarInfo(); \\
%     \hspace{2cm} System.\emph{\textcolor{green}{out}}.println(\textcolor{orange}{''Llamando a encenderMotor() ''}); \\
%     \hspace{2cm} a.encenderMotor(); \\
%     \hspace{1cm} \} \\
%     \hspace{1cm} \\
%     \}}
%   \end{block}

% \end{frame}

% \begin{frame}
%   \frametitle{Lenguajes de programación}
%   \framesubtitle{Ejercicios}

%   \begin{block}{Método: Mostrar atributos}
%     Desarrollo un programa en JAVA que permita saber los últimos 20
%     a\~nos biciestos (iniciando desde el a\~no actual). Desarrollo el
%     programa basado en el paradigma de la orientación a objeto.
%   \end{block}

% \end{frame}

% % \begin{frame}
% %			\frametitle{Lenguajes de programación}
% %			\framesubtitle{Estructuras de control - Excepciones.}

% %			\begin{itemize}
% %   \item Las excepciones son una forma avanzada de controlar el flujo de un programa.
% %   \item Con ellas se podrán realizar acciones especiales si se dan determinadas condiciones, justo en el momento en que esas condiciones se den.
% %			\end{itemize}
% % \end{frame}

\begin{frame}
  \frametitle{Preguntas}

  \hspace{4cm}\huge{Preguntas ?}
  
\end{frame}

\end{document}