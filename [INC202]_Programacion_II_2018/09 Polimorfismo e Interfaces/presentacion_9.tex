\documentclass{beamer}

\mode<presentation>
{
  \usetheme{CambridgeUS}
  % \setbeamercovered{transparent}
}

\usepackage[T1]{fontenc}
\usepackage[utf8]{inputenc}
\usepackage[spanish]{babel}
\usepackage{color}
\usepackage{hyperref}
\usepackage{algorithm,algorithmic}
\usepackage{colortbl}
\usepackage{graphicx}
\usepackage{multicol}
\usepackage{enumitem}
\setitemize{itemsep=1.2em,%
  label=\usebeamerfont*{itemize item}%
  \usebeamercolor[fg]{itemize item}%
  \usebeamertemplate{itemize item}
}

\usepackage{minted}
\usepackage[skins,minted]{tcolorbox}

\newcommand{\code}[1]{\mintinline{java}{#1}}
\newcommand{\codet}[1]{\texttt{#1}}

\setminted[java]{
  linenos=true,
  fontfamily=tt,
  fontsize=\small,
  frame=leftline,
  autogobble=True,
}

\newminted[jsmall]{java}{
  fontsize=\footnotesize
  , linenos = false
  , frame=single
  , autogobble=true
}

\newtcblisting{java}[2][]{
  minted language=java,
  enhanced, listing engine=minted,
  listing only, #1, title=#2, left=2em
}

\setminted[bash]{
  linenos=true,
  fontfamily=tt,
  fontsize=\small,
  frame=leftline
}

\newtcblisting{bash}[2][]{minted language=bash,
  enhanced, listing engine=minted,
  listing only, #1, title=#2, left=2em
}

\title[\textbf{Programación 2}]{\textbf{Programación 2}}
\subtitle{Polimorfismo e Interfaces}

\author[IF-EG]
{Profesores:\\
  Ismael Figueroa -  \texttt{\small ifigueroap@gmail.com} \\
  \vspace{0.5mm}
  Eduardo Godoy - \texttt{\small eduardo.gl@gmail.com}
}

\institute[Universidad de Valparaíso]

\date{}

\begin{document}

\begin{frame}
  \titlepage
\end{frame}

\section{Polimorfismo}

\begin{frame}
  \frametitle{¿Qué es el polimorfismo?}

  \begin{block}{Definición}
    La palabra \emph{polimorfismo} significa literalmente \emph{muchas
      formas}. En programación, es una técnica para \emph{programar de
      manera general}, procesando de manera uniforme objetos que
    comparten una misma superclase.
  \end{block}

  \begin{block}{Ejemplo}
    Por ejemplo, un editor de dibujos trabaja con figuras en 2D:
    círculos y cuadrados. Estas figuras se pueden mover,
    mediante el método \codet{mover}. ¿Cómo podemos implementar un
    método general para desplazar todas las figuras 100 pixeles hacia
    la derecha?
  \end{block}
  
\end{frame}

\begin{frame}[fragile]
  \frametitle{Código tentativo}

  \begin{jsmall}
    public class Editor2D {

      List<Cuadrado> cuadrados;
      List<Circulo> circulos;
      
      public void desplazar(int dx, int dy) {

        for(Cuadrado c : cuadrados) {
          c.mover(dx, dy);
        }

        for(Circulo c: circulos) {
          c.mover(dx, dy);
        }
      }      
    }    
  \end{jsmall}

  ¿Qué pasaría ahora si agregamos nuevas figuras 2D: rectangulo,
  elipse, triangulo, trapezoide?
  
\end{frame}


\begin{frame}[fragile]
  \frametitle{Código tentativo extendido}

  \begin{columns}
    \begin{column}{0.45\textwidth}
      \begin{jsmall}
    public class Editor2D {

      List<Cuadrado> cuadrados;
      List<Circulo> circulos;
      List<Elipse> elipses;
      List<Triangulo> triangulos;
      List<Trapezoide> trapezoides;
      
      public void
      desplazar(int dx, int dy) {

        for(Cuadrado c : cuadrados)
        {
          c.mover(dx, dy);
        }

        for(Circulo c: circulos) {
          c.mover(dx, dy);
        }
      \end{jsmall}
    \end{column}
    %
    \begin{column}{0.55\textwidth}
      \begin{jsmall}       

        for(Elipse e: elipses) {
          e.mover(dx, dy);
        }

        for(Triangulo t: triangulos) {
          t.mover(dx, dy);
        }

        for(Trapezoide t: trapezoide) {
          t.mover(dx, dy);
        }
        
      }      
    }    
  \end{jsmall}
\end{column}
\end{columns}  
\end{frame}

\begin{frame}
  \frametitle{Extensibilidad}

  \begin{alertblock}{El código no es extensible}
    Cada vez que agregamos una subclase en la jerarquía, tenemos que
    modificar el código que utiliza figuras 2D. No estamos ganando
    ningún beneficio por usar jerarquía de clases
  \end{alertblock}

  \begin{block}{Definición}
    Un software se considera extensible si para agregarle nueva
    funcionalidad se debe modificar una cantidad acotada de elementos,
    archivos, o clases. Es decir, se minimiza la cantidad de código a
    modificar ante la incorporación de nuevas clases o métodos.
  \end{block}
  
\end{frame}

\begin{frame}
  \frametitle{Polimorfismo basado en Herencia}

  \begin{itemize}

  \item En la programación orientada a objetos, y en Java en
    particular, el polimorfismo funciona explotando los elementos
    comunes de una jerarquía de herencia.   
    
  \item La idea es aprovechar la relación \emph{es un}. Entonces, el
    código se implementa de manera general haciendo referencia
    solamente al tipo de una superclase en específico.
    
  \item La gracia es que \emph{las instancias de las subclases \textbf{también
        son instancias de la super clase}}, y que se invoca el método
    ``correcto''...
    
  \end{itemize}
  
\end{frame}

\begin{frame}
  \frametitle{Tipo Aparente y Tipo Actual}

  \begin{alertblock}{Tipo Aparente}
    El \emph{tipo aparente} de una variable es aquel que se puede
    determinar estáticamente, es decir, en tiempo de compilación.
  \end{alertblock}

  \begin{alertblock}{Tipo Actual}
    El \emph{tipo actual} de una variable es aquel que realmente toma
    cuando se construye la instancia, y que siempre se puede conocer
    durante la ejecución del programa.
  \end{alertblock}
  
\end{frame}

\begin{frame}[fragile]
  \frametitle{Tipo Aparente y Tipo Actual}
  \begin{jsmall}
    public class Main {
      public static void main(String[] args) {
        // tipo aparente: Base / tipo actual: A
        Base b1 = new A();

        // tipo aparente: Base / tipo actual: B
        Base b2 = new B();

        // tipo aparente: A / tipo actual: B
        A a1 = ((A) b2); 
      }
    }
  \end{jsmall}  
\end{frame}

\begin{frame}
  \frametitle{Dispatch dinámico de mensajes en Java}
  \begin{alertblock}{Definición}
    Java posee \emph{dispatch dinámico de mensajes}. Esto significa
    que cuando se invoca un método, el código que se ejecuta
    corresponde \textit{\textbf{al método del TIPO ACTUAL al que
        pertenece la instancia que ejecutará el método.}}
  \end{alertblock}
\end{frame}

\begin{frame}[fragile]
  \frametitle{Dispatch dinámico de mensajes en Java}

  \begin{jsmall}
    abstract class Base {
      abstract void saludar();
    }
    
    class A extends Base {
      void saludar() { System.out.println("Hola A"); }
    }

    class B extends A {
      void saludar() { System.out.println("Hola B"); }
    }

    public class Main {
      public static void main(String[] args) {
        Base b1 = new A(); Base b2 = new B();
        b1.saludar(); b2.saludar();
        ((A) b2).saludar();        
      }
    }   
  \end{jsmall}
  
\end{frame}

\begin{frame}
  \frametitle{Polimorfismo = herencia + dispatch dinámico}
  \begin{alertblock}{}
    La combinación de herencia y dispatch dinámico es la que permite
    el polimorfismo en Java.
  \end{alertblock}
\end{frame}

\begin{frame}[fragile]
  \frametitle{Ejemplo Revisado}

  \begin{jsmall}
    public class Editor2D {
      List<Cuadrado> cuadrados;
      List<Circulo> circulos;
      List<Elipse> elipses;
      List<Triangulo> triangulos;
      List<Trapezoide> trapezoides;
      /* .... */
      public void desplazar(int dx, int dy) {
        List<Figura2D> figuras = new ArrayList<Figura2D>();
        figuras.addAll(cuadrados);
        figuras.addAll(circulos);
        figuras.addAll(elipses);
        figuras.addAll(triangulos);
        figuras.addAll(trapezoides);

        for(Figura f : figuras) {
          f.mover(dx, dy);
        }
      }      
    }    
  \end{jsmall}  
\end{frame}

\begin{frame}[fragile]
  \frametitle{Ejemplo Re-Revisado}

  \begin{jsmall}
    public class Editor2D {
      List<Figura2D> figuras;

      /* .... */
      
      public void desplazar(int dx, int dy) {      
        for(Figura f : figuras) {
          f.mover(dx, dy);
        }
      }      
    }    
  \end{jsmall}  
\end{frame}


\begin{frame}
  \frametitle{Determinando el tipo actual de un objeto}

  \begin{block}{Problema}
  ¿Cómo podríamos hacer para definir un método
  \codet{contarCuadrados()}, que dada la lista de figuras podamos
  contar cuántas de éstas son cuadrados?
\end{block}

\pause

  \begin{block}{Solución}
Java provee el operador \codet{instanceof} para determinar si el tipo
actual de un objeto es o no es igual a alguno que nosotros especifiquemos
\end{block}
  
\end{frame}

\begin{frame}[fragile]
  \frametitle{Contando cuadrados}

  \begin{jsmall}
    public class Editor2D {
      List<Figura2D> figuras;
      
      public int contarCuadrados() {
        int total = 0;
        
        for(Figura f : figuras) {
          if(f instanceof Cuadrado) {
            total = total + 1;
          }
        }

        return total;
      }      
    }    
  \end{jsmall}  
\end{frame}

\begin{frame}
  \frametitle{Métodos y clases finales}

  Mediante la declaración \codet{final} el programador puede indicar
  que:

  \begin{itemize}

  \item Un método no puede ser sobreescrito

  \item Una clase no puede ser extendida por una
    superclase. Implícitamente todos sus métodos son considerados \codet{final}
    
  \end{itemize}

  Esto es fundamental para garantizar ciertos comportamientos, por
  ejemplo en seguridad. \emph{¿Qué pasaría si puedo sobreescribir el
    método que verifica los passwords?}

  \begin{block}{Dispatch de métodos \codet{final}}
    Los métodos \codet{final} no utilizan dispatch dinámico: se invoca
    el método de acuerdo a la implementación final, que no puede
    sobreescribirse. Lo mismo pasa con los métodos privados que,
    implicitamente, también son finales.
  \end{block}  
  
\end{frame}

\section{Interfaces}

\begin{frame}
  \frametitle{¿Qué son las Interfaces?}

  En Java una interfaz es:

  \begin{itemize}

  \item Una especificación formal y verificada en compilación sobre un
    conjunto de valores constantes y métodos. Es similar a una clase
    abstracta, pero \emph{no necesita que exista una jerarquía de
      clases}.
    
  \item Un \emph{tipo de dato} que puede ser usado para programar con
    polimorfismo, en base a la especificación dada por la interfaz.
    
  \item Un mecanismo para implementar \emph{ocultamiento de la
      información}. Esto porque solo se conoce la interfaz pública,
    pero no la implementación interna, la cual puede ser cambiada de
    manera transparente.
    
  \item Un mecanismo para simular \emph{herencia múltiple}
    
  \end{itemize}
  
\end{frame}

\begin{frame}[fragile]
  \frametitle{Cómo se declara una interfaz}

  \begin{columns}
    \begin{column}{0.5\textwidth}
      \begin{small}
        Consideremos la funcionalidad de que un objeto puede ser
        \emph{archivable}, es decir, podemos guardarlo en un archivo
        de texto, y luego recuperar su valor.
       \end{small}

        \begin{jsmall}        
        public interface Archivable {
          void archivar(String filePath);
          void restaurar(String filePath);
        }        
      \end{jsmall}      
    \end{column}
    %
    \begin{column}{0.5\textwidth}
        \begin{itemize}
        \item El código define la interfaz \codet{Archivable}
          
        \item La interfaz tiene el método \codet{archivar} para
          guardar el estado del objeto en un archivo dado
          
        \item La interfaz tiene el método \codet{restaurar} para
          recuperar su estado interno desde un archivo dado
          
        \end{itemize}

    \end{column}
  \end{columns}  
\end{frame}

\begin{frame}
  \frametitle{¿Cuáles clases pueden ser \codet{Archivable}s?}

  \begin{block}{¿Quién puede implementar esta interfaz?}
    Cualquier clase puede, eventualmente, implementar la interfaz
    \codet{Archivable} o cualquier otra.
  \end{block}

  \begin{block}{Ejemplo}
    Por ejemplo, la clase \codet{Persona} y la clase \codet{Rectangulo}
    pueden ser archivables! Esto sin importar que no estén
    relacionados por herencia
  \end{block}

  \begin{block}{}
    Una clase declara que va a implementar una interface usando la
    palabra clave \codet{implements}
  \end{block}
  
\end{frame}


\begin{frame}[fragile]
  \frametitle{Implementando Persona Archivable}
  \begin{jsmall}
    public class Persona implements Archivable {
      String nombre;
      String apellido;
      Integer edad;

      public Persona(String nombre, String apellido, Integer edad) {
        this.nombre = nombre;
        this.apellido = apellido;
        this.edad = edad;
      }

      public void archivar(String filePath) { // desde Archivable
        /* abrir archivo y escribir datos de nombre, apellido y edad */
      }

      public void restaurar(String filePath) { // desde Archivable
        /* abrir archivo y leer datos para nombre, apellido y edad */
      }
    }    
  \end{jsmall}  
\end{frame}

\begin{frame}[fragile]
  \frametitle{Implementando Rectángulo Archivable}
  \begin{jsmall}
    public class Rectangulo extends Figura implements Archivable {
      Integer largo, ancho;
      
      public Rectangulo(Punto2D centro, Integer largo, Integer ancho) {
        /*constructor */
      }

      public Double area() { /* impl. area, desde Figura */ }
      
      public Double perimetro() { /* impl. perimetro, desde Figura */ }

      public void archivar(String filePath) { // desde Archivable
        /* abrir archivo y escribir datos de nombre, apellido y edad */
      }

      public void restaurar(String filePath) { // desde Archivable
        /* abrir archivo y leer datos para nombre, apellido y edad */
      }       
    }
  \end{jsmall}  
\end{frame}

\begin{frame}[fragile]
  \frametitle{Usando polimorfismo con interfaces}

  Podemos crear código general para objetos que sean
  \codet{Archivable}s:

  \begin{jsmall}
    void archivarTodo(List<Archivable> objs) {
      int index = 0;
      for(Archivable o : objs) {
        o.archivar("archivo_'' + index + ".txt");
      }      
    }    
  \end{jsmall}
\end{frame}

\begin{frame}[fragile]
  \frametitle{Usando polimorfismo con interfaces}
  También podemos usar \codet{instanceof} para saber si un objeto
  implementa o no implementa una determinada interfaz:

  \begin{jsmall}
    {
      Cuadrado c = new Cuadrado(40);
      Rectangulo r = new Rectangulo(20, 30);

      if(c instanceof Archivable) {
        System.out.println("El objeto " + c + " es archivable'');
      }

      if(r instanceof Archivable) {
        System.out.println("El objeto " + r + " es archivable'');
      }

      // instanceof puede ser verdadero en distintos casos
      System.out.println(r instanceof Archivable);
      System.out.println(r instanceof Rectangulo);
      System.out.println(r instanceof Figura);      
    }    
  \end{jsmall}
  
\end{frame}

\begin{frame}
  \frametitle{Especificación vs Implementación}

  \begin{block}{Especificación}
    Es una descripción formal sobre el comportamiento esperado por
    parte de una clase. Indica los métodos que se deben implementar,
    el tipo de retorno, el tipo de los parámetros
  \end{block}

  \begin{block}{Herramientas para Especificación}
    Las \textbf{clases abstractas} y las \textbf{interfaces} son
    mecanismos para especificar los métodos que una clase o sub-clase
    \textit{\textbf{deben implementar}}. En caso que falte alguna
    implementación, el código no va a compilar
  \end{block}

  \begin{block}{Implementación}
    Una clase \textit{\textbf{concreta}} que desea satisfacer una
    especificación debe implementar todos los métodos indicados por la
    especificación
  \end{block}
  
\end{frame}

\begin{frame}[fragile]
  \frametitle{Especificación Clases Abstractas con Interfaces}

  \begin{jsmall}
    public abstract class
    Figura2DArchivable extends Figura implements Archivable {
      /* las subclases deben implementar Figura y ademas Archivable */
    }

    public class RectanguloArchivable extends Figura2DArchivable {
      /* debe implementar los métodos de Figura: area, perimetro */
      /* debe implementar los métodos de Archivable: archivar, restaurar */
    }
  \end{jsmall}
  
\end{frame}

% Declarando una interfaz
%% Metodos son public abstract final
%% Cuerpo de una interfaz
%% Interfaces como APIs
% Implementando una interfaz en una clase
% Interfaces como tipos de dato

% Interfaces en Java 8
%% Métodos por defecto (Java 8) 
%%% Métodos estáticos
%

\begin{frame}
  \frametitle{Preguntas}
  \hspace{4cm}\huge{Preguntas?}  
\end{frame}

\end{document}