\documentclass{beamer}

\mode<presentation>
{
  \usetheme{CambridgeUS}
  % \setbeamercovered{transparent}
}

\usepackage[T1]{fontenc}
\usepackage[utf8]{inputenc}
\usepackage[spanish]{babel}
\usepackage{color}
\usepackage{hyperref}
\usepackage{algorithm,algorithmic}
\usepackage{colortbl}
\usepackage{graphicx}
\usepackage{multicol}
\usepackage{enumitem}
\setitemize{itemsep=1.2em,%
  label=\usebeamerfont*{itemize item}%
  \usebeamercolor[fg]{itemize item}%
  \usebeamertemplate{itemize item}
}

\usepackage{minted}
\usepackage[skins,minted]{tcolorbox}

\newcommand{\code}[1]{\mintinline{java}{#1}}
\newcommand{\codet}[1]{\texttt{#1}}

\setminted[java]{
  linenos=true,
  fontfamily=tt,
  fontsize=\small,
  frame=leftline,
  autogobble=True,
}

\newminted[jsmall]{java}{
  fontsize=\footnotesize
  , linenos = false
  , frame=single
  , autogobble=true
}

\newtcblisting{java}[2][]{
  minted language=java,
  enhanced, listing engine=minted,
  listing only, #1, title=#2, left=2em
}

\setminted[bash]{
  linenos=true,
  fontfamily=tt,
  fontsize=\small,
  frame=leftline
}

\newtcblisting{bash}[2][]{minted language=bash,
  enhanced, listing engine=minted,
  listing only, #1, title=#2, left=2em
}

\title[\textbf{Programación 2}]{\textbf{Programación 2}}
\subtitle{Entrada/Salida con Archivos en Java}

\author[IF-EG]
{Profesores:\\
  Ismael Figueroa -  \texttt{\small ifigueroap@gmail.com} \\
  \vspace{0.5mm} \\
  Eduardo Godoy - \texttt{\small eduardo.gl@gmail.com}
}

\institute[Universidad de Valparaíso]

\date{}

\begin{document}

\begin{frame}
  \titlepage
\end{frame}

\section{Operaciones de Entrada y Salida con Archivos}

\begin{frame}
  \frametitle{Librerías de Entrada/Salida para Archivos}

  En Java 7 y superior coexisten dos librerías para manejo de flujos
  con archivos:

  \begin{itemize}
  \item \codet{java.io}: librería introducida desde Java 1.0, hoy se
    considera como ``deprecated'', pero es relativamente sencilla de
    usar. Funciona de manera secuencial y bloqueante, en base a
    flujos, similar al uso de \codet{Scanner} o \codet{System.out}
    
  \item \codet{java.nio}: librería moderna de Entrada/Salida, muy
    flexible y configurable, pero en general más complicada de
    usar. Está orientada además al procesamiento concurrente de E/S.
    
  \end{itemize}

  \begin{alertblock}{Importante!}
      En el curso trabajaremos solamente con \codet{java.io}
  \end{alertblock}
  
\end{frame}

\begin{frame}
  \frametitle{Flujos de Entrada/Salida}

  \begin{block}{}
    Un flujo de entrada/salida representa una fuente de entrada o un
    destino de salida. Este concepto general puede representar
    diversos tipos de fuentes y destinos: archivos, dispositivos,
    otros programas, arreglos, etc.    
  \end{block}

  \begin{block}{}
    Los flujos soportan distintos tipos de datos: bytes, tipos de
    datos primitivos, u objetos. No obstante, \emph{todos los flujos
      tienen el mismo modelo de programación}
  \end{block}
\end{frame}

\begin{frame}
  \frametitle{Modelo de Flujos Entrada/Salida}

  Para el programador todos los flujos de entrada/salida funcionan de
  manera similar: los flujos son construcciones secuenciales de datos.

  \begin{itemize}
  \item Un \emph{flujo de entrada} se usa en un programa para
    \emph{leer datos desde la fuente}, de a un elemento cada vez.
    
  \item Un \emph{flujo de salida} se usa para \emph{escribir datos en
      un destino}, también de un elemento a la vez.
  \end{itemize}
  
\end{frame}

\section{Byte Streams}

\begin{frame}
  \frametitle{Byte Streams}

  \begin{block}{Definición}
    Se usan para trabajar directamente con bytes, o sea items de 8
    bits. Se usan en general para trabajar con datos/archivos
    binarios, como por ejemplo imagenes.
  \end{block}

  Clases para uso de Byte Streams:

  \begin{itemize}

  \item \codet{java.io.FileInputStream}: para leer archivos byte por
    byte. Es subclase de \codet{java.io.InputStream}.
    
  \item \codet{java.io.FileOutputStream}: para escribir archivos byte
    por byte. Es subclase de \codet{java.io.OutputStream}.
    
  \end{itemize}
    
\end{frame}

\begin{frame}[fragile]
  \frametitle{Ejemplo Byte Stream: Copiar archivo byte por byte}

  \begin{jsmall}
    import java.io.FileInputStream;
    import java.io.FileOutputStream;
    import java.io.IOException;

    public class Main {
      public static void main(String[] args) throws IOException {
        FileInputStream  fin  = null;
        FileOutputStream fout = null;
        int c;        
        try {
          fin  = new FileInputStream("hola.txt");
          fout = new FileOutputStream("salida.txt");
          while((c = fin.read()) != -1) {
            fout.write(c);
          }
        } finally {
          if (fin != null)  { fin.close(); }
          if (fout != null) { fout.close(); }
        }
      }}    
  \end{jsmall}  
\end{frame}

\section{Character Streams}

\begin{frame}
  \frametitle{Character Streams}

  \begin{block}{Definición}
    Se usan para trabajar con texto en base a caracteres. En Java los
    caracteres se almacenan usando Unicode, y cuando se procesan, el
    sistema considera la transformación al juego de caracteres local,
    según el país o región que esté configurada.
  \end{block}

  Clases para uso de Character Streams:

  \begin{itemize}
    
  \item \codet{java.io.FileReader}: para leer archivos de texto,
    caracter por caracter.
    
  \item \codet{java.io.FileWriter}: para escribir archivos de texto,
    caracter por caracter.
    
  \end{itemize}
    
\end{frame}

 \frametitle{Ejemplo Character Stream: Copiar archivo caracter por caracter}

  \begin{jsmall}
    import java.io.FileReader;
    import java.io.FileWriter;
    import java.io.IOException;

    public class Main {
      public static void main(String[] args) throws IOException {
        FileReader  fin  = null;
        FileWriter fout = null;
        int c;        
        try {
          fin  = new FileReader("hola.txt");
          fout = new FileWriter("salida.txt");
          while((c = fin.read()) != -1) {
            fout.write(c);
          }
        } finally {
          if (fin != null)  { fin.close(); }
          if (fout != null) { fout.close(); }
        }
      }}    
  \end{jsmall}  
\end{frame}

\section{Buffered Streams}

\begin{frame}
  \frametitle{Buffered Streams}

  El uso directo de los flujos de entrada o salida es poco eficiente
  porque cada invocación se delega directamente al sistema operativo,
  elemento por elemento. Java implementa un mechanismo de
  \emph{buffered streams} que acumula datos de entrada/salida en un
  área de memoria denominada \emph{buffer}. Entonces se invoca al
  sistema operativo cuando ya se tiene bastante información en el
  buffer, mejorando así la eficiencia.
  
\end{frame}

\begin{frame}
  \frametitle{Clases para BufferedStreams}

  Para trabajar con buffered streams en base a caracteres se usan las
  siguientes clases:

  \begin{itemize}
  \item \codet{java.io.BufferedReader} para trabajar con entrada de datos de caracteres    
  \item \codet{java.io.BufferedWriter} para trabajar con salida de datos de caracteres    
  \end{itemize}
  
\end{frame}

 \frametitle{Ejemplo Character Stream: Copiar archivo caracter por caracter}

  \begin{jsmall}
    import java.io.FileReader; import java.io.BufferedReader;
    import java.io.FileWriter; import java.io.BufferedWriter;
    import java.io.IOException;

    public class Main {
      public static void main(String[] args) throws IOException {
        BufferedReader  fin  = null;
        BufferedWriter fout = null;
        int c;        
        try {
          fin  = new BufferedReader(new FileReader("hola.txt"));
          fout = new BufferedWriter(new FileWriter("salida.txt"));
          while((c = fin.read()) != -1) {
            fout.write(c);
          }
        } finally {
          if (fin != null)  { fin.close(); }
          if (fout != null) { fout.close(); }
        }
      }}    
  \end{jsmall}  
\end{frame}

\section{Entrada y Salida Uniforme}
\begin{frame}
  \frametitle{Cómo nos gustaría que funcione la Entrada y Salida}

  \begin{block}{Mundo Ideal}
    Para no tener que pensar tanto cuando programamos, sería ideal que
    nuestros programas se escribieran y funcionaran de manera similar
    para:

    \begin{itemize}
    \item Leer desde el teclado o desde un archivo usando la clase
      \codet{Scanner}
      
    \item Escribir en la pantalla o en un archivo, usando métodos como
      \codet{println} u otros similares
      
    \item Además nos gustaría que esto fuera eficiente, utilizando
      Buffered Streams!
    \end{itemize}    
    
  \end{block}
    
\end{frame}

\section{Lectura de Archivos con \codet{Scanner}}

\begin{frame}[fragile]
  \frametitle{Leyendo archivos con \codet{Scanner}}

  Por suerte para nosotros, la clase \codet{Scanner} ya está preparada
  para lo que queremos hacer...
  
  \begin{itemize}
  \item En vez de ejecutar \codet{new Scanner(System.in)} para leer
    desde el teclado
    
  \item Ejecutamos:

    \begin{jsmall}
      Scanner s = new Scanner(
                      new BufferedReader(
                          new FileReader("archivo.txt")));

      \end{jsmall}
    
  \end{itemize}
  
\end{frame}

\section{Escritura de archivos con \codet{PrintWriter}}

\begin{frame}[fragile]
  \frametitle{Escribiendo archivos con \codet{PrintWriter}}

  Para escribir en un archivo de la misma forma que si estuvieramos
  usando \codet{System.out}, debemos crear un objeto de clase
  \codet{java.io.PrintWriter}:

  \begin{jsmall}
    PrintWriter writer = new PrintWriter(
                             new BufferedWriter(
                                 new FileWriter("archivo_salida.txt")));

   writer.println("Hola archivo!");
 \end{jsmall}

 \begin{itemize}[topsep]

 \item En la documentación de \codet{PrintWriter} están todos sus
   métodos, pero se destacan: \codet{print}, \codet{println},
   \codet{format}.
   
 \item Una desventaja es que los métodos nunca arrojan excepciones. Se
   debe usar el método \codet{checkError} para ver si hubo problemas
   en la escritura...
   
 \end{itemize}
   
\end{frame}

\section{Problema de Ejemplo}

\begin{frame}
  \frametitle{Ejemplo a desarrollar: Lectura datos de alumnos}

  \begin{block}{Enunciado}
    Considere un archivo de texto que contiene un listado de alumnos
    con los siguientes datos: nombre, apellido, nombre de asignatura,
    nota 1, nota 2 y nota 3. Cada campo está separado por un espacio.
  \end{block}

  \begin{block}{Problema}
    Escriba un programa en Java que lea el archivo con los datos y
    escriba un archivo ``informe.txt'' indicando los promedios de nota
    de cada alumno, indicando a qué asignatura corresponden. Use las
    clases \codet{Scanner} y \codet{PrintWriter} para leer y escribir
    los archivos, respectivamente.
  \end{block}

\end{frame}

\begin{frame}
  \frametitle{Preguntas}
  \hspace{4cm}\huge{Preguntas?}  
\end{frame}

\end{document}