\documentclass{beamer}

\mode<presentation>
{
  \usetheme{CambridgeUS}
  % \setbeamercovered{transparent}
}

\usepackage[T1]{fontenc}
\usepackage[utf8]{inputenc}
\usepackage[spanish]{babel}
\usepackage{color}
\usepackage{hyperref}
\usepackage{algorithm,algorithmic}
\usepackage{colortbl}
\usepackage{graphicx}
\usepackage{multicol}
\usepackage{enumitem}
\setitemize{itemsep=1.2em,%
  label=\usebeamerfont*{itemize item}%
  \usebeamercolor[fg]{itemize item}%
  \usebeamertemplate{itemize item}
}

\usepackage{minted}
\usepackage[skins,minted]{tcolorbox}

\newcommand{\code}[1]{\mintinline{java}{#1}}
\newcommand{\codet}[1]{\texttt{#1}}

\setminted[java]{
  linenos=true,
  fontfamily=tt,
  fontsize=\small,
  frame=leftline,
  autogobble=True,
}

\newminted[jsmall]{java}{
  fontsize=\footnotesize
  , linenos = false
  , frame=single
  , autogobble=true
}

\newtcblisting{java}[2][]{
  minted language=java,
  enhanced, listing engine=minted,
  listing only, #1, title=#2, left=2em
}

\setminted[bash]{
  linenos=true,
  fontfamily=tt,
  fontsize=\small,
  frame=leftline
}

\newtcblisting{bash}[2][]{minted language=bash,
  enhanced, listing engine=minted,
  listing only, #1, title=#2, left=2em
}

\title[\textbf{Programación 2}]{\textbf{Programación 2}}
\subtitle{Strings y Entrada/Salida Básica en Java}

\author[IF-EG]
{Profesores:\\
  Ismael Figueroa -  \texttt{\small ifigueroap@gmail.com} \\
  \vspace{0.5mm} \\
  Eduardo Godoy - \texttt{\small eduardo.gl@gmail.com}
}

\institute[Universidad de Valparaíso]

\date{}

\begin{document}

\begin{frame}
  \titlepage
\end{frame}

\section{La Clase \codet{String}}

\begin{frame}
  \frametitle{Strings en Java}

  Los strings, o en español \emph{cadenas}, representan secuencias de
  caracteres. En Java los strings siempre son objetos de clase
  \codet{String}. Existen muchas maneras de crear un String en
  Java\footnote{\url{https://docs.oracle.com/javase/8/docs/api/java/lang/String.html}}:

  \begin{itemize}    
  \item Asignar directamente un string literal
  \item Usar alguno de los constructores en base a arreglos de bytes o
    de caracteres.    
  \item Usar alguno de los constructores en base a otros strings y
    clases similares.
  \end{itemize}

  Las operaciones básicas son la concatenación, con el operador
  \codet{+}, y también conocer el largo del string, con la propiedad
  \codet{length}
  
\end{frame}

\begin{frame}[fragile]
  \frametitle{Inmutabilidad de los Strings}

  \begin{alertblock}{Inmutables}
    Por eficiencia, en Java los objetos \codet{String} son
    \textbf{inmutables}. Una vez creados, su valor no puede cambiar.
  \end{alertblock}

  \begin{jsmall}
    public class Hola {
      public static void main(String[] args) {
        String s1 = "Hola!";
        String s2 = s1;
        System.out.println(s1 + " / " + s2);
        
        s1 = "Ayudaaa";
        System.out.println(s1 + " / " + s2);
      }
    }    
  \end{jsmall}
  
\end{frame}

\begin{frame}[fragile]
  \frametitle{Igualdad de Strings}

  ¿Qué imprime por pantalla este programa?

  \begin{jsmall}
    public class Hola {
      public static void main(String[] args) {
        String s1 = new String("Hola!");
        String s2 = "Hola!";
        String s3 = s2;

        if(s1 == s2) {
          System.out.println("JAVA");
        }

        if(s2 == s3) {
          System.out.println("ES LO MEJOR");
        }
      }
    }    
  \end{jsmall}  
\end{frame}

\begin{frame}[fragile]
  \frametitle{Igualdad de Strings}

  \begin{block}{Igualdad por referencia o identidad}
    En Java el operador \codet{==} compara la \textbf{\textit{igualdad
        de referencias}}: ¿están las dos referencias apuntando a un
    mismo objeto? En el fondo este concepto está demasiado ligado a la
    \emph{identidad del objeto}.
  \end{block}

  \begin{block}{Igualdad por valor}
    En general es más práctico programar con la
    \textbf{\textit{igualdad por valor}}. Para esto se debe utilizar
    el método \codet{equals} o bien \codet{equalsIgnoreCase}.
  \end{block}

  \begin{jsmall}
    if(s1.equals(s2)) {
      System.out.println("JAVA ES LO MEJOR");
    }
  \end{jsmall}  
  
\end{frame}

\begin{frame}
  \frametitle{Comparación de Strings}

  Además de querer ver la igualdad entre strings, muchas necesitamos:

  \begin{itemize}
  \item Ver si un string comienza o termina con determinado
    prefijo/sufijo, con los métodos \codet{startsWith} y
    \codet{endsWith}
    
  \item Comparar dos strings para ver cuál es "mayor" o "menor", según
    orden lexicográfico, con los métodos \codet{compareTo} y
    \codet{compareToIgnoreCase}
    
  \item Determinar si un string calza con una \emph{expresión
      regular}, con el método \codet{matches}
  \end{itemize}
  
\end{frame}

\begin{frame}[fragile]
  \frametitle{Strings de Formato}

  Como alternativa a la concatenación, donde es algo engorroso abrir y
  cerrar comillas para cada elemento, en Java se pueden crear
  \emph{strings de formato}, que trabajan de forma similar a
  \codet{printf}.

  \begin{jsmall}
    public class Hola {
      public static void main(String[] args) {
        String nombre = "Juanito";
        int edad = 32;
        // Concatenacion
        String msg1 = "Hola soy " + nombre + " y tengo " + edad + " años";

        // String de formato
        String msg2 = String.format(
            "Hola soy %s y tengo %d años", nombre, edad);

        System.out.println(msg1);
        System.out.println(msg2);        
      }
    }
  \end{jsmall}
  
\end{frame}

\begin{frame}
  \frametitle{Conversión de String a número}

  Java provee 5 clases numéricas, que representan como objetos los
  tipos numéricos primitivos:

  \begin{itemize}
    \setlength\itemsep{0.2em}
  \item La clase \codet{Byte}
  \item La clase \codet{Integer}
  \item La clase \codet{Double}
  \item La clase \codet{Float}
  \item La clase \codet{Long}
  \item La clase \codet{Short}
  \end{itemize}

  Cada una de estas clases tiene el método \codet{valueOf} que
  convierte un string, asumiendo que tiene un número válido, en un
  objeto numérico.
  
\end{frame}

\begin{frame}
  \frametitle{Manipulación de Caracteres}
  A veces es necesario considerar los caracteres individuales de un
  string. Para esto tenemos los siguientes métodos:

  \begin{itemize}
  \item \codet{charAt}: retorna el caracter en la posición dada
  \item \codet{indexOf}/\codet{lastIndexOf}: retorna la primera/última
    posición del caracter en el string, o -1 si es que no se encuentra
  \item \codet{contains}: retorna verdader o falso según el caracter
    dado está o no en el string
  \item \codet{substring}: retorna el string entre las posiciones
    indicadas
  \end{itemize}
  
\end{frame}

% \begin{frame}
%   \frametitle{La clase \codet{StringBuilder}}
  
% \end{frame}

\begin{frame}
  \frametitle{Otros métodos de \codet{String}}

  \url{docs.oracle.com/javase/8/docs/api/java/lang/String.html}
  
\end{frame}

\section{Operaciones Básicas de Entrada y Salida}

\begin{frame}
  \frametitle{Flujos de Entrada/Salida}
  \begin{block}{}
    La programación de entrada/salida en Java se basa en la
    manipulación de \emph{flujos de entrada/salida}
  \end{block}

  \begin{block}{}
    Un flujo de entrada/salida representa una fuente de entrada o un
    destino de salida. Este concepto general puede representar
    diversos tipos de fuentes y destinos: archivos, dispositivos,
    otros programas, arreglos, etc.    
  \end{block}

  \begin{block}{}
    Los flujos soportan distintos tipos de datos: bytes, tipos de
    datos primitivos, u objetos. No obstante, \emph{todos los flujos
      tienen el mismo modelo de programación}
  \end{block}
\end{frame}

\begin{frame}
  \frametitle{Modelo de Flujos Entrada/Salida}

  Para el programador todos los flujos de entrada/salida funcionan de
  manera similar: los flujos son construcciones secuenciales de datos.

  \begin{itemize}
  \item Un \emph{flujo de entrada} se usa en un programa para
    \emph{leer datos desde la fuente}, de a un elemento cada vez.
    
  \item Un \emph{flujo de salida} se usa para \emph{escribir datos en
      un destino}, también de un elemento a la vez.
  \end{itemize}
  
\end{frame}

\begin{frame}
  \frametitle{Tipos Básicos de Flujos}

  \begin{itemize}
  \item \textbf{Bytes Streams}: se usan para trabajar directamente
    con bytes, o sea items de 8 bits. Se usan en general para trabajar
    con datos/archivos binarios.
    
  \item \textbf{Character Streams}: se usan para trabajar con
    caracteres de texto, con soporte Unicode directamente
    implementado. Es lo más conveniente para trabajar con
    datos/archivos de texto.
  \end{itemize}
  
\end{frame}

\begin{frame}
  \frametitle{Buffered Streams}

  El uso directo de los flujos de entrada o salida es poco eficiente
  porque cada invocación se delega directamente al sistema operativo,
  elemento por elemento. Java implementa un mechanismo de
  \emph{buffered streams} que acumula datos de entrada/salida en un
  área de memoria denominada \emph{buffer}. Entonces se invoca al
  sistema operativo cuando ya se tiene bastante información en el
  buffer, mejorando así la eficiencia.
  
\end{frame}

\begin{frame}
  \frametitle{Flujos Estándar del Sistema}

  Los flujos estándar son una característica de muchos sistemas
  operativos. Por defecto se usan para leer entrada desde el teclado y
  imprimir texto en la consola. Por defecto existen 3 streams
  disponibles automáticamente en Java:

  \begin{itemize}
  \item \codet{System.in}: entrada estándar, para lectura de datos
    desde el teclado.
    
  \item \codet{System.out}: salida estándar, imprime en la consola.

  \item \codet{System.err}: error estándar, también imprime en la
    consola, pero el sistema operativo puede diferenciar entre salida
    y error estándar.
  \end{itemize}
  
\end{frame}

\subsection{Escribir por Pantalla}

\begin{frame}
  \frametitle{\codet{System.out}}

  Hasta ahora hemos usado bastante el objeto
  \codet{System.out}. Algunos métodos clave son:

  \begin{itemize}
  \item \codet{print}: imprime un valor dado, que puede ser primitivo,
    o un objeto. En caso que sea un objeto se usa el método
    \codet{toString} implícitamente para la conversión
    
  \item \codet{println}: igual que \codet{print} pero agrega un salto
    de línea al final
    
  \item \codet{printf}: similar a la función de C, usa un string de
    formato y luego los argumentos respectivos
  
  \item \codet{format}: básicamente lo mismo que \codet{printf}
    
  \end{itemize}  
\end{frame}

\subsection{Leer desde el Teclado}

\begin{frame}
  \frametitle{¿Cúal clase usar para leer?}

  En Java existen al menos 3 formas distintas para leer desde el
  teclado:

  \begin{itemize}
  \item \codet{new BufferedReader(new InputStreamReader(System.in))}    
  \item Usando la clase \codet{Console}, disponible desde Java 6    
  \item Usando la clase \codet{Scanner}, disponible desde Java 5
  \end{itemize}

  Lo recomendado es usar siempre \codet{Scanner}, excepto dos casos
  donde es mejor usar \codet{Console}: lectura de passwords, y lectura
  caracter por caracter.
  
\end{frame}

\begin{frame}[fragile]
  \frametitle{Hola Mundo leyendo datos!}

   \begin{columns}
     \begin{column}{0.4\textwidth}
       \begin{small}
         \begin{itemize}
         \item Se debe importar \codet{java.util.Scanner}         
         \item En su constructor recibe \codet{System.in} para leer
           desde teclado.
         \item \codet{nextLine()} lee una línea como String           
         \item \codet{nextInt()} lee el siguiente entero
         \end{itemize}
       \end{small}
     \end{column}
    % 
    \begin{column}{0.6\textwidth}
      
      \begin{jsmall}
        import java.util.Scanner;
        
        public class Main {
          public static void main(String[] args) {
            Scanner s = new Scanner(System.in);
            String nombre;
            int edad;

            nombre = s.nextLine();           
            System.out.println(nombre);

            edad = s.nextInt();
            System.out.println(edad);
          }
        }
      \end{jsmall}
      
    \end{column}
  \end{columns}
  
\end{frame}

\begin{frame}
  \frametitle{Cómo funciona \codet{Scanner}}

  \begin{itemize}
  \item La clase \codet{Scanner} toma la entrada desde teclado    
  \item Considera como \emph{delimitador} el espacio en blanco    
  \item La entrada se separa según el delimitador, en una secuencia de
    \emph{tokens}    
  \item Se intenta interpretar los tokens según el tipo de dato
    pedido: \codet{nextInt}, \codet{nextDouble}, etc. Funciona con
    tipos primitivos y Strings.   
  \item Si no se puede, ocurre una excepción!    
  \item Por lo tanto se debe conocer (o determinar) de antemano la
    estructura de los datos que se van a leer desde teclado, para que
    la operación siempre sea correcta
  \end{itemize}
  
\end{frame}

\begin{frame}
  \frametitle{Métodos de Lectura de \codet{Scanner}}

  \begin{block}{Lectura de Strings}
    Para leer Strings desde \codet{Scanner}, basta usar el método
    \codet{next}. También, el método \codet{nextLine} toma la entrada
    hasta el siguiente salto de línea.
  \end{block}

  \begin{block}{Lectura de Tipos Primitivos}
    Para cada tipo primitivo, como \codet{int} o \codet{double},
    existe un método \codet{nextInt}, \codet{nextDouble}, etc.
  \end{block}

  \begin{block}{Finalizar la lectura}
    El método \codet{hasNext()} retorna falso si ya no hay más
    elementos por leer.
  \end{block}
  
\end{frame}

\begin{frame}
  \frametitle{Ejemplo a desarrollar: Lectura datos de alumnos}

  \begin{block}{Problema}
    Un usuario ingresará por teclado un listado con los siguientes
    datos: nombre, apellido, nombre de asignatura, nota 1, nota 2 y
    nota 3. Primeramente el usuario ingresará la cantidad de alumnos,
    y luego esa cantidad de filas con la información mencionada.

    \begin{itemize}
    \item \textbf{Escriba un programa en Java para imprimir los
        promedios de nota de cada alumno, indicando a qué asignatura
        corresponden.}
     
    \item \textbf{Escriba un programa en Java que además de lo
        anterior, imprime un ranking de alumnos, de mejor a peor
        nota.}      
    \end{itemize}

  \end{block}

\end{frame}

\begin{frame}
  \frametitle{Preguntas}
  \hspace{4cm}\huge{Preguntas?}  
\end{frame}

\end{document}