\documentclass{exam}
\usepackage[spanish,activeacute]{babel}
\usepackage[utf8]{inputenc}
\usepackage[T1]{fontenc}
\usepackage[newcommands]{ragged2e}

\usepackage{
    amsmath,
    amssymb,
    eso-pic,
    float,
    graphicx,
    lmodern,
    wrapfig,
    tabularx,
    multicol,
    multirow,
    color,
    colortbl,
    lastpage,
    titlesec,
    sectsty
}

\definecolor{azul}{RGB}{33,127,190}
\sectionfont{\color{azul}}
\subsectionfont{\color{azul}}
\renewcommand{\familydefault}{\sfdefault}

\footer{}{\thepage}{}

\makeatother

\title{\LARGE\color{azul}\textbf{ICI 202 Programaci\'on 2 - Ejercicio sumativo 01 }}
\author{\normalsize \color{gray}{Prof.} \color{black}{\textbf{Ismael Figueroa, Eduardo Godoy}}}
\date{\normalsize \em \today}

\begin{document}

\AddToShipoutPictureBG*{%
  \AtPageUpperLeft{\raisebox{-\height}{\includegraphics[scale=.95]{base/header.png}}}}

\maketitle

\begin{multicols}{2} \begin{flushleft} \textbf{Nombre:} \\ \vspace*{2mm} \textbf{Rut:} \\ \vspace*{2mm} \textbf{Paralelo:} \end{flushleft} \begin{center} \begin{table}[H] \begin{tabular}{p{4cm}|p{3cm}|} \arrayrulecolor{gray!50}\cline{2-2} ~ & {\em {\scriptsize \color{gray!50}{Puntaje:}}} \\ & ~ \\ ~ & \textbf{Nota:} \\ & ~ \\ \arrayrulecolor{gray!50}\cline{2-2} \end{tabular} \end{table} \end{center} \end{multicols}

\vspace*{-18mm}
\noindent
\textbf{\\Instrucciones:}
\begin{itemize}
    \item[-] El puntaje m\'aximo  es 100 puntos.
    \item[-] Tiempo m\'aximo: 90 minutos.
    \item[-] El trabajo es \underline{\textbf{individual}}. Cualquier intento de copia, ser\'a sancionado seg\'un dicta el reglamento de la carrera.
    \item[-] Se debe subir al aula virtual un archivo comprimido con el siguiente formato  Ej01\_<NombreApellidoEstudiante\_rut\_paralelo>.zip, dentro de el deben estar los ocdigos fuentes requeridos.
\end{itemize}

\noindent
\textbf{Resultados de aprendizaje a evaluar:} $\langle\langle$ {\em Enumerar los resultados de aprendizaje que se eval{\'u}an en este instrumento (programa de la asignatura).}$\rangle\rangle$
\vspace{2mm}

\noindent
\textbf{Contenido:} Este certamen eval\'ua los siguientes temas:

\vspace{-2mm}
\begin{table}[H]
\begin{tabular}{
    !{\color{gray!50}\vrule}l
    !{\color{gray!50}\vrule}c
    !{\color{gray!50}\vrule}c
    !{\color{gray!50}\vrule}} \arrayrulecolor{gray!50} \hline
    \multicolumn{1}{!{\color{gray!50}\vrule}c}{\multirow{2}{*}{\textbf{
        Tema
    }}} &
    \multicolumn{2}{!{\color{gray!50}\vrule}c!{\color{gray!50}\vrule}}{\textbf{
        Puntajes
    }} \\ \arrayrulecolor{gray!50}\cline{2-3} &
    \multicolumn{1}{!{\color{gray!50}\vrule}c!{\color{gray!50}\vrule}}{\textbf{
        Total
    }} &
    \multicolumn{1}{c!{\color{gray!50}\vrule}}{\textbf{
        Obtenido
    }} \\ \arrayrulecolor{gray!50} \hline
    Modelamiento del problema y ejemplo de una instancia de clase.
    & \multicolumn{1}{!{\color{gray!50}\vrule}c!{\color{gray!50}\vrule}}{\textbf{
        40 pts.
    }} & \\ \arrayrulecolor{gray!50} \hline
    Ejemplo de una instancia de clase.
    & \multicolumn{1}{!{\color{gray!50}\vrule}c!{\color{gray!50}\vrule}}{\textbf{
        10 pts.
    }} & \\ \arrayrulecolor{gray!50} \hline
    Codificaci\'on de Clase con atributos y m\'etodos en leguaje java.
    & \multicolumn{1}{!{\color{gray!50}\vrule}c!{\color{gray!50}\vrule}}{\textbf{
        40 pts.
    }} & \\ \arrayrulecolor{gray!50} \hline
    Cumple con el formato y compilaci\'on de cada clase de forma correcta
    & \multicolumn{1}{!{\color{gray!50}\vrule}c!{\color{gray!50}\vrule}}{\textbf{
        10 pts.
    }} & \\ \arrayrulecolor{gray!50} \hline

\end{tabular}
\end{table}

\vspace{-7mm}
\section{\textbf{Modelamiento en Orientaci\'on a Objetos (100 pts)}}
\noindent
\textbf{Plantamiento de problema: }

\begin{questions}

  \begin{enumerate}
		\item Modele el sistema Portal de Alumno de la Universidad de Valpara\'iso, seg\'un el siguiente plantamiento:
    \begin{itemize}
      \item El portal de almunos de la Universida Valpara\'iso actualmente posee 2 tipos de usuarios los cuales son Estudiantes y Profesores.
      \item Los Estudiantes pueden acceder a consultar su estado acad\'emico, sus asignaturas cursadas, sus calificaciones y su matricula.
      \item Los Acad\'emicos pueden ingresar para registrar notas de evaluaciones.
      \item Ambos usuarios pueden acceder al Aula Virtual para gestionar el material acad\'emico  generado para cada asignatura.
    \end{itemize}
    \item seleccionar una clase y generar su instancia de forma grafica.
    \item Codificar en lenguaje java cada clase del modelo junto a sus atributos y m\'etodos.
    \end{enumerate}
\end{questions}

\begin{table}[H]
\centering
\begin{tabular}{
!{\color{gray!50}\vrule}p{3.9cm}
!{\color{gray!50}\vrule}p{3.6cm}
!{\color{gray!50}\vrule}p{3.6cm}
!{\color{gray!50}\vrule}p{3.6cm}
!{\color{gray!50}\vrule}} \arrayrulecolor{gray!50} \hline
    \multicolumn{4}{!{\color{gray!50}\vrule}c!{\color{gray!50}\vrule}}{\textbf{?`C\'omo ser{\'e} evaluado en este trabajo?}} \\ \arrayrulecolor{gray!50} \hline
    \textbf{\'Item} & \textbf{Logrado} & \textbf{Suficiente} & \textbf{No Logrado}\\ \arrayrulecolor{gray!50} \hline
  \newline Creaci\'on de Clase con atributos. &
  \newline  Aplica de forma correcta 40\%  ... &
  \newline Aplica parcialmente con menos de 2 errores 25\% ... &
    \newline Aplica de forma incorrecta con 3 errores o m\'as 0-9\%\\ \arrayrulecolor{gray!50} \hline
    \newline  Implementaci\'on de m\'etodos (comportamientos) asociados y correcto funcionamiento. &
    \newline  Aplica de forma correcta 10\%  ... &
    \newline Aplica parcialmente con menos de 2 errores 5\%... &
    \newline Aplica de forma incorrecta con 3 errores o m\'as 0\%\\ \arrayrulecolor{gray!50} \hline
  \newline  Implementar clase instanciadora con m\'etodo main y crea instancias de objetos. &
  \newline  Aplica de forma correcta 40\%... &
    \newline  Aplica parcialmente con menos de 2 errores 25\%... &
    \newline Aplica de forma incorrecta con 3 errores o m\'as 0-9\%\\ \arrayrulecolor{gray!50} \hline
    \newline  Cumple con el formato y compilaci\'on correcta &
    \newline  Aplica de forma correcta 10\% ... &
    \newline Aplica parcialmente con menos de x errores 5\%... &
    \newline Aplica de forma incorrecta con x errores o m\'as 0-4\%\\ \arrayrulecolor{gray!50} \hline
    Total de la secci\'on &  100\% & 60\% & 0-22\%\\ \arrayrulecolor{gray!50} \hline
\end{tabular}
\label{tbl:1}
\end{table}
\vspace{-5mm}
\textbf{Nota:} En caso de que el {\'i}tem no est{\'e} presente, tiene ponderaci{\'o}n cero.

\end{document}
