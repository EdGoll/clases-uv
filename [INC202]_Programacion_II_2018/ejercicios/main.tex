\documentclass{exam}
\usepackage[spanish,activeacute]{babel}
\usepackage[utf8]{inputenc}
\usepackage[T1]{fontenc}
\usepackage[newcommands]{ragged2e}

\usepackage{
    amsmath,
    amssymb,
    eso-pic,
    float,
    graphicx,
    lmodern,
    wrapfig,
    tabularx,
    multicol,
    multirow,
    color,
    colortbl,
    lastpage,
    titlesec,
    sectsty
}

\definecolor{azul}{RGB}{33,127,190}
\sectionfont{\color{azul}}
\subsectionfont{\color{azul}}
\renewcommand{\familydefault}{\sfdefault}

\footer{}{\thepage}{}

\makeatother

\title{\LARGE\color{azul}\textbf{ICI YYY Programaci\'on 2 - Prueba Especial}}
\author{\normalsize \color{gray}{Prof.} \color{black}{\textbf{Eduardo Godoy}}}
\date{\normalsize \em \today}

\begin{document}

\AddToShipoutPictureBG*{%
  \AtPageUpperLeft{\raisebox{-\height}{\includegraphics[scale=.95]{base/header.png}}}}

\maketitle

\begin{multicols}{2} \begin{flushleft} \textbf{Nombre:} \\ \vspace*{2mm} \textbf{Rut:} \end{flushleft} \begin{center} \begin{table}[H] \begin{tabular}{p{4cm}|p{3cm}|} \arrayrulecolor{gray!50}\cline{2-2} ~ & {\em {\scriptsize \color{gray!50}{Puntaje:}}} \\ & ~ \\ ~ & \textbf{Nota:} \\ & ~ \\ \arrayrulecolor{gray!50}\cline{2-2} \end{tabular} \end{table} \end{center} \end{multicols}

\vspace*{-18mm}
\noindent
\textbf{Instrucciones:} 
\begin{itemize}
    \item[-] El puntaje m\'aximo del certamen es 100 puntos, siendo 60 puntos el m\'inimo requerido para aprobar.
    \item[-] Tiempo m\'aximo: 90 minutos.
    \item[-] El certamen es \underline{\textbf{individual}}. Cualquier intento de copia, ser\'a sancionado seg\'un dicta el reglamento de la carrera.
\end{itemize}

\noindent
\textbf{Resultados de aprendizaje a evaluar:} $\langle\langle$ {\em Enumerar los resultados de aprendizaje que se eval{\'u}an en este instrumento (programa de la asignatura).}$\rangle\rangle$
\vspace{2mm}

\noindent
\textbf{Contenido:} Este certamen eval\'ua los siguientes temas:

\vspace{-2mm}
\begin{table}[H]
\begin{tabular}{
    !{\color{gray!50}\vrule}l
    !{\color{gray!50}\vrule}c
    !{\color{gray!50}\vrule}c
    !{\color{gray!50}\vrule}} \arrayrulecolor{gray!50} \hline
    \multicolumn{1}{!{\color{gray!50}\vrule}c}{\multirow{2}{*}{\textbf{
        Tema
    }}} & 
    \multicolumn{2}{!{\color{gray!50}\vrule}c!{\color{gray!50}\vrule}}{\textbf{
        Puntajes
    }} \\ \arrayrulecolor{gray!50}\cline{2-3} &
    \multicolumn{1}{!{\color{gray!50}\vrule}c!{\color{gray!50}\vrule}}{\textbf{
        Total
    }} & 
    \multicolumn{1}{c!{\color{gray!50}\vrule}}{\textbf{
        Obtenido
    }} \\ \arrayrulecolor{gray!50} \hline
    Implementaci\'on de Componentes Java Swing.
    & \multicolumn{1}{!{\color{gray!50}\vrule}c!{\color{gray!50}\vrule}}{\textbf{
        40 pts.
    }} & \\ \arrayrulecolor{gray!50} \hline
    Implementaci\'on y operaciones b\'asicas con  arreglos.
    & \multicolumn{1}{!{\color{gray!50}\vrule}c!{\color{gray!50}\vrule}}{\textbf{
        20 pts.
    }} & \\ \arrayrulecolor{gray!50} \hline
    Conocimientos en Orientaci\'on a Objetos.
    & \multicolumn{1}{!{\color{gray!50}\vrule}c!{\color{gray!50}\vrule}}{\textbf{
        40 pts.
    }} & \\ \arrayrulecolor{gray!50} \hline
    Implementaci\'on de soluci\'on utilizando lenguaje de programaci\'on Java.
    & \multicolumn{1}{!{\color{gray!50}\vrule}c!{\color{gray!50}\vrule}}{\textbf{
        40 pts.
    }} & \\ \arrayrulecolor{gray!50} \hline
\end{tabular}
\end{table}

\vspace{-7mm}
\section{\textbf{Programaci\'on en Java (100 pts)}}
\noindent
\textbf{Respuestas breves: } Descripci{\'o}n de las reglas (L: X pts. S: Y pts. NL: Z pts.).

\begin{questions}
    
Desarrollar una herramienta, con interfaz de usuario en Java que permita transformar una palabra, frase u oraci\'on en c\'odigo morse y viceversa. Para ello, asuma que existe una clase llamada \textbf{TranslatorHelper} que pose\'e los m\'etodos est\'aticos \textbf{text2morse(String)} y \textbf{morse2text(String)} que transforma de texto a morse y de morse a texto, respectivamente. La interfaz de usuario debe contener al menos los componentes que se muestran en la Figura 

\end{questions}

\begin{table}[H]
    {\small
    \begin{tabular}{lll}
        L  & : Logrado    & : Describir lo que se considera como logrado por el estudiante. \\
        S  & : Suficiente & : Describir lo que se considera como suficiente. \\
        NL & : No Logrado & : Describir lo que se considera como no logrado, por ejemplo ausencia de respuesta.
    \end{tabular}}
\end{table}

\vspace{-7mm}
\section{\textbf{\'Item o Secci\'on (Puntaje de la secci\'on)}}
\noindent
\textbf{Respuestas de selecci\'on m\'ultiple: } Seleccione la alternativa correcta (L: X pts. NL: Z pts.).

\begin{questions}

    \question Pregunta 1.
    \begin{parts}
        \part Opci\'on 1
        \part Opci\'on 2
        \part $\ldots$
        \part $\ldots$
    \end{parts}

    \question Pregunta 2.
    \begin{parts}
        \part Opci\'on 1
        \part Opci\'on 2
        \part $\ldots$
        \part $\ldots$
    \end{parts}
    
\end{questions}

\begin{table}[H]
    {\small
    \begin{tabular}{lll}
        L  & : Logrado    & : Describir lo que se considera como logrado por el estudiante. \\
        NL & : No Logrado & : Describir lo que se considera como no logrado, por ejemplo ausencia de respuesta.
    \end{tabular}}
\end{table}

\vspace{-7mm}
\section{\textbf{\'Item o Secci\'on (Puntaje de la secci\'on)}}
\noindent
\textbf{Respuestas con dibujos: } Descripci{\'o}n de las reglas (L: X pts. S: Y pts. NL: Z pts.).

\begin{figure}[H]
    \centering
    \includegraphics{base/placeholder.png}
    \label{fig:1}
\end{figure}

\begin{table}[H]
\centering
\begin{tabular}{
!{\color{gray!50}\vrule}p{3.9cm}
!{\color{gray!50}\vrule}p{3.6cm}
!{\color{gray!50}\vrule}p{3.6cm}
!{\color{gray!50}\vrule}p{3.6cm}
!{\color{gray!50}\vrule}} \arrayrulecolor{gray!50} \hline
    \multicolumn{4}{!{\color{gray!50}\vrule}c!{\color{gray!50}\vrule}}{\textbf{?`C\'omo ser{\'e} evaluado en la secci\'on X?}} \\ \arrayrulecolor{gray!50} \hline
    \textbf{\'Item} & \textbf{Logrado} & \textbf{Suficiente} & \textbf{No Logrado}\\ \arrayrulecolor{gray!50} \hline
    XX\newline Descripci\'on del \'item &
    X\%\newline Aplica de forma correcta ... & 
    Y\%\newline Aplica parcialmente con menos de x errores ... & 
    Z\%\newline Aplica de forma incorrecta con x errores o m\'as\\ \arrayrulecolor{gray!50} \hline
    YY\newline Descripci\'on del \'item & 
    X\%\newline Aplica de forma correcta ... &
    Y\%\newline Aplica parcialmente con menos de x errores ... &
    Z\%\newline Aplica de forma incorrecta con x errores o m\'as\\ \arrayrulecolor{gray!50} \hline
    ZZ\newline Descripci\'on del \'item & 
    X\%\newline Aplica de forma correcta ... &
    Y\%\newline Aplica parcialmente con menos de x errores ... &
    Z\%\newline Aplica de forma incorrecta con x errores o m\'as\\ \arrayrulecolor{gray!50} \hline
    Total de la secci\'on &  100\% & 60\% & ?\%\\ \arrayrulecolor{gray!50} \hline
\end{tabular}
\label{tbl:1}
\end{table}
\vspace{-5mm}
\textbf{Nota:} En caso de que el {\'i}tem no est{\'e} presente, tiene ponderaci{\'o}n cero.

\end{document}