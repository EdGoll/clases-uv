\documentclass{exam}
\usepackage[spanish,activeacute]{babel}
\usepackage[utf8]{inputenc}
\usepackage[T1]{fontenc}
\usepackage[newcommands]{ragged2e}

\usepackage{
    amsmath,
    amssymb,
    eso-pic,
    float,
    graphicx,
    lmodern,
    wrapfig,
    tabularx,
    multicol,
    multirow,
    color,
    colortbl,
    lastpage,
    titlesec,
    sectsty
}

\definecolor{azul}{RGB}{33,127,190}
\sectionfont{\color{azul}}
\subsectionfont{\color{azul}}
\renewcommand{\familydefault}{\sfdefault}

\footer{}{\thepage}{}

\makeatother

\title{\LARGE\color{azul}\textbf{ICI 202 Programaci\'on 2 - Ejercicio sumativo 02 }}
\author{\normalsize \color{gray}{Prof.} \color{black}{\textbf{Ismael Figueroa, Eduardo Godoy}}}
\date{\normalsize \em \today}

\begin{document}

\AddToShipoutPictureBG*{%
  \AtPageUpperLeft{\raisebox{-\height}{\includegraphics[scale=.95]{base/header.png}}}}

\maketitle

\begin{multicols}{2} \begin{flushleft} \textbf{Nombre:} \\ \vspace*{2mm} \textbf{Rut:} \\ \vspace*{2mm} \textbf{Paralelo:} \end{flushleft} \begin{center} \begin{table}[H] \begin{tabular}{p{4cm}|p{3cm}|} \arrayrulecolor{gray!50}\cline{2-2} ~ & {\em {\scriptsize \color{gray!50}{Puntaje:}}} \\ & ~ \\ ~ & \textbf{Nota:} \\ & ~ \\ \arrayrulecolor{gray!50}\cline{2-2} \end{tabular} \end{table} \end{center} \end{multicols}

\vspace*{-18mm}
\noindent
\textbf{\\Instrucciones:}
\begin{itemize}
    \item[-] El puntaje m\'aximo  es 100 puntos.
    \item[-] Tiempo m\'aximo: 90 minutos.
    \item[-] El trabajo es \underline{\textbf{individual}}. Cualquier intento de copia, ser\'a sancionado seg\'un dicta el reglamento de la carrera.
    \item[-] Se debe subir al aula virtual un archivo comprimido con el siguiente formato  Ej02\_<NombreEstudiante\_rut\_paralelo>.zip, dentro de el deben estar los ocdigos fuentes requeridos.
\end{itemize}

\noindent
\textbf{Resultados de aprendizaje a evaluar:} $\langle\langle$ {\em Enumerar los resultados de aprendizaje que se eval{\'u}an en este instrumento (programa de la asignatura).}$\rangle\rangle$
\vspace{2mm}

\noindent
\textbf{Contenido:} Este certamen eval\'ua los siguientes temas:

\vspace{-2mm}
\begin{table}[H]
\begin{tabular}{
    !{\color{gray!50}\vrule}l
    !{\color{gray!50}\vrule}c
    !{\color{gray!50}\vrule}c
    !{\color{gray!50}\vrule}} \arrayrulecolor{gray!50} \hline
    \multicolumn{1}{!{\color{gray!50}\vrule}c}{\multirow{2}{*}{\textbf{
        Tema
    }}} &
    \multicolumn{2}{!{\color{gray!50}\vrule}c!{\color{gray!50}\vrule}}{\textbf{
        Puntajes
    }} \\ \arrayrulecolor{gray!50}\cline{2-3} &
    \multicolumn{1}{!{\color{gray!50}\vrule}c!{\color{gray!50}\vrule}}{\textbf{
        Total
    }} &
    \multicolumn{1}{c!{\color{gray!50}\vrule}}{\textbf{
        Obtenido
    }} \\ \arrayrulecolor{gray!50} \hline
    Creaci\'on de Clase con atributos.
    & \multicolumn{1}{!{\color{gray!50}\vrule}c!{\color{gray!50}\vrule}}{\textbf{
        30 pts.
    }} & \\ \arrayrulecolor{gray!50} \hline
    Implementaci\'on de m\'etodos (comportamientos) asociados y correcto funcionamiento.
    & \multicolumn{1}{!{\color{gray!50}\vrule}c!{\color{gray!50}\vrule}}{\textbf{
        30 pts.
    }} & \\ \arrayrulecolor{gray!50} \hline
    Implementar clase instanciadora con m\'etodo main.
    & \multicolumn{1}{!{\color{gray!50}\vrule}c!{\color{gray!50}\vrule}}{\textbf{
        30 pts.
    }} & \\ \arrayrulecolor{gray!50} \hline
    Cumple con el formato y compilaci\'on correcta
    & \multicolumn{1}{!{\color{gray!50}\vrule}c!{\color{gray!50}\vrule}}{\textbf{
        10 pts.
    }} & \\ \arrayrulecolor{gray!50} \hline

\end{tabular}
\end{table}

\vspace{-7mm}
\section{\textbf{Programaci\'on en Java (100 pts)}}
\noindent
\textbf{Plantamiento de problema: }

\begin{questions}

  \begin{enumerate}
		\item Desarrolle una clase \textbf{MedicionPersona} que posea:
		\begin{enumerate}
			\item Caracter\'isticas: \emph{nombre(s)}, \emph{apellido}, \emph{edad}, \emph{peso} y \emph{altura}. Considerar el tipo de dato id\'oneo para cada uno de los atributos y definir el valor por omisi\'on acorde a cada tipo.
			\item Comportamiento: \textbf{comprobarSexo()}, \textbf{esMayorDeEdad()}, \textbf{calcularIMC()} y \textbf{toString()} para mostrar los atributos del objeto.
			\begin{enumerate}
				\item \emph{esMayorDeEdad}(): Determina si la persona es o no mayor de edad mostrando mensaje por pantalla .
				\item \emph{calcularIMC}(): c\'alculo del \'indice de masa corporal ($\frac{peso}{altura^{2}}$). IMC ideal: $20 \leq IMC \leq 24,9$.
			\end{enumerate}
			\item Crear una clase externa que permita  instanciar 2 objetos con datos seg\'un la siguiente gu\'ia::

          \begin{enumerate}
            \item Primer Objeto:
            \begin{itemize}
              \item nombre = Pedro
              \item apellido=Figueroa
              \item peso=70
              \item altura=1.85.
            \end{itemize}
            \item Segundo Objeto:
            \begin{itemize}
              \item nombre = Juan.
              \item apellido=Godoy.
              \item peso=75.
              \item altura=1.75.
            \end{itemize}
          \end{enumerate}
				\item Para cada objeto, se debe comprobar si la persona est\'a en su peso ideal, tiene sobrepeso o esta\'a por debajo de su peso ideal (con un mensaje en la salida est\'andar).
				\item Indicar para cada objeto si la persona es mayor de edad.
				\item Por \'ultimo, para cada objeto, se debe mostrar la informaci\'on de la medici\'on.
			\end{enumerate}
		\end{enumerate}
    \end{enumerate}
\end{questions}

\begin{table}[H]
\centering
\begin{tabular}{
!{\color{gray!50}\vrule}p{3.9cm}
!{\color{gray!50}\vrule}p{3.6cm}
!{\color{gray!50}\vrule}p{3.6cm}
!{\color{gray!50}\vrule}p{3.6cm}
!{\color{gray!50}\vrule}} \arrayrulecolor{gray!50} \hline
    \multicolumn{4}{!{\color{gray!50}\vrule}c!{\color{gray!50}\vrule}}{\textbf{?`C\'omo ser{\'e} evaluado en este trabajo?}} \\ \arrayrulecolor{gray!50} \hline
    \textbf{\'Item} & \textbf{Logrado} & \textbf{Suficiente} & \textbf{No Logrado}\\ \arrayrulecolor{gray!50} \hline
  \newline Creaci\'on de Clase con atributos. &
  \newline  Aplica de forma correcta 30\%  ... &
  \newline Aplica parcialmente con menos de 2 errores 20\% ... &
    \newline Aplica de forma incorrecta con 3 errores o m\'as 0-9\%\\ \arrayrulecolor{gray!50} \hline
    \newline  Implementaci\'on de m\'etodos (comportamientos) asociados y correcto funcionamiento. &
    \newline  Aplica de forma correcta 30\%  ... &
    \newline Aplica parcialmente con menos de 2 errores 20\%... &
    \newline Aplica de forma incorrecta con 3 errores o m\'as 0-8\%\\ \arrayrulecolor{gray!50} \hline
  \newline  Implementar clase instanciadora con m\'etodo main y crea instancias de objetos. &
  \newline  Aplica de forma correcta 30\%... &
    \newline  Aplica parcialmente con menos de 2 errores 15\%... &
    \newline Aplica de forma incorrecta con 3 errores o m\'as 0-9\%\\ \arrayrulecolor{gray!50} \hline
    \newline  Cumple con el formato y compilaci\'on correcta &
    \newline  Aplica de forma correcta 10\% ... &
    \newline Aplica parcialmente con menos de x errores 5\%... &
    \newline Aplica de forma incorrecta con x errores o m\'as 0-4\%\\ \arrayrulecolor{gray!50} \hline
    Total de la secci\'on &  100\% & 60\% & 0-30\%\\ \arrayrulecolor{gray!50} \hline
\end{tabular}
\label{tbl:1}
\end{table}
\vspace{-5mm}
\textbf{Nota:} En caso de que el {\'i}tem no est{\'e} presente, tiene ponderaci{\'o}n cero.

\end{document}
