\documentclass{exam}
\usepackage[spanish,activeacute]{babel}
\usepackage[utf8]{inputenc}
\usepackage[T1]{fontenc}
\usepackage[newcommands]{ragged2e}

\usepackage{
    amsmath,
    amssymb,
    eso-pic,
    float,
    graphicx,
    lmodern,
    wrapfig,
    tabularx,
    multicol,
    multirow,
    color,
    colortbl,
    lastpage,
    titlesec,
    sectsty
}

\definecolor{azul}{RGB}{33,127,190}
\sectionfont{\color{azul}}
\subsectionfont{\color{azul}}
\renewcommand{\familydefault}{\sfdefault}

\footer{}{\thepage}{}

\makeatother

\title{\LARGE\color{azul}\textbf{ICI 202 Programaci\'on 2 - Ejercicio sumativo 06 }}
\author{\normalsize \color{gray}{Prof.} \color{black}{\textbf{Ismael Figueroa, Eduardo Godoy}}}
\date{\normalsize \em \today}

\begin{document}

\AddToShipoutPictureBG*{%
  \AtPageUpperLeft{\raisebox{-\height}{\includegraphics[scale=.95]{base/header.png}}}}

\maketitle

\begin{multicols}{2} \begin{flushleft} \textbf{Nombre:} \\ \vspace*{2mm} \textbf{Rut:} \\ \vspace*{2mm} \textbf{Paralelo:} \end{flushleft} \begin{center} \begin{table}[H] \begin{tabular}{p{4cm}|p{3cm}|} \arrayrulecolor{gray!50}\cline{2-2} ~ & {\em {\scriptsize \color{gray!50}{Puntaje:}}} \\ & ~ \\ ~ & \textbf{Nota:} \\ & ~ \\ \arrayrulecolor{gray!50}\cline{2-2} \end{tabular} \end{table} \end{center} \end{multicols}

\vspace*{-18mm}
\noindent
\textbf{\\Instrucciones:}
\begin{itemize}
    \item[-] El puntaje m\'aximo  es 100 puntos.
    \item[-] Tiempo m\'aximo: 90 minutos.
    \item[-] El trabajo es \underline{\textbf{individual}}. Cualquier intento de copia, ser\'a sancionado seg\'un dicta el reglamento de la carrera.
    \item[-] Se debe subir al aula virtual un archivo comprimido con el siguiente formato \\ Ej06\_<NombreApellidoEstudiante\_rut\_paralelo>.zip, dentro de el deben estar los c\'odigos fuentes requeridos.
\end{itemize}

\noindent
\textbf{Resultados de aprendizaje a evaluar:}
\begin{enumerate}
  \item Resoluci\'on de problemas utilizando Lenguaje Java.
  \item Implementación de Herencia.
  \item Uso de Polimorfismo
  \item Modelado de clases con herencia.
\end{enumerate}
\vspace{2mm}

\noindent
\textbf{Contenido:} Este Ejercicios eval\'ua los siguientes temas:

\vspace{-2mm}
\begin{table}[H]
\begin{tabular}{
    !{\color{gray!50}\vrule}l
    !{\color{gray!50}\vrule}c
    !{\color{gray!50}\vrule}c
    !{\color{gray!50}\vrule}} \arrayrulecolor{gray!50} \hline
    \multicolumn{1}{!{\color{gray!50}\vrule}c}{\multirow{2}{*}{\textbf{
        Tema
    }}} &
    \multicolumn{2}{!{\color{gray!50}\vrule}c!{\color{gray!50}\vrule}}{\textbf{
        Puntajes
    }} \\ \arrayrulecolor{gray!50}\cline{2-3} &
    \multicolumn{1}{!{\color{gray!50}\vrule}c!{\color{gray!50}\vrule}}{\textbf{
        Total
    }} &
    \multicolumn{1}{c!{\color{gray!50}\vrule}}{\textbf{
        Obtenido
    }} \\ \arrayrulecolor{gray!50} \hline
    Creaci\'on de Clase con atributos.
    & \multicolumn{1}{!{\color{gray!50}\vrule}c!{\color{gray!50}\vrule}}{\textbf{
        30 pts.
    }} & \\ \arrayrulecolor{gray!50} \hline
    Implementaci\'on de m\'etodos (comportamientos) asociados y correcto funcionamiento.
    & \multicolumn{1}{!{\color{gray!50}\vrule}c!{\color{gray!50}\vrule}}{\textbf{
        30 pts.
    }} & \\ \arrayrulecolor{gray!50} \hline
    Implementar clase instanciadora con m\'etodo main.
    & \multicolumn{1}{!{\color{gray!50}\vrule}c!{\color{gray!50}\vrule}}{\textbf{
        30 pts.
    }} & \\ \arrayrulecolor{gray!50} \hline
    Cumple con el formato y compilaci\'on correcta
    & \multicolumn{1}{!{\color{gray!50}\vrule}c!{\color{gray!50}\vrule}}{\textbf{
        10 pts.
    }} & \\ \arrayrulecolor{gray!50} \hline

\end{tabular}
\end{table}

\vspace{-7mm}
\newpage
\section{\textbf{Programaci\'on en Java (100 pts)}}
\noindent
\textbf{Plantamiento de problema: }

\begin{questions}

  \begin{itemize}
    \item  Se plantea desarrollar un programa Java que permita la gesti\'on de una empresa de alimentos que trabaja con tres tipos de productos: productos frescos, productos refrigerados y productos congelados. Todos los productos llevan esta informaci\'on com\'un: fecha de caducidad y n\'umero de lote. A su vez, cada tipo de producto lleva alguna informaci\'on espec\'ifica. Los productos frescos deben llevar la fecha de envasado y el pa\'is de origen. Los productos refrigerados deben llevar el c\'odigo del organismo de supervisi\'on alimentaria. Los productos congelados deben llevar la temperatura de congelaci\'on recomendada. Crear el c\'odigo de las clases Java implementando una relaci\'on de herencia desde la superclase \textbf{Producto} hasta las subclases \textbf{ProductoFresco}, \textbf{ProductoRefrigerado} y \textbf{ProductoCongelado}. Cada clase debe disponer de constructor, permitir establecer y recuperar el valor de sus atributos y tener un m\'etodo que permita mostrar la informaci\'on del objeto \emph{(toString)}. Crear una clase principal con el m\'etodo \emph{main} donde se cree un objeto de cada tipo y se ingresen y muestren los datos de cada uno de los objetos creados.}

  \end{itemize}

  \begin{enumerate}
    \item \emph{(20pts)} Cree un modelo de de clases (Gráfico) que represente el problema planteado.
    \item \emph{(70pts)} Codifique el modelo, según lo visto en clases. Debe ser representativo y dar solución a al problema planteado.
    \item \emph{(10pts)} Considera la clase entregada MainProductos para ejecutar el programa requerido.
  \end{enumerate}

\end{questions}

\begin{table}[H]
\centering
\begin{tabular}{
!{\color{gray!50}\vrule}p{3.9cm}
!{\color{gray!50}\vrule}p{3.6cm}
!{\color{gray!50}\vrule}p{3.6cm}
!{\color{gray!50}\vrule}p{3.6cm}
!{\color{gray!50}\vrule}} \arrayrulecolor{gray!50} \hline
    \multicolumn{4}{!{\color{gray!50}\vrule}c!{\color{gray!50}\vrule}}{\textbf{?`C\'omo ser{\'e} evaluado en este trabajo?}} \\ \arrayrulecolor{gray!50} \hline
    \textbf{\'Item} & \textbf{Logrado} & \textbf{Suficiente} & \textbf{No Logrado}\\ \arrayrulecolor{gray!50} \hline
  \newline Herencia y Polimorfismo, Creaci\'on de Clase con atributos. &
  \newline  Aplica de forma correcta 30\%  ... &
  \newline Aplica parcialmente con menos de 2 errores 20\% ... &
    \newline Aplica de forma incorrecta con 3 errores o m\'as 0-9\%\\ \arrayrulecolor{gray!50} \hline
    \newline  Implementaci\'on de m\'etodos (comportamientos) asociados y correcto funcionamiento. &
    \newline  Aplica de forma correcta 30\%  ... &
    \newline Aplica parcialmente con menos de 2 errores 20\%... &
    \newline Aplica de forma incorrecta con 3 errores o m\'as 0-8\%\\ \arrayrulecolor{gray!50} \hline
  \newline  Implementar clase instanciadora con m\'etodo main y crea instancias de objetos. &
  \newline  Aplica de forma correcta 30\%... &
    \newline  Aplica parcialmente con menos de 2 errores 15\%... &
    \newline Aplica de forma incorrecta con 3 errores o m\'as 0-9\%\\ \arrayrulecolor{gray!50} \hline
    \newline  Cumple con el formato y compilaci\'on correcta &
    \newline  Aplica de forma correcta 10\% ... &
    \newline Aplica parcialmente con menos de x errores 5\%... &
    \newline Aplica de forma incorrecta con x errores o m\'as 0-4\%\\ \arrayrulecolor{gray!50} \hline
    Total de la secci\'on &  100\% & 60\% & 0-30\%\\ \arrayrulecolor{gray!50} \hline
\end{tabular}
\label{tbl:1}
\end{table}
\vspace{-5mm}
\textbf{Nota:} En caso de que el {\'i}tem no est{\'e} presente, tiene ponderaci{\'o}n cero.

\end{document}
