\documentclass{beamer}

\mode<presentation>
{
  \usetheme{CambridgeUS}
  % \setbeamercovered{transparent}
}

\usepackage[T1]{fontenc}
\usepackage[utf8]{inputenc}
\usepackage[spanish]{babel}
\usepackage{color}
\usepackage{hyperref}
\usepackage{algorithm,algorithmic}
\usepackage{colortbl}
\usepackage{graphicx}
\usepackage{multicol}
\usepackage{enumitem}
\setitemize{itemsep=1.2em,%
  label=\usebeamerfont*{itemize item}%
  \usebeamercolor[fg]{itemize item}%
  \usebeamertemplate{itemize item}
}

\usepackage{minted}
\usepackage[skins,minted]{tcolorbox}

\newcommand{\code}[1]{\mintinline{java}{#1}}
\newcommand{\codet}[1]{\texttt{#1}}

\setminted[java]{
  linenos=true,
  fontfamily=tt,
  fontsize=\small,
  frame=leftline,
  autogobble=True,
}

\newminted[jsmall]{java}{
  fontsize=\footnotesize
  , linenos = false
  , frame=single
  , autogobble=true
}

\newtcblisting{java}[2][]{
  minted language=java,
  enhanced, listing engine=minted,
  listing only, #1, title=#2, left=2em
}

\setminted[bash]{
  linenos=true,
  fontfamily=tt,
  fontsize=\small,
  frame=leftline
}

\newtcblisting{bash}[2][]{minted language=bash,
  enhanced, listing engine=minted,
  listing only, #1, title=#2, left=2em
}

\title[\textbf{Programación 2}]{\textbf{Programación 2}}
\subtitle{Introducción al Lenguaje Java\\Clases en Java}

\author[IF-EG]
{Profesores:\\
  Ismael Figueroa -  \texttt{\small ifigueroap@gmail.com} \\
  \vspace{0.5mm} \\
  Eduardo Godoy - \texttt{\small eduardo.gl@gmail.com}
}

\institute[Universidad de Valparaíso]

\date{}

\begin{document}

\begin{frame}
  \titlepage
\end{frame}


\section{El Lenguaje de Programación Java}

\subsection{Recordatorio}

\begin{frame}
  \frametitle{¿Qué es un Objeto?}

  \begin{exampleblock}{}
    \begin{itemize}
    \item[] Cualquier cosa que vemos a nuestro alrededor. Ej: auto.
    \item[] Un objeto, en general, está compuesto por:
      \begin{itemize}
      \item \emph{Características} tales como: marca, modelo, color, etc.
      \item \emph{Comportamiento}, por ejemplo: encender, acelerar, frenar, retroceder, etc.
      \end{itemize}
    \end{itemize}
  \end{exampleblock}
\end{frame}

\begin{frame}
  \frametitle{¿Cómo se caracterizan los objetos?}

  \begin{block}{Características}
    Los objetos tienen un \emph{estado interno} que se representa
    mediante variables conocidas como \emph{atributos}.
  \end{block}

  \begin{block}{Comportamiento}
    Además, el comportamiento de los objetos se programa mediante
    \emph{métodos}, que son funciones o procedimientos que tienen acceso
    directo al estado interno del objeto.
  \end{block}

\end{frame}

\begin{frame}
  \frametitle{Clases e Instancias}

  \begin{itemize}
  \item \textbf{Clases:}    
    \begin{itemize}
    \item[] Modelo para múltiples objetos con características y
      comportamientos similares. Las clases comprenden todas las
      características de una serie particular de objetos. En
      \textbf{POO} se definen clases como un modelo abstracto de un
      objeto. Lo más parecido en C es un struct.
    \end{itemize}
    
  \item \textbf{Instancias de una clase:}
    \begin{itemize}
    \item[] Representación concreta de un objeto. Al definir una clase
      se pueden crear muchas instancias de la misma y cada instancia
      puede tener diferentes características mientras se comporte y
      reconozca como objeto de la clase. En programación estructurada
      sería una \emph{variable}.
    \end{itemize}
    
  \end{itemize}
\end{frame}

\begin{frame}
  \frametitle{Características de un Objeto} 

  \begin{itemize}
  \item \textbf{Atributos:}
    \begin{itemize}
    \item[] Características que diferencian a un objeto de otro y
      determinan la ''\emph{forma}'' de ese objeto.
      
    \item[] Los atributos se definen como \textbf{variables}. De hecho
      podrían, ser vistas como \textbf{variables globales del
        objeto}. Cada instancia de una clase puede tener diferentes
      valores para sus variables, por lo que a cada variable se le
      denomina \textbf{variable de instancia}. La clase define el tipo
      de atributo y cada instancia guarda su propio valor para ese
      atributo.
    \end{itemize}
  \end{itemize}
\end{frame}

\begin{frame}
  \frametitle{Características de un Objeto}

  \begin{itemize}
  \item \textbf{Métodos:}
    \begin{itemize}

    \item[] Comúnmente denominadas \textbf{funciones} o
      \textbf{procedimientos} definidas dentro de una clase, y que
      operan en sus instancias. Los objetos se \textbf{comunican}
      entre sí mediante el uso \textbf{llamadas a de métodos}. Se dice
      que una clase puede \textbf{llamar} al método de otra clase.
      
    \item[] Se pueden definir métodos de instancia (que operan en una
      instancia de la clase) y los métodos de clase que operan sobre
      la clase.
      
    \end{itemize}
  \end{itemize}
\end{frame}

\subsection{Programando Clases en Java}

\begin{frame}[fragile]
  \frametitle{Declarando una nueva clase}

    \begin{java}[autogobble]{Cuenta.java}
        public class Cuenta {
          
          private String nombre;

          public void
          establecerNombre(String nombre) {
            this.nombre = nombre;
          }

          public String obtenerNombre() {
            return nombre;
            // se refiere a this.nombre
          }
        }
      \end{java}
\end{frame}

\begin{frame}[fragile]
  \frametitle{Declaración de clase}

  La declaración de clase es:
  
  \begin{java}{Cuenta.java}
    public class Cuenta
  \end{java}

  \begin{enumerate}
  \item Por ahora usaremos siempre \code{public} como nivel de acceso    
  \item El nombre de la clase siempre debe comenzar con mayúscula    
  \item \textbf{El nombre define un nuevo tipo de dato}
  \item El nombre de la clase debe coincidir con el nombre del
    archivo, por lo tanto debemos trabajar en el archivo
    \codet{Cuenta.java}    
  \item El cuerpo de la clase es lo que está entre llaves, y sigue
    inmediatamente al nombre
  \end{enumerate}
  
\end{frame}

\begin{frame}[fragile]
  \frametitle{Estilo \emph{CamelCase}}

  Por convención, en Java todos los nombres de clases, variables,
  métodos, etc., se escriben usando un esquema que se conoce como
  \emph{CamelCase}:

  \begin{enumerate}
  \item Un nombre de clase parte con mayúscula, y se ocupa minúsculas
    hasta el comienzo de la siguiente palabra. Por ejemplo
    \code{CuentaAhorro}, \code{CuentaCorriente} son nombres de otras
    clases
    
  \item Los nombres de variables y métodos parten con mínusculas, y al
    cambiar palabras se usa una maýuscula en la nueva palabra. Por
    ejemplo: \code{obtenerNombre}, \code{rutTitular}, son nombres de
    método/variable que siguen la convención
  \end{enumerate}  

\end{frame}

\begin{frame}[fragile]
  \frametitle{Declaración de variables de instancia}

  La declaración de variable de instancia es:
  
  \begin{java}{Cuenta.java}
    private String nombre;
  \end{java}

  Sobre estas variables podemos decir:

  \begin{enumerate}
  \item Existen antes, durante, y después de la ejecución de los
    métodos
    
  \item Se definen en el cuerpo de la clase, pero fuera del cuerpo de
    los métodos
    
  \item Los métodos pueden---y en general es la idea que lo
    hagan---manipular los valores de las variables de instancia
    
  \item Cada \emph{instancia} tiene su propio juego de variables de
    instancia. Los valores específicos se conocen como \emph{el estado
      del objeto}
    
  \end{enumerate}
  
\end{frame}

\begin{frame}
  \frametitle{Acceso \code{private} y \code{public} para variables de
    instancia}

  Las variables de instancia también se pueden configurar con un
  modificar de acceso:

  \begin{itemize}
  \item \code{private}: la variable solamente puede utilizarse en los
    métodos de la clase que lo declara. En este caso objetos
    \code{Cuenta}
    
  \item \code{public}: la variable puede usarse en los métodos de su
    clase, pero también está disponible para otros objetos
  \end{itemize}
  
\end{frame}

\begin{frame}[fragile]
  \frametitle{Declaración de métodos}

  \begin{java}{Cuenta.java}
    public void establecerNombre(String nombre) {
      this.nombre = nombre;
    }
  \end{java}

  \begin{enumerate}
    
  \item El \emph{encabezado o firma} del método consiste en su acceso,
    tipo de retorno, parámetros, y excepciones lanzadas
    
  \item Los parámetros funcionan como variables locales dentro del
    cuerpo del método
    
  \item El parámetro \code{nombre} esconde a la variable de instancia
    que tiene el mismo nombre
    
  \item Por eso se usa el acceso \code{this.nombre} para referenciar
    explícitamente al atributo    
  \end{enumerate}  
  
\end{frame}

\begin{frame}[fragile]
  \frametitle{Usando nuestra clase en un programa}
  \setminted[java]{ fontsize=\footnotesize }
    \begin{java}{Main.java}
      public class Main {
        public static void main(String[] args) {
          // El operador new crea instancias de una clase dada
          Cuenta miCuenta = new Cuenta()
          System.out.println("Nombre cuenta: " +
                              miCuenta.obtenerNombre());

          miCuenta.establecerNombre("Juanito");
          System.out.println("Nombre cuenta: " +
                              miCuenta.obtenerNombre());
        }
      }
  \end{java}
  \setminted[java]{ fontsize=\small }
  
\end{frame}

\begin{frame}[fragile]
  \frametitle{Creando una instancia de Cuenta}

  \begin{java}{Main.java}
    Cuenta miCuenta;
  \end{java}

  \begin{exampleblock}{}
    Esta declaración define la variable \code{miCuenta} como una
    variable de tipo \code{Cuenta}. Todas las clases definen un nuevo
    tipo de dato, que es el tipo asociado a sus instancias.
  \end{exampleblock}
  
\end{frame}

\begin{frame}[fragile]
  \frametitle{Creando una instancia de Cuenta}

  \begin{java}{Main.java}
    Cuenta miCuenta = new Cuenta();
  \end{java}

  \begin{exampleblock}{}
    La variable puede inicializarse inmediatamente.\\
    El operador \code{new} \emph{se usa para construir una nueva
      instancia}.
  \end{exampleblock}
  
\end{frame}

\begin{frame}[fragile]
  \frametitle{El constructor por defecto}

  \begin{java}{Main.java}
    new Cuenta();
  \end{java}

  \begin{exampleblock}{}
    Cuando hacemos \code{new Cuenta()} estamos invocando un método
    especial que se llama \emph{constructor}, y que sirve para crear
    instancias.
  \end{exampleblock}

  \begin{exampleblock}{}
    Toda clase tiene un \emph{constructor por defecto}, que es el
    nombre de la clase y sin parámetros. Por eso se invoca como
    \code{new Cuenta ()}
  \end{exampleblock}

  \begin{exampleblock}{}
    Los constructores solamente pueden ser invocados al usar
    \code{new}
  \end{exampleblock}
  
\end{frame}

\begin{frame}[fragile]
  \frametitle{Constructores Explícitos}

  Podemos construir tantos constructores como queramos:
  \setminted[java]{ fontsize=\footnotesize }
  \begin{java}{Cuenta.java}
    public class Cuenta {
      private String nombre;

      public Cuenta(String nombre) {
        this.nombre = nombre;
      }

      /* ... idem anterior ... */
    }
    
  \end{java}
  \setminted[java]{ fontsize=\small }
  
\end{frame}

\begin{frame}[fragile]
  \frametitle{Usando constructores}
  \setminted[java]{ fontsize=\footnotesize }
    \begin{java}{Main2.java}
      public class Main2 {
        public static void main(String[] args) {
          // El operador new crea instancias de una clase dada
          Cuenta miCuenta1 = new Cuenta()
          System.out.println("Nombre cuenta 1: " +
                              miCuenta1.obtenerNombre());

          Cuenta miCuenta2 = new Cuenta("Juanito");
          System.out.println("Nombre cuenta 2: " +
                              miCuenta2.obtenerNombre());
        }
      }
  \end{java}
  \setminted[java]{ fontsize=\small }

  \begin{alertblock}{Error de compilacion!}
    Dado que definimos un constructor, Java ya no crea uno por defecto!
  \end{alertblock}
\end{frame}

\begin{frame}[fragile]
  \frametitle{Más Constructores Explícitos}

  Podemos construir tantos constructores como queramos:
  \setminted[java]{ fontsize=\footnotesize }
  \begin{java}{Cuenta.java}
    public class Cuenta {
      private String nombre;

      /* define explicitamente
      el constructor por defecto */
      public Cuenta() { /* nada */ } 
      
      public Cuenta(String nombre) {
        this.nombre = nombre;
      }

      /* ... idem anterior ... */
    }
    
  \end{java}
  \setminted[java]{ fontsize=\small }
  
\end{frame}

\begin{frame}[fragile]
  \frametitle{Agregando saldo a las cuentas}
  \begin{jsmall}
    public class Cuenta {
      private String nombre;
      private double saldo;
      public Cuenta(String nombre, double saldo) {
        this.nombre = nombre;
        if (saldo > 0.0) {
          this.saldo = saldo;
        }
      }

      public void depositar(double montoDeposito) {
        if (montoDeposito > 0.0) { saldo += montoDeposito; }        
      }

      public double obtenerSaldo() {
        return saldo;
      }
      /* ... idem anterior ... */
    }    
  \end{jsmall}  
\end{frame}

%%%%%%%%%%%%%%%%%%%

\subsection{Java y Principios de Orientación a Objetos

\begin{frame}
  \frametitle{Encapsulamiento}

    \begin{itemize}
      
    \item Técnica para asociar en un mismo lugar \emph{el estado
        interno} y \emph{el comportamiento} de los objetos

    \item En Java el encapsulamiento se realiza mediante la definición
      de clases      

    \end{itemize}
  \end{frame}		
  
\end{frame}

\begin{frame}
  \frametitle{Ocultamiento de Información}

    \begin{itemize}
      
    \item Es la capacidad de ocultar los detalles internos del
      comportamiento de una clase y exponer sólo los detalles que sean
      necesarios para el resto del sistema

    \item En Java esto se hace a través de los atributos de control de
      acceso: \code{private} y \code{public}

    \end{itemize}
  \end{frame}		
  
\end{frame}



\begin{frame}
  \frametitle{Preguntas}
  \hspace{4cm}\huge{Preguntas?}  
\end{frame}

\end{document}