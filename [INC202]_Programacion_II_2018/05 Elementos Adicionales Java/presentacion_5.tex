\documentclass{beamer}

\mode<presentation>
{
  \usetheme{CambridgeUS}
  % \setbeamercovered{transparent}
}

\usepackage[T1]{fontenc}
\usepackage[utf8]{inputenc}
\usepackage[spanish]{babel}
\usepackage{color}
\usepackage{hyperref}
\usepackage{algorithm,algorithmic}
\usepackage{colortbl}
\usepackage{graphicx}
\usepackage{multicol}
\usepackage{enumitem}
\setitemize{itemsep=1.2em,%
  label=\usebeamerfont*{itemize item}%
  \usebeamercolor[fg]{itemize item}%
  \usebeamertemplate{itemize item}
}

\usepackage{minted}
\usepackage[skins,minted]{tcolorbox}

\newcommand{\code}[1]{\mintinline{java}{#1}}
\newcommand{\codet}[1]{\texttt{#1}}

\setminted[java]{
  linenos=true,
  fontfamily=tt,
  fontsize=\small,
  frame=leftline,
  autogobble=True,
}

\newminted[jsmall]{java}{
  fontsize=\footnotesize
  , linenos = false
  , frame=single
  , autogobble=true
}

\newtcblisting{java}[2][]{
  minted language=java,
  enhanced, listing engine=minted,
  listing only, #1, title=#2, left=2em
}

\setminted[bash]{
  linenos=true,
  fontfamily=tt,
  fontsize=\small,
  frame=leftline
}

\newtcblisting{bash}[2][]{minted language=bash,
  enhanced, listing engine=minted,
  listing only, #1, title=#2, left=2em
}

\title[\textbf{Programación 2}]{\textbf{Programación 2}}
\subtitle{Elementos Adicionales de Programación Básica en Java}

\author[IF-EG]
{Profesores:\\
  Ismael Figueroa -  \texttt{\small ifigueroap@gmail.com} \\
  \vspace{0.5mm} \\
  Eduardo Godoy - \texttt{\small eduardo.gl@gmail.com}
}

\institute[Universidad de Valparaíso]

\date{}

\begin{document}

\begin{frame}
  \titlepage
\end{frame}

\section{Métodos y Atributos \code{static}}

\begin{frame}
  \frametitle{¿Qué es un método estático?}

  \begin{block}{}
    En Java los atributos y los métodos están asociados por defecto
    con \emph{instancias de una clase}. Esto significa que deben ser
    invocados desde una instancia, por ejemplo
    \codet{auto.acelerar(10)} ...
  \end{block}

  Pero a veces es deseable y/o necesario asociar métodos \emph{directamente a una clase}:

  \begin{itemize}
    
  \item Métodos utilitarios: por ejemplo un método \codet{millasAKilometros}, no tiene por
    qué estar asociado a una instancia, es algo general en el ámbito
    de los autos.
    
  \item Si el método en realidad no ocupa ninguna variable de
    instancia...
    
  \end{itemize}

\end{frame}

\begin{frame}[fragile]
  \frametitle{Métodos Estáticos vs No-Estáticos}

  \begin{columns}
    \begin{column}{0.5\textwidth}
      \begin{jsmall}
        public class AutoEst {
          String marca;
          int agno;

          public static double
          millasAKm(double m) {
            return m * 1.60934;
          }          
        }

        public class Auto {
          String marca;
          int agno;

          public double
          millasAKm(double m) {
            return m * 1.60934;
          }          
        }
     \end{jsmall}
    \end{column}
    % 
    \begin{column}{0.5\textwidth}
      
      \begin{jsmall}
        public class Main {
          public static void
          main(String[] args) {

            // Quiero calcular 102
            // 102 millas a KM

            // Con método no estático:
            System.out.println(
                new Auto().millasAKm(102));

            // Con método estático:
            System.out.println(
                AutoEst.millasAKm(102));
          }
        }
      \end{jsmall}
      
    \end{column}
  \end{columns}

\end{frame}

\section{Arreglos}

\begin{frame}
  \frametitle{¿Qué es un arreglo?}

  \begin{block}{}
    Un arreglo es una colección secuencial de elementos. En Java los
    arreglos pueden ser estáticos o dinámicos, y todos los elementos
    deben tener el mismo tipo de dato.
  \end{block}

  En general los pasos para trabajar con arreglos son:
  \begin{itemize}
  \item Declarar una variable para hacer referencia al arreglo
  \item Crear o instanciar un nuevo arreglo y asignarlo a la variable
    recién declarada
  \item Almacenar, actualizar, manipular los valores en el arreglo
  \end{itemize}

\end{frame}


\subsection{Arreglos Estáticos}

\begin{frame}[fragile]
  \frametitle{Arreglos Estáticos}

  \begin{columns}
    \begin{column}{0.4\textwidth}
      \begin{small}
        \begin{itemize}
        \item Tienen un largo fijo determinado en su creación.
          
        \item Los arreglos están indexados por número, desde el número 0
          en adelante.
          
        \item Un arreglo de elementos de tipo \code{T} se escribe
          \code{T[]}, y se pueden inicializar directamente, o construir
          usando \code{new}.
          
        \end{itemize}
      \end{small}
    \end{column}
    % 
    \begin{column}{0.6\textwidth}
      
      \begin{jsmall}
        public class Main {
          public static void main(String[] args) {

            String[] nombres = { "P", "J", "D" };
            double[] notas   = new double[3];

            notas[0] = 4.2;
            notas[1] = 7.0;
            notas[2] = 6.6;

            System.out.println(nombres.length);
            System.out.println(nombres);
            System.out.println(notas);
          }
        }
      \end{jsmall}
      
    \end{column}
  \end{columns}

\end{frame}

\begin{frame}[fragile]
  \frametitle{Propiedades Clave}

  \begin{columns}
    \begin{column}{0.4\textwidth}
      \begin{small}
        \begin{itemize}

        \item El acceso a la posición \code{i} del arreglo \code{a} se
          escribe \code{a[i]}.
          
        \item Los arreglos son un objeto bastante simple, que tiene como
          principal propiedad su largo.
          
        \item El largo está almacenado en la propiedad \code{length}, la
          cual se define en la construcción del arreglo.
          
        \end{itemize}
      \end{small}
    \end{column}
    % 
    \begin{column}{0.6\textwidth}
      \begin{jsmall}
        public class Main {
          public static void main(String[] args) {
            
            String[] nombres = { "P", "J", "D" };
            double[] notas   = new double[3];
            
            notas[0] = 4.2;
            notas[1] = 7.0;
            notas[2] = 6.6;
            
            for(int i=0; i < notas.length; i++) {
              System.out.println(
              "N[" + i + "]: " + notas[i]);
            }
          }
        }
      \end{jsmall}      
    \end{column}
  \end{columns}

\end{frame}

\begin{frame}[fragile]
  \frametitle{La clase utilitaria \codet{java.util.Arrays}}

  \begin{columns}
    \begin{column}{0.4\textwidth}
      \begin{small}
        \begin{itemize}

        \item La clase \code{java.util.Arrays} contiene muchos métodos
          utilitarios para la manipulación de arreglos estáticos.
          
        \item La clase tiene solamente \emph{métodos estáticos} por lo
          que no es necesario tener una instancia---sólo se llaman los métodos.
          
        \item Como ejemplo, veamos los métodos \code{fill},
          \code{sort}, y \code{toString}.
          
        \end{itemize}
      \end{small}
    \end{column}
    % 
    \begin{column}{0.6\textwidth}
      \begin{jsmall}
        import java.util.Arrays;
        
        public class Main {
          public static void main(String[] args) {            
            double[] notas   = new double[10];
            // Rellenamos con valor 3.3
            Arrays.fill(notas, 3.3);
            
            // Imprimimos representacion en String
            System.out.println(
                Arrays.toString(notas));

            // Modificamos un valor y ordenamos
            notas[9] = 0; Arrays.sort(notas);

            System.out.println(
                Arrays.toString(notas));
          }
        }
      \end{jsmall}      
    \end{column}
  \end{columns}

\end{frame}

\subsection{Arreglos Dinámicos}

\begin{frame}[fragile]
  \frametitle{Arreglos Dinámicos}

  \begin{columns}
    \begin{column}{0.4\textwidth}
      \begin{footnotesize}
        \begin{itemize}
        \item Tienen un largo variable, que se ajusta automáticamente
          a medida que se agregan o eliminan elementos.
          
        \item Los elementos siguen estando indexados por posición
          desde el número 0 en adelante.
          
        \item En Java corresponden a la implementación de la interface
          \code{List}, que representa una colección ordenada de
          elementos.
          
        \item La implementación más usada es la clase \code{ArrayList}
          
        \end{itemize}
      \end{footnotesize}
    \end{column}
    % 
    \begin{column}{0.6\textwidth}
      
      \begin{jsmall}
        import java.util.List;
        import java.util.ArrayList;
        
        public class Main {
          public static void main(String[] args) {
            List<String> nombres =
                new ArrayList<String>();

            nombres.add("P");
            nombres.add("J");
            nombres.add("D");

            System.out.println(nombres);
          }
        }
      \end{jsmall}
      
    \end{column}
  \end{columns}

\end{frame}

\begin{frame}[fragile]
  \frametitle{Métodos Clave para Listas\footnote{\url{https://docs.oracle.com/javase/8/docs/api/java/util/List.html}}}

  \begin{columns}
    \begin{column}{0.5\textwidth}
      \begin{footnotesize}
        \begin{itemize}
        \item \code{add}: agrega un elemento al final de la lista, o
          en una posición específica dada
          
        \item \code{contains}: retorna un booleano indicando si un
          elemento pertenece o no a la lista
          
        \item \code{get}: retorna el elemento que está en una posición
          dada de la lista
          
        \item \code{isEmpty}: retorna verdadero si la lista no tiene
          elementos
          
        \item \code{size}: retorna el tamaño de la lista
          
        \end{itemize}
      \end{footnotesize}
    \end{column}
    % 
    \begin{column}{0.5\textwidth}
      \begin{footnotesize}
      \begin{itemize}
        \item \code{indexOf}: retorna el índice del primer elemento
          que es igual al dado como parámetro, o retorna \code{-1} si
          no hay
          
        \item \code{remove}: remueve el elemento en la posición dada
          de la lista
          
        \item \code{subList}: retorna una nueva lista que corresponde
          al segmento dado como parámetro
          
        \item \code{toArray}: retorna un arreglo estático con los
          elementos de la lista
          
        \end{itemize}
       \end{footnotesize}
    \end{column}
  \end{columns}

\end{frame}

\begin{frame}[fragile]
  \frametitle{Iteración sobre una Lista}

  \begin{columns}
    \begin{column}{0.4\textwidth}
        \begin{itemize}
        \item La interface \code{List} es una colección
          \emph{iterable}...
          
        \item Por lo tanto se puede usar el constructo \code{for}
          especial para recorrer los elementos de la lista, sin
          preocuparnos por el largo
          
        \end{itemize}
    \end{column}
    % 
    \begin{column}{0.6\textwidth}
      
      \begin{jsmall}
        import java.util.List;
        import java.util.ArrayList;
        
        public class Main {
          public static void main(String[] args) {
            List<String> nombres =
                new ArrayList<String>();

            nombres.add("P");
            nombres.add("J");
            nombres.add("D");

            for(String n : nombres) {
              System.out.println(n);
            }
          }
        }
      \end{jsmall}
      
    \end{column}
  \end{columns}

\end{frame}

\subsection{El Parámetro \code{args}}

\begin{frame}[fragile]
  \frametitle{Parámetros desde Línea de Comandos}
  \begin{itemize}
    
  \item El método \code{main} toma obligatoriamente como argumento un
    valor \code{String[] args}
    
  \item Ahora sabemos que eso es un arreglo estático. Haga un programa
    que imprima sus valores...
    
  \item El arreglo toma sus valores de los parámetros usados en la
    invocación del programa desde la línea de comandos. Por ejemplo:

    \begin{jsmall}
      > java Main hola 123
    \end{jsmall}
    
  \end{itemize}

\end{frame}

\subsection{La Clase \codet{Object} y el método \codet{toString}}

\begin{frame}
  \frametitle{La Clase \codet{Object}\footnote{\url{https://docs.oracle.com/javase/8/docs/api/java/lang/Object.html}}}

  \begin{block}{}
    La clase \codet{Object} está en la raíz de la jerarquía de
    herencia de Java. Todas las clases heredan, directa o
    indirectamente, desde \codet{Object}. Esta clase representa el
    tipo de dato más general posible.
  \end{block}
  
\end{frame}

\begin{frame}[fragile]
  \frametitle{El método \codet{toString}}

  \begin{columns}
    \begin{column}{0.35\textwidth}
      \begin{small}
        \begin{itemize}
        \item La clase \codet{Object} define el métod
          \codet{toString}, el cual puede ser sobre-escrito por las
          clases descendientes...
          
        \item Esto es útil para imprimir información relevante sobre
          nuestras propias clases, ya que hay muchos lugares donde se
          invoca implícitamente el método \codet{toString}
          
        \end{itemize}
       \end{small}
    \end{column}
    % 
    \begin{column}{0.65\textwidth}
      
      \begin{jsmall}        
        public class Auto {
          String marca;
          int agno;

          public Auto(String m, int a) {
            marca = m;
            agno = a,
          }

          public String toString() {
            return "Auto m:" + marca + " a:" + agno;
          }
        }
      \end{jsmall}
      
    \end{column}
  \end{columns}  
  
\end{frame}



\begin{frame}
  \frametitle{Preguntas}
  \hspace{4cm}\huge{Preguntas?}  
\end{frame}

\end{document}